
\documentclass[english,11pt,twoside,toc=bib,toc=idx]{scrreprt}
\usepackage{scrhack}

% --- Version ---

\newcommand{\Version}{0.29}

% --- Packages ---

\usepackage{babel}
\usepackage{graphicx}
\usepackage{array}
\usepackage{tabularx}
\usepackage{multirow}
\usepackage{threeparttable}
\usepackage{longtable}
\usepackage{booktabs}
\usepackage{makeidx}
\usepackage{framed}
\usepackage{alltt}
\usepackage{barracuda}
\usepackage{underscore}
\usepackage{longtable}
% \usepackage{filecontents}


% \begin{filecontents}{beforetable}
%   \begin{supertabular}{cc}
%   \end{filecontents}
  
%   \begin{filecontents}{aftertable}
%     \\
%   \end{supertabular}
% \end{filecontents}
%--- Indexing ---

\makeindex

% --- Fonts ---

\ifdefined\ifHtml\else

\usepackage{unicode-math}
\usepackage{fontspec}

\usepackage{microtype}

% Use the IBM Plex font family.

\setmainfont{IBMPlexSerif-Regular.otf}[
  BoldFont = IBMPlexSerif-Bold.otf,
  ItalicFont = IBMPlexSerif-Italic.otf,
  BoldItalicFont = IBMPlexSerif-BoldItalic.otf]
\setsansfont{IBM Plex Sans}
% \setmonofont{IBM Plex Sans Condensed}
\setmonofont{IBM Plex Mono }


\setkomafont{captionlabel}{%
  \usekomafont{descriptionlabel}%
}

% There's no math font for Plex yet, so choose a font with similar
% proportions and scale it to match the base font's uppercase letters.
% Once a math font for Plex exists (check https://github.com/IBM/plex), it
% should be activated here instead.

\setmathfont[Scale=MatchUppercase]{TeX Gyre DejaVu Math}

\fi

% --- Layout definitions ---

\KOMAoptions{paper=a4}
\KOMAoptions{DIV=13,BCOR=1cm}

\setlength{\parskip}{\smallskipamount}
\setlength{\parindent}{0pt}
\setcounter{secnumdepth}{3}
\setcounter{tocdepth}{3}
\DeclareTOCStyleEntry[numwidth=2em]{default}{chapter}
\DeclareTOCStyleEntry[indent=2em,numwidth=3em]{default}{section}
\DeclareTOCStyleEntry[indent=5em,numwidth=3.5em]{default}{subsection}
\DeclareTOCStyleEntry[indent=8.5em,numwidth=4em]{default}{subsubsection}
\DeclareTOCStyleEntry[numwidth=3em]{default}{figure}
\DeclareTOCStyleEntry[numwidth=3em]{default}{table}

% --- Custom commands ---

\newcommand{\jumplabel}[1]{\textsf{‹#1›}}
\newcommand{\stackit}[2][l]{\setlength{\tabcolsep}{0mm}\begin{tabular}{#1}%
    #2\end{tabular}}

\ifzseries
\newcommand{\NBITS}{64}
\newcommand{\ADDRBITS}{64}
\newcommand{\NBYTES}{8}
\newcommand{\STACKSIZE}{160}
\newcommand{\ABINAME}{s390x}
\newcommand{\ARCH}{z/\kern-1pt Ar\-chi\-tec\-ture}
\newcommand{\ARCHarch}{\ARCH}
\newcommand{\aARCH}{a \ARCH}
\newcommand{\theARCH}{the \ARCH}
\else
\newcommand{\NBITS}{32}
\newcommand{\ADDRBITS}{31}
\newcommand{\NBYTES}{4}
\newcommand{\STACKSIZE}{96}
\newcommand{\ABINAME}{s390}
\newcommand{\ARCH}{ESA/390}
\newcommand{\ARCHarch}{the \ARCH{} ar\-chi\-tec\-ture}
\newcommand{\aARCH}{an \ARCH}
\newcommand{\theARCH}{\ARCHarch}
\fi
\newcommand{\crexx}{c\textsc{Rexx}}
\newcommand{\keyword}[1]{\texttt{#1}}
\newcommand{\code}[1]{\texttt{#1}}

\newenvironment{DIFnomarkup}{}{} % For latexdiff

% --- Listings ---

\usepackage{xcolor}
\definecolor{lstnumbers}{rgb}{0.5,0.5,0.5}
\definecolor{shadecolor}{rgb}{0.9,0.9,0.9}
\definecolor{nrblue}{RGB}{38,139,210}
\definecolor{nrgreen}{RGB}{65,133,153}
\definecolor{nrcyan}{RGB}{42,161,152 }
\definecolor{nrorange}{RGB}{203 ,75,22}
\definecolor{nrgrey}{RGB}{101,123,131}


\usepackage{listings}
\lstset{%
    basicstyle=\fontsize{10}{10}\fontspec{Hack},emph={label},       
  numberstyle={\sffamily\footnotesize\color{lstnumbers}},
  showstringspaces=false,
  captionpos=b,
  abovecaptionskip=2ex,
  language=C}
\lstdefinestyle{float}{%
  frame=single,
  float=tbp}
\lstdefinestyle{embed}{%
  frame=single}
\lstdefinestyle{long}{%
  frame=single,
  captionpos=t}
\lstdefinestyle{short}{%
  aboveskip=-1.2ex,
  belowskip=-\baselineskip}
\lstdefinelanguage{simpleasm}{%
  comment=[l]\#,
  string=[b]{"}}

\lstdefinelanguage{crexx}
{morekeywords={abstract,adapter,binary,case,catch,class,constant,dependent,deprecated,digits,do,else,end,engineering,extends,final,finally,for,forever,if,implements,indirect,import,indirect,inheritable,interface,iterate,label,leave,loop,method,native,nop,numeric,options,otherwise,over,package,parent,parse,private,properties,protect,public,return,returns,rexx,say,scientific,set,digits,form,select,shared,signal,signals,sourceline,static,super,then,this,until,used,upper,volatile,when,where,while},
sensitive=false,
extendedchars=true,
morecomment=[s]={/*}{*/},
morecomment=[l]{--},
morecomment=[s]{/**}{*/},
morestring=[b]",
morestring=[d]",
morestring=[b]',
morestring=[d]'}

\lstset{language=crexx,
  captionpos=none,
  tabsize=3,
  alsolanguage=Rexx,
  keywordstyle=\color{nrorange},
  commentstyle=\color{nrgrey},
  stringstyle=\color{nrgreen},
  numbers=none,
  numberstyle=\tiny,
  numbersep=5pt,
  breaklines=true,
  showstringspaces=false,
  index=[1][keywords],
  columns=fixed,
  basicstyle=\fontsize{10}{10}\fontspec{Hack},emph={label}
}

% --- Hyperref ---

\usepackage[unicode=true,pdfusetitle,
bookmarks=true,bookmarksnumbered=true,bookmarksopen=true,bookmarksopenlevel=1,
breaklinks=true,pdfborder={0 0 0},backref=false,colorlinks=true]
{hyperref}
\hypersetup{linkcolor=black,pdfstartview={XYZ 0 0 1}}

\usepackage{cleveref}

% Fixed definition of \refstepcounter@noarg from cleveref.sty.  This
% version fixes an endless loop with tex4ht when used together with
% listings and hyperref.

\makeatletter
\def\refstepcounter@noarg#1{%
  \cref@old@refstepcounter{#1}%
  \cref@constructprefix{#1}{\cref@result}%
  \@ifundefined{cref@#1@alias}%
    {\def\@tempa{#1}}%
    {\def\@tempa{\csname cref@#1@alias\endcsname}}%
  \protected@xdef\cref@currentlabel{%   <-- xdef instead of edef
    [\@tempa][\arabic{#1}][\cref@result]%
    \csname p@#1\endcsname\csname the#1\endcsname}}%
\makeatother

% Call out figures with "figure" instead of "fig.".

\crefname{figure}{figure}{figures}
\Crefname{figure}{Figure}{Figures}

% --- TikZ support ---

\newif\ifSkipTikZ
\ifdefined\ifHtml
\SkipTikZtrue
\else
\SkipTikZfalse
\fi

\ifSkipTikZ\else          % BEGIN skip TikZ

\usepackage{tikz}
\usetikzlibrary{arrows}
\usetikzlibrary{decorations.pathreplacing}
\usetikzlibrary{fit}
\usetikzlibrary{patterns}
\usetikzlibrary{positioning}
\usetikzlibrary{shadows}
\usetikzlibrary{shapes}

% --- PGF layer definitions ---

\pgfdeclarelayer{background}

% --- Misc PGF/TikZ style definitions ---

\tikzset{memory layout/.style={fill=yellow!10,draw}}
\tikzset{inactive layout/.style={text=black!60}}

% --- Style definitions for bit layout charts ---

\tikzset{bitnum/.style={text=black!60, font={\footnotesize\sffamily}}}
\tikzset{bytenum/.style={font={\footnotesize}}}
\tikzset{bitchart box/.style={fill=blue!7, draw}}
\tikzset{bitfield label/.style={}}
\tikzset{bitfield padding/.style={draw=black, very thick,
    line cap=round, shorten <=0.5ex, shorten >=0.5ex,
    dash pattern=on 0pt off 6pt, dash phase=3pt}}
\tikzset{>=latex}

% --- Helpers for bit layout charts ---

\newcommand{\bitchartsep} {
  \draw (0, 0) -- (last SE);
}

\makeatletter

\def\one@bitchartfield#1#2{%
  \pgfmathsetmacro{\xlow}{\xprev-\xfirst}
  \pgfmathsetmacro{\xhigh}{#1-\xfirst}
  \pgfmathtruncatemacro{\xhighminus}{#1-1}
  \path (\xlow, 0) node [bitnum, above right] {\xprev};
  \ifx\\#2\\
  \path [bitfield padding] (\xlow, 0.5) -- (\xhigh, 0.5);
  \else
  \path (\xlow, 0) -- node [bitfield label,pos=0.5]
  {\texttt{#2}} (\xhigh, 1);
  \fi
  \path (\xhigh, 0) coordinate (last SE) {}
  node [bitnum, above left] {\xhighminus};
  \edef\xprev{#1}}

\def\bitchartfield@n#1/#2/#3{%
  \one@bitchartfield{#1}{#2}
  \draw (\xlow, 0) -- (\xlow, 1);
  \if,#3\let\next\bitchartfield@n\else\let\next\egroup\fi\next}

\def\bitchartfields#1 #2/#3/#4{%
  \bgroup
  \edef\xfirst{#1}
  \let\xprev\xfirst
  \one@bitchartfield{#2}{#3}
  \if,#4\let\next\bitchartfield@n\else\let\next\egroup\fi\next}

\def\one@bytechartfield#1#2{%
  \pgfmathsetmacro{\xlow}{8*(\prevbyte-\firstbyte)}
  \pgfmathsetmacro{\xhigh}{8*(#1-\firstbyte)}
  \path (\xlow, 1) node [bytenum, below right] (bnum) {\prevbyte};
  \ifx\\#2\\
  \path [bitfield padding] (\xhigh, 0.5) coordinate (m){} -- (bnum|-m);
  \else
  \path (\xlow, 0) -- node [bitfield label,pos=0.5]
  {\texttt{#2}} (\xhigh, 1);
  \fi
  \path (\xhigh, 0) coordinate (last SE) {};
  \edef\prevbyte{#1}}

\def\bytechartfield@n#1/#2/#3{%
  \one@bytechartfield{#1}{#2}
  \draw (\xlow, 1) -- (\xlow, 0);
  \if,#3\let\next\bytechartfield@n\else\let\next\egroup\fi\next}

\def\bytechartfields#1 #2/#3/#4{%
  \bgroup
  \edef\firstbyte{#1}
  \edef\prevbyte{#1}
  \one@bytechartfield{#2}{#3}
  \if,#4\let\next\bytechartfield@n\else\let\next\egroup\fi\next}

\newcommand{\bitchartbytes}[2]{%
  \bgroup
  \edef\first@byte{#1}
  \foreach \b in {#2} {%
    \pgfmathsetmacro{\bit@}{8*(\b-#1)}
    \path (\bit@, 1) node [bytenum, below right] {\b};
    \ifx\b\first@byte\else
    \path (\bit@, 1) -- (\bit@, 0);
    \fi
  }
  \egroup
}

\makeatother

\fi                             % END skip TikZ

% Hyphenation rules for compound words.

\hyphenation{big=endi-an}
\hyphenation{cia=rel-a-tive}
\hyphenation{float-ing=point}
\hyphenation{func-tion=call-ing}
\hyphenation{lit-tle=endi-an}
\hyphenation{non=ze-ro}
\hyphenation{pa-ram-e-ter=pass-ing}
\hyphenation{po-si-tion=inde-pen-dence}
\hyphenation{po-si-tion=inde-pen-dent}
\hyphenation{sig-nal=han-dling}
\hyphenation{sign=ex-tend-ed}
\hyphenation{sys-tem=sup-plied}
\hyphenation{two=el-e-ment}
\hyphenation{ze-ro=ex-tend-ed}

% ------------------------------------------------------------
\begin{document}

\newcommand{\myTitle}{\crexx{} Principles of Operation}

\begin{DIFnomarkup}
\title{\myTitle}
\subtitle{Version \Version}

\author{Mike Grossmann \and Peter Jacobs \and René Vincent Jansen%
  \and Adrian Sutherland}
\publishers{The \textsc{Rexx} Language Association\textregistered}

\lowertitleback{%
  \noindent \textbf{\myTitle}

  \noindent Version \Version\\
  \\
  ISBN 978-90-819090-0-6\\
  \barracuda{ean-13}{9789081909006}\\
  \\
  \noindent © Copyright RexxLA 2021,2022

  \medskip
  }

\maketitle
\end{DIFnomarkup}

\tableofcontents
\listoffigures
\listoftables
\lstlistoflistings

\chapter*{About This Book}
The \crexx{} language is a further development, and version of the
\textsc{Rexx} language\footnote{Cowlishaw, 1979}. This book aims to
document the workings of this implementation and serves as reference
for users and implementors alike.

% The \ABINAME{} supplement to the Executable and Linkage
% Format Application Binary Interface (or ELF ABI) defines a system
% interface for compiled application programs.  Its purpose is to
% establish a standard binary interface for application programs on
% Linux\textregistered{} for \ARCH{}\textregistered{} systems.

% This book is a supplement to the generic ``System V Application Binary
% Interface'' and should be read in conjunction with it.

\section*{History}

\begin{description}
\item[mvp] This version documents the Minimally Viable Product
  release, Q1 2022 and is intended for developers only. It documents
  the RXAS instruction set and \crexx{} level B, which is a typed
  subset of Classic Rexx.
\item[f29] First version, Git feature F0029]
\end{description}

\chapter{Low-Level System Information}
The execution environment of a \crexx{} program is a
threaded\footnote{an alternative, non-threaded executable is available
  under the \emph{rxbvm} name} virtual
machine that is designed for optimal performance. This virtual
machine, implemented in the \emph{rxvm} executable, executes
machine instructions produced by the \emph{rxc} \crexx{}
compiler, or written by hand, assembled into an .rxbin binary file by
the \emph{rxas} assembler.

\section{REXX VM Specification}

\emph{Page Status: This page is work in progress, incomplete, inconsistent and full of errors ... read with care!}

\subsection{Acknowledgement}

Much of this work is based on the excellent \textquotedbl{}Language Implementation Patterns\textquotedbl{}
by Terence Parr.

https://pragprog.com/titles/tpdsl/language-implementation-patterns/

\subsection{REXX VM Components - Summary}

The REXX interpreter simulates a computer with the following components:

\begin{itemize}
\item Code memory: This 64 or 32-bit unsigned integer array holds a program's \textquotedbl{}bytecode\textquotedbl{} instructions
(bytecodes plus operands). Addresses are integers.

\item ip register: The instruction pointer is a special-purpose ``register'' that
points into code memory at the next instruction to execute.

\item CPU: To execute instructions, the interpreter has a simulated CPU that
conceptually amounts to a loop around a giant ``switch on bytecode'' statement.
This is called the instruction dispatcher. We will use a Threaded VM instead.

\item Constant pool: Anything that we can't store as an integer operand
goes into the constant pool. This includes strings, arbitrary precision numbers,
and function symbols. Instructions like \texttt{say {[}STRING{]}} and use an index into
the constant pool instead of the actual operand.

\item Function call stack: The interpreter has a stack to hold function call return
addresses as well as parameters and local variables.

\item fp register: A special-purpose register called the frame pointer that points
to the top of the function call stack. StackFrame represents the information
we need to invoke functions.

\item An infinite and regular register set per function call. Functions can access
any element in the register array, whereas a stack can only access elements
at the end. Registers hold integer, float, string and object values, plus flags
that can be used to dynamically store which value formats are valid.

\end{itemize}

The following capabilities will be implemented in REXX Level B - REXX called by the virtual machine

\begin{itemize}
\item Variable Pool: Holds slots for variables, each function/procedure has its own
pool, and a function/procedure can \textquotedbl{}link\textquotedbl{} to variables from the parents pool (to
support dynamic expose scenarios).
The memory slots can point at Integer, Float, String, and struct instances.
Variables are accessed via a linked register. This linking can be done statically
or dynamically via a variable search capability

\item Runtime libraries that care used as \textquotedbl{}exits\textquotedbl{}; for example, REXX code to format error messages.

\end{itemize}

\subsection{REXX Machine Architecture}

cREXX will target a bytecode interpreter tuned for the specific needs of
the REXX language. This has two advantages

\begin{itemize}
\item Provides platform independence and portability

\item Provides a much simpler target for the cREXX compile compared to real CPU\textquotesingle{}s
instruction sets (which after all have to be implemented in hardware)

\end{itemize}

A bytecode interpreter is like a real processor. The biggest difference is
that bytecode instruction sets are much simpler. Also, for example, we can
assume there are an infinite number of registers.

The cREXX VM is a register based bytecode interpreter that uses Threading
and Super-Instructions, meaning:

\subsubsection{Register Based}

A Register Based VM instructions use registers. The cREXX VM gives each
stack frame an ``infinite'' set of registers.

\subsubsection{Threading}

Threading means that the \textquotedbl{}bytecode\textquotedbl{} loader converts instruction opcodes to
the address of the subroutine that emulates the instruction. This reduces
an indirection to each instruction (i.e. instead of
\texttt{switch(opcode) \{case ...\})} we can do \texttt{goto opcode\_address}).

In addition, the code to dispatch to the next instruction is repeated at the
end of each instruction subroutine rather than having a single dispatcher.
This allows the real CPU\textquotesingle{}s pipelining logic to work better.

The following two examples demonstrate the concept.

\paragraph{Traditional Interpreter}

\begin{verbatim}
char code[] = {
  ICONST_1, ICONST_2,
  IADD, ...
}
char *pc = code;

/* dispatch loop */
while(true) {
  switch(*pc++) {
    case ICONST_1: *++sp = 1; break;
    case ICONST_2: *++sp = 2; break;
    case IADD:
      sp[-1] += *sp; --sp; break;
   ...
  }
}
\end{verbatim}

\paragraph{Equivalent Threaded Interpreter}

\begin{verbatim}
void *code[] = {
  &&ICONST_1, &&ICONST_2,
  &&IADD, ...
}
void **pc = code;

/* implementations */
goto **(pc);

ICONST_1: pc++; *++sp = 1; goto **(pc);
ICONST_2: pc++; *++sp = 2; goto **(pc);
IADD:
  pc++; sp[-1] += *sp; --sp; goto **(pc);
...
\end{verbatim}

Note how the next instruction pointer \texttt{pc} is calculated up front (\texttt{pc++}) before
the instruction logic. This is to allow the CPU pipeline to work; by the
time the \texttt{goto} is being decoded the value of \texttt{pc} \emph{should} have completed.

\subsubsection{Super Instructions}

In any interpreter the dispatch to the next instruction is an overhead. The
fewer instructions needed to execute logic then the less dispatching overhead.
Moreover a complex instruction\textquotesingle{}s native implementation code running linearly
maximises the effectiveness of the real CPU\textquotesingle{}s pipeline.

We therefore will have a large number of instructions to cater for common
sequences, for example a decrement (decr) instruction might often be followed by a
branch if not zero instruction (brnz), e.g. for a loop - a super-instruction combines
these two instructions to one (e.g. decrbrnz).

\subsubsection{Low-level Function Instructions}

The VM provides low-level functions as instruction, for example covering
aspects like variable manipulation (like substr()), IO functions,
Environment access, and virtual hardware like the timer. This ensures that:

\begin{itemize}
\item Platform independence / platform drivers / porting etc. is
achieved only by implementing these instructions for each platform.

\item Higher level functionality can all be implemented in REXX.

\end{itemize}

\subsubsection{Instruction coding}

There is a balance here - memory usage verses performance, made more complicated
because reducing memory usage also has performance benefits. Tests will be
needed across different CPUs types to validate our optimisations.

\paragraph{Compiled Mode}

REXX VM binary code can be generated for storage (i.e. compiled) to be loaded
and run later. In this case the code will have to be loaded and then threaded.

The threading process converts the op\_codes into addresses, branches to addresses
constant indexes to addresses - everything is converted into real addresses
of the machine running the REXX VM, so that when executed (for example) data is
accessed via its real address with no lookups/indexes needed. This needs to
be all done at load time as the exact real addresses will be different on each machine
and for each run (modern OS\textquotesingle{}s randomise load addresses).

We will investigate if there is value in compressing the saved REXX VM binary
to speed load times. Java uses ZIP for this and cREXX could do the same - and/or
we could pack the instruction coding.

In the short term we will save it as 32 or 64 bit instructions (see following).

This means that the threading process can simply
replace the opcode with the opcode address - the 32 or 64 bit memory address can
fit in the 32 or 64 bit int instruction code location, we overwrite the opcode
integer with the opcode function address.

\paragraph{Interpreter Mode}

When interpreting a REXX program the compiler will emit code that will be
executed there and then. In principle all the real addresses will be known and
this means that the compiler can generate threaded code from the get-go. This
way the loading / threading stage can be skipped.

Initial versions will only use the Compiler Mode while the Interpreter Mode is
checked for feasibility and to see if there is a noticeable performance gain. One
issue we may have is that the loader/threader may be needed anyway for other reasons
like late binding/linking of libraries.

\paragraph{Endianness}

Where the compiler and runtime are known to be on the same machine then the
Endianness and float format will be machine specific.

Where the REXX VM is to be saved then big-endianness / \textquotedbl{}network order\textquotedbl{} will be
used. And for floats we will use big-endianness double (IEEE 754 / binary64).

The loader / threader will need to convert as need be.

\paragraph{32 / 64 bit}

Although we are only expecting a few hundred opcodes (if that!!), we need to store
them in an integer of the same size as the machine address pointer size because
of the need to overwrite the op\_code with its implementation address.

For a 64bit OS we therefore need to store all opcodes and register numbers as 64-bit
integers.

\textbf{Note: If not for threading, there could have been many schemes to compress this,
for example a 64bit integer could have held a 16-bit opcode and 3 16-bit register
numbers as parameters - very efficient. Therefore we will confirm that the
threaded interpreter is still faster than a classic interpreter using this more
efficient scheme. Modern CPU\textquotesingle{}s may be much smarter at pipelining!}

The 64 bit machine code format is therefore

\begin{itemize}
\item A 64bit Opcode, followed by

\item Zero or more (depending on opcode) 64bit parameters. Each parameter (depending on
the opcode) can be one of:

\begin{itemize}
\item An integer constant

\item An double float constant

\item Index to an item in the constant pool (e.g. a String, High-precision Number
or Object Constant)

\item Register Number

\item An instruction Number (i.e. a virtual address for a jump)

\end{itemize}

\end{itemize}

For 32-bit machines we can use two options

\begin{enumerate}
\item Have an alternative but equivalent format using 32bit integers/floats/addresses.
This reduces the size of binaries but reduces compatibility between 64-bit and
32-bit machines. 2 versions of binaries may be needed, or the loader could convert
on load. An issue is that Float constants would need to be put in the Constant
pool.

\item Use the 64-bit format for 32 bit machines. This means that 50\% of the program
space will be wasted but will provide compatibility. Note that only
4,294,967,295 registers per stack frame will be supported!

\end{enumerate}

We will explore the different options here including looking at program size
and performance.

\subsection{Standards/Rules for VM Instruction Implementation}

The following are the rules that should be followed when implementing VM Instructions.

\subsubsection{Key Principles}

\begin{enumerate}
\item To maximise CPU performance with respect to speculative execution (including simple pipelining) branches should be avoided and if required (e.g. when dispatching to the next instruction) the target address should be calculated and assigned as early as possible.

\item The compiler (or human writing the assembler code) is responsible for producing valid assembler including ensuring that assembler registers are initiated and have the required type, and ensuring (for example) then bound checks have been done. Run time validation at the VM Instruction levels should be avoided except for

\end{enumerate}

\begin{itemize}
\item When in debug mode

\item When \textquotedbl{}safe\textquotedbl{} versions of instructions are used by the compiler - see later.

\end{itemize}

Badly formed REXX Assembler code will have undefined runtime behaviour.

Justification for this approach is that SPECTRE attacks mean that it is possible for a program to access all process memory in a VM despite any access checks implemented by the VM (see Google\textquotesingle{}s SPECTRE JavaScript demonstrator).

For information, a SPECTRE attack uses the non-functional performance improvement of a data read associated with a speculative execution of an invalid instruction that moved data into the L1 cache. My assessment (and that of security advice) is that designers should assume untrusted code can read across an entire process memory space (and across a CPU core too) and that countermeasures will have limited impact because of the fundamental need for the performance boost provided by speculative execution and caches (we are not giving those up!)

Therefore when deploying cREXX code in a secure manner each VM instance will need to be in its own process space - and that is the way we will need to protect against nefarious code. All the cleaver work done in the Java and .NET VMs to prove code is safe might very well be proved to be pointless :-(

\subsubsection{Rules}

\begin{enumerate}
\item Calculate the next instruction address as early as possible.

\item Do not use \textquotedbl{}if\textquotedbl{} statements and validation

\item Conversion errors should cause SIGNALS (\textquotedbl{}goto SIGNAL;\textquotedbl{})

\item When needed by the compiler we will define large instructions that combine several instructions into one - this avoid multiple instruction dispatches.

\item Use DEBUG Macros to store debug validation - these will be not included in production/release compiled code.

\end{enumerate}


\section{RXVM machine instructions}

% \section{Machine Interface}
This section describes the processor-specific information for
\emph{RXVM} processors. The instruction set can be seen as the ISA
(instruction set architecture) for
an RXVM processor, of which the microcode is implemented in the C99 language.

% \subsection{Processor Architecture}
\index{processor architecture}
\index{instruction set}
Programs intended to execute directly on the processor use the
\emph{RXVM} instruction set and the
instruction encoding and semantics of this architecture.

An application program can assume that all instructions defined by the
architecture and that are neither privileged nor optional, exist and work
as documented.

To be ABI (application binary interface) conforming, the processor must implement the instructions of
the architecture, perform the specified operations, and produce the
expected results.  The ABI neither places performance constraints on
systems nor specifies what instructions must be implemented in
hardware.  A software implementation of the architecture conforms to
the ABI; likewise, the architecture could be implemented in hardware,
e.g. an FPGA.

\subsection{The Register}
\index{register}
RXVM is a Register based virtual machine, as opposed to a Stack based
VM. The number of registers is only limited by memory and, for
practical purposes, can be considered unlimited. The address size of
the fields in a register is, for the virtual machine implementations,
implied by the address size that the host OS can handle. For hardware
the size is undefined and can follow the hardware address generation
capacity.

\begin{lstlisting}[style=embed,label=crexxregister,caption={The
\crexx{} Register implementation in C}]
struct value {
    /* bit field to store value status - these are explicitly set */
    value_type status;

    /* Value */
    rxinteger int_value;
    double float_value;
    void *decimal_value; /* TODO */
    char *string_value;
    size_t string_length;
    size_t string_buffer_length;
    size_t string_pos;
#ifndef NUTF8
    size_t string_chars;
    size_t string_char_pos;
#endif
    void *object_value;
\end{lstlisting}

\subsubsection{Conversions of register values}
\index{conversions, data type in assembler}
\index{say instruction, rxas}
\index{int_value, register}
The status field determines for instructions that expect a data type,
which field is the field to act upon. Conversions are possible but
never implicit. So for example, when the register contains an int
value in field \code{int_value}, but it needs to be printed with the rxas
\keyword{say} instruction, the
\keyword{itos} takes care of the conversion and the population of the
memory area the \code{*string_value} points to. 

\begin{lstlisting}[style=embed,label=crexxregister,caption={Values of
the Status field}]
typedef union {
    struct {
        unsigned int type_object : 1;
        unsigned int type_string : 1;
        unsigned int type_decimal : 1;
        unsigned int type_float : 1;
        unsigned int type_int : 1;
    };
    unsigned int all_type_flags;
} value_type;
\end{lstlisting}

\subsubsection{Layout of a Register}
  \begin{tikzpicture}
    \matrix [memory layout, inner sep=0pt, nodes={inner sep=1ex},
    description/.style={text width=12em, text badly centered},
    cells={anchor=center}] (m) {
      % \node {\texttt{\stackit{f6\\f4\\f2\\f0}}}; &
      \node [description] {Floating-point argument register save area};
      \\
      \coordinate (A);\\
      % \path (0,1.5ex) node [above] (r15) {\texttt{r15}};
      % \path (0,-1.5ex) node [below] (r7) {\texttt{r7}}; &
      \node [description] {Other register save area};
      \\
      \coordinate (B);\\
      % \path (0,1.5ex) node [above] (r6) {\texttt{r6}};
      % \path (0,-1.5ex) node [below] (r2) {\texttt{r2}}; &
      \node [description] {Argument register save area};
      \\
      \coordinate (C);\\
      &
      \node [description] {Unused};
      \\
    };
    % \draw [<->] (r7) -- (r15);
    % \draw [<->] (r2) -- (r6);
    \foreach \Node in {A, B, C} {
      \draw (\Node -| m.west) -- (\Node -| m.east);
    };
    % \ifzseries
    % \foreach \where/\offs in {m.north/160, A/128, B/56, C/16, m.south/8} {
    %   \path (\where -| m.west) +(-1ex,0)
    %   node [left] {\texttt{\offs}};
    % }
    % \else
    % \foreach \where/\offs in {m.north/96, A/64, B/28, C/8, m.south/4} {
    %   \path (\where -| m.west) +(-1ex,0)
    %   node [left] {\texttt{\offs}};
    % }
    % \fi
  \end{tikzpicture}

\subsubsection{Operand Types}

The following operand types are possible for assembler
instructions. The operand(s), if needed, always follow the instruction
mnemonic. A small subset of the instructions takes no operands, like
\keyword{RET}, \keyword{INC0}, \keyword{INC1}, \keyword{INC2}.

\begin{table}
  \centering
  \begin{DIFnomarkup}
  \begin{threeparttable}
    \begin{tabularx}{\textwidth}{lXl}
      \toprule
      Operand Type & Role &  \\
      \midrule
      \texttt{REG} & a register & \\
      \texttt{STRING} & a sequence of Unicode characters & \\
      \texttt{CHAR} & one Unicode character \\
      \texttt{FUNC} & a global or local function & \\
      \texttt{ID} & a label & \\
      \texttt{INT} & a number of type integer \dagger & \\
      \texttt{FLOAT} & a number of type float \dagger & \\
      \bottomrule
    \end{tabularx}
    \medskip
    \begin{tablenotes}
    \item [\dagger] maximum value implied by the host hardware.
    \end{tablenotes}
  \end{threeparttable}
  \end{DIFnomarkup}
  \caption{RXAS Assembler Instruction Operands}
  \label{tab:rxasoperands}
\end{table}


\subsection{RXVM Instruction Set}

\subsubsection{Fixed Point Arithmetic}

\subsubsection{Floating Point Arithmetic}

\subsubsection{Logical Operations}

\subsubsection{Branching}

\subsubsection{Input/Output Operations}

\subsection{RXAS Directives}

The \emph{rxas} assembler language has a number of statements that are
directives to the assembler as opposed to machine instructions.

\begin{table}
  \centering
  \begin{DIFnomarkup}
  \begin{threeparttable}
    \begin{tabularx}{\textwidth}{lXl}
      \toprule
      Directive & Role &  \\
      \midrule
      \texttt{*} & comment until end of line & \\
      \texttt{/* */} & delimited comment & \\
      \texttt{.globals=\emph{n}} & specifies number of globals & \\
      \texttt{.expose} & expose (make globally known) & \\
      \texttt{.locals=\emph{n}} &  specifies number of locals & \\
      \bottomrule
    \end{tabularx}
    \medskip
    % \begin{tablenotes}
    % \item [\dagger] maximum value implied by the host hardware.
    % \end{tablenotes}
  \end{threeparttable}
  \end{DIFnomarkup}
  \caption{RXAS Assembler Directives}
  \label{tab:rxasdirectives}
\end{table}

\hypertarget{rexx-assembler-specification}{%
\section{REXX Assembler
Specification}\label{rexx-assembler-specification}}

\hypertarget{overview}{%
\subsection{Overview}\label{overview}}

{[}Work in Progress{]}

\href{Logical-REXX-VM}{Additional details of the RXVM virtual machine}

\hypertarget{features-required-to-support-crexx-languages}{%
\subsection{Features Required to Support cREXX
Languages}\label{features-required-to-support-crexx-languages}}

The following table maps REXX features to RXAS capabilities to explain
the motivation for RXAS capabilities and approach for implementing REXX
features. Details of the REXX Features themselves are documented in the
REXX specifications, and more details of the RXAS capabilities follow.

\begin{longtable}[]{@{}llll@{}}
\toprule
REXX~Features & RXAS~Capabilities & Implementation~Approach &
Available \\
\midrule
\endhead
Program Flow & Branches & Generate loops in RX & Yes \\
Simple Variables & Local Registers & & Yes \\
STEM Variables & See Classes & STEMs implemented as class & N/A \\
PROCEDURES & Procedures & & Yes \\
SIGNAL / Labels & branches / labels & may need duplicate code & Yes \\
Static EXPOSE & Pass by Reference args & EXPOSED vars by reference &
Yes \\
Dynamic EXPOSE & No Additional & Pool implemented in REXX & N/A \\
VALUE() & No Additional & Pool implemented in REXX & N/A \\
INTERPRET & Dynamically created rxbin & Complex\footnote{Version 1 will
  require the compiler / assembler linked into the runtime. Version 2
  (REXX on REXX) will require advanced REXX parsing.} & No \\
REXXb Classes & Regs contain sub-regs & \footnote{Class Attributes
  (always private) are stored in sub-registers. Member functions are
  statically linked and use a naming convention ``class.member''. The
  first argument to the member is the object} & No \\
REXXc Singleton Classes & global regs & \footnote{Singleton Classes
  replace static classes / members in other languages. The singleton is
  stored in a global register and exposed with a naming convention
  ``class.@1''. Note that ``@'' cannot be used from REXX programs
  directly} & Yes \\
REXXb Interface & Regs hold function pointer & No & \\
Dynamic / Late binding & footnote & by REXXb class(s) & No \\
Arbitrary Precision Maths & No Additional & in REXXb class(s) & N/A \\
ADDRESS & Platform specific & & No \\
External Functions & Platform specific & \footnote{Dynamic discovery and
  linking, call-backs to access variable pool. Classic REXX
  implementation will need to ensure all required variables are
  available in the variable pool} & No \\
Error Reporting & Annotation and debug pool & \footnote{Source line
  information stored in a ``debug'' pool to allow source line error
  reporting. Run time error conditions (signals) call REXX Exit
  functions} & No \\
\bottomrule
\end{longtable}

\hypertarget{rxas-source-syntax-and-structure}{%
\subsection{RXAS Source Syntax and
Structure}\label{rxas-source-syntax-and-structure}}

\hypertarget{structure}{%
\subsubsection{Structure}\label{structure}}

\begin{enumerate}
\def\labelenumi{\arabic{enumi}.}
\tightlist
\item
  Global Variable definition or Declaration
\item
  Procedure Definition or Declaration (repeated)
\end{enumerate}

\hypertarget{comments}{%
\subsubsection{Comments}\label{comments}}

\begin{verbatim}
/* Block Comment */

* Line Comment
\end{verbatim}

\hypertarget{instructions}{%
\subsubsection{Instructions}\label{instructions}}

These have the format OP\_CODE {[}ARG1{[},ARG2{[},ARG3{]}{]}{]} where
each argument can be a

\begin{itemize}
\tightlist
\item
  Register - e.g.~r0\ldots1 (for locals), a0\ldots n (for arguments) or
  g0\ldots n (for globals)
\item
  String - e.g.~``hello''
\item
  Integer - e.g.~5
\item
  Float - e.g.~5.0
\item
  Function - e.g.~proc()
\item
  Identifiers / Label - e.g.~label1
\end{itemize}

\hypertarget{directives}{%
\subsubsection{Directives}\label{directives}}

RXAS supports the following directives.

\begin{longtable}[]{@{}ll@{}}
\toprule
Directive & Description \\
\midrule
\endhead
.globals = \{INT\} & Defines the number of globals defined for the
file \\
.locals = \{INT\} & Defines the number of local registers in a
procedure \\
.expose = \{ID\} & Defines the exposed index of a global register
\footnote{or Procedure. These are used for linking between files/modules} \\
\bottomrule
\end{longtable}

\hypertarget{file-scope-global-registers}{%
\subsubsection{File Scope Global
Registers}\label{file-scope-global-registers}}

\begin{verbatim}
.globals={int}
\end{verbatim}

Defines \{int\} global variable g0 \ldots{} gn. These can be used within
any procedure in the file.

\hypertarget{global-registers}{%
\subsubsection{Global Registers}\label{global-registers}}

Any global register marked as exposed is available to any file which
also has the corresponding exposed index/name.

\hypertarget{file-1}{%
\paragraph{File 1}\label{file-1}}

\begin{verbatim}
.globals=2            * 2 Global Registers
g0 .expose=namespace.var_name   * 
\end{verbatim}

\hypertarget{file-2}{%
\paragraph{File 2}\label{file-2}}

\begin{verbatim}
.globals=3            * 3 Global Registers
g2 .expose=namespace.var_name   * 
\end{verbatim}

In this case file 1 g0 is mapped to file 2 g2 under the index/name of
``namespace.var\_name''.

\hypertarget{file-scope-procedure}{%
\subsubsection{File Scope Procedure}\label{file-scope-procedure}}

The locals define how many local registers, r0 to r(locals-1), are
needed by the procedure.

\begin{verbatim}
* The ".locals" shows the procedure is defined in here
file_scope_proc() .locals=3
...
ret
\end{verbatim}

\hypertarget{global-procedures}{%
\subsubsection{Global Procedures}\label{global-procedures}}

Global Procedures can be called between file/modules.

\hypertarget{file-1-1}{%
\paragraph{File 1}\label{file-1-1}}

\begin{verbatim}
* The ".locals" shows the procedure is defined in here
proc() .locals=3 .expose=namespace.proc
ret
\end{verbatim}

\hypertarget{file-2-1}{%
\paragraph{File 2}\label{file-2-1}}

\begin{verbatim}
* No ".locals" here! Showing that the procedure is only being declared 
rproc() .expose=namespace.proc

main() .locals=3 
call rproc()
ret
\end{verbatim}

In this case main() in File 2, calls rproc() which is globally provided
under the index/name of ``namespace.proc''. In File 1, proc() is exposed
under this name and hence called from File 1.

Note: that ``namespace'' hints at the use of namespaces as part of
exposed names; this facility is used by the compiler to define classes.

Also, as shown names can be mapped - they don't have to be the same in
the source and in the target.

\hypertarget{rxas-capabilities-alphabetical}{%
\subsection{RXAS Capabilities
(alphabetical)}\label{rxas-capabilities-alphabetical}}

\hypertarget{branching-and-labels}{%
\subsubsection{Branching and Labels}\label{branching-and-labels}}

Within a procedure labels can be defined as branch targets. Conditional
and unconditional branch instructions can target these labels. The
following example shows a loop structure.

\begin{verbatim}
   ...                   * Code before loop
l75:                     * Loop start label
   igt r0,r1,r4          * Does a integer compare of r1 and r4 - puts true (1) or false (0) into r0
   brt l37,r0            * Branch if true to l37 (i.e. branch out of the loop)
   ...                   * Instructions in the loop
   inc r1                * Increment r1 (the loop counter)
   br l75                * Unconditional Branch to the start of the loop
l37:                     * Loop end Label 
   ...                   * Instructions following the loop
\end{verbatim}

\hypertarget{code-annotation-and-debug-pool}{%
\subsubsection{Code Annotation and Debug
Pool}\label{code-annotation-and-debug-pool}}

NOT IMPLEMENTED

cREXX needs to support appropriate error messages (including source line
number / text), breakpoints, and tracing. RXAS directives (.file and
.line) allow the source file, source line to be defined. The .clause
directive allow clause boundaries to be defined.

\begin{verbatim}
.file = "testfile.rexx"   * Source file name 
proc()   .locals=3
   .regname r1,a          * Maps r1 to an id for debugging purposes 
   .regname r2,b
   .line 9 "a=5; b=6"     * The line number and source string
   .clause                * REXX clause boundary
   load r1,5
   .clause
   load r2,6
   .line 10 "say a+b"
   .clause
   iadd r0,r1,r2
   itos r0
   say r0
   .line 11 "return"
   .clause
   ret
\end{verbatim}

The directives are processed at ``build time'' and the debug constant
pool is created which allow assembler instructions to be mapped back to
source lines. In this way error messages can be generated as if the
source file was being interpreted ``classically'' but with no run-time
overhead.

In addition, tracing/debugging is implemented by the VM machine using
the clause boundaries stored in the debug pool. The VM can set a
breakpoint by replacing the instruction at the appropriate address with
a breakpoint instruction. When the breakpoint is reached it uses the
clause boundary information to determine where the next breakpoint
should be set. Tools can be made available to allow a REXX programmer to
set a breakpoint at a REXX source code line number.

The debug pool also contains the information to display the rexx
variable name stored in a register.

Note that accessing debug information is a significant overhead as the
debug pool will need repeated searching, and will only be used for
debugging/tracing (or creating an error message) where performance is
not critical. The reason this approach is used is that when there is no
debugging in action there is no runtime performance overhead at all
(obviously the size of the rxbin binary file is made larger with the
debug pool).

\hypertarget{constant-pool}{%
\subsubsection{Constant Pool}\label{constant-pool}}

Each File/Module has a constant pool that stores: * String Constants *
Procedure Details * PTable information (mapping class to interface
procedures for a class and objects)

\hypertarget{dynamic-access-to-registers-including-arguments}{%
\subsubsection{Dynamic Access to Registers (including
Arguments)}\label{dynamic-access-to-registers-including-arguments}}

Dynamic access to a register is enabled by additional members of the
link family of instructions these allow a register to point to the same
value as a dynmically number primary register.

\begin{verbatim}
alink secondary_reg, arg_reg_num * Links secondary_reg to the argument register with number stored in the int value of arg_reg_num 
glink secondary_reg, global_reg_num * Links secondary_reg to the global register with number stored in the int value of global_reg_num 
\end{verbatim}

\hypertarget{dynamic-type-instructions}{%
\subsubsection{Dynamic Type
Instructions}\label{dynamic-type-instructions}}

The compiler will be able to manage which registers have what values
most of the time but there will be certain dynamic situations where the
value type or status is not known. To handle this the compiler use the
registers type flag:

\begin{verbatim}
gettp - gets the register type flag (op1 = op2.typeflag)
settp - sets the register type flag (op1.typeflag = op2)
setortp - or the register type flag (op1.typeflag = op1.typeflag || op2)
brtpt - if op2.typeflag true then goto op1
brtpandt - if op2.typeflag && op3 true then goto op1
\end{verbatim}

The typeflag is a 64bit integer and its usage is defined by convention
only \protect\hyperlink{crexx-calling-convention}{see cREXX Calling
Convention}.

\hypertarget{dynamic-procedure-pointers}{%
\subsubsection{Dynamic Procedure
Pointers}\label{dynamic-procedure-pointers}}

This capability is to support interfaces. Where the compiler knows the
object's class it can link statically to the correct member by using the
procedure name (i.e.~class\_name.member\_name(), however when the object
is only known to implement an interface (i.e.~its class is not known)
then the VM needs to dynamically link interface members to the object's
class specific implementation.

Each register, that contains an object whose call implements an
interface, has a pointer to a entry in the constant pool. This entry
allows the the interface name and member number to be searched at
runtime, returning its implementation procedure pointer. This is known
as the register's static ptable. The static ptable also stores the name
of the objects class.

\emph{Note: Where an object members are dynamically assigned at runtime
(not an immediately required capability) the dynamic mapping from member
name to procedure pointer will be done within the REXX runtime library
(i.e.~not applicable to RXAS).}

A directive defines the entry

\begin{verbatim}
.ptable class_name interface1_name(impl1_1(), impl1_2(), ...) interface2_name(impl2_1() ...) ...
\end{verbatim}

This creates the entry into the constant pool with links to the
procedure implementing an interfaces members \#1,\#2 etc.

The object is linked to the entry with an instruction:

\begin{verbatim}
setptable r1,class_name
\end{verbatim}

This sets the register r1 ptable to the entry ``class\_name'' in the
constant pool.

Finally the entry can be used at runtime:

\begin{verbatim}
srcptable r2,"interface_name",3
\end{verbatim}

In this example r2 is a class instance (object) implementing interface
``interface\_name''. This instruction looks up the object's procedure
implementing member \#3 of interface ``interface\_name'' and sets the
procedure pointer of r2 to this. Then

\begin{verbatim}
dyncall r0, r2, r3
\end{verbatim}

calls the procedure in r2, with arguments from r3 and puts the result in
r0.

\hypertarget{external-functions}{%
\subsubsection{External Functions}\label{external-functions}}

\hypertarget{injecting-dynamic-code}{%
\subsubsection{Injecting Dynamic Code}\label{injecting-dynamic-code}}

\hypertarget{instructions-1}{%
\subsubsection{Instructions}\label{instructions-1}}

\hypertarget{type-coding}{%
\paragraph{Type coding}\label{type-coding}}

Instructions have prefix to determine type: s=string, i=integer,
f=float, o=object

\hypertarget{maths}{%
\paragraph{Maths}\label{maths}}

As an example, the add family will have * iadd reg,reg,reg\\
* fadd reg,reg,reg * etc.

Each function just uses the corresponding registers value (int, float,
etc).

\hypertarget{load}{%
\paragraph{Load}\label{load}}

\begin{itemize}
\tightlist
\item
  {[}s/i/f/d{]}load - lost loads the corresponding type value only
\item
  load (i.e.~with no prefix) copies all values and the type flag to the
  target register
\end{itemize}

\hypertarget{conversion}{%
\paragraph{Conversion}\label{conversion}}

Converting means setting a value for a type based on the value on
another type in the same register, e.g. * itos reg - sets the string
value to the string representation of the integer value of reg * ftos
reg * stof reg - This converts the string to a float, or triggers a
signal if it can't

Note: this replaces prime/master.

\hypertarget{say-address-etc.}{%
\paragraph{SAY / ADDRESS etc.}\label{say-address-etc.}}

Where an instruction needs a string it will only have a string
``version''. For clarity we will have

\begin{itemize}
\tightlist
\item
  say
\item
  address
\end{itemize}

but there will not be a isay etc. Instead the compiler might need to do
a ``itos'' first.

\hypertarget{procedures-and-arguments}{%
\subsubsection{Procedures and
Arguments}\label{procedures-and-arguments}}

A procedures registers are independent to the caller's registers. What
happens is that the VM maps its registers to the registers in the
caller.

Each time a procedure/function is called a new ``stack frame'' is
provided. This means that the called function has its own set of
registers.

The function header defines how many registers (called locals) the
function can access - for practical purposes we can consider that any
number of registers can be assigned to a function.

In addition, each file defines a number of global registers that can be
shared between procedures.

In a function with `a' arguments, `n' locals, and `m' globals: * R0
\ldots{} R(n-1) - are local registers to be used by the function * R(n)
\ldots{} R(n+m-1) - are the global registers, i.e.~g(0) \ldots{} g(m-1)
* R(n+m) - holds the number of arguments (a) * R(n+m+1) \ldots{}
R(n+m+a) holds the arguments, i.e.~a(1) \ldots{} a(a)

This ordering allows a dynamic numbers of arguments.

\hypertarget{crexx-calling-convention}{%
\paragraph{cREXX Calling Convention}\label{crexx-calling-convention}}

All arguments within RXAS are pass by reference, therefore arguments
needs to be copied to another register if pass by reference is not
wanted. This approach is a way to support moves rather than copies -
especially important to avoid slow object and string copies.

It is mandatory to use this calling convention between REXX and RXAS
procedures. Although not necessary, it is recommended to also use this
convention between RXAS procedures.

In this convention the caller is responsible for setting argument
registers' typeflag. This is used to indicate if an optional argument is
present, and if a pass-by-value string or object argument needs
preserving.

The callee (procedure) is responsible for applying default values for
optional arguments, and for ensuring that pass-by-value arguments are
kept constant (so they are not changed, effecting the caller logic) if
required. The callee uses the typeflag for this.

\textbf{Register Type Flag Byte Values}

The register typeflag is used to optimise function arguments.

\begin{itemize}
\item
  Bit 1 - REGTP\_VAL - ONLY used for optional arguments; setting (1)
  means the register has a specified value
\item
  Bit 2 - REGTP\_NOTSYM - ONLY used for ``pass be value'' and ONLY large
  (strings, objects) registers; setting (2) means that it is not a
  symbol so does not need copying as, even if it is changed, the caller
  will not use its original value. Note: Small registers (int, float)
  are always copied as this is faster than setting and checking this
  flag; the REGTP\_NOTSYM flag is not set or read for integers and
  floats.
\end{itemize}

The following examples demonstrate the calling convention.

\textbf{Basic Call by Reference}

REXX Program

Annotated Generated RXAS

\textbf{Optional Call by Reference}

REXX Program

Annotated Generated RXAS

\textbf{Call By Value Integer and Optimisations}

In this example, REGTP\_NOTSYM is not used as the parameter is an
integer.

REXX Program

Annotated Generated RXAS

\textbf{Call By Value Strings and Optimisations (optional arguments)}

In this example, REGTP\_NOTSYM is used as the parameter is an string in
optional arguments.

REXX Program

Annotated Generated RXAS

\hypertarget{arbitrary-number-of-arguments-with}{%
\paragraph{Arbitrary number of arguments with
\ldots{}}\label{arbitrary-number-of-arguments-with}}

\textbf{TO BE IMPLEMENTED} (requires arrays)

\hypertarget{expose}{%
\paragraph{EXPOSE}\label{expose}}

\textbf{TO BE IMPLEMENTED}

Syntax candy to provide familiar (but not the same) EXPOSE experience
for REXX programmers.

In this example:

\begin{verbatim}
exp = 100

say test( "hello")
exit

test: procedure = .string expose exp = .int
  arg message = .string
  return message || exp
\end{verbatim}

Is converted by the compiler to:

\begin{verbatim}
exp = 100

say test(exp, "hello")
exit

test: procedure = .string 
  arg expose exp = .int, message = .string
  return message || exp
\end{verbatim}

This is designed to provide a familiar (but not the same) experience for
REXX programmers

\hypertarget{procedure-lookup-tables}{%
\subsubsection{Procedure Lookup Tables}\label{procedure-lookup-tables}}

\hypertarget{registers}{%
\subsubsection{Registers}\label{registers}}

\hypertarget{register-data}{%
\paragraph{Register Data}\label{register-data}}

Each register holds 5 values - String, float, integer and object, and a
type flag which is used to indicate which values are valid. Note that
arbitrary position maths is handled as objects.

In most cases it is for the compiler to decide what values to use, and
how/if to set the type flag. Only a few dynamic scenarios will need some
extended functions (see following). The type values are 0=unset,
1=string, 2=float, 4=integer, 8=object, 16=interface. Each register can
have multiple types set as valid. The compiler sets the valid types
explicitly with instructions - this is not an automatic runtime
capability. At runtime the VM has no need nor the \emph{ability} to
validate data. Any caching has to be achieved by compiler logic.

An Object has an pointer to its static ptable entry in the constant pool
as well as an array of sub-registers. These sub-registers are the
private attributes of the object.

\hypertarget{register-initialisation}{%
\paragraph{Register Initialisation}\label{register-initialisation}}

All registers are initialized on entry to a procedure on the register
``stack''. The rationale is that all the memory can be malloced at once
which is faster/safer. The pointers to globals and arguments are also
setup.

In addition a shadow set of pointers to the procedure's registers are
setup. These are used by the unlink instruction, they holding the
base/initial register pointer, and unlink sets the register pointer to
the pointer held in the respective shadow value.

Registers hold references to their parent/owner for memory freeing
purposes. The owner can change, for example the owner of a returned
register is set to the caller. When an object, stack frame, global pool
is being deleted the registers are also freed/deleted if the registers
owner is the same as the container being deleted.

\hypertarget{register-re-mapping-facilities}{%
\paragraph{Register Re-Mapping
Facilities}\label{register-re-mapping-facilities}}

There are a few scenarios where the contents of a register is needed in
another register number: a call requires the arguments to be in
consecutive registers, object sub-registers need to be copied to
registers, or accessing an arbitrary argument registers (i.e.~when the
number of arguments is only known at runtime).

Copying the contents of registers to achieve this would be slow (and for
large strings or objects, very slow). Also it is inconsistent because an
integer copy and an object pointer copy (which is a copy by reference)
behave differently (integers become independent but a change to the
object changes it in each ``copy'').

We provide 2 facilities to allow very fast and safe register moves:

\begin{itemize}
\item
  SWAP. This swaps two registers. This is very fast as it requires only
  6 pointer copies (3 swapping the two register pointers and 3 swapping
  the two shadow pointers). It allows the programmer to swap two
  registers (arguments, globals, registers) to get the register into a
  convenient register number (perhaps for a call). Doing the swap again
  restores the register numbers.
\item
  LINK/UNLINK. The link instruction makes two register numbers point to
  the same register (one is primary, the other secondary). Unlink makes
  the \emph{secondary} register revert back to it original state by
  using the shadow register pointer. Each instruction only requires one
  pointer copy.
\end{itemize}

The behaviour should normally be quite simple - however swapping or
linking already linked or swapped registers may cause complex outcomes.
Developers should consider the above behaviour descriptions to untangle
this!

\hypertarget{runtime-error-conditions-and-exit-functions}{%
\subsubsection{Runtime Error Conditions and Exit
Functions}\label{runtime-error-conditions-and-exit-functions}}

\hypertarget{shell-instructions}{%
\subsubsection{Shell Instructions}\label{shell-instructions}}

\hypertarget{sub-registers}{%
\subsubsection{Sub-Registers}\label{sub-registers}}

\hypertarget{super-instructions}{%
\subsubsection{Super Instructions}\label{super-instructions}}

Once we see what the code generated by the compiler looks like we may
combine some of these instructions in to super instructions for
performance reasons.

\hypertarget{variable-pool-in-rexx}{%
\subsection{Variable Pool in REXX}\label{variable-pool-in-rexx}}

In the current code for the VM we have registers which point to a
variable structure (that can contain string, integers etc). These can be
considered to be ``anonymous variables'' -- the compiler will assign a
register to hold a variable at compile time.

In this way ``80\%'' of the needs for REXX programs will be handled --
but not all. Some aspects of REXX needs dynamic variable names --
e.g.~Some EXPOSE scenarios, INTERPRET, VALUE, and the REXXCOMM / SAA
API. When these are needed the compiler needs to work in what I am
calling ``Pedantic'' mode. In this mode variables are also given a
string index in a variable pool, this index can be searched for
dynamically at runtime.

Each stack frame will also include its own variable pool. This is a
name/value index (via a HASH or TREE) whereby a variable can be found
via the index string. \emph{This will be implemented in REXXb.}

\begin{itemize}
\tightlist
\item
  When a procedure exits and the stack frame is torn down, the
  corresponding variable pool and variables also need freeing.
\item
  To facilitate EXPOSE, a variable pool index can be linked to a parent
  pool variable.
\item
  A register can point to an anonymous variable (as implemented today in
  the code) or mapped to a variable in the variable pool.
\end{itemize}


% {\ifzseries\else
%   There are some instructions in \ARCHarch{}
%   which are described as ``optional.''  This ABI
%   requires some of these to be available; in particular:
%   \begin{itemize}
%   \item additional floating-point facilities, % checksum instruction
%   \item compare and move extended,
%   \item immediate and relative instructions,
%   \item square root,
%   \item string instructions.
%   \end{itemize}

%   The ABI guarantees that these instructions are present.  In order to
%   comply with the ABI the operating system must emulate these
%   instructions on machines which do not support them in the hardware.
%   Other instructions are not available in some current models; programs
%   using these instructions do not conform to the \ABINAME{} ABI and
%   executing them on machines without the extra capabilities will result
%   in undefined behavior.  \fi}

% In \ARCHarch{} a
% processor runs in big-endian mode.  (See \cref{byteordering}.)

% \subsection{Data Representation}
% \subsubsection{Byte Ordering}
% \label{byteordering}
% \index{byte ordering}

% The architecture defines an 8-bit byte\index{byte},
% a 16-bit halfword\index{halfword},
% a 32-bit word\index{word},{\ifzseries\else{} and\fi}
% a 64-bit doubleword\index{doubleword}{\ifzseries ,
%   and a 128-bit quadword\index{quadword}\fi}.
% Byte ordering defines how the bytes that make up halfwords,
% words,{\ifzseries\else{} and\fi} doublewords{\ifzseries , and
%   quadwords\fi} are ordered in memory.  Most significant byte (MSB)
% ordering, also called ``big-endian,''\index{big-endian} means that
% the most significant byte
% of a structure is located in the lowest addressed byte position in a
% storage unit (byte 0).  By contrast, least significant byte (LSB)
% ordering, or ``little-endian,''\index{little-endian} refers to the
% reverse byte order, where
% the lowest addressed byte position holds the least significant byte.

% \Crefrange{fig:halfword}{\ifzseries fig:quadwords\else
%   fig:doublewords\fi} illustrate the conventions for bit and byte
% numbering within storage units of various widths.  These conventions
% apply to both integer data and floating-point data, where the most
% significant byte of a floating-point value holds the sign and the
% exponent (or at least the start of the exponent).  The figures show
% big-endian byte numbers in the upper left corners and bit numbers in
% the lower corners.

% \begin{figure}
%   \centering
%   \ifSkipTikZ
% \begin{verbatim}
% +-------------+-------------+
% | 0           | 1           |
% |     msb     |     lsb     |
% | 0         7 | 8        15 |
% +-------------+-------------+
% \end{verbatim}
%   \else
  % \begin{tikzpicture}[x=1.3ex,y=3em]
  %   \path [bitchart box] (0, 0) rectangle (16, 1);
  %   \bitchartfields /\textrm{add}/, /\textrm{r1}/;
  %   \bitchartbytes{0}{0,1}
  % \end{tikzpicture}
%   \fi
%   \caption{Bit and byte numbering in halfwords}
%   \label{fig:halfword}
% \end{figure}

% \begin{figure}
%   \centering
%   \ifSkipTikZ
% \begin{verbatim}
% +-------------+-------------+-------------+-------------+
% | 0           | 1           | 2           | 3           |
% |     msb     |             |             |     lsb     |
% | 0         7 | 8        15 | 16       23 | 24       31 |
% +-------------+-------------+-------------+-------------+
% \end{verbatim}
% %   \else
%   \begin{tikzpicture}[x=1.3ex,y=3em]
%     \path [bitchart box] (0, 0) rectangle (32, 1);
%         \bitchartfields x/\textrm{msb}/, 16/~/, 24/~/, 32/~/;
%     \bitchartbytes{0}{0,...,3}
%     \bitchartsep

%      \end{tikzpicture}
%   \fi
%   \caption{Bit and byte numbering in words}
%   \label{fig:words}
% \end{figure}

% \begin{figure}
%   \centering
%   \ifSkipTikZ
% \begin{verbatim}
% +-------------+-------------+-------------+-------------+
% | 0           | 1           | 2           | 3           |
% |     msb     |             |             |             |
% | 0         7 | 8        15 | 16       23 | 24       31 |
% +-------------+-------------+-------------+-------------+
% | 4           | 5           | 6           | 7           |
% |             |             |             |     lsb     |
% | 32       39 | 40       47 | 48       55 | 56       63 |
% +-------------+-------------+-------------+-------------+
% \end{verbatim}
%   \else
  \begin{tikzpicture}[x=1.3ex,y=3em]
    \path [bitchart box] (0, -1) rectangle (32, 1);
    \bitchartfields 0 8/\textrm{msb}/, 16/~/, 24/~/, 32/~/;
    \bitchartbytes{0}{0,...,3}
    \bitchartsep
    \begin{scope}[shift={(0,-1)}]
      \bitchartfields 32 40/~/, 48/~/, 56/~/, 64/\textrm{lsb}/;
      \bitchartbytes{4}{4,...,7}
    \end{scope}
  \end{tikzpicture}
%   % \fi
%   \caption{Bit and byte numbering in doublewords}
%   \label{fig:doublewords}
% \end{figure}

% \ifzseries
% \begin{figure}
%   \centering
%   \ifSkipTikZ
% \begin{verbatim}
% +-------------+-------------+-------------+-------------+
% | 0           | 1           | 2           | 3           |
% |     msb     |             |             |             |
% | 0         7 | 8        15 | 16       23 | 24       31 |
% +-------------+-------------+-------------+-------------+
% | 4           | 5           | 6           | 7           |
% |             |             |             |             |
% | 32       39 | 40       47 | 48       55 | 56       63 |
% +-------------+-------------+-------------+-------------+
% | 8           | 9           | 10          | 11          |
% |             |             |             |             |
% | 64       71 | 72       79 | 80       87 | 88       95 |
% +-------------+-------------+-------------+-------------+
% | 12          | 13          | 14          | 15          |
% |             |             |             |     lsb     |
% | 96      103 | 104     111 | 112     119 | 120     127 |
% +-------------+-------------+-------------+-------------+
% \end{verbatim}
%   \else
%   \begin{tikzpicture}[x=1.3ex,y=3.2em]
%     \path [bitchart box] (0, -3) rectangle (32, 1);
%     \bitchartfields 0 8/\textrm{msb}/, 16/~/, 24/~/, 32/~/;
%     \bitchartbytes{0}{0,...,3}
%     \bitchartsep
%     \begin{scope}[shift={(0,-1)}]
%       \bitchartfields 32 40/~/, 48/~/, 56/~/, 64/~/;
%       \bitchartbytes{4}{4,...,7}
%       \bitchartsep
%     \end{scope}
%     \begin{scope}[shift={(0,-2)}]
%       \bitchartfields 64 72/~/, 80/~/, 88/~/, 96/~/;
%       \bitchartbytes{8}{8,...,11}
%       \bitchartsep
%     \end{scope}
%     \begin{scope}[shift={(0,-3)}]
%       \bitchartfields 96 104/~/, 112/~/, 120/~/, 128/\textrm{lsb}/;
%       \bitchartbytes{12}{12,...,15}
%     \end{scope}
%   \end{tikzpicture}
%   \fi
%   \caption{Bit and byte numbering in quadwords}
%   \label{fig:quadwords}
% \end{figure}
% \fi

% \subsubsection{Fundamental Types}
% \index{type!scalar}
% \Cref{tab:scalar} shows how ISO C scalar types correspond to those of
% \aARCH{} processor.  To comply with this ABI, objects stored in memory
% must be aligned\index{alignment!scalar} as indicated, even though the
% architecture permits unaligned storage operands for most instructions.

% For all types, a null pointer\index{null pointer} has the value zero (binary).

% A Boolean\index{Boolean} object is represented in memory as a single byte
% with a value of 0 or 1.  If a byte with any other value is evaluated as a
% Boolean, the behavior is undefined.

% For each binary floating-point type, there is a corresponding complex
% type\index{complex type}.  It is represented as a two-element array with
% the real part as its first and the imaginary part as its second element.

% Some C dialects permit enumeration constants that exceed the range of an
% \texttt{int}.  Then the enumeration type\index{enumeration type} shall be
% encoded as the smallest unsigned or signed C integer type that can
% represent all of its enumeration constants and is not smaller than
% \texttt{int}.

% \begin{table}
%   \centering
%   \begin{DIFnomarkup}
%   \begin{threeparttable}
%     \begin{tabularx}{\textwidth}{XlrrX}
%       \toprule
%       \multirow{2}{*}{Type}
%       & \multirow{2}{*}{ISO C}
%       & Size in & Align- & \ARCH{} \\
%       &
%       & bytes   & ment   & type \\
%       \midrule
%       \multirow{8}{\hsize}{Unsigned integer}
%       & \texttt{\_Bool} & 1 & 1
%       & \multirow{8}{\hsize}{$n$-bit unsigned binary integer\tnote{\dagger}}
%       \\
%       & \texttt{unsigned char} & 1 & 1 \\
%       & \texttt{char} & 1 & 1 \\
%       & \texttt{unsigned short} & 2 & 2 \\
%       & \texttt{unsigned int} & 4 & 4 \\
%       & \texttt{unsigned long} & \NBYTES{} & \NBYTES{} \\
%       & \texttt{unsigned long long} & 8 & 8 \\
%       & \texttt{unsigned \_\_int128}\tnote{\dagger\dagger} & 16 & 8 \\
%       \midrule
%       \multirow{12}{\hsize}{Signed integer}
%       & \texttt{signed char} & 1 & 1
%       & \multirow{12}{\hsize}{$n$-bit signed binary integer\tnote{\dagger}}
%       \\
%       & \texttt{signed short} & 2 & 2 \\
%       & \texttt{short} & 2 & 2 \\
%       & \texttt{signed int} & 4 & 4 \\
%       & \texttt{int} & 4 & 4 \\
%       & \texttt{enum} & 4 & 4 \\
%       & \texttt{signed long} & \NBYTES{} & \NBYTES{} \\
%       & \texttt{long} & \NBYTES{} & \NBYTES{} \\
%       & \texttt{signed long long} & 8 & 8 \\
%       & \texttt{long long} & 8 & 8 \\
%       & \texttt{\_\_int128}\tnote{\dagger\dagger} & 16 & 8 \\
%       & \texttt{signed \_\_int128}\tnote{\dagger\dagger} & 16 & 8 \\
%       \midrule
%       \multirow{2}{\hsize}{Pointer}
%       & \textit{any-type}\texttt{ *} & \NBYTES{} & \NBYTES{}
%       & \multirow{2}{\hsize}{\ADDRBITS{}-bit address} \\
%       & \textit{any-type}\texttt{ (*) ()} & \NBYTES{} & \NBYTES{} \\
%       \midrule
%       \multirow{3}{\hsize}{Binary floating-point} &
%       \texttt{float} & 4 & 4 & short BFP \\
%       & \texttt{double} & 8 & 8 & long BFP \\
%       & \texttt{long double} & 16 & 8 & extended BFP \\
%       \midrule
%       \multirow{3}{\hsize}{Decimal floating-point} &
%       \texttt{\_Decimal32}\tnote{\dagger\dagger} & 4 & 4 & short DFP \\
%       & \texttt{\_Decimal64}\tnote{\dagger\dagger} & 8 & 8 & long DFP \\
%       & \texttt{\_Decimal128}\tnote{\dagger\dagger} & 16 & 8 & extended DFP \\
%       \bottomrule
%     \end{tabularx}
%     \medskip
%     \begin{tablenotes}
%     \item [\dagger] Here $n$ denotes the bit size, which equals the byte
%       size multiplied by 8.
%     \item [\dagger\dagger] These types are an extension to C (ISO/IEC
%       9899:2011).
%     \end{tablenotes}
%   \end{threeparttable}
%   \end{DIFnomarkup}
%   \caption{Scalar types}
%   \label{tab:scalar}
% \end{table}

% \subsubsection{Aggregates and Unions}
% Aggregates\index{aggregate}\index{type!aggregate}
% (structures\index{structure} and arrays\index{array}) and
% unions\index{union}\index{type!union} assume the
% alignment\index{alignment!aggregate or union} of their most strictly
% aligned component---that is, the component with the largest alignment.
% The size\index{size!aggregate or union} of any object, including
% aggregates and unions, is always a multiple of the alignment of the
% object.  An array uses the same alignment as its elements.  Structure and
% union objects may require padding to meet these size and alignment
% constraints:

% \begin{itemize}
% \item An entire structure or union object is aligned on the same
%   boundary as its most strictly aligned member.
% \item Each member is assigned to the lowest available offset with the
%   appropriate alignment.  This may require internal
%   padding\index{padding}, depending on the previous member.
% \item If necessary, a structure's size is increased to make it a
%   multiple of the structure's alignment.  This may require tail padding
%   if the last member does not end on the appropriate boundary.
% \end{itemize}

% In the examples shown in \crefrange{fig:struct1}{fig:struct5},
% member byte offsets (for the big-endian implementation) appear in the
% upper left corners.

% \begin{figure}
%   \centering
%   \ifSkipTikZ
% \begin{verbatim}
%              Byte aligned, sizeof is 1
%              +-------------+
% struct {     | 0           |
%     char c;  |             |
% };           |      c      |
%              +-------------+
% \end{verbatim}
%   \else
%   \begin{tabular}{>{\texttt\bgroup}l<{\texttt\egroup}}
%     struct \{\\
%     ~~~~char c;\\
%     \};\\
%   \end{tabular}
%   \quad
%   \begin{tikzpicture}[baseline=(mid),x=1.3ex,y=3em]
%     \path [bitchart box] (0, 0) rectangle (8, 1);
%     \bytechartfields 0 1/c/;
%     \path (current bounding box.center) coordinate (mid) {};
%     \path (current bounding box.north)
%     node [above] {Byte aligned, \texttt{sizeof} is 1};
%   \end{tikzpicture}
%   \fi
%   \caption{Structure smaller than a word}
%   \label{fig:struct1}
% \end{figure}

% \begin{figure}
%   \centering
%   \ifSkipTikZ
% \begin{verbatim}
%               Word aligned, sizeof is 8
%               +-------------+-------------+---------------------------+
% struct {      | 0           | 1           | 2                         |
%     char c;   |             |             |                           |
%     char d;   |       c     |      d      |             s             |
%     short s;  |-------------+-------------+---------------------------|
%     int n;    | 4                                                     |
% };            |                                                       |
%               |                           n                           |
%               +-------------------------------------------------------+
% \end{verbatim}
%   \else
%   \begin{tabular}{>{\texttt\bgroup}l<{\texttt\egroup}}
%     struct \{\\
%     ~~~~char c;\\
%     ~~~~char d;\\
%     ~~~~short s;\\
%     ~~~~{\ifzseries int\else long\fi} n;\\
%     \};\\
%   \end{tabular}
%   \quad
%   \begin{tikzpicture}[baseline=(mid),x=1.3ex,y=3em]
%     \path [bitchart box] (0, -1) rectangle (32, 1);
%     \bytechartfields 0 1/c/, 2/d/, 4/s/;
%     \bitchartsep
%     \begin{scope}[shift={(0,-1)}]
%       \bytechartfields 4 8/n/;
%     \end{scope}
%     \path (current bounding box.center) coordinate (mid) {};
%     \path (current bounding box.north)
%     node [above] {Word aligned, \texttt{sizeof} is 8};
%   \end{tikzpicture}
%   \fi
% \caption{No padding}
% \label{fig:struct2}
% \end{figure}

% \begin{figure}
%   \centering
%   \ifSkipTikZ
% \begin{verbatim}
%               +-------------+-------------+---------------------------+
% struct {      | 0           | 1           | 2                         |
%     char c;   |             |             |                           |
%     short s;  |      c      |     pad     |             s             |
% };            +-------------+-------------+---------------------------+
% \end{verbatim}
%   \else
%   \begin{tabular}{>{\texttt\bgroup}l<{\texttt\egroup}}
%     struct \{\\
%     ~~~~char c;\\
%     ~~~~short s;\\
%     \};\\
%   \end{tabular}
%   \quad
%   \begin{tikzpicture}[baseline=(mid),x=1.3ex,y=3em]
%     \path [bitchart box] (0, 0) rectangle (32, 1);
%     \bytechartfields 0 1/c/, 2//, 4/s/;
%     \path (current bounding box.center) coordinate (mid) {};
%     \path (current bounding box.north)
%     node [above] {Halfword aligned, \texttt{sizeof} is 4};
%   \end{tikzpicture}
%   \fi
%   \caption{Internal padding}
%   \label{fig:struct3}
% \end{figure}

% \begin{figure}
%   \centering
%   \ifSkipTikZ
% \begin{verbatim}
%                Doubleword aligned, sizeof is 24
%                +-------------+-----------------------------------------+
%                | 0           | 1                                       |
%                |             |                                         |
%                |      c      |                  pad                    |
%                |-------------+-----------------------------------------|
%                | 4                                                     |
%                |                                                       |
%                |                          pad                          |
%                |-------------------------------------------------------|
% struct {       | 8                                                     |
%     char c;    |                                                       |
%     double d;  |                           d                           |
%     short s;   |-------------------------------------------------------|
% };             | 12                                                    |
%                |                                                       |
%                |                           d                           |
%                |---------------------------+---------------------------|
%                | 16                        | 18                        |
%                |                           |                           |
%                |             s             |            pad            |
%                |---------------------------+---------------------------|
%                | 20                                                    |
%                |                                                       |
%                |                          pad                          |
%                +-------------------------------------------------------+
% \end{verbatim}
%   \else
%   \begin{tabular}{>{\texttt\bgroup}l<{\texttt\egroup}}
%     struct \{\\
%     ~~~~char c;\\
%     ~~~~double d;\\
%     ~~~~short s;\\
%     \};\\
%   \end{tabular}
%   \quad
%   \begin{tikzpicture}[baseline=(mid),x=1.3ex,y=2.5em]
%     \path [bitchart box] (0, -5) rectangle (32, 1);
%     \bytechartfields 0 1/c/, 4//;
%     \bitchartsep
%     \begin{scope}[shift={(0,-1)}]
%       \bytechartfields 4 8//;
%       \bitchartsep
%     \end{scope}
%     \begin{scope}[shift={(0,-2)}]
%       \bytechartfields 8 12/d/;
%       \bitchartsep
%     \end{scope}
%     \begin{scope}[shift={(0,-3)}]
%       \bytechartfields 12 16/d/;
%       \bitchartsep
%     \end{scope}
%     \begin{scope}[shift={(0,-4)}]
%       \bytechartfields 16 18/s/, 20//;
%       \bitchartsep
%     \end{scope}
%     \begin{scope}[shift={(0,-5)}]
%       \bytechartfields 20 24//;
%       \bitchartsep
%     \end{scope}
%     \path (current bounding box.center) coordinate (mid) {};
%     \path (current bounding box.north)
%     node [above] {Doubleword aligned, \texttt{sizeof} is 24};
%   \end{tikzpicture}
%   \fi
%   \caption{Internal and tail padding}
%   \label{fig:struct4}
% \end{figure}

% \begin{figure}
%   \centering
%   \ifSkipTikZ
% \begin{verbatim}
%               Word aligned, sizeof is 4
%               +-------------+-----------------------------------------+
%               | 0           | 1                                       |
%               |             |                                         |
%               |      c      |                   pad                   |
%               +-------------+-----------------------------------------+
% union {       +---------------------------+---------------------------+
%     char c;   | 0                         | 2                         |
%     short s;  |                           |                           |
%     int j;    |             s             |            pad            |
% };            +---------------------------+---------------------------+
%               +-------------------------------------------------------+
%               | 0                                                     |
%               |                                                       |
%               |                           j                           |
%               +-------------------------------------------------------+
% \end{verbatim}
%   \else
%   \begin{tabular}{>{\texttt\bgroup}l<{\texttt\egroup}}
%     union \{\\
%     ~~~~char c;\\
%     ~~~~short s;\\
%     ~~~~int j;\\
%     \};\\
%   \end{tabular}
%   \quad
%   \begin{tikzpicture}[baseline=(mid),x=1.3ex,y=2.5em]
%     \path [bitchart box] (0, 0) rectangle (32, 1);
%     \bytechartfields 0 1/c/, 4//;
%     \begin{scope}[shift={(0,-1.2)}]
%       \path [bitchart box] (0, 0) rectangle (32, 1);
%       \bytechartfields 0 2/s/, 4//;
%     \end{scope}
%     \begin{scope}[shift={(0,-2.4)}]
%       \path [bitchart box] (0, 0) rectangle (32, 1);
%       \bytechartfields 0 4/j/;
%     \end{scope}
%     \path (current bounding box.south west) coordinate (sw) {}
%     (current bounding box.north west) coordinate (nw) {};
%     \path (current bounding box.center) coordinate (mid) {};
%     \path (current bounding box.north)
%     node [above] {Word aligned, \texttt{sizeof} is 4};
%     \path (nw) +(-1em,0) coordinate (nw) {};
%     \draw [decorate,decoration=brace, thick] (sw) + (-1em,0) -- (nw);
%   \end{tikzpicture}
%   \fi
%   \caption{Union padding}
%   \label{fig:struct5}
% \end{figure}

% \subsubsection{Bit-Fields}
% C struct and union definitions may have ``bit-fields,'' defining
% integral objects with a specified number of bits
% (see \cref{tab:bitfields}).

% \begin{table}
%   \centering
%   \begin{DIFnomarkup}
%   \begin{tabular}[t]{lcr@{~\ldots~}l}
%     \toprule
%     Bit-field type & Width $n$ & \multicolumn{2}{c}{Range} \\
%     \midrule
%     \texttt{signed char} & \multirow{3}{*}{1…8} & $-2^{n-1}$ & $2^{n-1}-1$ \\
%     \texttt{char} &  & $0$ & $2^n-1$ \\
%     \texttt{unsigned char} &  & $0$ & $2^n-1$ \\
%     \midrule
%     \texttt{signed short} & \multirow{3}{*}{1…16} & -$2^{n-1}$ & $2^{n-1}-1$ \\
%     \texttt{short} &  & -$2^{n-1}$ & $2^{n-1}-1$ \\
%     \texttt{unsigned short} &  & $0$ & $2^n-1$ \\
%     \midrule
%     \texttt{signed int} & \multirow{\ifzseries 3\else 6\fi}{*}{1…32} &
%     $-2^{n-1}$ & $2^{n-1}-1$ \\
%     \texttt{int} &  & $-2^{n-1}$ & $2^{n-1}-1$ \\
%     \texttt{unsigned int} &  & $0$ & $2^n-1$ \\
%     \ifzseries \midrule \fi
%     \texttt{signed long} & {\ifzseries\multirow{3}{*}{1…64}\fi} &
%     $-2^{n-1}$ & $2^{n-1}-1$ \\
%     \texttt{long} &  & $-2^{n-1}$ & $2^{n-1}-1$ \\
%     \texttt{unsigned long} &  & $0$ & 2$^n-1$ \\
%     \midrule
%     \texttt{signed long long} & \multirow{3}{*}{1…64} & $-2^{n-1}$ & $2^{n-1}-1$ \\
%     \texttt{long long} &  & $-2^{n-1}$ & $2^{n-1}-1$ \\
%     \texttt{unsigned long long} &  & $0$ & $2^n-1$ \\
%     \bottomrule
%   \end{tabular}
%   \end{DIFnomarkup}
%   \caption{Bit-fields}
%   \label{tab:bitfields}
% \end{table}

% Bit-fields\index{bit-field} have the signedness of their underlying type.
% For example, a bit-field of type \texttt{long} is signed, whereas a
% bit-field of type \texttt{char} is unsigned.

% Bit-fields obey the same size and alignment rules as other structure and
% union members, with the following additions:

% \begin{itemize}
% \item Bit-fields are allocated from left to right (most to least
%   significant).
% \item A bit-field must entirely reside in a storage unit appropriate
%   for its declared type.  Thus, a bit-field never crosses its unit
%   boundary.
% \item Bit-fields must share a storage unit with other structure and
%   union members (either bit-field or non-bit-field) if and only if
%   there is sufficient space within the storage unit.
% \item Unnamed bit-fields' types do not affect the alignment of a
%   structure or union, although an individual bit-field's member
%   offsets obey the alignment constraints.  An unnamed, zero-width
%   bit-field shall prevent any further member, bit-field or other, from
%   residing in the storage unit corresponding to the type of the
%   zero-width bit-field.
% \end{itemize}

% The examples in \crefrange{fig:bitnum}{fig:unnbitf} show
% structure and union member byte offsets in the upper left corners.
% Bit numbers appear in the lower corners.

% \begin{figure}
%   \centering
%   \ifSkipTikZ
% \begin{verbatim}
%              +-------------+-------------+-------------+-------------+
%              | 0           | 1           | 2           | 3           |
% 0x01020304   |      01     |      02     |      03     |      04     |
%              | 0         7 | 8        15 | 16       23 | 24       31 |
%              +-------------+-------------+-------------+-------------+
% \end{verbatim}
%   \else
%   \begin{tikzpicture}[x=1.3ex,y=3em]
%     \path [bitchart box] (0, 0) rectangle (32, 1);
%     \bitchartfields 0 8/01/, 16/02/, 24/03/, 32/04/;
%     \bitchartbytes{0}{0,...,3}
%     \path (current bounding box.west) node[left=1em]{\texttt{0x01020304}};
%   \end{tikzpicture}
%   \fi
%   \caption{Bit numbering}
%   \label{fig:bitnum}
% \end{figure}

% \begin{figure}
%   \centering
%   \ifSkipTikZ
% \begin{verbatim}
%                Word aligned, sizeof is 4
% struct {      +--------+---------+------------+-----------------------+
%     int j:5;  | 0      |         |            |                       |
%     int k:6;  |   j    |    k    |     m      |          pad          |
%     int m:7;  | 0    4 | 5    10 | 11      17 | 18                 31 |
% };            +--------+---------+------------+-----------------------+
% \end{verbatim}
%   \else
%   \begin{tabular}{>{\texttt\bgroup}l<{\texttt\egroup}}
%     struct \{ \\
%     ~~~~~int~j:5; \\
%     ~~~~~int~k:6; \\
%     ~~~~~int~m:7; \\
%     \}; \\
%   \end{tabular}
%   \quad
%   \begin{tikzpicture}[baseline=(mid),x=1.3ex,y=3em]
%     \path [bitchart box] (0, 0) rectangle (32, 1);
%     \bitchartfields 0 5/j/, 11/k/, 18/m/, 32//;
%     \bitchartbytes{0}{0}
%     \path (current bounding box.center) coordinate (mid) {};
%     \path (current bounding box.north)
%     node [above] {Word aligned, \texttt{sizeof} is 4};
%   \end{tikzpicture}
%   \fi
%   \caption{Left-to-right allocation}
%   \label{fig:lralloc}
% \end{figure}

% \begin{figure}
%   \centering
%   \ifSkipTikZ
% \begin{verbatim}
%                  Word aligned, sizeof is 12
% struct {        +--------------+--------------+-----------+-------------+
%     short s:9;  | 0            |              |           | 3           |
%     int   j:9;  |       s      |      j       |    pad    |      c      |
%     char  c;    | 0          8 | 9         17 | 18     23 | 24       31 |
%     short t:9;  |--------------+------------+-+-----------++------------|
%     short u:9;  | 4            |            | 6            |            |
%     char  d;    |       t      |    pad     |       u      |    pad     |
% };              | 32        40 | 41      47 | 48        56 | 57      63 |
%                 |-------------++------------+--------------+------------|
%                 | 8           | 9                                       |
%                 |      d      |                    pad                  |
%                 | 64       71 | 72                                   95 |
%                 +-------------+-----------------------------------------+
% \end{verbatim}
%   \else
%   \begin{tabular}{>{\texttt\bgroup}l<{\texttt\egroup}}
%     struct \{ \\
%     ~~~~~short~s:9; \\
%     ~~~~~int~~~j:9; \\
%     ~~~~~char~~c; \\
%     ~~~~~short~t:9; \\
%     ~~~~~short~u:9; \\
%     ~~~~~char~~d; \\
%     \}; \\
%   \end{tabular}
%   \quad
%   \begin{tikzpicture}[baseline=(mid),x=1.3ex,y=3em]
%     \path [bitchart box] (0, -2) rectangle (32, 1);
%     \bitchartfields 0 9/s/, 18/j/, 24//, 32/c/;
%     \bitchartbytes{0}{0,3}
%     \bitchartsep
%     \begin{scope}[shift={(0,-1)}]
%       \bitchartfields 32 41/t/, 48//, 57/u/, 64//;
%       \bitchartbytes{4}{4,6}
%       \bitchartsep
%     \end{scope}
%     \begin{scope}[shift={(0,-2)}]
%       \bitchartfields  64 72/d/, 96//;
%       \bitchartbytes{8}{8,9}
%     \end{scope}
%     \path (current bounding box.center) coordinate (mid) {};
%     \path (current bounding box.north)
%     node [above] {Word aligned, \texttt{sizeof} is 12};
%   \end{tikzpicture}
%   \fi
%   \caption{Boundary alignment}
%   \label{fig:balign}
% \end{figure}

% \begin{figure}
%   \centering
%   \ifSkipTikZ
% \begin{verbatim}
%                  Halfword aligned, sizeof is 2
% struct {        +-------------+-------------+
%     char  c;    | 0           | 1           |
%     short s:8;  |      c      |      s      |
% };              | 0         7 | 8        15 |
%                 +-------------+-------------+
% \end{verbatim}
%   \else
%   \begin{tabular}{>{\texttt\bgroup}l<{\texttt\egroup}}
%     struct \{ \\
%     ~~~~~char~~c; \\
%     ~~~~~short~s:8; \\
%     \}; \\
%   \end{tabular}
%   \quad
%   \begin{tikzpicture}[baseline=(mid),x=1.3ex,y=3em]
%     \path [bitchart box] (0, 0) rectangle (16, 1);
%     \bitchartfields 0 8/c/, 16/s/;
%     \bitchartbytes{0}{0,1}
%     \path (current bounding box.center) coordinate (mid) {};
%     \path (current bounding box.north)
%     node [above] {Halfword aligned, \texttt{sizeof} is 2};
%   \end{tikzpicture}
%   \fi
% \caption{Storage unit sharing}
% \label{fig:sushar}
% \end{figure}

% \begin{figure}
%   \centering
%   \ifSkipTikZ
% \begin{verbatim}
%                  Halfword aligned, sizeof is 2
%                 +-------------+-------------+
%                 | 0           | 1           |
%                 |      c      |     pad     |
% union {         | 0         7 | 8        15 |
%     char  c;    +-------------+-------------+
%     short s:8;  +-------------+-------------+
% };              | 0           | 1           |
%                 |      s      |     pad     |
%                 | 0         7 | 8        15 |
%                 +-------------+-------------+
% \end{verbatim}
%   \else
%   \begin{tabular}{>{\texttt\bgroup}l<{\texttt\egroup}}
%     union \{ \\
%     ~~~~~char~~c; \\
%     ~~~~~short~s:8; \\
%     \}; \\
%   \end{tabular}
%   \quad
%   \begin{tikzpicture}[baseline=(mid),x=1.3ex,y=3em]
%     \path [bitchart box] (0, 0) rectangle (16, 1);
%     \bitchartfields 0 8/c/, 16//;
%     \bitchartbytes{0}{0,1}
%     \begin{scope}[shift={(0,-1.2)}]
%       \path [bitchart box] (0, 0) rectangle (16, 1);
%       \bitchartfields 0 8/s/, 16//;
%       \bitchartbytes{0}{0,1}
%     \end{scope}
%     \path (current bounding box.south west) coordinate (sw) {}
%     (current bounding box.north west) coordinate (nw) {};
%     \path (current bounding box.center) coordinate (mid) {};
%     \path (current bounding box.north)
%     node [above] {Halfword aligned, \texttt{sizeof} is 2};
%     \path (nw) +(-1em,0) coordinate (nw) {};
%     \draw [decorate,decoration=brace, thick] (sw) + (-1em,0) -- (nw);
%   \end{tikzpicture}
%   \fi
%   \caption{Union allocation}
%   \label{fig:unalloc}
% \end{figure}

% \begin{figure}
%   \centering
%   \ifSkipTikZ
% \begin{verbatim}
%                 Byte aligned, sizeof is 9
%                +-------------+-----------------------------------------+
%                | 0           | 1                                       |
%                |      c      |                     :0                  |
% struct {       | 0         7 | 8                                    31 |
%     char  c;   |-------------+-------------+--------------+------------|
%     int   :0;  | 4           | 5           | 6            | 7          |
%     char  d;   |      d      |     pad     |       :9     |    pad     |
%     short :9;  | 32       39 | 40       47 | 48        55 | 56      63 |
%     char  e;   |-------------+-------------+--------------+------------+
% };             | 8           |
%                |      e      |
%                | 64       71 |
%                +-------------+
% \end{verbatim}
%   \else
%   \begin{tabular}{>{\texttt\bgroup}l<{\texttt\egroup}}
%     struct \{ \\
%     ~~~~~char~~c; \\
%     ~~~~~int~~~:0; \\
%     ~~~~~char~~d; \\
%     ~~~~~short~:9; \\
%     ~~~~~char~~e; \\
%     \}; \\
%   \end{tabular}
%   \quad
%   \begin{tikzpicture}[baseline=(mid),x=1.3ex,y=3em]
%     \path [bitchart box] (0, -2) -| (8, -1) -| (32, 1) -- (0, 1) --cycle;
%     \bitchartfields 0 8/c/, 32/:0/;
%     \bitchartbytes{0}{0,1}
%     \bitchartsep
%     \begin{scope}[shift={(0,-1)}]
%       \bitchartfields 32 40/d/, 48//, 56/:9/, 64//;
%       \bitchartbytes{4}{4,...,7}
%       \draw (0,0) -- (8,0);
%     \end{scope}
%     \begin{scope}[shift={(0,-2)}]
%       \bitchartfields  64 72/e/;
%       \bitchartbytes{8}{8}
%     \end{scope}
%     \path (current bounding box.center) coordinate (mid) {};
%     \path (current bounding box.north)
%     node [above] {Byte aligned, \texttt{sizeof} is 9};
%   \end{tikzpicture}
%   \fi
%   \caption{Unnamed bit-fields}
%   \label{fig:unnbitf}
% \end{figure}

% \subsubsection{Vector Types}
% \index{vector type}
% Vector types are used for SIMD\index{SIMD} (single-instruction,
% multiple-data) programming.  They are not part of the C language, but
% defined by a language extension, such as the ``vector extensions''
% described in the respective section in the GCC manual~\cite{gnu-vec}.

% A vector\index{vector} holds multiple values (``elements'') of a given
% base type (``element type'').  Valid element types include the scalar
% types shown in \cref{tab:scalar}, except for pointer types and the Boolean
% type \texttt{\_Bool}.  The number of elements in a vector must be a power
% of two.  Each allowed combination of base type and number of elements
% forms a distinct vector type.  A single-element vector type is not
% compatible with its base type.

% The size of a vector type is the size of the base type multiplied by the
% number of elements.  Vectors with a size of 1, 2, 4, or $\ge 8$ bytes are
% aligned to a 1-, 2-, 4-, or 8-byte boundary, respectively.

% \section{Function-Calling Sequence}
% \index{function call}
% This section discusses the standard function-calling sequence,
% including stack frame layout, register usage, and parameter passing.

% \subsection{Registers}
% \index{registers}
% The ABI makes the assumption that the processor has 16 general registers,
% \texttt{r0} through \texttt{r15}, and 16 floating-point registers,
% \texttt{f0} through \texttt{f15}.  \ARCH{} processors {\ifzseries have
%   these registers; each register is 64 bits wide.\else have 16 general
%   registers; newer models have 16 IEEE floating-point registers but older
%   systems have only four non-IEEE floating-point registers.  On these
%   older machines Linux emulates 16 IEEE registers within the kernel.  The
%   width of the general registers is 32 bits, and the width of the
%   floating-point registers is 64 bits.\fi}%

% \index{vector registers}
% Optionally, \ARCH{} processors may have a vector facility installed, which
% extends the 64-bit floating-point registers to 128-bit vector registers,
% \texttt{v0} through \texttt{v15}, and provides 16 additional 128-bit
% vector registers, \texttt{v16} through \texttt{v31}.

% In addition, the processor state includes 32-bit access registers
% \texttt{a0} through \texttt{a15}, a 2-bit \index{condition code}condition
% code \texttt{cc}, a 4-bit \index{program mask}program mask \texttt{pm},
% and a 32-bit \index{FPC}floating-point control register \texttt{fpc}.

% \subsubsection{Register Preservation Rules}
% \begin{table}
%   \centering
%   \begin{DIFnomarkup}
%   \begin{threeparttable}
%     \begin{tabularx}{\textwidth}{lXl}
%       \toprule
%       Register name & Role(s) & Call effect\tnote{\dagger} \\
%       \midrule
%       \texttt{r0}, \texttt{r1} & -- & Volatile \\
%       \texttt{r2}{\ifzseries\else , \texttt{r3}\fi}&
%       Argument / return value & Volatile \\
%       {\ifzseries\texttt{r3}, \fi}\texttt{r4}, \texttt{r5} &
%       Arguments & Volatile \\
%       \texttt{r6} & Argument & Saved \\
%       \texttt{r7}…\texttt{r11} & -- & Saved \\
%       \texttt{r12} & (Commonly used as GOT pointer) & Saved \\
%       \texttt{r13} & (Commonly used as literal pool pointer) & Saved \\
%       \texttt{r14} & Return address & Volatile \\
%       \texttt{r15} & Stack pointer & Saved \\
%       \texttt{f0} & Argument / return value & Volatile \\
%       \ifzseries %----------------------------------------
%       \texttt{f2}, \texttt{f4}, \texttt{f6} & Arguments & Volatile \\
%       \texttt{f1}, \texttt{f3}, \texttt{f5}, \texttt{f7} & -- & Volatile \\
%       \texttt{f8}…\texttt{f15} & -- & Saved \\
%       \texttt{v0}…\texttt{v7} & (Extend \texttt{f0}…\texttt{f7}) & Volatile \\
%       \texttt{v8}…\texttt{v15} & (Extend \texttt{f8}…\texttt{f15}) &
%       Volatile\tnote{\dagger\dagger} \\
%       \else %---------------------------------------------
%       \texttt{f2} & Argument & Volatile \\
%       \texttt{f4}, \texttt{f6} & -- & Saved \\
%       \texttt{f1}, \texttt{f3}, \texttt{f5}, \texttt{f7}…\texttt{f15} &
%       -- & Volatile \\
%       \texttt{v0}…\texttt{v3}, \texttt{v5}, \texttt{v7}…\texttt{v15} &
%       (Extend respective FP registers) & Volatile \\
%       \texttt{v4}, \texttt{v6} & (Extend \texttt{f4}, \texttt{f6}) &
%       Volatile\tnote{\dagger\dagger} \\
%       \fi %-----------------------------------------------
%       \texttt{v16}…\texttt{v23} & -- & Volatile \\
%       \texttt{v24} & Argument / return value & Volatile \\
%       \texttt{v25}…\texttt{v31} & Arguments & Volatile \\
%       \texttt{cc} & Condition code & Volatile \\
%       \texttt{pm} & Program mask & Cleared \\
%       \texttt{a0}{\ifzseries , \texttt{a1}\fi} &
%       Reserved for system use & Reserved \\
%       \texttt{a\ifzseries 2\else 1\fi}…\texttt{a15} & -- & Volatile \\
%       \bottomrule
%     \end{tabularx}
%     \medskip
%     \begin{tablenotes}
%     \item [\dagger] Volatile: These registers' values are not preserved across
%       function calls.
%     \item Saved: These registers' values are preserved across function
%       calls.
%     \item Cleared: These registers must be 0 before entering/leaving a
%       function.
%     \item Reserved: These registers must not be modified by
%       ABI-compliant functions.
%     \item [\dagger\dagger] Except for bytes 0--7, which are aliased to
%       {\ifzseries \texttt{f8}…\texttt{f15}\else \texttt{f4},
%         \texttt{f6}\fi}.
%     \end{tablenotes}
%   \end{threeparttable}
%   \end{DIFnomarkup}
%   \caption{Register usage across function calls}
%   \label{tab:regsacrosscall}
% \end{table}

% \Cref{tab:regsacrosscall} summarizes the roles of registers and their
% persistence\index{registers!across function call} across function calls.
% Registers marked as ``saved'' are also referred to as
% ``nonvolatile''\index{nonvolatile}; they ``belong'' to the calling
% function and must retain their values over the function call.  A called
% function modifying these registers must restore their original values
% before returning.  By contrast, ``volatile''\index{volatile} registers
% need not be restored.  To preserve such a register's value across the
% function call, the caller must take care of saving and restoring the value
% by itself.  ``Reserved'' registers are reserved for system use and must
% not be modified at all.

% Using these definitions, the registers are categorized as follows:
% \begin{itemize}
% \item Registers \texttt{r6} through \texttt{r13}, \texttt{r15},
%   {\ifzseries and \texttt{f8} through \texttt{f15}\else \texttt{f4}
%     and \texttt{f6}\fi} are nonvolatile.
% \item Access register{\ifzseries s\fi} \texttt{a0} {\ifzseries and
%     \texttt{a1} are\else is\fi} reserved.
% \item The left halves of vector registers {\ifzseries \texttt{v8} through
%     \texttt{v15}\else \texttt{v4} and \texttt{v6}\fi} are nonvolatile,
%   since they are aliased to {\ifzseries \texttt{f8} through
%     \texttt{f15}\else \texttt{f4} and \texttt{f6}\fi}.  The right halves
%   are volatile.
% \item The program mask \texttt{pm} must be zero before entering and before
%   leaving a function.
% \item All other registers are volatile.
% \item Furthermore the values in registers \texttt{r0} and \texttt{r1}
%   may be altered by the interface code in cross-module calls, so a
%   function cannot depend on the values in these registers having the
%   same values that were placed in them by the caller.
%   % FIXME: What about the FPC?  -> See the FPSCR in the POWER ABI.
% \end{itemize}

% \subsubsection{Register Roles}
% The roles\index{registers!roles} mentioned in \cref{tab:regsacrosscall}
% have the following meaning:
% \begin{description}
% \item[Argument:] \index{argument register} When calling a function, such a
%   register may hold {\ifzseries\else (part of) \fi}%
%   an argument to that function, according to the parameter-passing rules
%   defined in \cref{parameterpassing}.
% \item[Return value:] \index{return value!register} When a called function
%   returns, such a register may hold {\ifzseries\else (part of)\fi}%
%   the return value of that function, according to the rules defined in
%   \cref{retvalues}.
% \item[GOT pointer:] \index{GOT pointer} Global Offset Table pointer.  In a
%   position-independent module, such a register may point to the start of
%   that module's GOT, described in \cref{globaloffsettable}.  If
%   instructions like ``Load Relative'' can be used, no GOT pointer may be
%   needed.
% \item[Literal pool pointer:] \index{literal pool pointer} Some constant
%   local data objects (``literals'') can be encoded in the instructions
%   themselves, using immediate values.  Others are typically grouped into
%   pools of such literals, in which case a register may be set up as a base
%   pointer to such a pool.
% \item[Return address:] \index{return address register} When entering a
%   function, this register, \texttt{r14}, contains the address of the
%   instruction that the function must return to.  Except at function entry,
%   no special role is assigned to \texttt{r14}.
% \item[Stack pointer:] \index{stack pointer} This register, \texttt{r15},
%   always points to the lowest allocated valid stack frame.  It shall
%   maintain an 8-byte alignment.  A function may decrement \texttt{r15} to
%   allocate a new stack frame or to enlarge the current one.  Before
%   returning, \texttt{r15} must be restored to its original value.  For
%   more information about stack frames, see \cref{stackframe}.
% \end{description}

% \subsubsection{Registers And Signal Handling}
% Signals can interrupt processes.  Functions called during signal
% handling have no unusual restrictions on their use of registers.
% Moreover, if a signal-handling function returns, the process will
% resume its original execution path with all registers restored to
% their original values.  Thus programs and compilers may freely use all
% registers listed above, except those reserved for system use, without
% the danger of signal handlers inadvertently changing their values.

% \subsubsection{Register Usage in Inline Assemblies}
% \index{inline assembly!register usage}

% With these calling conventions, the following usage of the registers
% for inline assemblies is recommended:
% \begin{itemize}
% \item General registers \texttt{r0} and \texttt{r1} should be used
%   internally whenever possible.
% \item General registers \texttt{r2} to \texttt{r5} should be second
%   choice.
% \item General registers \texttt{r12} to \texttt{r15} should only be
%   used for their standard function.
% \end{itemize}

% \subsection{The Stack Frame}
% \label{stackframe}
% \index{stack frame}

% A function will be passed a frame on the runtime stack by the function
% which called it, and may allocate a new stack frame.  A new stack frame is
% required if the called function will in turn call further functions (which
% must be passed the address of the new frame).  The stack grows downward
% from high addresses.  Each stack frame is aligned on an 8-byte boundary.
% General register \texttt{r15} holds the stack pointer and always
% points to the first byte of the lowest allocated stack frame.
% \Cref{fig:stackframe} shows the stack frame organization.

% \begin{figure}
%   \centering
%   \ifSkipTikZ
% \begin{verbatim}
%          |          ...             |
%          |                          |
%          |   Previous stack frame   |
% +------> ============================ High address
% |        | Local and spill variable |
% |        | area of calling function |
% |        +--------------------------+
% |        | Parameter area passed to |
% |        |     called function      |
% | SP+160 +--------------------------+
% |        |  Register save area for  |
% |        |   called function use    |
% |   SP+8 +--------------------------+
% +--------|  Back chain (optional)   |
%    SP -> ============================ Low address
% \end{verbatim}
%   \else
%   \pgfsetlayers{background,main}
%   \begin{tikzpicture}
%     \matrix [inner sep=0pt,
%     nodes={text width=12em, text badly centered, inner sep=1ex}] (m)
%     {
%       \node (prev)     {\stackit[c]{\ldots\\\strut\\ Previous stack frame}}; \\
%       \node (spill)    {Local and spill variable area of calling function}; \\
%       \node (param)    {Parameter area passed to called function}; \\
%       \node (save)     {Register save area for called function use}; \\
%       \node (back)     {Back chain (optional)}; \\
%     };
%     \draw \foreach \Node in {prev, spill, param, save} {
%       (\Node.south -| m.west) -- (\Node.south -| m.east)
%     };
%     \draw [very thick, shorten <=-1ex, shorten >=-1ex] (prev.south -| m.west)
%     -- (prev.south -| m.east) node [right=1em] {High address};
%     \path
%     (m.south east) node [right=1em] {Low address}
%     (param.south -| m.west) node [left]
%     {\footnotesize\texttt{SP+\STACKSIZE{}}}
%     (save.south -| m.west) node [left]
%     {\footnotesize\texttt{SP+\NBYTES}}
%     (m.south west) node [left=2em] (sp) {\texttt{SP}};
%     \draw [very thick, shorten <=-1ex, shorten >=-1ex]
%     (m.south west) -- (m.south east);
%     \draw [->, shorten >=1.5ex] (sp) -- (m.south west);
%     \draw [rounded corners, shorten >=1.5ex, ->]
%     (back -| m.west)  -- +(-4em,0) |- (prev.south -| m.west);
%     \begin{pgfonlayer}{background}
%       \path [memory layout] (m.north west) |- (m.south east) --
%       (m.north east);
%     \end{pgfonlayer}
%   \end{tikzpicture}
%   \fi
%   \caption[Standard stack frame]{Standard stack frame.  \texttt{SP}
%     denotes the value of \texttt{r15} upon entering the called function.}
%   \label{fig:stackframe}
% \end{figure}

% \subsubsection{Back-Chain Slot}
% \index{back chain}
% The first {\ifzseries double\fi}word of a calling function's stack frame
% is preserved across function calls.  It may be used for maintaining a back
% chain for stack unwinding, in which case it must hold the address of the
% previously allocated stack frame (toward higher addresses), or zero
% (\texttt{NULL}) if there is none.

% Maintenance of the stack back chain is optional.  If a function chooses to
% maintain the back chain, it should also store the values of \texttt{r14}
% and \texttt{r15} at function entry into the register save area, using
% their standard save slots as shown in \cref{fig:regsave}.

% \subsubsection{Register Save Area}
% \index{register save area}
% The first \STACKSIZE{} bytes of a calling function's stack frame,
% excluding the initial {\ifzseries double\fi}word, are referred to as the
% register save area.  This area must be allocated by the caller and may be
% used by the called function in any way.  For example, if the called
% function is going to modify any nonvolatile registers, it may use the
% register save area for saving these registers' original values first.  It
% is customary to assign a standard save slot to each register, as shown in
% \cref{fig:regsave}.

% \begin{figure}
%   \centering
%   \ifSkipTikZ
% \begin{verbatim}
% 160 +-------------------------------+
%     | f6                            |
%     | f4   Floating-point argument  |
%     | f2     register save area     |
%     | f0                            |
% 128 +-------------------------------+
%     | r15                           |
%     |  .                            |
%     |  .   Other register save area |
%     |  .                            |
%     | r7                            |
%  56 +-------------------------------+
%     | r6                            |
%     |  .                            |
%     |  .   Argument register save   |
%     |  .            area            |
%     | r2                            |
%  16 +-------------------------------+
%     |              Unused           |
%   8 +-------------------------------+
% \end{verbatim}
%   \else
%   \begin{tikzpicture}
%     \matrix [memory layout, inner sep=0pt, nodes={inner sep=1ex},
%     description/.style={text width=12em, text badly centered},
%     cells={anchor=center}] (m) {
%       \node {\texttt{\stackit{f6\\f4\\f2\\f0}}}; &
%       \node [description] {Floating-point argument register save area};
%       \\
%       \coordinate (A);\\
%       \path (0,1.5ex) node [above] (r15) {\texttt{r15}};
%       \path (0,-1.5ex) node [below] (r7) {\texttt{r7}}; &
%       \node [description] {Other register save area};
%       \\
%       \coordinate (B);\\
%       \path (0,1.5ex) node [above] (r6) {\texttt{r6}};
%       \path (0,-1.5ex) node [below] (r2) {\texttt{r2}}; &
%       \node [description] {Argument register save area};
%       \\
%       \coordinate (C);\\
%       &
%       \node [description] {Unused};
%       \\
%     };
%     \draw [<->] (r7) -- (r15);
%     \draw [<->] (r2) -- (r6);
%     \foreach \Node in {A, B, C} {
%       \draw (\Node -| m.west) -- (\Node -| m.east);
%     };
%     \ifzseries
%     \foreach \where/\offs in {m.north/160, A/128, B/56, C/16, m.south/8} {
%       \path (\where -| m.west) +(-1ex,0)
%       node [left] {\texttt{\offs}};
%     }
%     \else
%     \foreach \where/\offs in {m.north/96, A/64, B/28, C/8, m.south/4} {
%       \path (\where -| m.west) +(-1ex,0)
%       node [left] {\texttt{\offs}};
%     }
%     \fi
%   \end{tikzpicture}
%   \fi
%   \caption[Register save area usage example]
%   {Register save area usage example.  The slots for \texttt{r2} through
%     \texttt{r5} and for the floating-point argument registers are used
%     when the called function receives varying arguments.}
%   \label{fig:regsave}
% \end{figure}

% \subsubsection{Parameter Area}
% \index{parameter area}
% The parameter area shall be allocated by a calling function if some
% parameters cannot be passed in registers, but must be passed on the stack
% instead (see \cref{parameterpassing}).  This area starts at byte offset
% \STACKSIZE{} of the calling function's stack frame and consists of as many
% \NBYTES{}-byte parameter slots as needed.  The calling function cannot
% rely on the contents of these slots to be preserved across the function
% call.

% \subsubsection{Stack Frame Allocation}
% \index{stack frame!allocation}
% A function may allocate a new stack frame by decrementing the stack
% pointer by the size of the new frame.  The stack pointer must be restored
% prior to return.  By restoring the stack pointer, the allocated stack
% frame is deallocated and may not be accessed after that.

% A new stack frame is required if the function calls further functions.
% Then the stack frame must at least contain the back chain slot, the
% register save area, and the parameter area (if needed).  The remaining
% space in the stack frame is called the ``local-variable area.''  It
% immediately follows the parameter area and can have arbitrary size,
% provided that it contains any padding necessary to make the entire frame a
% multiple of 8 bytes in length.

% If a function does not call any other functions and does not require more
% stack space than available in the register save area, it need not
% establish a stack frame.

% \subsection{Parameter Passing}
% \label{parameterpassing}
% \index{parameter passing}
% Arguments to called functions are passed in registers.  Since all
% computations must be performed in registers, memory traffic can be
% eliminated if the caller can compute arguments into registers and pass
% them in the same registers to the called function, where the called
% function can then use these arguments for further computation in the
% same registers.  The number of registers implemented in a processor
% architecture naturally limits the number of arguments that can be
% passed in this manner.

% This ABI defines that the following registers shall be used for parameter
% passing:\index{registers!parameter passing}
% \begin{itemize}
% \item General registers \texttt{r2} to \texttt{r5} (volatile)
% \item General register \texttt{r6} (nonvolatile)
% \item Floating-point registers {\ifzseries\texttt{f0}, \texttt{f2},
%     \texttt{f4} and \texttt{f6}\else \texttt{f0} and \texttt{f2}\fi}
%   (volatile)
% \item Vector registers \texttt{v24} to \texttt{v31} (volatile)
% \end{itemize}

% If needed, more arguments are passed in the parameter area, which starts
% \STACKSIZE{} bytes above the stack pointer (see \cref{fig:prmarea}).

% \begin{figure}
%   \centering
%   \ifSkipTikZ
% \begin{verbatim}
%                               High Address
%       |        ...         |
%       |                    | \
%  +16  |  Parameter slot 3  |  |
%       +--------------------+  |
%   +8  |  Parameter slot 2  |   > Parameter area
%       +--------------------+  |
%  160  |  Parameter slot 1  |  |
%       +--------------------+ /
%       |                    |
%       | Register save area |
%    8  |                    |
%       +--------------------+
%       |   Back chain slot  |
%       +--------------------+  Low Address
% \end{verbatim}
%   \else
%   \pgfsetlayers{background,main}
%   \begin{tikzpicture}
%     \matrix [inner sep=0pt, nodes={inner sep=1ex}] (m) {
%      \node (A0) {\stackit[c]{\ldots\\\strut\\ Parameter slot 3}};\\
%       \coordinate (A) {}; \\
%       \node {Parameter slot 2};\\
%       \coordinate (B) {}; \\
%       \node {Parameter slot 1};\\
%       \coordinate (C) {}; \\
%       \node [minimum height=4em] {Register save area};\\
%       \coordinate (D) {}; \\
%       \node {Back chain slot};\\
%     };
%     \foreach \Node in {A, ..., D} {
%       \draw (\Node -| m.west) -- (\Node -| m.east);
%     };
%     \ifzseries
%     \foreach \Node/\offs in {A/+16, B/+8, C/160, D/8} {
%       \path (\Node -| m.west) + (-1ex,0)
%       node [above left] {\texttt{\offs}};
%     };
%     \else
%     \foreach \Node/\offs in {A/+8, B/+4, C/96, D/4} {
%       \path (\Node -| m.west) + (-1ex,0)
%       node [above left] {\texttt{\offs}};
%     };
%     \fi
%     \path (m.south east) +(1ex,0) node [above right] {Low Address};
%     \path (m.north east) +(1ex,0) node [below right] {High Address};
%     \path (C -| m.east) ++ (1ex,0) coordinate (C right) {};
%     \draw [decorate,decoration=brace, thick]
%     (A0 -| m.east) + (1ex,0) -- (C right)
%     node [midway, right=1ex] {Parameter area};
%     \begin{pgfonlayer}{background}
%       \path [memory layout] (m.north west) |- (m.south east) --
%       (m.north east);
%     \end{pgfonlayer}
%   \end{tikzpicture}
%   \fi
%   \caption{Parameter area}
%   \label{fig:prmarea}
% \end{figure}

% The following algorithm\index{parameter passing!algorithm} specifies where
% argument data is passed for the C language.  For this purpose, consider
% the arguments as ordered from left (first argument) to right, although the
% order of evaluation of the arguments is unspecified.  In this algorithm
% \texttt{fr} contains the number of the next available floating-point
% register, \texttt{gr} contains the number of the next available general
% register, and \texttt{starg} is the address of the next available stack
% argument word.

% \begin{description}
% \item[\jumplabel{initialize}:] Allocate a sufficiently large parameter
%   area for the arguments that will be passed according to the
%   \jumplabel{more} and \jumplabel{more\_vec} descriptions that follow.
%   Set $\mbox{\texttt{fr}}=0$, $\mbox{\texttt{gr}}=2$,
%   $\mbox{\texttt{vr}}=24$, and \texttt{starg} to the address of the
%   parameter area.
% \item[\jumplabel{return\_parameter}:] If the called function's return
%   value is not passed in a register (according to \cref{retvalues}), then
%   allocate a return value buffer, store its address in \texttt{r2}, and
%   set $\mbox{\texttt{gr}}=3$.
% \item[\jumplabel{scan}:] If there are no more arguments, terminate.
%   Otherwise, select one of the following depending on the type of the next
%   argument:
%   \begin{description}
%   \item[\jumplabel{double\_or\_float}:] A double\_or\_float is
%     one of the following:
%     \begin{itemize}
%     \item A \texttt{float} or \texttt{\_Decimal32}.
%     \item A \texttt{double} or \texttt{\_Decimal64}.
%     \item A structure equivalent to one of the above.  A structure is
%       equivalent to a type $T$ if and only if it has exactly one member,
%       which is either of type $T$ itself or a structure equivalent to
%       type~$T$.
%     \end{itemize}
%     If $\mbox{\texttt{fr}}>{\ifzseries 6\else 2\fi}$, that is, if there are no
%     more floating-point registers available for parameter passing, go to
%     \jumplabel{more}.  Otherwise, load the argument value into
%     floating-point register \texttt{fr}, set \texttt{fr} to
%     $\mbox{\texttt{fr}}+2$, and go to \jumplabel{scan}.
%   \item[\jumplabel{vector\_arg}:] A vector\_arg has one of the following
%     types:
%     \begin{itemize}
%     \item Any vector type whose size is 16 bytes or less.
%     \item A structure equivalent to such a vector type, where
%       ``equivalent'' has the same meaning as for double\_or\_float.
%     \end{itemize}
%     If the argument is part of the varying arguments (see \cref{varargs}),
%     or if $\mbox{\texttt{vr}}=\mbox{\emph{nil}}$, go to
%     \jumplabel{more\_vec}.  Otherwise, load the value left-justified into
%     vector register \texttt{vr}, set \texttt{vr} to the next entry in the
%     list
%     \[ 24, 26, 28, 30, 25, 27, 29, 31, \mbox{\emph{nil}} \]
%     and go to \jumplabel{scan}.
%     \ifzseries\else
%   \item[\jumplabel{double\_arg}:] A double\_arg is one of type
%     \texttt{long long}, or is a struct or a union of size 8 bytes which is
%     not a double\_or\_float.%
%     \\
%     If $\mbox{\texttt{gr}}>5$ set \texttt{gr} to 7 and go to
%     \jumplabel{more}.  Else load the argument's lower-addressed word into
%     \texttt{gr} and the higher-addressed word into $\mbox{\texttt{gr}}+1$,
%     set \texttt{gr} to $\mbox{\texttt{gr}}+2$, and go to \jumplabel{scan}.
%     \fi
%   \item[\jumplabel{simple\_arg}:] A simple\_arg is one of the following:
%     \begin{itemize}
%     \item One of the simple integer types no more than \NBITS{} bits wide.
%       This includes \texttt{signed} \texttt{char}, \texttt{short},
%       \texttt{int}, \texttt{long},{\ifzseries{} \texttt{long}
%         \texttt{long},\fi} their unsigned counterparts, \texttt{\_Bool},
%       and any \texttt{enum} type.  If such an argument is shorter than
%       \NBITS{} bits, replace it by a full \NBITS{}-bit integer
%       representing the same number, using sign or zero extension, as
%       appropriate.
%     \item Any pointer type.
%     \item A struct or a union of 1, 2, {\ifzseries 4, or 8\else or 4\fi}
%       bytes that is not a double\_or\_float (see above).  If such a
%       struct or union is strictly smaller than \NBYTES{} bytes, extend it
%       to \NBYTES{} bytes by adding padding bytes with unspecified contents
%       on the left.
%     \item A struct or union of any other size, a complex type, an
%       \texttt{\_\_int128}, a \texttt{long} \texttt{double}, a
%       \texttt{\_Decimal128}, or a vector whose size exceeds 16 bytes.
%       Replace such an argument by a pointer to the object, or to a copy
%       where necessary to enforce call-by-value semantics.  Only if the
%       caller can ascertain that the object is ``constant'' can it pass a
%       pointer to the object itself.
%     \end{itemize}
%     If $\mbox{\texttt{gr}}>6$, go to \jumplabel{more}.  Otherwise load the
%     argument value (now \NBITS{} bits wide) into general register
%     \texttt{gr}, set \texttt{gr} to $\mbox{\texttt{gr}}+1$, and go to
%     \jumplabel{scan}.
%   \end{description}
% \item[\jumplabel{more}:] The argument cannot be passed in registers; it
%   will be passed in the parameter area of the caller's stack frame
%   instead.  After having applied the replacement rules previously
%   explained as appropriate, the argument now has a size of {\ifzseries 8
%     bytes, except when its type is equivalent to \texttt{float} or
%     \texttt{\_Decimal32}, in which case it has 4 bytes.\else 4 or 8
%     bytes.\fi}%
%   \\
%   Copy the argument value {\ifzseries right-aligned into the 8-byte
%     parameter slot at the current stack position \texttt{starg}, leaving
%     the skipped bytes (if any) at unspecified values.\else to the current
%     stack position \texttt{starg}.\fi} Increment \texttt{starg} by
%   {\ifzseries 8\else the argument size\fi}, then go to \jumplabel{scan}.
% \item[\jumplabel{more\_vec}:] The argument cannot be passed in vector
%   registers, but will be passed in the parameter area.  Copy its value to
%   the current stack position \texttt{starg}, increment \texttt{starg} by
%   the argument size, align \texttt{starg} to the next \NBYTES{}-byte
%   boundary, and go to \jumplabel{scan}.
% \end{description}

% As an example, assume the declarations and the function call shown in
% \cref{lst:prmpass}.  The corresponding register allocation and
% storage would be as shown in \cref{tab:prmpass}.

% \ifzseries\else In this example \texttt{r6} is unused as the \texttt{long}
% \texttt{long} variable \texttt{ll} will not fit into a single register.\fi

% \begin{table}
%   \centering
%   \begin{lstlisting}[style=embed, label=lst:prmpass,
%     caption={Parameter-passing example}]
% typedef float __attribute__((vector_size(8)) v2f_t;

% int i, j, k, l;
% long long ll;
% double f, g, h;
% v2f_t v1, v2;
% int m;
% x = func(i, j, g, k, l, ll, f, h, m, v1, v2);
%   \end{lstlisting}
%   \begin{DIFnomarkup}
%   \par\medskip
%   \newdimen\mycolwidth\mycolwidth=.22\hsize
%   \begin{tabular}[t]{%
%       >{\texttt\bgroup}l<{\egroup :~}>{\texttt\bgroup}l<{\egroup}
%       >{\texttt\bgroup}l<{\egroup :~}>{\texttt\bgroup}l<{\egroup}
%       >{\texttt\bgroup}l<{\egroup :~}>{\texttt\bgroup}l<{\egroup}
%       r<{:~}>{\texttt\bgroup}l<{\egroup}}
%     \toprule
%     \multicolumn{2}{>{\raggedright}p{\mycolwidth}}{General registers} &
%     \multicolumn{2}{>{\raggedright}p{\mycolwidth}}{Floating-point registers} &
%     \multicolumn{2}{>{\raggedright}p{\mycolwidth}}{Vector registers} &
%     \multicolumn{2}{>{\raggedright}p{\mycolwidth}}{Stack frame offset} \\
%     \midrule
%     \ifzseries
%     r2 & i  & f0    & g & v24   & v1 & 160   & m  \\
%     r3 & j  & f2    & f & v26   & v2 & \omit &    \\
%     r4 & k  & f4    & h & \omit &    & \omit &    \\
%     r5 & l  & \omit &   & \omit &    & \omit &    \\
%     r6 & ll & \omit &   & \omit &    & \omit &    \\
%     \else
%     r2 & i  & f0    & g & v24   & v1 & 96    & ll \\
%     r3 & j  & f2    & f & v26   & v2 & 104   & h  \\
%     r4 & k  & \omit &   & \omit &    & 112   & m  \\
%     r5 & l  & \omit &   & \omit &    & \omit &    \\
%     r6 & -- & \omit &   & \omit &    & \omit &    \\
%     \fi
%     \bottomrule
%   \end{tabular}
%   \end{DIFnomarkup}
%   \caption{Parameter-passing example: register allocation}
%   \label{tab:prmpass}
% \end{table}

% \subsection{Variable Argument Lists}
% \label{varargs}
% \index{variable argument list}
% If a C function declaration has a parameter type list that terminates with
% an ellipsis ``\texttt{...}\,,'' a call to that function can have varying
% numbers and types of arguments corresponding to the ellipsis.  Except for
% vector arguments of 16 bytes or less, these varying arguments are passed
% to the called function as if the ellipsis were replaced with a parameter
% type list of the actual arguments.  Varying vector arguments are always
% passed in the parameter area.

% \begin{lstlisting}[style=float,caption={\texttt{va\_list}
%     declaration example},label={lst:valist}]
% typedef struct __va_list_tag {
%     long __gpr;
%     long __fpr;
%     void *__overflow_arg_area;
%     void *__reg_save_area;
% } va_list[1];
% \end{lstlisting}

% The called function can store the varying arguments in a variable of type
% \texttt{va\_list}, defined in \texttt{<stdarg.h>}.  Such a variable
% represents the list of remaining arguments to be processed and can be
% passed down to further functions.  The \ABINAME{} ABI defines
% \texttt{va\_list} to be equivalent to a structure with four {\ifzseries
%   double\fi}word members, or to an array whose single element is such a
% structure, like the declaration shown in \cref{lst:valist}.  The
% declaration as an array reduces copying of the structure when used as an
% argument.  The structure members have the following meaning:

% \begin{description}
% \item[\texttt{\_\_gpr}] holds the number (0 to 5) of general argument
%   registers that have already been processed.
% \item[\texttt{\_\_fpr}] holds the number (0 to {\ifzseries 4\else 2\fi})
%   of floating-point argument registers that have already been processed.
% \item[\texttt{\_\_overflow\_arg\_area}] points to the first ``overflow
%   argument'' (passed via the parameter area) that has not been processed
%   yet.
% \item[\texttt{\_\_reg\_save\_area}] points to the start of a
%   \STACKSIZE{}-byte memory region that contains the saved values of all
%   argument registers, with the general registers (\texttt{r2} to
%   \texttt{r6}) starting at offset {\ifzseries 16\else 8\fi} and the
%   floating-point registers ({\ifzseries\texttt{f0}, \texttt{f2},
%     \texttt{f4}, and \texttt{f6}\else \texttt{f0} and \texttt{f2}\fi})
%   starting at offset {\ifzseries 128\else 64\fi}.  These offsets
%   correspond to the layout shown in \cref{fig:regsave}.  The argument
%   registers that have already been processed do not actually need to be
%   saved in their slots.
% \end{description}

% \paragraph{Note:}
% Since \texttt{va\_list} may be defined as an array, a variable of this
% type cannot be copied by a simple C assignment.  The standard C header
% \texttt{<stdarg.h>} defines the macro \texttt{va\_copy} for this purpose
% instead.  Any C code that intends to be portable across platforms should
% use this macro for copying a \texttt{va\_list} variable.

% \subsection{Return Values}
% \label{retvalues}
% \index{return value!passing}
% \index{registers!return value passing}
% A function must pass its return value either in general register
% \texttt{r2},{\ifzseries\else{} in the register pair
%   \texttt{r2}/\texttt{r3},\fi} in floating-point register \texttt{f0}, in
% vector register \texttt{v24}, or in a return value buffer allocated by the
% caller, depending on the return value type:

% \begin{itemize}
% \item A value of type \texttt{double} or \texttt{\_Decimal64} is returned
%   in \texttt{f0}.
% \item A value of type \texttt{float} or \texttt{\_Decimal32} is returned
%   in the left half of \texttt{f0} and encoded in short BFP format or short
%   DFP format, respectively.  The right half of \texttt{f0} is unspecified.
% \item Any integer type with \NBITS{} or fewer bits, including
%   \texttt{\_Bool}, as well as any \texttt{enum} type, is returned in
%   \texttt{r2}.  The return value is zero- or sign-extended to \NBITS{}
%   bits, as appropriate.
% \item A pointer to any type is returned in \texttt{r2}.
%   \ifzseries\else
% \item A value of type \texttt{long} \texttt{long} or \texttt{unsigned}
%   \texttt{long} \texttt{long} is returned with the lower addressed half in
%   \texttt{r2} and the higher in \texttt{r3}.\fi
% \item A vector of 16 or fewer bytes is returned left-aligned in
%   \texttt{v24}.  The padding bits' values are unspecified.
% \item Any other type, such as \texttt{long} \texttt{double},
%   \texttt{\_Decimal128}, \texttt{\_\_int128}, a complex type, a structure,
%   a union, or a vector larger than 16 bytes, is returned in a return value
%   buffer allocated by the caller.  This buffer's address is treated like a
%   ``hidden argument'' and passed by the caller in \texttt{r2}.
% \end{itemize}

% \section{Operating System Interface}
% This section describes various interfaces with the operating system that
% are specific to the \ABINAME{} ABI\@.

% \subsection{Signal Context}
% \index{signal context}
% A signal handler that was installed with \texttt{sigaction} using the
% \texttt{SA\_SIGINFO} flag receives three arguments, as follows:
% \begin{center}
%   \lstinline@void handler(int sig, siginfo_t *info, void *ucontext);@
% \end{center}
% The second argument \texttt{info} is a pointer to a structure containing
% additional signal information, including the number \texttt{si\_code} that
% indicates why the signal \texttt{sig} was sent.

% The third argument \texttt{ucontext} points to a \texttt{ucontext\_t}
% structure on the stack where signal-related context information has been
% saved by the operating system.  It contains the processing context to be
% restored when resuming the interrupted program, including the
% architecture-dependent register state.  Although most signal handlers will
% ignore this information, some may access it for debugging purposes such as
% printing the registers, or when their logic depends on that state.

% \Cref{lst:ucontext} shows the declaration of \texttt{ucontext\_t} on
% systems implementing the \ABINAME{} ABI.

% \begin{lstlisting}[style=float,caption={[The \texttt{ucontext\_t}
%     structure]The \texttt{ucontext\_t} structure.  The size of
%     \texttt{uc\_sigmask} may vary, and additional information may be
%     stored after it.},label={lst:ucontext}]
% typedef struct {
%     unsigned long      mask;       /* PSW mask */
%     unsigned long      addr;       /* PSW address */
% } __psw_t;

% typedef union {
%     double             d;
%     float              f;
% } fpreg_t;

% typedef struct {
%     unsigned int       fpc;        /* floating-point control register */
%     fpreg_t            fprs[16];   /* floating-point registers */
% } fpregset_t;

% typedef struct {
%     __psw_t            psw;
%     unsigned long      gregs[16];  /* general registers */
%     unsigned int       aregs[16];  /* access registers */
%     fpregset_t         fpregs;
% } mcontext_t;

% typedef struct {
%     void              *ss_sp;
%     int                ss_flags;
%     size_t             ss_size;
% } stack_t;

% typedef ... sigset_t;               /* opaque type */

% struct ucontext_t {
%     unsigned long      uc_flags;
%     struct ucontext_t *uc_link;
%     stack_t            uc_stack;
%     mcontext_t         uc_mcontext; /* machine-specific context */
%     sigset_t           uc_sigmask;  /* blocked signals */
% };
% \end{lstlisting}

% \subsection{Exception Interface}
% \label{exceptionint}
% \index{exception}
% When the CPU detects an exceptional condition while a process is executing
% instructions, an \index{interruption}interruption may occur, transferring
% control to the operating system.  The operating system then handles the
% interruption either in a manner transparent to the application, or by
% delivering a signal.

% If such an exception and its corresponding interruption are immediately
% caused by the execution of an instruction, the exception is called
% ``synchronous''.  Program interruptions generally fall into this category.
% They may give rise to \texttt{SIGILL}, \texttt{SIGSEGV}, \texttt{SIGBUS},
% \texttt{SIGTRAP}, or \texttt{SIGFPE}\@.  If one of these signals is
% generated due to an exception when the signal is blocked, the behavior is
% undefined.

% When a signal handler other than for \texttt{SIGSEGV} or \texttt{SIGBUS}
% gets control after a synchronous exception, the \texttt{si\_addr} field in
% the signal handler's \texttt{siginfo\_t} argument points to the
% instruction that caused the exception, while the \index{PSW address!after
%   signal}PSW address in the signal context points to the next instruction.

% In the case of \texttt{SIGSEGV} or \texttt{SIGBUS}, the PSW address points
% to the faulting instruction instead, whereas \texttt{si\_addr} points to
% the address of the memory access causing the fault, possibly rounded down
% to a page boundary.

% The correspondence between the causes of program interruptions and the
% resulting signals\index{signal!from exception} is shown in
% \cref{tab:exceptions}.

% \begin{table}
%   \centering
%   \begin{DIFnomarkup}
%   \begin{threeparttable}
%     \begin{tabular}{llll}
%       \toprule
%       \ARCH{} exception
%       & Signal & \texttt{si\_code} \\
%       \midrule
%       Addressing & \multirow{10}{*}{\texttt{SIGILL}} & \texttt{ILL\_ILLADR} \\
%       Data, general-operand & & \texttt{ILL\_ILLOPN} \\
%       Execute & & \texttt{ILL\_ILLOPN} \\
%       Operand & & \texttt{ILL\_ILLOPN} \\
%       Operation, no breakpoint\tnote{\dagger} & & \texttt{ILL\_ILLOPC} \\
%       Privileged-operation & & \texttt{ILL\_PRVOPC} \\
%       Special-operation & & \texttt{ILL\_ILLOPN} \\
%       Space-switch & & \texttt{ILL\_PRVOPC} \\
%       Specification & & \texttt{ILL\_ILLOPN} \\
%       Transaction-constraint & & \texttt{ILL\_ILLOPN} \\
%       \midrule
%       Operation, breakpoint\tnote{\dagger} & \texttt{SIGTRAP}
%                & \texttt{TRAP\_BRKPT} \\
%       \midrule
%       Data, (simulated) IEEE invalid operation
%       & \multirow{18}{*}{\texttt{SIGFPE}} & \texttt{FPE\_FLTINV} \\
%       Data, (simulated) IEEE division by zero & & \texttt{FPE\_FLTDIV} \\
%       Data, any (simulated) IEEE overflow & & \texttt{FPE\_FLTOVF} \\
%       Data, any (simulated) IEEE underflow & & \texttt{FPE\_FLTUND} \\
%       Data, any (simulated) IEEE inexact\tnote{\ddagger} &
%                & \texttt{FPE\_FLTRES} \\
%       Data, neither IEEE nor general-operand & & \texttt{SI\_USER} \\
%       Fixed-point/decimal divide & & \texttt{FPE\_INTDIV} \\
%       Fixed-point/decimal overflow & & \texttt{FPE\_INTOVF} \\
%       HFP divide & & \texttt{FPE\_FLTDIV} \\
%       HFP exp\@. overflow & & \texttt{FPE\_FLTOVF} \\
%       HFP exp\@. underflow & & \texttt{FPE\_FLTUND} \\
%       HFP square root & & \texttt{FPE\_FLTINV} \\
%       HFP significance & & \texttt{FPE\_FLTRES} \\
%       Vector-processing, invalid operation & & \texttt{FPE\_FLTINV} \\
%       Vector-processing, division by zero & & \texttt{FPE\_FLTDIV} \\
%       Vector-processing, overflow & & \texttt{FPE\_FLTOVF} \\
%       Vector-processing, underflow & & \texttt{FPE\_FLTUND} \\
%       Vector-processing, inexact & & \texttt{FPE\_FLTRES} \\
%       \midrule
%       Protection
%       & \multirow{2}{*}{\texttt{SIGSEGV}} & \texttt{SEGV\_ACCERR} \\
%       Any translation\tnote{*} & & \texttt{SEGV\_MAPERR} \\
%       \midrule
%       Any translation\tnote{*} & \texttt{SIGBUS} & \texttt{BUS\_ADDRERR} \\
%       \bottomrule
%     \end{tabular}
%     \medskip
%     \begin{tablenotes}
%     \item [\dagger] A breakpoint\index{breakpoint} is recognized when a
%       \texttt{ptrace} target executes the special illegal instruction
%       \texttt{0x0001}.
%     \item [\ddagger] Except if an overflow or underflow condition is
%       indicated as well.
%     \item [*] For a translation exception the operating system may yield
%       SIGSEGV or SIGBUS, or it may handle the fault without a signal.
%     \end{tablenotes}
%   \end{threeparttable}
%   \end{DIFnomarkup}
%   \caption[Exceptions and signals]{Exceptions and signals.
%     \texttt{si\_code} refers to the respective field in
%     \texttt{siginfo\_t}.}
%   \label{tab:exceptions}
% \end{table}

% \subsection{Virtual Address Space}
% \index{address space}
% Processes execute in a \ADDRBITS{}-bit virtual address
% space.  Memory management translates virtual addresses to physical
% addresses, hiding physical addressing and letting a process run
% anywhere in the system's real memory.  Processes typically begin with
% three logical segments, commonly called ``text,'' ``data,'' and
% ``stack.''  An object file may contain more segments (for example, for
% debugger use), and a process can also create additional segments for
% itself with system services.

% \paragraph{Note:}
% \index{virtual address}
% The term ``virtual address'' as used in this document refers to a
% \ADDRBITS{}-bit address generated by a program, as
% contrasted with the physical address to which it is mapped.

% \subsection{Page Size}
% \index{memory page}
% \index{page size}
% Memory is organized into pages, which are the system's smallest units
% of memory allocation.  The hardware page size for \ARCHarch{}
% is 4096 bytes.

% \subsection{Virtual Address Assignments}
% Processes have {\ifzseries a 42, 53, or 64\else the full 31\fi}-bit
% address space available to them{\ifzseries, depending on the Linux
%   kernel level\fi}.

% \Cref{fig:vac} shows the virtual address configuration on \theARCH{}.
% The segments with different properties are typically
% grouped in different areas of the address space.  The loadable segments
% may begin at zero (\texttt{0}); the exact addresses depend on the
% executable file format (see \cref{chobjfiles,chprogload}).  The process's
% stack resides at the end of the virtual memory and grows downwards.
% Processes can control the amount of virtual memory allotted for stack
% space, as described below.

% \begin{figure}
%   \centering
%   \ifSkipTikZ
% \begin{verbatim}
% 0x3ffffffffff+----------------------------+ End of memory
%              |                            |
%              |           Stack            |
%              |                            |
%              +----------------------------+
%              |                            |
%              |      Dynamic segments      |
%   Anonymous  |                            |
% mapping base +----------------------------+
%              |                            |
%              |            Heap            |
%              |                            |
%              +----------------------------+
%              |                            |
%              |      Executable file       |
%              |                            |
% Program base +----------------------------+
%              |                            |
%              |         Unmapped           |
%              |                            |
% 0x00000000   +----------------------------+ Beginning of memory
% \end{verbatim}
%   \else
%   \begin{tikzpicture}
%     \matrix [memory layout,nodes={minimum height=3em}] (m) {
%       \node (stack)  {Stack}; \\
%       \node (dynseg) {Dynamic segments}; \\
%       \node (heap)   {Heap}; \\
%       \node (exec)   {Executable file}; \\
%       \node (unmap)  {Unmapped}; \\
%     };
%     \draw (stack.south -| m.west) -- (stack.south -| m.east);
%     \draw (dynseg.south -| m.west)
%     node [left=1em, text width=6em, align=right] {Anonymous mapping base}
%     -- (dynseg.south -| m.east);
%     \draw (heap.south -| m.west) -- (heap.south -| m.east);
%     \draw (exec.south -| m.west)
%     node [left=1em, text width=6em, align=right] {Program base}
%     -- (exec.south -| m.east);
%     \path (m.south west) node [left=1em] {\texttt{0}}
%     (m.north west) node [left=1em]
%     {\texttt{\ifzseries 0x3ffffffffff\else 0x7fffffff\fi}}
%     (m.south east) node [right=1em] {Beginning of memory}
%     (m.north east) node [right=1em] {End of memory};
%   \end{tikzpicture}
%   \fi
%   \caption{{\ifzseries 42-bit virtual\else Virtual\fi} address
%     configuration}
%   \label{fig:vac}
% \end{figure}

% \paragraph{Note:}
% Although application programs may begin at virtual address 0, they
% conventionally begin above \texttt{0x1000} (4$\,$Kbytes), leaving the
% initial 4$\,$Kbytes with an invalid address mapping.  Processes that
% reference this invalid memory (for example by de-referencing a
% null pointer) generate a translation exception as described in
% \cref{exceptionint}.

% Although applications may control their memory assignments, the
% typical arrangement follows \cref{fig:vac}.

% \subsection{Managing the Process Stack}
% \Cref{procinit} describes the initial stack contents.
% Stack addresses can change from one system to the next---even from one
% process execution to the next on a single system.  A program,
% therefore, should not depend on finding its stack at a particular
% virtual address.

% A tunable configuration parameter controls the system maximum stack size.
% A process can also use \texttt{setrlimit} to set its own maximum stack
% size, up to the system limit.  The stack segment is both readable and
% writable.

% \subsection{Coding Guidelines}
% Operating system facilities, such as \texttt{mmap}, allow a process to
% establish address mappings in two ways.  Firstly, the program can let
% the system choose an address.  Secondly, the program can request the
% system to use an address the program supplies.  The second alternative
% can cause application portability problems because the requested
% address might not always be available.  Differences in virtual address
% space can be particularly troublesome between different architectures,
% but the same problems can arise within a single architecture.

% Processes' address spaces typically have three segments that can
% change size from one execution to the next: the stack (through
% \texttt{setrlimit}); the data segment (through \texttt{malloc}); and
% the dynamic segment area (through \texttt{mmap}).  Changes in one area
% may affect the virtual addresses available for another.  Consequently
% an address that is available in one process execution might not be
% available in the next.  Thus a program that used \texttt{mmap} to
% request a mapping at a specific address could appear to work in some
% environments and fail in others.  For this reason programs that want
% to establish a mapping in their address space should let the system
% choose the address.

% Despite these warnings about requesting specific addresses, the
% facility can be used properly.  For example, a multiprocess
% application might map several files into the address space of each
% process and build relative pointers among the files' data.  This could
% be done by having each process ask for a certain amount of memory at
% an address chosen by the system.  After each process received its own
% private address from the system it would map the desired files into
% memory at specific addresses within the original area.  This
% collection of mappings could be at different addresses in each process
% but their relative positions would be fixed.  Without the ability to
% ask for specific addresses, the application could not build shared
% data structures because the relative positions for files in each
% process would be unpredictable.

% \subsection{Processor Execution Modes}
% Two execution modes exist in \ARCHarch{}: problem (user) state and
% supervisor state.  Processes run in problem state (the less privileged).
% The operating system kernel runs in supervisor state.  A program executes
% a ``Supervisor Call'' (\texttt{SVC}) instruction to change execution
% modes.

% Note that the ABI does not define the implementation of individual
% system calls.  Instead programs should use the system libraries.
% Programs with embedded \texttt{SVC} instructions do not conform
% to the ABI.

% \section{Process Initialization}
% \label{procinit}
% \index{process initialization}
% \index{initialization!process}
% This section describes the machine state that \texttt{exec} creates
% for ``infant'' processes, including argument passing, register usage,
% and stack frame layout.  Programming language systems use this initial
% program state to establish a standard environment for their
% application programs.  For example, a C program begins executing at a
% function named \texttt{main}, conventionally declared as follows:
% \begin{center}
%   \lstinline@extern int main (int argc, char *argv[ ], char *envp[ ]);@
% \end{center}

% Its parameters are passed from the C programming language system when
% invoking \texttt{main}.  They are:
% \begin{description}
% \item[\texttt{argc}] a non-negative argument count
% \item[\texttt{argv}] an array of argument strings, with
%   \begin{center}
%     \lstinline@argv[argc] == NULL@
%   \end{center}
% \item[\texttt{envp}] an array of environment strings, also terminated by a
%   null pointer
% \end{description}

% Although this section does not describe C program initialization, it
% gives the information necessary to implement the call to \texttt{main}
% or to the entry point for a program in any other language.

% \subsection{Registers}
% \index{registers!process startup}
% \index{process initialization!registers}
% When a process is first entered (from an \texttt{exec} system call),
% the contents of registers other than those listed below are
% unspecified.  Consequently, a program that requires registers to have
% specific values must set them explicitly during process
% initialization.  It should not rely on the operating system to set all
% registers to 0.  Following are the registers whose contents are
% specified:
% \begin{description}
% \item[\texttt{r15}] The initial stack pointer, aligned to an 8-byte
%   boundary and pointing to a stack location that contains the
%   argument count (see \cref{processstack} for further
%   information about the initial stack layout).
% \item[\texttt{fpc}] The floating-point control register contains 0,
%   specifying ``round to nearest'' mode and the disabling of
%   floating-point exceptions.
% \end{description}

% \subsection{Process Stack}
% \label{processstack}
% \index{process initialization!stack}
% Every process has a stack, but the system defines no fixed stack
% address.  Furthermore, a program's stack address can change from one
% system to another---even from one process invocation to another.
% Thus the process initialization code must use the stack address in
% general register \texttt{r15}.  Data in the stack segment at
% addresses below the stack pointer contain undefined values.

% When a process receives control, its stack holds the arguments,
% environment, and auxiliary vector (see \cref{auxvector}) from
% \texttt{exec}.  Argument strings, environment strings, and the auxiliary
% information appear in no specific order within the information block; the
% system makes no guarantees about their relative arrangement.  The system
% may also leave an unspecified amount of memory between the \texttt{NULL}
% auxiliary vector entry and the beginning of the information block.  A
% sample initial stack is shown in \cref{fig:inistack}.

% \begin{figure}
%   \centering
%   \ifSkipTikZ
% \begin{verbatim}
%       +--------------------------------+  Top of Stack
%       |  Information block, including  |
%       |  argument and environment      |
%       |  strings and auxiliary         |
%       |  information (size varies)     |
%       +--------------------------------+
%       |        Unspecified             |
%       +--------------------------------+
%       | AT_NULL auxiliary vector entry |
%       +--------------------------------+
%       |       Auxiliary vector         |
%       |       (4-word entries)         |
%       +--------------------------------+
%       |       Zero doubleword          |
%       +--------------------------------+
%       |       Environment pointers     |
%       |          (2-word each)         |
%       +--------------------------------+
%       |       Zero doubleword          |
%       +--------------------------------+
%       |        Argument pointers       |
%       |          (2-word each)         |
%       +--------------------------------+
%       |   Argument count doubleword    |
% %r15  +--------------------------------+  Low Address
% \end{verbatim}
%   \else
%   \begin{tikzpicture}
%     \matrix [memory layout,inner sep=0pt,
%     nodes={text width=16em, text badly centered, inner sep=1ex}] (m) {
%       \node (A) {Information block, including argument and
%         environment strings and auxiliary information (size varies)}; \\
%       \node (B) {Unspecified}; \\
%       \node (C) {\texttt{AT\_NULL} auxiliary vector entry}; \\
%       \node (D) {Auxiliary vector ({\ifzseries 4\else 2\fi}-word entries)}; \\
%       \node (E) {Zero {\ifzseries double\fi}word}; \\
%       \node (F) {Environment pointers ({\ifzseries 2\else 1\fi}-word each)}; \\
%       \node (G) {Zero {\ifzseries double\fi}word}; \\
%       \node (H) {Argument pointers ({\ifzseries 2\else 1\fi}-word each)}; \\
%       \node (I) {Argument count {\ifzseries double\fi}word}; \\
%     };
%     \foreach \Node in {A,...,H} {
%       \draw (\Node.south -| m.west) -- (\Node.south -| m.east);
%     }
%     \path (m.south west) node [left=1em] (r15) {\texttt{r15}}
%     (m.south east) node [right=1em] {Low address}
%     (m.north east) node [right=1em] {Top of stack};
%     \draw [->, shorten >=1pt] (r15) -- (m.south west);
%   \end{tikzpicture}
%   \fi
%   \caption{Initial process stack}
% \label{fig:inistack}
% \end{figure}

% \subsection{Auxiliary Vector}
% \label{auxvector}
% \index{auxiliary vector}
% \index{process initialization!auxiliary vector}
% Whereas the argument and environment vectors transmit information from
% one application program to another, the auxiliary vector conveys
% information from the operating system to the program.  This vector is
% an array of structures, which are defined in \cref{lst:auxstruct}.

% \begin{lstlisting}[style=float,label=lst:auxstruct,
%   caption=Auxiliary vector structure,escapechar=@]
% typedef struct {
%     @\ifzseries long\else int\fi@ a_type;
%     union {
%         long a_val;
%         void *a_ptr;
%         void (*a_fcn)();
%     } a_un;
% } auxv_t;
% \end{lstlisting}

% The structures are interpreted according to the \texttt{a\_type}
% member, as shown in \cref{tab:auxtypes}.

% \begin{table}
%   \centering
%   \begin{DIFnomarkup}
%   \begin{tabular}{lrl!{\qquad}lrl}
%     \toprule
%     Name & Value & \texttt{a\_un}
%     & Name & Value & \texttt{a\_un} \\
%     \midrule
%     \texttt{AT\_NULL} & 0 & ignored
%     & \texttt{AT\_UID} & 11 & \texttt{a\_val} \\
%     \texttt{AT\_IGNORE} & 1 & ignored
%     & \texttt{AT\_EUID} & 12 & \texttt{a\_val} \\
%     \texttt{AT\_EXECFD} & 2 & \texttt{a\_val}
%     & \texttt{AT\_GID} & 13 & \texttt{a\_val} \\
%     \texttt{AT\_PHDR} & 3 & \texttt{a\_ptr}
%     & \texttt{AT\_EGID} & 14 & \texttt{a\_val} \\
%     \texttt{AT\_PHENT} & 4 & \texttt{a\_val}
%     & \texttt{AT\_PLATFORM} & 15 & \texttt{a\_ptr} \\
%     \texttt{AT\_PHNUM} & 5 & \texttt{a\_val}
%     & \texttt{AT\_HWCAP} & 16 & \texttt{a\_val} \\
%     \texttt{AT\_PAGESZ} & 6 & \texttt{a\_val}
%     & \texttt{AT\_CLKTCK} & 17 & \texttt{a\_val} \\
%     \texttt{AT\_BASE} & 7 & \texttt{a\_ptr}
%     & \texttt{AT\_SECURE} & 23 & \texttt{a\_val} \\
%     \texttt{AT\_FLAGS} & 8 & \texttt{a\_val}
%     & \texttt{AT\_RANDOM} & 25 & \texttt{a\_ptr} \\
%     \texttt{AT\_ENTRY} & 9 & \texttt{a\_ptr}
%     & \texttt{AT\_EXECFN} & 31 & \texttt{a\_ptr} \\
%     \texttt{AT\_NOTELF} & 10 & \texttt{a\_val}
%     & \texttt{AT\_SYSINFO\_EHDR} & 33 & \texttt{a\_ptr} \\
%     \bottomrule
%   \end{tabular}
%   \end{DIFnomarkup}
%   \caption{Auxiliary vector types, \texttt{a\_type}}
%   \label{tab:auxtypes}
% \end{table}

% \begin{table}
%   \centering
%   \begin{DIFnomarkup}
%   \begin{tabularx}{\textwidth}{lr>{\raggedright\arraybackslash}X}
%     \toprule
%     Name & Value & Description \\
%     \midrule
%     \texttt{HWCAP\_S390\_ZARCH} & \texttt{0x2}
%     & Running in z/Architecture mode \\
%     \texttt{HWCAP\_S390\_STFLE} & \texttt{0x4}
%     & Store-facility-list-extended facility installed \\
%     \texttt{HWCAP\_S390\_MSA} & \texttt{0x8}
%     & Message-security assist available \\
%     \texttt{HWCAP\_S390\_LDISP} & \texttt{0x10}
%     & Long-displacement facility installed \\
%     \texttt{HWCAP\_S390\_EIMM} & \texttt{0x20}
%     & Extended-immediate facility installed \\
%     \texttt{HWCAP\_S390\_DFP} & \texttt{0x40}
%     & Decimal floating-point facility and perform floating-point
%     facility (PFPO) installed \\
%     \texttt{HWCAP\_S390\_HPAGE} & \texttt{0x80}
%     & Huge page support available \\
%     \texttt{HWCAP\_S390\_ETF3EH} & \texttt{0x100}
%     & Extended-translation facility 3 and ETF3-enhancement
%     facility installed \\
%     \texttt{HWCAP\_S390\_TE} & \texttt{0x400}
%     & Transactional-execution facility installed \\
%     \texttt{HWCAP\_S390\_VXRS} & \texttt{0x0800}
%     & Vector facility installed \\
%     \texttt{HWCAP\_S390\_VXRS\_BCD} & \texttt{0x1000}
%     & Vector packed-decimal facility installed \\
%     \texttt{HWCAP\_S390\_VXRS\_EXT} & \texttt{0x2000}
%     & Vector-enhancements facility 1 installed \\
%     \texttt{HWCAP\_S390\_GS} & \texttt{0x4000}
%     & Guarded-storage facility installed \\
%     \texttt{HWCAP\_S390\_VXRS\_EXT2} & \texttt{0x8000}
%     & Vector-enhancements facility 2 installed \\
%     \texttt{HWCAP\_S390\_VXRS\_PDE} & \texttt{0x10000}
%     & Vector-packed-decimal enhancement facility installed \\
%     \texttt{HWCAP\_S390\_DFLT} & \texttt{0x40000}
%     & Deflate-conversion facility installed \\
%     \bottomrule
%   \end{tabularx}
%   \end{DIFnomarkup}
%   \caption{Hardware capabilities}
%   \label{tab:hwcap}
% \end{table}

% \texttt{a\_type} auxiliary vector types are described in the
% following:
% \begin{description}
% \item[\texttt{AT\_NULL}] The auxiliary vector has no fixed length, so
%   an entry of this type is used to denote the end of the vector.  The
%   corresponding value of \texttt{a\_un} is undefined.
% \item[\texttt{AT\_IGNORE}] This type indicates the entry has no
%   meaning.  The corresponding value of \texttt{a\_un} is undefined.
% \item[\texttt{AT\_EXECFD}] \texttt{exec} may pass control to an interpreter
%   program.  When this happens, the system places either an entry of
%   type \texttt{AT\_EXECFD} or one of type \texttt{AT\_PHDR} in the
%   auxiliary vector.  The \texttt{a\_val} field in the
%   \texttt{AT\_EXECFD} entry contains a file descriptor for the
%   application program's object file.
% \item[\texttt{AT\_PHDR}] Under some conditions, the system creates the
%   memory image of the application program before passing control to an
%   interpreter program.  When this happens, the \texttt{a\_ptr} field of
%   the \texttt{AT\_PHDR} entry tells the interpreter where to find the
%   program header table in the memory image.  If the \texttt{AT\_PHDR}
%   entry is present, entries of types \texttt{AT\_PHENT},
%   \texttt{AT\_PHNUM}, and \texttt{AT\_ENTRY} must also be present.  See
%   \cref{chprogload} for more information about the program header
%   table.
% \item[\texttt{AT\_PHENT}] The \texttt{a\_val} field of this entry
%   holds the size, in bytes, of one entry in the program header table
%   at which the \texttt{AT\_PHDR} entry points.
% \item[\texttt{AT\_PHNUM}] The \texttt{a\_val} field of this entry
%   holds the number of entries in the program header table at which the
%   \texttt{AT\_PHDR} entry points.
% \item[\texttt{AT\_PAGESZ}] If present, this entry's \texttt{a\_val}
%   field gives the system page size in bytes.  The same information is
%   also available through \texttt{sysconf}.
% \item[\texttt{AT\_BASE}] The \texttt{a\_ptr} member of this entry
%   holds the base address at which the interpreter program was loaded
%   into memory.
% \item[\texttt{AT\_FLAGS}] If present, the \texttt{a\_val} field of
%   this entry holds 1-bit flags.  Undefined bits are set to zero.
% \item[\texttt{AT\_ENTRY}] The \texttt{a\_ptr} field of this entry
%   holds the entry point of the application program to which the
%   interpreter program should transfer control.
% \item[\texttt{AT\_NOTELF}] The \texttt{a\_val} field of this entry is
%   non-zero if the program is in another format than ELF, for example
%   in the old COFF format.
% \item[\texttt{AT\_UID}] The \texttt{a\_ptr} field of this entry holds
%   the real user id of the process.
% \item[\texttt{AT\_EUID}] The \texttt{a\_ptr} field of this entry holds
%   the effective user id of the process.
% \item[\texttt{AT\_GID}] The \texttt{a\_ptr} field of this entry holds
%   the real group id of the process.
% \item[\texttt{AT\_EGID}] The \texttt{a\_ptr} field of this entry holds
%   the effective group id of the process.
% \item[\texttt{AT\_PLATFORM}] The \texttt{a\_ptr} field of this entry holds
%   the address of a string that identifies the platform the program runs
%   on.
% \item[\texttt{AT\_HWCAP}] The \texttt{a\_val} field of this entry holds a
%   bit map of hardware capabilities\index{hardware capabilities} hints.
%   \Cref{tab:hwcap} lists some of the assigned bits and their meaning.
% \item[\texttt{AT\_CLKTCK}] The \texttt{a\_val} field of this entry holds
%   the number of clock ticks per second.  The function \texttt{times()},
%   which measures execution time, reports all times in clock ticks.  The
%   number of clock ticks per second is also available through
%   \texttt{sysconf}.
% \item[\texttt{AT\_SECURE}] The \texttt{a\_val} field of this entry holds a
%   Boolean that indicates whether the program shall be locked into a secure
%   environment, such as when access rights have been upgraded by executing
%   a setuid/setgid executable.
% \item[\texttt{AT\_RANDOM}] The \texttt{a\_ptr} field of this entry holds
%   the address of 16 random bytes.
% \item[\texttt{AT\_EXECFN}] The \texttt{a\_ptr} field of this entry holds
%   the address of a string that contains the executable's file name.
% \item[\texttt{AT\_SYSINFO\_EHDR}] The \texttt{a\_ptr} field of this entry
%   holds the address at which the system-supplied dynamic shared object
%   (DSO), specifically its ELF header, is mapped in the program's virtual
%   address space.
% \end{description}

% Other auxiliary vector types are reserved.  No flags are currently
% defined for \texttt{AT\_FLAGS} on \ABINAME{}.

% \section{Coding Examples}
% \label{codingexamples}
% This section describes example code sequences for fundamental
% operations such as calling functions, accessing static objects, and
% transferring control from one part of a program to another.  Previous
% sections discussed how a program may use the machine or the operating
% system, and they specified what a program may and may not assume about
% the execution environment.  Unlike previous material, the information
% in this section illustrates how operations \emph{may} be done,
% not how they \emph{must} be done.

% As before, examples use the ISO C language.  Other programming
% languages may use the same conventions displayed below, but failure to
% do so does not prevent a program from conforming to the ABI\@.  Two main
% object code models are available:
% \begin{description}
% \item[Absolute code:] Instructions can hold absolute addresses under
%   this model.  To execute properly, the program must be loaded at a
%   specific virtual address, making the program's absolute addresses
%   coincide with the process's virtual addresses.
% \item[Position-independent code:] Instructions under this model hold
%   relative addresses, not absolute addresses.  Consequently, the code
%   is not tied to a specific load address, allowing it to execute
%   properly at various positions in virtual memory.
% \end{description}

% The following sections describe the differences between these models.
% When different, code sequences for the models appear together for
% easier comparison.

% \paragraph{Note:}
% The examples below show code fragments with various simplifications.
% They are intended to explain addressing modes, not to show optimal
% code sequences or to reproduce compiler output.

% \subsection{Code Model Overview}
% When the system creates a process image, the executable file
% portion of the process has fixed addresses and the system chooses
% shared object library virtual addresses to avoid conflicts with other
% segments in the process.  To maximize text sharing, shared objects
% conventionally use position-independent code, in which instructions
% contain no absolute addresses.  Shared object text segments can be
% loaded at various virtual addresses without having to change the
% segment images.  Thus multiple processes can share a single shared
% object text segment, even if the segment resides at a different
% virtual address in each process.

% Position-independent code relies on two techniques:
% \begin{itemize}
% \item Control transfer instructions hold addresses relative to the
%   Current Instruction Address (CIA), or use registers that hold the
%   transfer address.  A CIA-relative branch computes its destination
%   address in terms of the CIA, not relative to any absolute address.
% \item When the program requires an absolute address, it computes the
%   desired value.  Instead of embedding absolute addresses in
%   instructions (in the text segment), the compiler generates code to
%   calculate an absolute address (in a register or in the stack or data
%   segment) during execution.
% \end{itemize}

% Because \ARCHarch{}
% provides CIA-relative branch instructions and also branch instructions
% using registers that hold the transfer address, compilers can satisfy
% the first condition easily.

% A Global Offset Table (GOT) provides information for address
% calculation.  Position-independent object files (executable and shared
% object files) have a table in their data segment that holds
% addresses.  When the system creates the memory image for an object
% file, the table entries are relocated to reflect the absolute virtual
% address as assigned for an individual process.  Because data segments
% are private for each process, the table entries can change---unlike
% those of text segments, which multiple processes share.

% Two position-independent models give programs a choice between more
% efficient code with some size restrictions and less efficient code
% without those restrictions.  Because of the processor architecture, a
% GOT with no more than {\ifzseries 512\else 1024\fi} entries (4096
% bytes) is more efficient than a larger one.  Programs that need more
% entries must use the larger, more general code.  In the following
% sections, the term ``small model position-independent code'' is used
% to refer to code that assumes the smaller GOT, and ``large model
% position-independent code'' is used to refer to the general code.

% \subsection{Function Prologue and Epilogue}
% This section describes the prologue and epilogue code of functions.  A
% function's prologue establishes a stack frame, if necessary, and may
% save any nonvolatile registers it uses.  A function's epilogue generally
% restores registers that were saved in the prologue code, restores the
% previous stack frame, and returns to the caller.

% \subsubsection{Prologue}
% The prologue of a function has to save the state of the calling
% function and set up the base register for the code of the function
% body.  The following is in general done by the function
% prologue:
% \begin{itemize}
% \item Save all registers used within the function which the calling
%   function assumes to be nonvolatile.
% \item Set up the base register for the literal pool, if needed.
% \item Allocate stack space by decrementing the stack pointer.
% \item Set up the dynamic chain by storing the old stack pointer value
%   at stack location zero if the ``back chain'' is implemented.
% \item Set up the GOT pointer if the compiler is generating
%   position-independent code.

%   (Usually the GOT pointer is loaded into a nonvolatile register.  This
%   may be omitted if the function makes no external data references.  If
%   external data references are only made within conditional code, loading
%   the GOT pointer may be deferred until it is known to be needed.)
% \item Set up the frame pointer if the function allocates stack space
%   dynamically (with \texttt{alloca}).
% \end{itemize}

% The compiler tries to do as little as possible of the above; the ideal
% case is to do nothing at all (for a leaf function without symbolic
% references).

% \ifzseries
% \begin{lstlisting}[language=simpleasm,style=float,label=lst:prolcode,
%   caption={[Prologue and epilogue example]{Prologue and epilogue example.
%       This example stores the optional backchain.}}]
%           .section .rodata
%           .align  2
% .LC0:     .string "hello, world!"

%           .text
%           .align  8
%           .globl  main
%           .type   main, @function
% main:
%                                         # Prologue
%           stmg    %r14,%r15,112(%r15)   # Save caller's registers
%           lgr     %r1,%r15              # Load stack pointer into r1
%           aghi    %r15,-160             # Allocate new stack frame
%           stg     %r1,0(%r15)           # Store back chain
%                                         # Prologue end
%           larl    %r2,.LC0
%           brasl   %r14,puts
%           lghi    %r2,0
%                                         # Epilogue
%           lmg     %r14,%r15,272(%r15)   # Restore registers
%           br      %r14                  # Branch back to caller
%                                         # Epilogue end
% \end{lstlisting}
% \else
% \begin{lstlisting}[language=simpleasm,style=float,label=lst:prolcode,
%   caption=Prologue and epilogue example]
%           .string "hello, world\n"
%           .align  4
%           .globl  main
%           .type   main,@function
% main:
%                                        # Prologue
%           STM     11,15,44(15)         # Save callers registers
%           BRAS    13,.LTN0_0           # Set up literal pool
%                                        #   and branch over
% .LT0_0:
% .LC21:
%           .long   .LC18
% .LC22:
%           .long   printf
% .LTN0_0:
%           LR      1,15                 # Load stack pointer in GPR 1
%           AHI     15,-96               # Allocate stack space
%           ST      1,0(15)              # Save backchain
%                                        # Prologue end
%           L       2,.LC21-.LT0_0(13)
%           L       1,.LC22-.LT0_0(13)
%           BASR    14,1
%           SLR     2,2
%                                        # Epilogue
%           L       4,152(15)            # Load return address
%           LM      11,15,140(15)        # Restore registers
%           BR      4                    # Branch back to caller
%                                        # Epilogue end
% \end{lstlisting}
% \fi

% \subsubsection{Epilogue}
% The epilogue of a function restores the registers saved in the prologue
% (which include the stack pointer) and branches to the return address.

% The small program in \cref{lst:prolcode} shows a simple example of a
% function prologue and epilogue.

% \subsection{Profiling}
% \index{profiling}
% This section shows a way of providing profiling (entry counting) for
% \ABINAME{} applications.  An ABI-conforming system is not required to
% provide profiling; however, if it does, this is one possible (not
% required) implementation.

% If a function is to be profiled, it has to call the \texttt{\_mcount}
% routine before the function prologue.  This routine has a special linkage.
% Its return address is passed in \texttt{r14} as usual.  However, instead
% of register arguments it receives the caller's return address in the first
% slot of the register save area, which is located \NBYTES{} bytes above the
% current stack pointer.  And it preserves more registers than a normal
% function, treating all the usual argument registers as nonvolatile as
% well.  Since \texttt{\_mcount} gets invoked before the caller's prologue,
% no additional frame needs to be allocated for it.  It may overwrite the
% caller's register save area, except for the first slot, which it will
% preserve.

% \Cref{lst:profcode} shows an example of a function prologue preceded by a
% call to \texttt{\_mcount}.

% \ifzseries
% \begin{lstlisting}[language=simpleasm,style=float,label=lst:profcode,
%   caption=Code for profiling]
%           stg     %r14,8(%r15)          # Pass r14 in first regsave slot
%           brasl   %r14,_mcount          # Branch to _mcount
%           lg      %r14,8(%r15)          # Restore r14
%           stmg    %r7,%r15,56(%r15)     # Save caller's registers
%           aghi    %r15,-160             # Allocate new frame
%           ...
% \end{lstlisting}
% \else
% \begin{lstlisting}[language=simpleasm,style=float,label=lst:profcode,
%   caption=Code for profiling]
%           STM     7,15,28(15)          # Save callers registers
%           BRAS    13,.LTN0_0           # Jump to function prologue
% .LT0_0:
% .LC3:     .long   _mcount              # Literal pool entry for _mcount
% .LC4:     .long   .LP0                 # Literal pool entry
%                                        #   for profile counter
% .LTN0_0:
%           LR      1,15                 # Stack pointer
%           AHI     15,-96               # Allocate new
%           ST      1,0(15)              # Save backchain
%           LR      11,15                # Local stack pointer
%           .data
%           .align 4
% .LP0:     .long   0                    # Profile counter
%           .text
%                                        # Function profiler
%           ST    14,4(15)               # Preserve r14
%           L     14,.LC3-.LT0_0(13)     # Load address of _mcount
%           L     1,.LC4-.LT0_0(13)      # Load address of profile counter
%           BASR  14,14                  # Branch to _mcount
%           L     14,4(15)               # Restore r14
% \end{lstlisting}
% \fi

% \subsection{Data Objects}
% This section describes only objects with static storage duration.  It
% excludes stack-resident objects because programs always compute their
% virtual addresses relative to the stack or frame pointers.

% % TODO: The use of literal pool entries for relative symbols is outdated.
% % Better describe the modern approach here.
% Because \ARCH{} instructions cannot hold \ADDRBITS{}-bit addresses
% directly, a program has to build an address in a register and access
% memory through that register.  In order to do so, a function may contain a
% literal pool that holds the addresses of data objects used by the
% function.  Then \texttt{r13} is typically set up in the function prologue
% to point to the start of this literal pool.

% Position-independent code cannot contain absolute addresses.  In order
% to access a local symbol, the literal pool contains the (signed) offset
% of the symbol relative to the start of the pool.  Combining the offset
% loaded from the literal pool with the address in \texttt{r13} gives the
% absolute address of the local symbol.  In the case of a global symbol
% the address of the symbol has to be loaded from the Global Offset
% Table.  The offset in the GOT can either be contained in the
% instruction itself or in the literal pool.

% \Crefrange{tab:addresses}{tab:largegot} show sample assembly
% language equivalents to C language code for absolute and
% position-independent compilations.  It is assumed that all shared
% objects are compiled as position-independent and only executable
% modules may have absolute addresses.  The
% function prologue is not shown, and it is assumed that it has loaded the
% address of the literal pool in \texttt{r13}.

% \begin{table}
%   \centering
%   \begin{DIFnomarkup}
%   \begin{tabular}{p{0.35\textwidth}p{0.60\textwidth}}
%     \toprule
%     C & \ARCH{} machine instructions (Assembler) \\
%     \midrule
% \begin{lstlisting}[style=short]
% extern int src;
% extern int dst;
% extern int *ptr;
% dst = src;
% ptr = &dst;
% \end{lstlisting}
%     &
% \ifzseries
% \begin{lstlisting}[style=short,language=simpleasm]
% larl  %r1,src
% larl  %r2,dst
% larl  %r3,ptr
% mvc   0(4,%r2),0(%r1)   # dst = src
% stg   %r2,0(%r3)        # ptr = &dst
% \end{lstlisting}
% \else
% \begin{lstlisting}[style=short,language=simpleasm]
%            # Literal pool
% .LT0:
% .LC1:      .long dst
% .LC2:      .long src
%            # Code
%            L     2,.LC1-.LT0(13)
%            L     1,.LC2-.LT0(13)
%            MVC   0(4,2),0(1)
%            # Literal pool
% .LT0:
% .LC1:      .long ptr
% .LC2:      .long dst
%            # Code
%            L     1,.LC1-.LT0(13)
%            MVC   0(4,1),.LC2-.LT0(13)
%            # Literal pool
% .LT0:
% .LC1:      .long ptr
% .LC2:      .long src
%            # Code
%            L     1,.LC1-.LT0(13)
%            L     2,.LC2-.LT0(13)
%            L     3,0(1)
%            MVC 0(4,3),0(2)
% \end{lstlisting}
% \fi \\
%     \bottomrule
%   \end{tabular}
%   \end{DIFnomarkup}
%   \caption{Absolute addressing}
%   \label{tab:addresses}
% \end{table}

% \begin{table}
%   \centering
%   \begin{DIFnomarkup}
%   \begin{tabular}{p{0.35\textwidth}p{0.60\textwidth}}
%     \toprule
%     C & \ARCH{} machine instructions (Assembler) \\
%     \midrule
% \begin{lstlisting}[style=short]
% extern int src;
% extern int dst;
% extern int *ptr;
% dst = src;
% ptr = &dst;
% *ptr = src;
% \end{lstlisting}
%     &
% \ifzseries
% \begin{lstlisting}[style=short,language=simpleasm]
% larl  %r12,_GLOBAL_OFFSET_TABLE_
% lg    %r1,dst@GOT12(%r12)
% lg    %r2,src@GOT12(%r12)
% lgf   %r3,0(%r2)
% st    %r3,0(%r1)
% larl  %r12,_GLOBAL_OFFSET_TABLE_
% lg    %r1,ptr@GOT12(%r12)
% lg    %r2,dst@GOT12(%r12)
% stg   %r2,0(%r1)
% larl  %r12,_GLOBAL_OFFSET_TABLE_
% lg    %r2,ptr@GOT12(%r12)
% lg    %r1,0(%r2)
% lg    %r2,src@GOT12(%r12)
% lgf   %r3,0(%r2)
% st    %r3,0(%r1)
% \end{lstlisting}
% \else
% \begin{lstlisting}[style=short,language=simpleasm]
%            # Literal pool
% .LT0:
% .LC1:      .long _GLOBAL_OFFSET_TABLE_-.LT0
%            # Code
%            L     12,.LC1-.LT0(13)
%            LA    12,0(12,13)
%            L     2,dst@GOT(12)
%            L     1,src@GOT(12)
%            MVC   0(4,2),0(1)
%            # Literal pool
% .LT0:
% .LC1:      .long _GLOBAL_OFFSET_TABLE_-.LT0
%            # Code
%            L     12,.LC1-.LT0(13)
%            LA    12,0(12,13)
%            L     1,ptr@GOT(12)
%            L     2,dst@GOT(12)
%            ST    2,0(1)
%            # Literal pool
% .LT0:
% .LC1:      .long _GLOBAL_OFFSET_TABLE_-.LT0
%            # Code
%            L     12,.LC1-.LT0(13)
%            LA    12,0(12,13)
%            L     1,ptr@GOT(12)
%            L     2,src@GOT(12)
%            L     3,0(1)
%            MVC 0(4,3),0(2)
% \end{lstlisting}
% \fi \\
%     \bottomrule
%   \end{tabular}
%   \end{DIFnomarkup}
%   \caption{Small model position-independent addressing}
% \end{table}

% \begin{table}
%   \centering
%   \begin{DIFnomarkup}
%   \begin{tabular}{p{0.35\textwidth}p{0.60\textwidth}}
%     \toprule
%     C & \ARCH{} Assembler \\
%     \midrule
% \begin{lstlisting}[style=short]
% extern int src;
% extern int dst;
% extern int *ptr;
% dst = src;
% ptr = &dst;
% *ptr = src;
% \end{lstlisting}
%     &
% \ifzseries
% \begin{lstlisting}[style=short,language=simpleasm]
% larl  %r2,dst@GOT
% lg    %r2,0(%r2)
% larl  %r3,src@GOT
% lg    %r3,0(%r3)
% mvc   0(4,%r2),0(%r3)
% larl  %r2,ptr@GOT
% lg    %r2,0(%r2)
% larl  %r3,dst@GOT
% lg    %r3,0(%r3)
% stg   %r3,0(%r2)
% larl  %r2,ptr@GOT
% lg    %r2,0(%r2)
% larl  %r3,src@GOT
% lg    %r3,0(%r3)
% mvc   0(4,%r3),0(%r2)
% \end{lstlisting}
% \else
% \begin{lstlisting}[style=short,language=simpleasm]
%            # Literal pool
% .LT0:
% .LC1:      .long dst@GOT
% .LC2:      .long src@GOT
% .LC3:      .long _GLOBAL_OFFSET_TABLE_-.LT0
%            # Code
%            L     12,.LC3-.LT0(13)
%            LA    12,0(12,13)
%            L     2,.LC1-.LT0(13)
%            L     1,.LC2-.LT0(13)
%            L     2,0(2,12)
%            L     1,0(1,12)
%            MVC   0(4,2),0(1)
%            # Literal pool
% .LT0:
% .LC1:      .long ptr@GOT
% .LC2:      .long dst@GOT
% .LC3:      .long _GLOBAL_OFFSET_TABLE_-.LT0
%            # Code
%            L     12,.LC3-.LT0(13)
%            LA    12,0(12,13)
%            L     2,.LC1-.LT0(13)
%            L     1,.LC2-.LT0(13)
%            L     2,0(2,12)
%            L     1,0(1,12)
%            ST    1,0(2)
%            # Literal pool
% .LT0:
% .LC1:      .long ptr@GOT
% .LC2:      .long src@GOT
% .LC3:      .long _GLOBAL_OFFSET_TABLE_-.LT0
%            # Code
%            L   12,.LC1-.LT0(13)
%            LA  12,0(12,13)
%            L   1,.LC1-.LT0(13)
%            L   2,.LC2-.LT0(13)
%            L   1,0(1,12)
%            L   2,0(2,12)
%            L   3,0(1)
%            MVC 0(4,3),0(2)
% \end{lstlisting}
% \fi \\
%     \bottomrule
%   \end{tabular}
%   \end{DIFnomarkup}
%   \caption{Large model position-independent addressing}
%   \label{tab:largegot}
% \end{table}

% \subsection{Function Calls}
% Programs can use the \ARCH{} {\ifzseries\texttt{BRASL}\else
%   \texttt{BRAS}\fi} instruction to make direct function calls.
% A {\ifzseries \texttt{BRASL}\else \texttt{BRAS}\fi} instruction has a
% self-relative branch displacement that can reach {\ifzseries
%   4$\,$GBytes\else 64$\,$Kbytes\fi} in either direction.  {\ifzseries
%   To call functions beyond this limit (inter-module calls),\else Hence
%   the use of the \texttt{BRAS} instruction is limited to very rare
%   cases.  The usual method of calling a function is to\fi} load the
% address in a register and use the \texttt{BASR} instruction for the
% call.  Register \texttt{r14} is used as the first operand of \texttt{BASR}
% to hold the return address as shown in \cref{tab:fncalldirect}.

% The called function may be in the same module (executable or shared
% object) as the caller, or it may be in a different module.  In the
% former case, if the called function is not in a shared object, the
% linkage editor resolves the symbol.  In all other cases the linkage
% editor cannot directly resolve the symbol.  Instead the linkage editor
% generates ``glue'' code and resolves the symbol to point to the glue
% code.  The dynamic linker will provide the real address of the
% function in the Global Offset Table.  The glue code loads this address
% and branches to the function itself.  See
% \cref{procedurelinkagetable} for more details.

% \begin{table}
%   \centering
%   \begin{DIFnomarkup}
%   \begin{tabular}{p{0.35\textwidth}p{0.60\textwidth}}
%     \toprule
%     C & \ARCH{} machine instructions (Assembler) \\
%     \midrule
% \begin{lstlisting}[style=short]
% extern void func();
% extern void (*ptr)();
% ptr = func;
% func();
% (*ptr) ();
% \end{lstlisting}
%     &
% \ifzseries
% \begin{lstlisting}[style=short,language=simpleasm]
% larl  %r1,ptr
% larl  %r2,func
% stg   %r2,0(%r1)
% brasl %r14,func
% larl  %r1,ptr
% lg    %r1,0(%r1)
% basr  %r14,%r1
% \end{lstlisting}
% \else
% \begin{lstlisting}[style=short,language=simpleasm]
%            # Literal pool
% .LT0:
% .LC1:      .long ptr
% .LC2:      .long func
%            # Code
%            L     1,.LC1-.LT0(13)
%            MVC   0(4,1),.LC2-.LT0(13)
%            # Literal pool
% .LT0:
% .LC1:      .long func
%            # Code
%            L     1,.LC1-.LT0(13)
%            BASR  14,1
%            # Literal pool
% .LT0:
% .LC1:      .long ptr
%            # Code
%            L     1,.LC1-.LT0(13)
%            L     1,0(1)
%            BASR  14,1
% \end{lstlisting}
% \fi \\
%     \bottomrule
%   \end{tabular}
%   \end{DIFnomarkup}
%   \caption{Absolute {\ifzseries\else direct\fi} function call}
%   \label{tab:fncalldirect}
% \end{table}

% \begin{table}
%   \centering
%   \begin{DIFnomarkup}
%   \begin{tabular}{p{0.35\textwidth}p{0.60\textwidth}}
%     \toprule
%     C & \ARCH{} machine instructions (Assembler) \\
%     \midrule
% \begin{lstlisting}[style=short]
% extern void func();
% extern void (*ptr)();
% ptr = func;
% func();
% (*ptr) ();
% \end{lstlisting}
%     &
% \ifzseries
% \begin{lstlisting}[style=short,language=simpleasm]
% larl  %r12,_GLOBAL_OFFSET_TABLE_
% lg    %r1,ptr@GOT12(%r12)
% lg    %r2,func@GOT12(%r12)
% stg   %r2,0(%r1)
% brasl %r14,func@PLT
% larl  %r12,_GLOBAL_OFFSET_TABLE_
% lg    %r1,ptr@GOT12(%r12)
% lg    %r1,0(%r1)
% basr  %r14,%r1
% \end{lstlisting}
% \else
% \begin{lstlisting}[style=short,language=simpleasm]
%            # Literal pool
% .LT0:
% .LC1:      .long _GLOBAL_OFFSET_TABLE_-.LT0
%            # Code
%            L     12,.LC1-.LT0(13)
%            LA    12,0(12,13)
%            L     1,ptr@GOT(12)
%            L     2,func@GOT(12)
%            ST    2,0(1)
%            # Literal pool
% .LT0:
% .LC1:      .long _GLOBAL_OFFSET_TABLE_-.LT0
% .LC2:      .long func@PLT-.LT0
%            # Code
%            L     12,.LC1-.LT0(13)
%            LA    12,0(12,13)
%            L     1,.LC2-.LT0(13)
%            BAS   14,0(1,13)
%            # Literal pool
% .LT0:
% .LC1:      .long _GLOBAL_OFFSET_TABLE_-.LT0
%            # Code
%            L     12,.LC1-.LT0(13)
%            LA    12,0(12,13)
%            L     1,ptr@GOT(12)
%            L     2,0(1)
%            BASR  14,2
% \end{lstlisting}
% \fi \\
%     \bottomrule
%   \end{tabular}
%   \end{DIFnomarkup}
%   \caption{Small model position-independent {\ifzseries\else direct\fi}
%     function call}
%   \label{tab:fnsmalldirect}
% \end{table}

% \begin{table}
%   \centering
%   \begin{DIFnomarkup}
%   \begin{tabular}{p{0.35\textwidth}p{0.60\textwidth}}
%     \toprule
%     C & \ARCH{} machine instructions (Assembler) \\
%     \midrule
% \begin{lstlisting}[style=short]
% extern void func();
% extern void (*ptr)();
% ptr = func;
% func();
% (*ptr) ();
% \end{lstlisting}
%     &
% \ifzseries
% \begin{lstlisting}[style=short,language=simpleasm]
% larl  %r2,ptr@GOT
% lg    %r2,0(%r2)
% larl  %r3,func@GOT
% lg    %r3,0(%r3)
% stg   %r3,0(%r2)
% brasl %r14,func@PLT
% larl  %r2,ptr@GOT
% lg    %r2,0(%r2)
% lg    %r2,0(%r2)
% basr  %r14,%r2
% \end{lstlisting}
% \else
% \begin{lstlisting}[style=short,language=simpleasm]
%            # Literal pool
% .LT0:
% .LC1:      .long ptr@GOT
% .LC2:      .long func@GOT
% .LC3:      .long _GLOBAL_OFFSET_TABLE_-.LT0
%            # Code
%            L     12,.LC3-.LT0(13)
%            LA    12,0(12,13)
%            L     2,.LC1-.LT0(13)
%            L     1,.LC2-.LT0(13)
%            L     2,0(2,12)
%            L     1,0(1,12)
%            ST    1,0(2)
%            # Literal pool
% .LT0:
% .LC1:      .long _GLOBAL_OFFSET_TABLE_-.LT0
% .LC2:      .long func@PLT-.LT0
%            # Code
%            L     12,.LC1-.LT0(13)
%            LA    12,0(12,13)
%            L     1,.LC2-.LT0(13)
%            BAS   14,0(1,13)
%            # Literal pool
% .LT0:
% .LC1:      .long ptr@GOT
% .LC2:      .long _GLOBAL_OFFSET_TABLE_-.LT0
%            # Code
%            L     12,.LC2-.LT0(13)
%            LA    12,0(12,13)
%            L     1,.LC1-.LT0(13)
%            L     1,0(1,12)
%            L     2,0(1)
%            BASR  14,2
% \end{lstlisting}
% \fi \\
%     \bottomrule
%   \end{tabular}
%   \end{DIFnomarkup}
%   \caption{Large model position-independent {\ifzseries\else direct
%       \fi}function call}
%   \label{tab:fnlargedirect}
% \end{table}

% \ifzseries\else
% \begin{table}
%   \centering
%   \begin{DIFnomarkup}
%   \begin{tabular}{p{0.35\textwidth}p{0.60\textwidth}}
%     \toprule
%     C & \ARCH{} machine instructions (Assembler) \\
%     \midrule
% \begin{lstlisting}[style=short]
% extern void func();
% extern void (*ptr)();
% ptr = func;
% func();
% (*ptr) ();
% \end{lstlisting}
%     &
% \begin{lstlisting}[style=short,language=simpleasm]
%           # Literal pool
% .LT0:
% .LC1:     .long ptr
% .LC2:     .long func
%           # Code
%           L     1,.LC1-.LT0(13)
%           MVC   0(4,1),.LC2-.LT0(13)
%           # Literal pool
% .LT0:
% .LC1:     .long ptr
%           # Code
%           L     1,.LC1-.LT0(13)
%           L     1,0(1)
%           BASR  14,1
% \end{lstlisting} \\
%   \end{tabular}
%   \end{DIFnomarkup}
%   \caption{Absolute indirect function call}
%   \label{tab:fncallabsindirect}
% \end{table}
% \fi

% \ifzseries\else
% \begin{table}
%   \centering
%   \begin{DIFnomarkup}
%   \begin{tabular}{p{0.35\textwidth}p{0.60\textwidth}}
%     \toprule
%     C & \ARCH{} machine instructions (Assembler) \\
%     \midrule
% \begin{lstlisting}[style=short]
% extern void func();
% extern void (*ptr)();
% ptr = func;
% func();
% (*ptr) ();
% \end{lstlisting}
%     &
% \begin{lstlisting}[style=short,language=simpleasm]
%            # Literal pool
% .LT0:
% .LC1:      .long _GLOBAL_OFFSET_TABLE_-.LT0
%            # Code
%            L     12,.LC2-.LT0(13)
%            LA    12,0(12,13)
%            L     1,ptr@GOT(12)
%            L     2,func@GOT(12)
%            ST    2,0(1)
%            # Literal pool
% .LT0:
% .LC1:      .long _GLOBAL_OFFSET_TABLE_-.LT0
%            # Code
%            L     12,.LC1-.LT0(13)
%            LA    12,0(12,13)
%            L     1,ptr@GOT(12)
%            L     2,0(1)
%            BASR  14,2
% \end{lstlisting} \\
%   \end{tabular}
%   \end{DIFnomarkup}
%   \caption{Small model position-independent indirect function call}
%   \label{tab:fncallpicsmall}
% \end{table}
% \fi

% \ifzseries\else
% \begin{table}
%   \centering
%   \begin{DIFnomarkup}
%   \begin{tabular}{p{0.35\textwidth}p{0.60\textwidth}}
%     \toprule
%     C & \ARCH{} machine instructions (Assembler) \\
%     \midrule
% \begin{lstlisting}[style=short]
% extern void func();
% extern void (*ptr)();
% ptr = func;
% func();
% (*ptr) ();
% \end{lstlisting}
%     &
% \begin{lstlisting}[style=short,language=simpleasm]
%            # Literal pool
% .LT0:
% .LC1:      .long ptr@GOT
% .LC2:      .long func@GOT
% .LC3:      .long _GLOBAL_OFFSET_TABLE_-.LT0
%            # Code
%            L     12,.LC3-.LT0(13)
%            LA    12,0(12,13)
%            L     2,.LC1-.LT0(13)
%            L     1,.LC2-.LT0(13)
%            L     2,0(2,12)
%            L     1,0(1,12)
%            ST    1,0(2)
%            # Literal pool
% .LT0:
% .LC1:      .long ptr@GOT
% .LC2:      .long _GLOBAL_OFFSET_TABLE_-.LT0
%            # Code
%            L     12,.LC2-.LT0(13)
%            LA    12,0(12,13)
%            L     1,.LC1-.LT0(13)
%            L     1,0(1,12)
%            L     2,0(1)
%            BASR  14,2
% \end{lstlisting} \\
%   \end{tabular}
%   \end{DIFnomarkup}
%   \caption{Large model position-independent indirect function call}
%   \label{tab:fncallpiclarge}
% \end{table}
% \fi

% \subsection{Branching}
% Programs use branch instructions to control their execution flow.
% {\ifzseries \ARCH{}\else The \ARCH{} architecture\fi} has a
% variety of branch instructions.  The most commonly used of these
% performs a self-relative jump with a 128-Kbyte range (up to 64 Kbytes
% in either direction).  {\ifzseries For large functions, another
%   self-relative jump is available with a range of 4$\,$Gbytes (up to
%   2$\,$Gbytes in either direction).\fi}%

% \begin{table}
%   \centering
%   \begin{DIFnomarkup}
%   \begin{tabular}{p{0.35\textwidth}p{0.60\textwidth}}
%     \toprule
%     C & \ARCH{} machine instructions (Assembler) \\
%     \midrule
%     \ifzseries
% \begin{lstlisting}[style=short]
% label:
%         ...
%         goto label;
%         ...
%         ...
%         ...
% farlabel:
%         ...
%         ...
%         ...
%         goto farlabel;
% \end{lstlisting}
%     &
% \begin{lstlisting}[style=short,language=simpleasm]
% .L01:
%            ...
%            j    .L01
%            ...
%            ...
%            ...
% .L02:
%            ...
%            ...
%            ...
%            jg   .L02
% \end{lstlisting} \\
%     \else
% \begin{lstlisting}[style=short]
% label:
%         ...
%         goto label;
% \end{lstlisting}
%     &
% \begin{lstlisting}[style=short,language=simpleasm]
% .L01:
%            ...
%            BRC 15,.L01
% \end{lstlisting}
% \fi \\
%     \bottomrule
%   \end{tabular}
%   \end{DIFnomarkup}
%   \caption{Branch instruction}
%   \label{tab:branchinsn}
% \end{table}

% C language switch statements provide multi-way selection.  When the case
% labels of a switch statement satisfy grouping constraints, the compiler
% implements the selection with an address table.  The examples shown in
% \cref{tab:absswitch,tab:indswitch} use several simplifying conventions to
% hide irrelevant details:
% \begin{enumerate}
% \item The selection expression resides in \texttt{r2}.
% \item The case label constants begin at zero.
% \item The case labels, the default, and the address table use assembly
%   names \texttt{.Lcasei}, \texttt{.Ldef}, and \texttt{.Ltab} respectively.
% \end{enumerate}

% \begin{table}
%   \centering
%   \begin{DIFnomarkup}
%   \begin{tabular}{p{0.3\textwidth}p{0.65\textwidth}}
%     \toprule
%     C & \ARCH{} machine instructions (Assembler) \\
%     \midrule
% \begin{lstlisting}[style=short]
% switch(j)
%   {
%   case 0:
%     /* ... */
%   case 1:
%     /* ... */
%   case 3:
%     /* ... */
%   default:
%   }
% \end{lstlisting}
%     &
% \ifzseries
% \begin{lstlisting}[style=short,language=simpleasm]
%            lghi  %r1,%r3
%            clgr  %r2,%r1
%            brc   2,.Ldef
%            sllg  %r2,%r2,3
%            larl  %r1,.Ltab
%            lg    %r3,0(%r1,%r2)
%            br    %r3
% .Ltab:     .quad .Lcase0
%            .quad .Lcase1
%            .quad .Ldef
%            .quad .Lcase3
% \end{lstlisting}
% \else
% \begin{lstlisting}[style=short,language=simpleasm]
%            # Literal pool
% .LT0:
% .LC1:      .long .Ltab
%            # Code
%            LHI    1,3
%            CLR    2,1
%            BRC    2,.Ldef
%            SLL    2,2
%            A      2,.LC1-.LT0(13)
%            L      1,0(2)
%            BR     1
% .Ltab:     .long .Lcase0
%            .long .Lcase1
%            .long .Ldef
%            .long .Lcase3
% \end{lstlisting}
% \fi \\
%     \bottomrule
%   \end{tabular}
%   \end{DIFnomarkup}
%   \caption{Absolute switch code}
%   \label{tab:absswitch}
% \end{table}

% \begin{table}
%   \centering
%   \begin{DIFnomarkup}
%   \begin{tabular}{p{0.3\textwidth}p{0.65\textwidth}}
%     \toprule
%     C & \ARCH{} machine instructions (Assembler) \\
%     \midrule
% \begin{lstlisting}[style=short]
% switch(j)
%   {
%   case 0:
%     /* ... */
%   case 1:
%     /* ... */
%   case 3:
%     /* ... */
%   default:
%   }
% \end{lstlisting}
%     &
% \ifzseries
% \begin{lstlisting}[style=short,language=simpleasm]
%             # Literal pool
% .LT0:
%             # Code
%             lghi  %r1,3
%             clgr  %r2,%r1
%             brc   2,.Ldef
%             sllg  %r2,%r2,3
%             larl  %r1,.Ltab
%             lg    %r3,0(%r1,%r2)
%             agr   %r3,%r13
%             br    %r3
% .Ltab:      .quad .Lcase0-.LT0
%             .quad .Lcase1-.LT0
%             .quad .Ldef-.LT0
%             .quad .Lcase3-.LT0
% \end{lstlisting}
% \else
% \begin{lstlisting}[style=short,language=simpleasm]
%            # Literal pool
% .LT0:
% .LC1:      .long .Ltab-.LT0
%            # Code
%            LHI   1,3
%            CLR   2,1
%            BRC   2,.Ldef
%            SLL   2,2
%            L     1,.LC1-.LT0(13)
%            LA    1,0(1,13)
%            L     2,0(1,2)
%            LA    2,0(2,13)
%            BR    2
% .Ltab:     .long .Lcase0-.LT0
%            .long .Lcase1-.LT0
%            .long .Ldef-.LT0
%            .long .Lcase3-.LT0
% \end{lstlisting}
% \fi \\
%     \bottomrule
%   \end{tabular}
%   \end{DIFnomarkup}
%   \caption{Position-independent switch code, all models}
%   \label{tab:indswitch}
% \end{table}

% \subsection{Dynamic Stack Space Allocation}
% \label{dynamicstack}
% The GNU C compiler, and most recent compilers, support dynamic stack
% space allocation via \texttt{alloca}.

% \Cref{fig:dynstackalloc} shows the stack frame before and after dynamic
% stack allocation.  The local variables area is used for storage of
% function data, such as local variables, whose sizes are known to the
% compiler.  This area is allocated at function entry and does not
% change in size or position during the function's activation.

% The parameter area holds ``overflow'' arguments passed in calls to
% other functions.  (See the \jumplabel{more} label in
% \cref{parameterpassing}.)  Its size is also known to the compiler and can
% be allocated along with the fixed frame area at function entry.  However,
% the standard calling sequence requires that the parameter area begins
% at a fixed offset (\STACKSIZE{}) from the stack pointer, so this area must
% move when dynamic stack allocation occurs.

% Data in the parameter area are naturally addressed at
% constant offsets from the stack pointer.  However, in the presence of
% dynamic stack allocation, the offsets from the stack pointer to the
% data in the local-variable area are not constant.  To provide
% addressability, a frame pointer is established to locate the local
% variables area consistently throughout the function's
% activation.

% Dynamic stack allocation is accomplished by ``opening'' the stack
% just above the parameter area.  The following steps show the
% process in detail:

% \begin{enumerate}
% \item After a new stack frame is acquired, and before the
% first dynamic space allocation, a new register, the frame pointer or
% FP, is set to the value of the stack pointer.  The frame pointer is
% used for references to the function's local, non-static variables.  The
% frame pointer does not change during the execution of a function, even
% though the stack pointer may change as a result of dynamic
% allocation.

% \item The amount of dynamic space to be allocated is rounded
% up to a multiple of 8 bytes, so that 8-byte stack alignment is
% maintained.

% \item The stack pointer is decreased by the rounded byte
% count, and the address of the previous stack frame (the back chain)
% may be stored at the word addressed by the new stack pointer.  The back
% chain is not necessary to restore from this allocation at the end of
% the function since the frame pointer can be used to restore the stack
% pointer.

% \end{enumerate}

% \Cref{fig:dynstackalloc} is a snapshot of the stack layout after the
% prologue code has dynamically extended the stack frame.

% \begin{figure}
%   \centering
%   \ifSkipTikZ
% \begin{verbatim}
%           Before Dynamic               After Dynamic
%          Stack Allocation             Stack Allocation

%        |  Previous stack  |         |  Previous stack  |
%        |      frame       |         |      frame       |
%        +------------------+         +------------------+
%        |    Back chain    |         |    Back chain    |
%        |    (optional)    |         |    (optional)    |
%   .--> ====================    .--> ====================
%  /     | Local and spill  |   /     | Local and spill  |
% |      | variable area of |  |      | variable area of |
% |      | calling function |  |      | calling function |
% |160+n +------------------+  |      +------------------+ <---  FP+160+n
% |      |  Parameter area  |  |      |                  |     ^
% |      | passed to called |  |      |                  |     :
% |      |    functions     |  |      |                  |     :
% |  160 +------------------+  |      |                  |     :
% |      |  Register save   |  |      |                  |     :
% |      | area for called  |  |      |                  |     :
% |      |    functions     |  |      |                  |     :
%  \   8 +------------------+  |      |     Dynamic      |     :
%   `----+    Back chain    |  |      | Allocation Area  |     :
%        |    (optional)    |  |      |                  |     :
% SP-> 0 ====================- | - - -|                  | <-- FP
%                              |      |                  |
%                              |      |                  |
%                              |      |                  |
%                              |      |                  |
%                              |      |                  |
%                              |160+n +------------------+
%                              |      |  Parameter area  |
%                              |      | passed to called |
%                              |      |    functions     |
%                              |  160 +------------------+
%                              |      |  Register save   |
%                              |      | area for called  |
%                              |      |    functions     |
%                               \   8 +------------------+
%                                `----+    Back chain    |
%                                     |    (optional)    |
%                              SP-> 0 ====================
% \end{verbatim}
%   \else
%   \pgfsetlayers{background,main}
%   \begin{tikzpicture}
%     \matrix [memory layout,inner sep=0pt,
%     nodes={text width=10em, align=center, inner sep=1ex}] (ma)
%     {
%       \node (prev)     {Previous stack frame}; \\
%       \node (prevback) {Back chain (optional)}; \\
%       \node (aspill)    {Local and spill variable area of calling function}; \\
%       \node [minimum height=9em] (dynalloc) {\vfill Dynamic Allocation Area\par\vfill}; \\
%       \node [inactive layout] (param) {Parameter area passed to called functions}; \\
%       \node [inactive layout] (save)  {Register save area for called functions}; \\
%       \node (back)     {Back chain (optional)}; \\
%     };
%     \draw \foreach \Node in {prev, aspill, param, dynalloc, save} {
%       (\Node.south -| ma.west) -- (\Node.south -| ma.east)
%     };
%     \begin{scope}[very thick, shorten <=-1ex, shorten >=-1ex]
%       \draw (prevback.south -| ma.west) -- (prevback.south -| ma.east);
%       \draw (ma.south west) -- (ma.south east);
%     \end{scope}
%     \path [every node/.style={left=1pt,fill=white,font={\footnotesize}}]
%     (dynalloc.south -| ma.west) node {\texttt{~SP+\STACKSIZE{}+n}}
%     (param.south -| ma.west) node {\texttt{SP+\STACKSIZE{}}}
%     (save.south -| ma.west) node {\texttt{SP+{\ifzseries 16\else 8\fi}}};
%     \path (ma.south west) node [left=2em] (sp) {\texttt{SP}};
%     \draw [->, shorten >=1.5ex] (sp) -- (ma.south west);
%     \draw [rounded corners, shorten >=1.5ex, ->]
%     (back -| ma.west)  -- (back -| current bounding box.west)
%     |- (prevback.south -| ma.west);
%     % ---
%     \path (current bounding box.north west) +(-2em,0)
%     node [matrix, memory layout, inner sep=0pt, below left,
%     nodes={text width=10em, align=center, inner sep=1ex}] (mb)
%     {
%       \node (prev)     {Previous stack frame}; \\
%       \node (prevback) {Back chain (optional)}; \\
%       \node (spill)    {Local and spill variable area of calling function}; \\
%       \node [inactive layout] (param) {Parameter area passed to called functions}; \\
%       \node [inactive layout] (save)  {Register save area for called functions}; \\
%       \node (back)     {Back chain (optional)}; \\
%     };
%     \draw \foreach \Node in {prev, spill, param, save} {
%       (\Node.south -| mb.west) -- (\Node.south -| mb.east)
%     };
%     \begin{scope}[very thick, shorten <=-1ex, shorten >=-1ex]
%       \draw (prevback.south -| mb.west) -- (prevback.south -| mb.east);
%       \draw (mb.south west) -- (mb.south east);
%     \end{scope}
%     \path [every node/.style={left=1pt,font={\footnotesize}}]
%     (spill.south -| mb.west) node {\texttt{~SP+\STACKSIZE{}+n}}
%     (param.south -| mb.west) node {\texttt{SP+\STACKSIZE{}}}
%     (save.south -| mb.west) node {\texttt{SP+{\ifzseries 16\else 8\fi}}};
%     \path (mb.south west) node [left=2em] (sp) {\texttt{SP}};
%     \draw [->, shorten >=1.5ex] (sp) -- (mb.south west);
%     \draw [rounded corners, shorten >=1.5ex, ->]
%     (back -| mb.west)  -- (back -| current bounding box.west)
%     |- (prevback.south -| mb.west);
%     % ---
%     \path [every node/.style={right=2em}]
%     (back.south -| ma.east) node (fp) {\texttt{FP}}
%     (aspill.south -| ma.east) node [font={\footnotesize}] (newfp)
%     {\texttt{FP+\STACKSIZE{}+n}};
%     \draw [->] (fp) -- (fp -| ma.east);
%     \draw [->] (newfp) -- (newfp -| ma.east);
%     \draw [loosely dotted, ->] (fp) -- (fp |- newfp.south);
%     \begin{pgfonlayer}{background}
%       \begin{scope}[dashed, thin]
%         \draw (newfp -| ma.east) -- (spill.south -| mb.east);
%         \draw (fp -| ma.east) -- (mb.south east);
%       \end{scope}
%     \end{pgfonlayer}
%     % ---
%     \path [every node/.style={above=1ex}]
%     (ma.north) node {After Dynamic Stack Allocation}
%     (mb.north) node {Before Dynamic Stack Allocation};
%   \end{tikzpicture}
%   \fi
%   \caption{Dynamic stack space allocation}
%   \label{fig:dynstackalloc}
% \end{figure}

% The above process can be repeated as many times as desired
% within a single function activation.  When it is time to return, the
% stack pointer is set to the value of the back chain, thereby removing
% all dynamically allocated stack space along with the rest of the stack
% frame.  Naturally, a program must not reference the dynamically
% allocated stack area after it has been freed.

% Even in the presence of signals, the above dynamic allocation
% scheme is ``safe.'' If a signal interrupts allocation, one of three
% things can happen:
% \begin{itemize}
% \item The signal handler can return.  The process then resumes the
%   dynamic allocation from the point of interruption.
% \item The signal handler can execute a non-local goto or a jump.  This
%   resets the process to a new context in a previous stack frame,
%   automatically discarding the dynamic allocation.
% \item The process can terminate.
% \end{itemize}

% Regardless of when the signal arrives during dynamic allocation, the
% result is a consistent (though possibly dead) process.

% \section{DWARF Definition}
% \index{DWARF}
% This section defines the ``Debug With Attributed Record Format''
% (DWARF) debugging format for \ARCH{} processors.  The
% \ABINAME{} ABI does not define a debug format.  However, all systems that
% do implement DWARF shall use the following definitions.

% DWARF is a specification developed for symbolic source-level
% debugging.  The debugging information format does not favor the design
% of any compiler or debugger.

% The DWARF definition requires some machine-specific definitions.  The
% register number mapping\index{DWARF!register
%   numbers}\index{registers!DWARF numbers} is specified for the \ARCH{}
% processors in \cref{tab:dwarfreg}.

% \begin{table}
%   \centering
%   \begin{DIFnomarkup}
%   \begin{threeparttable}
%     \begin{tabular}[t]{rl!{\qquad}rl}
%       \toprule
%       DWARF  & \ARCH{}  & DWARF  & \ARCH{} \\
%       number & register & number & register \\
%       \midrule
%       0--15 & \texttt{r0}--\texttt{r15}
%                         & 65 & PSW address \\
%       16 & \texttt{f0} / \texttt{v0}
%                         & 66 & \emph{reserved} (z/OS) \\
%       17 & \texttt{f2} / \texttt{v2}
%                         & 67 & \emph{reserved} (z/OS) \\
%       18 & \texttt{f4} / \texttt{v4}
%                         & 68 & \texttt{v16} \\
%       19 & \texttt{f6} / \texttt{v6}
%                         & 69 & \texttt{v18} \\
%       20 & \texttt{f1} / \texttt{v1}
%                         & 70 & \texttt{v20} \\
%       21 & \texttt{f3} / \texttt{v3}
%                         & 71 & \texttt{v22} \\
%       22 & \texttt{f5} / \texttt{v5}
%                         & 72 & \texttt{v17} \\
%       23 & \texttt{f7} / \texttt{v7}
%                         & 73 & \texttt{v19} \\
%       24 & \texttt{f8} / \texttt{v8}
%                         & 74 & \texttt{v21} \\
%       25 & \texttt{f10} / \texttt{v10}
%                         & 75 & \texttt{v23} \\
%       26 & \texttt{f12} / \texttt{v12}
%                         & 76 & \texttt{v24} \\
%       27 & \texttt{f14} / \texttt{v14}
%                         & 77 & \texttt{v26} \\
%       28 & \texttt{f9} / \texttt{v9}
%                         & 78 & \texttt{v28} \\
%       29 & \texttt{f11} / \texttt{v11}
%                         & 79 & \texttt{v30} \\
%       30 & \texttt{f13} / \texttt{v13}
%                         & 80 & \texttt{v25} \\
%       31 & \texttt{f15} / \texttt{v15}
%                         & 81 & \texttt{v27} \\
%       32--47 & \texttt{cr0}--\texttt{cr15}\tnote{\dagger}
%                         & 82 & \texttt{v29} \\
%       48--63 & \texttt{a0}--\texttt{a15}
%                         & 83 & \texttt{v31} \\
%       64 & PSW mask \\
%       \bottomrule
%     \end{tabular}
%     \medskip
%     \begin{tablenotes}
%     \item [\dagger] Control registers cannot be referenced by user-space
%       applications.  They are reserved for use by operating system code.
%     \end{tablenotes}
%   \end{threeparttable}
%   \end{DIFnomarkup}
%   \caption{DWARF register number mapping}
%   \label{tab:dwarfreg}
% \end{table}

% % FIXME: Should document the CFA and its offset.

% For the placement of a piece within a composite location description, as
% defined by the byte piece operation \texttt{DW\_OP\_piece} or the bit
% piece operation \texttt{DW\_OP\_bit\_piece}, the following applies:
% \begin{itemize}
% \item Pieces of a floating-point or vector register are taken from the
%   left.  This means that a bit piece with offset \(t\) and size \(n\)
%   consists of the register's bits numbered from \(t\) to \(t+n-1\),
%   according to big-endian bit numbering.  And a byte piece of a
%   floating-point or vector register of size \(n\) consists of the
%   register's \(n\) leftmost bytes.
% \item For any other register, pieces are taken from the right.  This means
%   that a bit piece with offset \(t\) and size \(n\) consists of the bits
%   numbered from \(w-t-n\) to \(w-t-1\), where \(w\) is the register's bit
%   width.  And a byte piece of size \(n\) consists of the register's \(n\)
%   rightmost bytes.
% \end{itemize}

% Whenever interpreting a register as a given type, such as when using the
% register value operation \texttt{DW\_OP\_regval\_type} or the register
% location description \texttt{DW\_OP\_regx}, the resulting value consists
% of the same bits as the bit piece starting at offset zero and having the
% size of the given type.

\chapter{The rxbin file layout}
\label{rxbinfiles}
\index{.rxbin layout}
% \index{object file}
This section describes the \crexx{} binary file, \emph{.rxbin} format.

\section{rxbin header}
Note: at the moment, there is no rxbin header. This is a proposed format.
% \subsection{Machine Information}
% For file identification in \texttt{e\_ident} the \ARCH{} processor
% family requires the values shown in \cref{tab:eident}.

\begin{table}
  \centering
  \begin{DIFnomarkup}
  \begin{tabular}{lll}
    \toprule
    Position & Value & Comments \\
    \midrule
    \texttt{e\_ident[EI\_CLASS]} & \texttt{ELFCLASS\NBITS{}} &
    For all \NBITS{}$\,$bit implementations \\
    \texttt{e\_ident[EI\_DATA]} & \texttt{ELFDATA\NBITS{}MSB} &
    For all Big-Endian implementations \\
    \bottomrule
  \end{tabular}
  \end{DIFnomarkup}
  \caption{Machine-specific ELF identification fields}
  \label{tab:eident}
\end{table}

% The ELF header's \texttt{e\_flags} field holds bit flags associated
% with the file.  Since the \ARCH{} processor family defines no flags,
% this member contains zero.

% Processor identification resides in the ELF header's
% \texttt{e\_machine} field and must have the value 22, defined as the
% name \texttt{EM\_S390}.

% \section{Sections}
% \subsection{Special Sections}
% Various sections hold program and control information.  The following
% sections, whose types and attributes are listed in \cref{tab:sections},
% are used by the system:
% \begin{description}
% \item[\texttt{.got}] \index{got section@\texttt{.got} section} This
%   section holds the Global Offset Table, or GOT\@.  See
%   \cref{codingexamples,globaloffsettable} for more information.
% \item[\texttt{.plt}] \index{plt section@\texttt{.plt} section} This
%   section holds the Procedure Linkage Table, or PLT.  See
%   \cref{procedurelinkagetable} for more information.
% \end{description}

% \begin{table}
%   \centering
%   \begin{DIFnomarkup}
%   \begin{tabular}{lll}
%     \toprule
%     Name & Type & Attributes \\
%     \midrule
%     \texttt{.got} & \texttt{SHT\_PROGBITS} &
%     \texttt{SHF\_ALLOC + SHF\_WRITE} \\
%     \texttt{.plt} & \texttt{SHT\_PROGBITS} &
%     \texttt{SHF\_ALLOC + SHF\_WRITE + SHF\_EXECINSTR} \\
%     \bottomrule
%   \end{tabular}
%   \end{DIFnomarkup}
%   \caption{Special sections}
%   \label{tab:sections}
% \end{table}

% \section{Symbol Table}
% \index{symbol table}
% \subsection{Symbol Values}
% \label{symbolvalues}
% A symbol table entry's \texttt{st\_value} field is the symbol value.  If
% that value represents a section offset or a virtual address, it must be
% halfword aligned.  This enables use of CIA-relative addressing
% instructions such as \texttt{LARL}.

% If an executable file contains a reference to a function defined in
% one of its associated shared objects, the symbol table section for the
% file will contain an entry for that symbol.  The \texttt{st\_shndx}
% field of that symbol table entry contains \texttt{SHN\_UNDEF}.  This
% informs the dynamic linker that the symbol definition for that
% function is not contained in the executable file itself.  If that
% symbol has been allocated a Procedure Linkage Table entry in the
% executable file, and the \texttt{st\_value} field for that symbol
% table entry is nonzero, the value is the virtual address of the first
% instruction of that PLT entry.  Otherwise the \texttt{st\_value} field
% contains zero.  This PLT entry address is used by the dynamic linker
% in resolving references to the address of the function.  See
% \cref{functionaddresses} for details.

% \section{Relocation}
% \index{relocation}
% \subsection{Relocation Types}
% Relocation entries describe how to alter the instruction and data
% relocation fields listed below.  \Cref{fig:relocfields} illustrates the
% affected bits of each field type.

% \begin{figure}
%   \centering
%   \ifSkipTikZ
% \begin{verbatim}
% +-------+-------+-------+-------+-------+-------+-------+-------+
% |0      |1      |2      |3      |4      |5      |6      |7      |
% |                            quad64                             |
% |0                                                            63|
% +---------------------------------------------------------------+

% +-------+-------+-------+-------+
% |0      |1      |2      |3      |
% |             word32            |
% |0                            31|
% +-------------------------------+

% +-------+-------+-------+-------+
% |0      |1      |2      |3      |
% |              pc32             |
% |0                            31|
% +-------------------------------+

% +-------+-------+-------+-------+
% |0      |1      |2      |3      |
% |       |      pc24             |
% |0     7|8                    31|
% +-------+-----------------------+

% +----+--+-------+-------+-------+
% |0   |  |1      |2      |3      |
% |    |      mid20       |       |
% |0  3|4               23|24   31|
% +----+------------------+-------+

% +-------+-------+
% |0      |1      |
% |     half16    |
% |0            15|
% +---------------+

% +-------+-------+
% |0      |1      |
% |     pc16      |
% |0            15|
% +---------------+

% +---+---+-------+
% |0  |   |1      |
% |   |   low12   |
% |0 3|4        15|
% +---+-----------+

% +---+---+-------+
% |0  |   |1      |
% |   |    pc12   |
% |0 3|4        15|
% +---+-----------+

% +-------+
% |0      |
% | byte8 |
% |0     7|
% +-------+
% \end{verbatim}
%   \else
%   \begin{tikzpicture}[x=1.2ex,y=3em]
%     \ifzseries
%     \foreach \fields/\nbytes/\shifted in {
%       {0 64/quad64/}/8/{0,0},
%       {0 32/word32/}/4/{0,-1.3},
%       {0 32/pc32/}/4/{0,-2.6},
%       {0 8/ /,32/pc24/}/4/{0,-3.9},
%       {0 4/ /,24/mid20/, 32/ /}/4/{0,-5.2},
%       {0 16/half16/}/2/{0,-6.5},
%       {0 16/pc16/}/2/{44,-1.3},
%       {0 4/ /,16/low12/}/2/{44,-2.6},
%       {0 4/ /,16/pc12/}/2/{44,-3.9},
%       {0 8/\vbox{\hbox{byte8}\vskip 1em }/}/1/{44,-5.2}}
%     {
%       \begin{scope}[shift={(\shifted)}]
%         \pgfmathsetmacro{\nbits}{8*\nbytes}
%         \path [bitchart box] (0, 0) rectangle (\nbits, 1);
%         \begin{scope}[every path/.style={draw,shorten >=2em}]
%           \pgfmathsetmacro{\bytesm}{\nbytes-1}
%           \bitchartbytes{0}{0,...,\bytesm}
%         \end{scope}
%         \begin{scope}[bitfield label/.style={shift={(0,-0.2)}}]
%           \csname bitchartfields\expandafter\endcsname\fields;
%         \end{scope}
%       \end{scope}
%     }
%     \else
%     \foreach \fields/\nbytes/\shifted in {
%       {0 32/word32/}/4/{0,0},
%       {0 16/half16/}/2/{40,0},
%       {0 16/pc16/}/2/{0,-1.3},
%       {0 4/ /,16/low12/}/2/{24,-1.3},
%       {0 8/\vbox{\hbox{byte8}\vskip 1em }/}/1/{48,-1.3}}
%     {
%       \begin{scope}[shift={(\shifted)}]
%         \pgfmathsetmacro{\nbits}{8*\nbytes}
%         \path [bitchart box] (0, 0) rectangle (\nbits, 1);
%         \begin{scope}[every path/.style={draw,shorten >=2em}]
%           \pgfmathsetmacro{\bytesm}{\nbytes-1}
%           \bitchartbytes{0}{0,...,\bytesm}
%         \end{scope}
%         \begin{scope}[bitfield label/.style={shift={(0,-0.2)}}]
%           \csname bitchartfields\expandafter\endcsname\fields;
%         \end{scope}
%       \end{scope}
%     }
%     \fi
%   \end{tikzpicture}
%   \fi
%   \caption[Relocation fields]{Relocation fields.  Bit numbers appear in
%     the lower box corners; byte numbers appear in the upper left box
%     corners.}
%   \label{fig:relocfields}
% \end{figure}

% \begin{description}
% \ifzseries
% \item[\texttt{quad64}] This specifies a 64-bit field occupying 8
%   bytes, the alignment of which is 4 bytes unless otherwise specified.
% \fi
% \item[\texttt{word32}] This specifies a 32-bit field occupying 4 bytes, the
%   alignment of which is 4 bytes unless otherwise specified.
% \ifzseries
% \item[\texttt{pc32}] This specifies a 32-bit field occupying 4 bytes
%   with 2-byte alignment.  The signed value in this field is shifted to
%   the left by 1 before it is used as a program counter relative
%   displacement (for example, the immediate field of a ``Load Address
%   Relative Long'' instruction).
% \item[\texttt{pc24}] This specifies a 24-bit field contained within 4
%   consecutive bytes with 2-byte alignment.  The signed value in this field
%   is shifted to the left by 1 before it is used as a program counter
%   relative displacement (for example, the third immediate field of a
%   ``Branch Prediction Relative Preload'' instruction).
% \item[\texttt{mid20}] This specifies a 20-bit field contained within 4
%   consecutive bytes with 2-byte alignment.  The 20-bit signed value is the
%   ``long displacement'' of a memory reference.
% \fi
% \item[\texttt{half16}] This specifies a 16-bit field occupying 2 bytes with
%   2-byte alignment (for example, the immediate field of an ``Add
%   Halfword Immediate'' instruction).
% \item[\texttt{pc16}] This specifies a 16-bit field occupying 2 bytes
%   with 2-byte alignment.  The signed value in this field is shifted to
%   the left by 1 before it is used as a program counter relative
%   displacement (for example, the immediate field of an ``Branch
%   Relative'' instruction).
% \item[\texttt{low12}] This specifies a 12-bit field contained within a
%   halfword with 2-byte alignment.  The 12 bit unsigned value is the
%   displacement of a memory reference.
% \item[\texttt{pc12}] This specifies a 12-bit field contained within a
%   halfword with 2-byte alignment.  The signed value in this field is
%   shifted to the left by 1 before it is used as a program counter relative
%   displacement (for example, the second immediate field of a ``Branch
%   Prediction Relative Preload'' instruction).
% \item[\texttt{byte8}] This specifies an 8-bit field with 1-byte
%   alignment.
% \end{description}

% Calculations in \cref{tab:relocations} assume the actions are
% transforming a relocatable file into either an executable or a shared
% object file.  Conceptually, the linkage editor merges one or more
% relocatable files to form the output.  It first determines how to
% combine and locate the input files, next it updates the symbol values,
% and then it performs relocations.

% Relocations applied to executable or shared object files are similar
% and accomplish the same result.  The following notations are used in
% \cref{tab:relocations}:

% \begin{description}
% \item[$A$] Represents the addend used to compute the value of the
%   relocatable field.
% \item[$B$] Represents the base address at which a shared object has
%   been loaded into memory during execution.  Generally, a shared object
%   file is built with a 0 base virtual address, but the execution
%   address will be different.
% \item[$G$] Represents the section offset or address of the Global
%   Offset Table.  See \cref{codingexamples,globaloffsettable}
%   for more information.
% \item[$L$] Represents the section offset or address of the Procedure
%   Linkage Table entry for a symbol.  A PLT entry redirects a function
%   call to the proper destination.  The linkage editor builds the
%   initial PLT\@.  See \cref{procedurelinkagetable} for more
%   information.
% \item[$O$] Represents the offset into the GOT at which the address of
%   the relocation entry's symbol will reside during execution.  See
%   \cref{codingexamples,globaloffsettable} for more
%   information.
% \item[$P$] Represents the place (section offset or address) of the
%   storage unit being relocated (computed using \texttt{r\_offset}).
% \item[$R$] Represents the offset of the symbol within the section in
%   which the symbol is defined (its section-relative address).
% \item[$S$] Represents the value of the symbol whose index resides in
%   the relocation entry.
% \item[$T$] Similar to $O$, except that the address that is stored may be
%   the address of the PLT entry for the symbol.
% \end{description}

% Relocation entries apply to bytes, halfwords, {\ifzseries words, or
%   doublewords\else or words\fi}.  In either case, the \texttt{r\_offset}
% value designates the offset or virtual address of the first byte of the
% affected storage unit.  The relocation type specifies which bits to change
% and how to calculate their values.  The \ARCH{} family uses only the
% \texttt{Elf\NBITS{}\_Rela} relocation entries with explicit addends.  For
% the relocation entries, the \texttt{r\_addend} field serves as the
% relocation addend.  In all cases, the offset, addend, and the computed
% result use the byte order specified in the ELF header.

% The following general rules apply to the interpretation of the relocation
% types in \cref{tab:relocations}:

% \begin{itemize}
% \item ``$+$'' and ``$-$'' denote \NBITS{}-bit modulus addition and
%   subtraction, respectively.  ``$>>$'' denotes arithmetic right-shifting
%   (shifting with sign copying) of the value of the left operand by the
%   number of bits given by the right operand.
% \item Reference in a calculation to the value $G$, $O$, or $T$ implicitly
%   creates a GOT entry for the indicated symbol, and a reference to $L$
%   implicitly creates a PLT entry.
% \item A computed value must be suited for the relocation field it is used
%   for.  In particular:
%   \begin{description}
%   \item[\texttt{half16}:] The upper {\ifzseries 48\else 16\fi} bits must
%     be all ones or all zeroes.
%   \item[\texttt{pc16}:] The upper {\ifzseries 47\else 15\fi} bits must be
%     all ones or all zeroes and the lowest bit must be zero.
%     %
%     {\ifzseries{} \item[\texttt{pc32}:] The upper 31 bits must be all ones
%       or all zeroes and the lowest bit must be zero.\fi}
%     %
%   \item[\texttt{low12}:] The upper {\ifzseries 52\else 20\fi} bits must
%     all be zero.
%   \item[\texttt{byte8}:] The upper {\ifzseries 56\else 24\fi} bits must
%     all be zero.
%   \end{description}
% \end{itemize}

% \begin{DIFnomarkup}
% \begin{longtable}{lrll}
%   \caption{Relocation types\label{tab:relocations}}\\[\medskipamount]
%   \toprule
%   Name & Value & Field & Calculation \\
%   \midrule
%   \endfirsthead
%   \caption[]{Relocation types \emph{-- continued}}\\[\medskipamount]
%   \toprule
%   Name & Value & Field & Calculation \\
%   \midrule
%   \endhead
%   \bottomrule
%   \endfoot
%   \texttt{R\_390\_NONE} & 0 & \emph{none} & \emph{none} \\
%   \texttt{R\_390\_8} & 1 & \texttt{byte8} & $S + A$ \\
%   \texttt{R\_390\_12} & 2 & \texttt{low12} & $S + A$ \\
%   \texttt{R\_390\_16} & 3 & \texttt{half16} & $S + A$ \\
%   \texttt{R\_390\_32} & 4 & \texttt{word32} & $S + A$ \\
%   \texttt{R\_390\_PC32} & 5 & \texttt{word32} & $S + A - P$ \\
%   \texttt{R\_390\_GOT12} & 6 & \texttt{low12} & $O + A$ \\
%   \texttt{R\_390\_GOT32} & 7 & \texttt{word32} & $O + A$ \\
%   \texttt{R\_390\_PLT32} & 8 & \texttt{word32} & $L + A$ \\
%   \texttt{R\_390\_COPY}\textsuperscript{ \dagger} & 9 & \emph{none} & \\
%   \texttt{R\_390\_GLOB\_DAT}\textsuperscript{ \dagger} & 10 & \texttt{quad64} & $S + A$ \\
%   \texttt{R\_390\_JMP\_SLOT}\textsuperscript{ \dagger} & 11 & \emph{none} & \\
%   \texttt{R\_390\_RELATIVE}\textsuperscript{ \dagger} & 12 & \texttt{quad64} & $B + A$ \\
%   \texttt{R\_390\_GOTOFF32} & 13 & \texttt{word32} & $S + A - G$ \\
%   \texttt{R\_390\_GOTPC} & 14 & \texttt{quad64} & $G + A - P$ \\
%   \texttt{R\_390\_GOT16} & 15 & \texttt{half16} & $O + A$ \\
%   \texttt{R\_390\_PC16} & 16 & \texttt{half16} & $S + A - P$ \\
%   \texttt{R\_390\_PC16DBL} & 17 & \texttt{pc16} & $(S + A - P) >> 1$ \\
%   \texttt{R\_390\_PLT16DBL} & 18 & \texttt{pc16} & $(L + A - P) >> 1$ \\
%   \ifzseries
%   \texttt{R\_390\_PC32DBL} & 19 & \texttt{pc32} & $(S + A - P) >> 1$ \\
%   \texttt{R\_390\_PLT32DBL} & 20 & \texttt{pc32} & $(L + A - P) >> 1$ \\
%   \texttt{R\_390\_GOTPCDBL} & 21 & \texttt{pc32} & $(G + A - P) >> 1$ \\
%   \texttt{R\_390\_64} & 22 & \texttt{quad64} & $S + A$ \\
%   \texttt{R\_390\_PC64} & 23 & \texttt{quad64} & $S + A - P$ \\
%   \texttt{R\_390\_GOT64} & 24 & \texttt{quad64} & $O + A$ \\
%   \texttt{R\_390\_PLT64} & 25 & \texttt{quad64} & $L + A - P$ \\
%   \texttt{R\_390\_GOTENT} & 26 & \texttt{pc32} & $(G + O + A - P) >> 1$ \\
%   \texttt{R\_390\_GOTOFF16} & 27 & \texttt{half16} & $S + A - G$ \\
%   \texttt{R\_390\_GOTOFF64} & 28  & \texttt{quad64} & $S + A - G$\\
%   \texttt{R\_390\_GOTPLT12} & 29 & \texttt{low12} & $T + A$ \\
%   \texttt{R\_390\_GOTPLT16} & 30 & \texttt{half16} & $T + A$ \\
%   \texttt{R\_390\_GOTPLT32} & 31 & \texttt{word32} & $T + A - P$ \\
%   \texttt{R\_390\_GOTPLT64} & 32 & \texttt{quad64} & $T + A$ \\
%   \texttt{R\_390\_GOTPLTENT} & 33 & \texttt{pc32} & $(G + T + A - P) >> 1$ \\
%   \texttt{R\_390\_PLTOFF16} & 34 & \texttt{half16} & $L - G + A$ \\
%   \texttt{R\_390\_PLTOFF32} & 35 & \texttt{word32} & $L - G + A$ \\
%   \texttt{R\_390\_PLTOFF64} & 36 & \texttt{quad64} & $L - G + A$ \\
%   \texttt{R\_390\_TLS\_LOAD}\textsuperscript{ \dagger} & 37 & \emph{none} & \\
%   \texttt{R\_390\_TLS\_GDCALL}\textsuperscript{ \dagger} & 38 & \emph{none} & \\
%   \texttt{R\_390\_TLS\_LDCALL}\textsuperscript{ \dagger} & 39 & \emph{none} & \\
%   \texttt{R\_390\_TLS\_GD64}\textsuperscript{ \dagger} & 41 & \texttt{quad64} & \\
%   \texttt{R\_390\_TLS\_GOTIE12}\textsuperscript{ \dagger} & 42 & \texttt{low12} & \\
%   \texttt{R\_390\_TLS\_GOTIE64}\textsuperscript{ \dagger} & 44 & \texttt{quad64} & \\
%   \texttt{R\_390\_TLS\_LDM64}\textsuperscript{ \dagger} & 46 & \texttt{quad64} & \\
%   \texttt{R\_390\_TLS\_IE64}\textsuperscript{ \dagger} & 48 & \texttt{quad64} & \\
%   \texttt{R\_390\_TLS\_IEENT}\textsuperscript{ \dagger} & 49 & \texttt{pc32} & \\
%   \texttt{R\_390\_TLS\_LE64}\textsuperscript{ \dagger} & 51 & \texttt{quad64} & \\
%   \texttt{R\_390\_TLS\_LDO64}\textsuperscript{ \dagger} & 53 & \texttt{quad64} & \\
%   \texttt{R\_390\_TLS\_DTPMOD}\textsuperscript{ \dagger} & 54 & \texttt{quad64} & \\
%   \texttt{R\_390\_TLS\_DTPOFF}\textsuperscript{ \dagger} & 55 & \texttt{quad64} & \\
%   \texttt{R\_390\_TLS\_TPOFF}\textsuperscript{ \dagger} & 56 & \texttt{quad64} & \\
%   \texttt{R\_390\_20} & 57 & \texttt{mid20} & $S + A$ \\
%   \texttt{R\_390\_GOT20} & 58 & \texttt{mid20} & $O + A$ \\
%   \texttt{R\_390\_GOTPLT20} & 59 & \texttt{mid20} & $T + A$ \\
%   \texttt{R\_390\_TLS\_GOTIE20}\textsuperscript{ \dagger} & 60 & \texttt{mid20} & \\
%   \texttt{R\_390\_IRELATIVE}\textsuperscript{ \dagger} & 61 & \texttt{quad64} & $\char`*(B + A)()$ \\
%   \texttt{R\_390\_PC12DBL} & 62 & \texttt{pc12} & $(S + A - P) >> 1$ \\
%   \texttt{R\_390\_PLT12DBL} & 63 & \texttt{pc12} & $(L + A - P) >> 1$ \\
%   \texttt{R\_390\_PC24DBL} & 64 & \texttt{pc24} & $(S + A - P) >> 1$ \\
%   \texttt{R\_390\_PLT24DBL} & 65 & \texttt{pc24} & $(L + A - P) >> 1$ \\
%   \fi
% \end{longtable}
% \end{DIFnomarkup}

% The relocation types marked with ``\dagger'' in \cref{tab:relocations} are
% handled specially:

% \begin{description}
% \item[\texttt{R\_390\_COPY}] The linkage editor creates this relocation
%   type for dynamic linking.  Its offset member refers to a location in a
%   writable segment.  The symbol table index specifies a symbol that should
%   exist both in the current object file and in a shared object.  During
%   execution, the dynamic linker copies data associated with the shared
%   object's symbol to the location specified by the offset.
% \item[\texttt{R\_390\_GLOB\_DAT}] This relocation type resembles
%   \texttt{R\_390\_\NBITS{}}, except that it sets a Global Offset Table
%   entry to the address of the specified symbol.  This special relocation
%   type allows one to determine the correspondence between symbols and GOT
%   entries.
% \item[\texttt{R\_390\_JMP\_SLOT}] The linkage editor creates this
%   relocation type for dynamic linking.  Its offset member gives the
%   location of a Global Offset Table entry.  The dynamic linker modifies
%   the GOT entry to transfer control to the designated symbol's address
%   (see \cref{procedurelinkagetable}).
% \item[\texttt{R\_390\_RELATIVE}] The linkage editor creates this
%   relocation type for dynamic linking.  Its offset member gives a location
%   within a shared object that contains a value representing a virtual
%   address.  The dynamic linker computes the virtual address by adding the
%   shared object's base address to the addend.  Relocation entries for this
%   type must specify 0 for the symbol table index.
% \item[\texttt{R\_390\_IRELATIVE}] The linkage editor creates this
%   relocation type for dynamic linking.  The dynamic linker computes an
%   address as for the \texttt{R\_390\_RELATIVE} relocation and then invokes
%   the function residing at that address, passing the value of
%   \texttt{AT\_HWCAP} from the auxiliary vector as its single argument (see
%   \cref{auxvector}).  The return value resulting from that invocation is
%   written into the location described by the offset.  Such a function is
%   also known as an ``IFUNC resolver'' and has the following signature:
%   \begin{center}
%     \lstinline@void *f (unsigned long hwcap);@
%   \end{center}
% \item[\texttt{R\_390\_TLS\_\char`*}] These relocation types are used for
%   thread-local storage handling.  They are described in
%   \cite{tlshandling}.
% \end{description}

\chapter{Program Loading and Linking}
% \label{chprogload}
% This section describes how the Executable and Linking Format (ELF) is
% used in the construction and execution of programs.

% \section{Program Loading}
% \index{program loading}
% As the system creates or augments a process image, it logically copies
% a file's segment to a virtual memory segment.  When---and if---the
% system physically reads the file depends on the program's execution
% behavior, on the system load, and so on.  A process does not require a
% physical page until it references the logical page during execution,
% and processes commonly leave many pages unreferenced.  Therefore, if
% physical reads can be delayed they can frequently be dispensed with,
% improving system performance.  To obtain this efficiency in practice,
% executable and shared object files must have segment images of which
% the offsets and virtual addresses are congruent modulo the page size.

% Virtual addresses and file offsets for the \ARCH{} processor family
% segments are congruent modulo {\ifzseries the system page size\else
%   4$\,$Kbytes\fi}.  The value of the \texttt{p\_align} field of each
% program header in a shared object file must be {\ifzseries a multiple
%   of the system page size\else \texttt{0x1000} (4$\,$Kbytes)\fi}.
% \Cref{fig:execfile} is an example of an executable file assuming an
% executable program linked with a base address of {\ifzseries
%   \texttt{0x80000000} (2$\,$Gbytes)\else \texttt{0x00400000}
%   (4$\,$Mbytes)\fi}.

% \begin{figure}
%   \centering
%   \ifSkipTikZ
% \begin{verbatim}
% File Offset                               Virtual Address
%          0 +----------------------------+ 0x80000000
%            |         ELF header         |
%            |    Program header table    |
%            |     Other information      |
%            |                            |
%            |         Text segment       |
%            |           . . .            |
%            |        0x1bf58 bytes       |
%            |                            | 0x8001bfff
%            +----------------------------+
%    0x1bf58 |                            | 0x8001cf58
%            |        Data segment        |
%            |           . . .            |
%            |        0x17c4 bytes        |
%            |                            | 0x8001d72b
%            +----------------------------+
%    0x1d71c |     Other Information      |
%            +----------------------------+
% \end{verbatim}
%   \else
%   \begin{tikzpicture}
%     \matrix [memory layout,inner sep=0pt,
%     nodes={align=center, inner sep=0.7ex}] (m) {
%       \node (A1) {ELF header}; \\
%       \node (A2) {Program header table}; \\
%       \node (A3) {Other information}; \\
%       \node (A4) {\stackit[c]{\noalign{\medskip}
%           Text segment\\ \ldots\\\noalign{\medskip}}}; \\
%       \node (A5) {\texttt{0x1bf58} bytes}; \\
%       \node (B1) {\stackit[c]{Data segment\\ \ldots\\
%           \noalign{\medskip}}}; \\
%       \node (B2) {\texttt{0x17c4} bytes}; \\
%       \node (C1) {Other information\strut}; \\
%     };
%     \foreach \Node in {A5, B2} {
%       \draw (\Node.south -| m.west) -- (\Node.south -| m.east);
%     }
%     \path (m.north west) + (-1ex,0.5ex) node [above left]
%     (foffs) {File Offset};
%     \path (m.north east) + (1ex,0.5ex) node [above right]
%     (vaddr) {Virtual Address};
%     \foreach \Node/\text in {A1/0, B1/0x1bf58, C1/0x1d71c} {
%       \path (\Node.north west -| foffs.east) node [below left]
%       {\texttt{\text}};
%     }
%     \ifzseries
%     \foreach \Node/\text in {A1/0x80000000, B1/0x8001cf58} {
%       \path (\Node.north east -| vaddr.west) node [below right]
%       {\texttt{\text}};
%     }
%     \foreach \Node/\text in {A5/0x8001bfff, B2/0x8001e71b} {
%       \path (\Node.south east -| vaddr.west) node [above right]
%       {\texttt{\text}};
%     }
%     \else
%     \foreach \Node/\text in {A1/0x00400000, B1/0x0041cf58} {
%       \path (\Node.north east -| vaddr.west) node [below right]
%       {\texttt{\text}};
%     }
%     \foreach \Node/\text in {A5/0x0041bfff, B2/0x0041e71b} {
%       \path (\Node.south east -| vaddr.west) node [above right]
%       {\texttt{\text}};
%     }
%     \fi
%   \end{tikzpicture}
%   \fi
%   \caption{Executable file example}
%   \label{fig:execfile}
% \end{figure}

% \begin{table}
%   \centering
%   \begin{DIFnomarkup}
%   \begin{tabular}{lll}
%     \toprule
%     Member & Text & Data \\
%     \midrule
%     \texttt{p\_type} & \texttt{PT\_LOAD} & \texttt{PT\_LOAD} \\
%     \texttt{p\_offset} & \texttt{0x0} & \texttt{0x1bf58} \\
%     \texttt{p\_vaddr} & \texttt{\ifzseries 0x80000000\else 0x400000\fi}
%     & \texttt{\ifzseries 0x8001cf58\else 0x41cf58\fi} \\
%     \texttt{p\_paddr} & unspecified & unspecified \\
%     \texttt{p\_filesz} & \texttt{0x1bf58} & \texttt{0x17c4} \\
%     \texttt{p\_memsz} & \texttt{0x1bf58} & \texttt{0x2578} \\
%     \texttt{p\_flags} & \texttt{PF\_R+PF\_X} & \texttt{PF\_R+PF\_W} \\
%     \texttt{p\_align} & \texttt{0x1000} & \texttt{0x1000} \\
%     \bottomrule
%   \end{tabular}
%   \end{DIFnomarkup}
%   \caption{Program header segments}
%   \label{tab:phdr}
% \end{table}

% Although the file offsets and virtual addresses are congruent modulo
% 4$\,$Kbytes for both text and data, up to four file pages can hold
% impure text or data (depending on page size and file system block
% size).

% \begin{itemize}
% \item The first text page contains the ELF header, the program header
%   table, and other information.
% \item The last text page may hold a copy of the beginning of data.
% \item The first data page may have a copy of the end of text.
% \item The last data page may contain file information not relevant to
%   the running process.
% \end{itemize}

% Logically, the system enforces memory permissions as if each segment
% were complete and separate; segment addresses are adjusted to ensure
% that each logical page in the address space has a single set of
% permissions.  In the example in \cref{tab:phdr} the file region
% holding the end of text and the beginning of data is mapped twice; at
% one virtual address for text and at a different virtual address for
% data.

% The end of the data segment requires special handling for
% uninitialized data, which the system defines to begin with zero
% values.  Thus if the last data page of a file includes information
% beyond the logical memory page, the extraneous data must be set to
% zero by the loader, rather than to the unknown contents of the
% executable file.  ``Impurities'' in the other three segments are not
% logically part of the process image, and whether the system clears
% them is unspecified.  The memory image for the program in
% \cref{tab:phdr} is presented in \cref{fig:pimgseg}.

% \begin{figure}
%   \centering
%   \ifSkipTikZ
% \begin{verbatim}
% Virtual Address                             Segment
%   0x80000000 +----------------------------+
%              |         ELF header         |
%              |    Program header table    |
%              |     Other information      |
%              |                            |  Text
%              |         Text segment       |
%              |           . . .            |
%              |        0x1bf58 bytes       |
%              +----------------------------+
%   0x8001bf58 |        Page padding        |
%              |         0xa8 bytes         |
%              +----------------------------+

%              +----------------------------+
%   0x8001c000 |          Padding           |
%              |         0xf58 bytes        |
%              +----------------------------+
%   0x8001cf58 |                            |
%              |        Data segment        |
%              |           . . .            |  Data
%              |        0x17c4 bytes        |
%              |                            |
%              +----------------------------+
%   0x8001e71c |     Uninitialized data     |
%              |        0xdb4 bytes         |
%              +----------------------------+
%   0x8001f4d0 |        Page padding        |
%              |        0xb30 bytes         |
%   0x8001ffff +----------------------------+
% \end{verbatim}
%   \else
%   \pgfsetlayers{background,main}
%   \begin{tikzpicture}
%     \matrix [inner sep=0pt,
%     nodes={align=center, inner sep=0.7ex}] (m) {
%       \node (A1) {ELF header}; \\
%       \node (A2) {Program header table}; \\
%       \node (A3) {Other information}; \\
%       \node (A4) {\stackit[c]{\noalign{\medskip} Text segment\\ \ldots}}; \\
%       \node (A5) {\texttt{0x1bf58} bytes}; \\
%       \node (B1) {Page padding}; \\
%       \node (B2) {\texttt{0xa8} bytes}; \\
%       \node (C1) {\strut}; \\
%       \node (D1) {Padding}; \\
%       \node (D2) {\texttt{0xf58} bytes}; \\
%       \node (E1) {\stackit[c]{Data segment\\ \ldots}}; \\
%       \node (E2) {\texttt{0x17c4} bytes}; \\
%       \node (F1) {Uninitialized data}; \\
%       \node (F2) {\texttt{0xdb4} bytes}; \\
%       \node (G1) {Page padding}; \\
%       \node (G2) {\texttt{0xb30} bytes}; \\
%     };
%     \begin{pgfonlayer}{background}
%       \path [memory layout] (B2.south west -| m.west)
%       rectangle (A1.north east -| m.east);
%       \path [memory layout] (G2.south west -| m.west)
%       rectangle (D1.north east -| m.east);
%     \end{pgfonlayer}
%     \foreach \Node in {A5, D2, E2, F2} {
%       \draw (\Node.south -| m.west) -- (\Node.south -| m.east);
%     }
%     \path (m.north west) + (-1ex,0.5ex) node [above left]
%     (vaddr) {Virtual Address};
%     \path (m.north east) + (1ex,0.5ex) node [above right]
%     (segm) {Segment};
%     \ifzseries
%     \foreach \Node/\text in {%
%       A1/0x80000000,
%       B1/0x8001bf58,
%       D1/0x8001c000,
%       E1/0x8001cf58,
%       F1/0x8001e71c,
%       G1/0x8001f4d0} {
%       \path (\Node.north -| vaddr.east) node [below left] {\texttt{\text}};
%     }
%     \path (m.south -| vaddr.east) node [above left] {\texttt{0x8001ffff}};
%     \else
%     \foreach \Node/\text in {%
%       A1/0x00400000,
%       B1/0x0041bf58,
%       D1/0x0041c000,
%       E1/0x0041cf58,
%       F1/0x0041e71c,
%       G1/0x0041f4d0} {
%       \path (\Node.north -| vaddr.east) node [below left] {\texttt{\text}};
%     }
%     \path (m.south -| vaddr.east) node [above left] {\texttt{0x0041ffff}};
%     \fi
%     \foreach \from/\to/\text in {%
%       A1/B2/Text, D1/G2/Data
%     } {
%       \path (\from.north -| segm.west) -- node [pos=0.5,right] {\text}
%       (\to.south -| segm.west);
%     }
%   \end{tikzpicture}
%   \fi
% \caption{Process image segments}
% \label{fig:pimgseg}
% \end{figure}

% One aspect of segment loading differs between executable files and
% shared objects.  Executable file segments may contain absolute code.
% For the process to execute correctly, the segments must reside at the
% virtual addresses assigned when building the executable file, with the
% system using the \texttt{p\_vaddr} values unchanged as virtual
% addresses.

% On the other hand, shared object segments typically contain
% position-independent code.  This allows a segment's virtual address to
% change from one process to another, without invalidating execution
% behavior.  Though the system chooses virtual addresses for individual
% processes, it maintains the ``relative positions'' of the
% segments.  Because position-independent code uses relative addressing
% between segments, the difference between virtual addresses in memory
% must match the difference between virtual addresses in the
% file.  \Cref{tab:soseg} shows possible shared object virtual
% address assignments for several processes, illustrating constant
% relative positioning.  The table also illustrates the base address
% computations.

% \begin{table}
%   \centering
%   \begin{DIFnomarkup}
%   \begin{tabular}{llll}
%     \toprule
%     Source & Text & Data & Base Address \\
%     \midrule
%     \ifzseries
%     File & \texttt{0x00000000000} & \texttt{0x0000002a400} & \\
%     Process 1 & \texttt{0x20000000000} & \texttt{0x2000002a400} &
%     \texttt{0x20000000000} \\
%     Process 2 & \texttt{0x20000010000} & \texttt{0x2000003a400} &
%     \texttt{0x20000010000} \\
%     Process 3 & \texttt{0x20000020000} & \texttt{0x2000004a400} &
%     \texttt{0x20000020000} \\
%     Process 4 & \texttt{0x20000030000} & \texttt{0x2000005a400} &
%     \texttt{0x20000030000} \\
%     \else
%     File & \texttt{0x00000200} & \texttt{0x0002a400} & \\
%     Process 1 & \texttt{0x40000000} & \texttt{0x4002a400} &
%     \texttt{0x40000000} \\
%     Process 2 & \texttt{0x40010000} & \texttt{0x4003a400} &
%     \texttt{0x40010000} \\
%     Process 3 & \texttt{0x40020000} & \texttt{0x4004a400} &
%     \texttt{0x40020000} \\
%     Process 4 & \texttt{0x40030000} & \texttt{0x4005a400} &
%     \texttt{0x40030000} \\
%     \fi
%     \bottomrule
%   \end{tabular}
%   \end{DIFnomarkup}
%   \caption{Shared object segment example\ifzseries{} for 42-bit
%     address space\fi}
%   \label{tab:soseg}
% \end{table}

% \section{Dynamic Linking}
% \label{dynamiclinking}
% \index{dynamic linking}
% \subsection{Dynamic Section}
% Dynamic section entries give information to the dynamic linker.  Some
% of this information is processor-specific, including the
% interpretation of some entries in the dynamic structure.

% \begin{description}
% \item[\texttt{DT\_PLTGOT}] The \texttt{d\_ptr} field of this entry gives
%   the address of the first byte in the Global Offset Table.  See
%   \cref{globaloffsettable} for more information.

% \item[\texttt{DT\_JMPREL}] This entry is associated with a table of
%   relocation entries for the PLT\@.  For \ABINAME{} this entry is
%   mandatory both for executable and shared object files.  Moreover,
%   the relocation table's entries must have a one-to-one correspondence
%   with the PLT\@.  The table of \texttt{DT\_JMPREL} relocation entries
%   is wholly contained within the \texttt{DT\_RELA} referenced table.
%   See \cref{procedurelinkagetable} for more information.
% \end{description}

% \subsection{Global Offset Table}
% \label{globaloffsettable}
% \index{global offset table}
% \index{GOT}
% Position-independent code cannot, in general, contain absolute virtual
% addresses.  Global Offset Tables hold absolute addresses in private
% data, thus making the addresses available without compromising the
% position-independence and sharability of a program's text.  A program
% references its GOT using position-independent addressing and extracts
% absolute values, thus redirecting position-independent references to
% absolute locations.

% When the dynamic linker creates memory segments for a loadable object
% file, it processes the relocation entries, some of which will be of
% type \texttt{R\_390\_GLOB\_DAT}, referring to the GOT\@.  The dynamic
% linker determines the associated symbol values, calculates their
% absolute addresses, and sets the GOT entries to the proper
% values.  Although the absolute addresses are unknown when the linkage
% editor builds an object file, the dynamic linker knows the addresses
% of all memory segments and can thus calculate the absolute addresses
% of the symbols contained therein.

% A GOT entry provides direct access to the absolute address of a symbol
% without compromising position-independence and sharability.  Because
% the executable file and shared objects have separate GOTs, a symbol
% may appear in several tables.  The dynamic linker processes all the
% GOT relocations before giving control to any code in the process
% image, thus ensuring the absolute addresses are available during
% execution.

% The dynamic linker may choose different memory segment addresses for
% the same shared object in different programs; it may even choose
% different library addresses for different executions of the same
% program.  Nevertheless, memory segments do not change addresses once
% the process image is established.  As long as a process exists, its
% memory segments reside at fixed virtual addresses.

% The format and interpretation of the Global Offset Table is processor
% specific.  For \ABINAME{} the symbol \texttt{\_GLOBAL\_OFFSET\_TABLE\_}
% may be used to access the table.  The symbol refers to the start of
% the \texttt{.got} section.  Two words in the GOT are reserved:
% \begin{itemize}
% \item The word at \texttt{\_GLOBAL\_OFFSET\_TABLE\_[0]} is set by the
%   linkage editor to hold the address of the dynamic structure,
%   referenced with the symbol \texttt{\_DYNAMIC}.  This allows a
%   program, such as the dynamic linker, to find its own dynamic
%   structure without having yet processed its relocation entries.  This
%   is especially important for the dynamic linker, because it must
%   initialize itself without relying on other programs to relocate its
%   memory image.
% \item The word at \texttt{\_GLOBAL\_OFFSET\_TABLE\_[1]} is reserved
%   for future use.
% \end{itemize}

% The Global Offset Table resides in the ELF \texttt{.got} section.

% \subsection{Function Addresses}
% \label{functionaddresses}
% References to a function address from an executable file and from the
% shared objects associated with the file must resolve to the same
% value.  References from within shared objects will normally be resolved
% (by the dynamic linker) to the virtual address of the function itself.
% References from within the executable file to a function defined in a
% shared object will normally be resolved (by the linkage editor) to the
% address of the Procedure Linkage Table entry for that function within
% the executable file.

% To allow comparisons of function addresses to work as expected, if an
% executable file references a function defined in a shared object, the
% linkage editor will place the address of the PLT entry for that
% function in its associated symbol table entry.  See
% \cref{symbolvalues} for details.  The dynamic linker treats
% such symbol table entries specially.  If the dynamic linker is
% searching for a symbol and encounters a symbol table entry for that
% symbol in the executable file, it normally follows these rules:

% \begin{itemize}
% \item If the \texttt{st\_shndx} field of the symbol table entry is not
%   \texttt{SHN\_UNDEF}, the dynamic linker has found a definition for
%   the symbol and uses its \texttt{st\_value} field as the symbol's
%   address.
% \item If the \texttt{st\_shndx} field is \texttt{SHN\_UNDEF} and the
%   symbol is of type \texttt{STT\_FUNC} and the \texttt{st\_value}
%   field is not zero, the dynamic linker recognizes this entry as
%   special and uses the \texttt{st\_value} field as the symbol's
%   address.
% \item Otherwise, the dynamic linker considers the symbol to be
%   undefined within the executable file and continues processing.
% \end{itemize}

% Some relocations are associated with PLT entries.  These entries are
% used for direct function calls rather than for references to function
% addresses.  These relocations are not treated specially as described
% above because the dynamic linker must not redirect PLT entries to
% point to themselves.

% \subsection{Procedure Linkage Table}
% \label{procedurelinkagetable}
% \index{procedure linkage table}
% \index{PLT}
% Much as the Global Offset Table redirects position-independent address
% calculations to absolute locations, the Procedure Linkage Table
% redirects position-independent function calls to absolute
% locations.  The linkage editor cannot resolve execution transfers (such
% as function calls) from one executable or shared object to another, so
% instead it arranges for the program to transfer control to entries in
% the PLT\@.  The dynamic linker determines the absolute addresses of the
% destinations and stores them in the GOT, from which they are loaded by
% the PLT entry.  The dynamic linker can thus redirect the entries
% without compromising the position-independence and sharability of the
% program text.  Executable files and shared object files have separate
% PLTs.

% As mentioned above, a relocation table is associated with the PLT\@.
% The \texttt{DT\_JMPREL} entry in the \texttt{\_DYNAMIC} array gives
% the location of the first relocation entry.  The relocation table
% entries match the PLT entries in a one-to-one correspondence
% (relocation table entry 1 applies to PLT entry 1 and so on).  The
% relocation type for each entry shall be
% \texttt{R\_390\_JMP\_SLOT}.  The relocation offset shall specify the
% address of the GOT entry containing the address of the function, and
% the symbol table index shall reference the appropriate symbol.

% To illustrate Procedure Linkage Tables, \cref{lst:pltex} shows how
% the linkage editor might initialize the PLT when linking a shared
% executable or shared object.

% % FIXME below: the layout of the listing below is difficult to understand:
% % Two different PLT entries in one listing; untypical assembler source
% % style, registers specified as mere numbers, use of general mnemonics
% % instead of extended ones, etc.
% %
% % Also:
% %
% % * "L" -> "LG".  This seems like a cut & paste error from the ESA/390
% %   ABI.
% %
% % * Listing should better be split before second asterisk.
% %
% % * Last comment: "Offset into symbol table" -> "offset into .rela.plt".

% \ifzseries
% \begin{lstlisting}[style=float,language=simpleasm,
%   caption=Procedure Linkage Table example,label=lst:pltex]
% *                                  # PLT for executables (not
%                                    #   position-independent)
% PLT1      BASR  1,0                # Establish base
% BASE1     L     1,AGOTENT-BASE1(1) # Load address of the GOT entry
%           L     1,0(0,1)           # Load function address from the
%                                    #   GOT to r1
%           BCR   15,1               # Jump to address
% RET1      BASR  1,0                # Return from GOT first time
%                                    #   (lazy binding)
% BASE2     L     1,ASYMOFF-BASE2(1) # Load offset in symbol table to r1
%           BRC   15,-x              # Jump to start of PLT
%           .word 0                  # Filler
% AGOTENT   .long ?                  # Address of the GOT entry
% ASYMOFF   .long ?                  # Offset into the symbol table
% *                                  # PLT for shared objects
%                                    #   (position-independent)
% PLT1      LARL  1,<fn>@GOTENT      # Load address of GOT entry in r1
%           LG    1,0(1)             # Load function address from the
%                                    #   GOT to r1
%           BCR   15,1               # Jump to address
% RET1      BASR  1,0                # Return from GOT first time
%                                    #   (lazy binding)
% BASE2     LGF   1,ASYMOFF-BASE2(1) # Load offset in symbol table to r1
%           BRCL  15,-x              # Jump to start of PLT
% ASYMOFF   .long ?                  # Offset into symbol table
% \end{lstlisting}
% \else
% \begin{lstlisting}[style=float,language=simpleasm,
%   caption=Procedure Linkage Table example,label=lst:pltex]
% *                                  # PLT for executables (not
%                                    #   position-independent)
% PLT1      BASR  1,0                # Establish base
% BASE1     L     1,AGOTENT-BASE1(1) # Load address of the GOT entry
%           L     1,0(0,1)           # Load function address from the
%                                    #   GOT to r1
%           BCR   15,1               # Jump to address
% RET1      BASR  1,0                # Return from GOT first time
%                                    #   (lazy binding)
% BASE2     L     1,ASYMOFF-BASE2(1) # Load offset in symbol table to r1
%           BRC   15,-x              # Jump to start of PLT
%           .word 0                  # Filler
% AGOTENT   .long ?                  # Address of the GOT entry
% ASYMOFF   .long ?                  # Offset into the symbol table
% *                                  # PLT for shared objects
%                                    #   (position-independent)
% PLT1      BASR  1,0                # Establish base
% BASE1     L     1,AGOTOFF-BASE1(1) # Load offset into the GOT to r1
%           L     1,(1,12)           # Load address from the GOT to r1
%           BCR   15,1               # Jump to address
% RET1      BASR  1,0                # Return from GOT first time
%                                    #   (lazy binding)
% BASE2     L     1,ASYMOFF-BASE2(1) # Load offset in symbol table to r1
%           BRC   15,-x              # Jump to start of PLT
%           .word 0                  # Filler
% AGOTOFF   .long ?                  # Offset in the GOT
% ASYMOFF   .long ?                  # Offset in the symbol table
% \end{lstlisting}
% \fi

% As described below, the dynamic linker and the program cooperate to
% resolve symbolic references through the PLT\@.  Again, the details
% described below are for explanation only.  The precise execution-time
% behavior of the dynamic linker is not specified.

% \begin{enumerate}
% \item The caller of a function in a different shared object transfers
%   control to the start of the PLT entry associated with the function.
% \item The first part of the PLT entry loads the address from the GOT
%   entry associated with the function to be called.  Control is
%   transferred to the code referenced by the address.  If the function
%   has already been called at least once, or if lazy binding is not used,
%   then the address found in the GOT is the address of the function.
% \item If a function has never been called and lazy binding is used,
%   the address in the GOT points to the second half of the PLT\@.
%   The second half loads the offset in the symbol table associated with
%   the called function.  Control is then transferred to the special
%   first entry of the PLT\@.
% \item This first entry of the PLT entry (see \cref{lst:plt0ex})
%   calls the dynamic linker, giving it the offset into the symbol table
%   % FIXME: symbol table -> .rela.plt
%   and the address of a structure that identifies the location of the
%   caller.
% \item The dynamic linker finds the real address of the symbol.  It
%   will store this address in the GOT entry of the function in the
%   object code of the caller and it will then transfer control to the
%   function.
% \item Subsequent calls to the function from this object will find the
%   resolved address in the first half of the PLT entry and will
%   transfer control directly without invoking the dynamic linker.
% \end{enumerate}

% \ifzseries
% % FIXME bad 32-bit operations, should be 64-bit
% \begin{lstlisting}[style=float,language=simpleasm,label=lst:plt0ex,
%   caption=Special first entry in Procedure Linkage Table]
% *                               # PLT0 for static object (not
%                                 #   position-independent)
% PLT0      ST    1,28(15)        # R1 has offset into symbol table
%           BASR  1,0             # Establish base
% BASE1     L     1,AGOT-BASE1(1) # Get address of GOT
%           MVC   24(4,15),4(1)   # Move loader info to stack
%           L     1,8(1)          # Get address of loader
%           BR    1               # Jump to loader
%           .word 0               # Filler
% AGOT      .long got             # Address of GOT

%                                 # PLT0 for shared object
%                                 #   (position-independent)
% PLT0      STG   1,56(15)        # R1 has offset into symbol table
%           LARL  1,_GLOBAL_OFFSET_TABLE_
%           MVC   48(8,15),8(1)   # move loader info (object struct
%                                 #   address) to stack
%           LG    1,16(12)        # Entry address of loader in R1
%           BCR   15,1            # Jump to loader
% \end{lstlisting}
% \else
% \begin{lstlisting}[style=float,language=simpleasm,label=lst:plt0ex,
%   caption=Special first entry in Procedure Linkage Table]
% *                               # PLT0 for static object (not
%                                 #   position-independent)
% PLT0      ST    1,28(15)        # R1 has offset into symbol table
%           BASR  1,0             # Establish base
% BASE1     L     1,AGOT-BASE1(1) # Get address of GOT
%           MVC   24(4,15),4(1)   # Move loader info to stack
%           L     1,8(1)          # Get address of loader
%           BR    1               # Jump to loader
%           .word 0               # Filler
% AGOT      .long got             # Address of GOT

%                                 # PLT0 for shared object
%                                 #   (position-independent)
% PLT0      ST    1,28(15)        # R1 has offset into symbol table
%           L     1,4(12)         # Get loader info (object struct
%                                 #   address)
%           ST    1,24(15)        # Store address
%           L     1,8(12)         # Entry address of loader in R1
%           BR    1               # Jump to loader
% \end{lstlisting}
% \fi

% The \texttt{LD\_BIND\_NOW} environment variable can change dynamic
% linking behavior.  If set to a nonempty string, the dynamic linker
% resolves the function call binding at load time, before transferring
% control to the program.  In other words, the dynamic linker processes
% relocation entries of type \texttt{R\_390\_JMP\_SLOT} during process
% initialization.  If \texttt{LD\_BIND\_NOW} is not set, the dynamic
% linker evaluates PLT entries lazily, delaying symbol resolution and
% relocation until the first execution of a table entry.

% \paragraph{Note:}
% Lazy binding generally improves overall application performance
% because unused symbols do not incur the overhead of dynamic
% linking.  Nevertheless, two situations make lazy binding undesirable
% for some applications:
% \begin{enumerate}
% \item The initial reference to a shared object function takes longer
%   than subsequent calls because the dynamic linker intercepts the call
%   to resolve the symbol, and some applications cannot tolerate this
%   unpredictability.
% \item If an error occurs and the dynamic linker cannot resolve the
%   symbol, the dynamic linker will terminate the program.  Under lazy
%   binding, this might occur at arbitrary times.  Once again, some
%   applications cannot tolerate this unpredictability.  By turning off
%   lazy binding, the dynamic linker forces the failure to occur during
%   process initialization, before the application receives control.
% \end{enumerate}


\chapter{Logical Grammar Specification}
\section{Grammar Specification}

\emph{Page Status: This page is ready for review for Phase 0 (PoC). It may have errors
and be changed based on feedback and implementation experience}

\subsection{ANSI Standard}

The current REXX ANSI specification describes the grammar in terms of
implementation details, specifically pulling out details across the interactions
between a stateful lexer and parser (although it does not describe it in these
terms). And although not explicitly stated, it is possible that the parser is assumed
to be a LALR one.

The difficulty with this approach is that the specification logic is split up and
therefore hard to interpret or indeed validate. Also it is often not clear where
certain rules are:

\begin{itemize}
\item Essential to the REXX Grammar being described

\item Needed to remove ambiguities in the grammar inherent in REXX itself

\item Needed to overcome weaknesses in the the LALR parser (i.e. with one lookahead)

\end{itemize}

\subsection{Parsing Expression Grammar (PEG)}

We want a format that describes the grammar in an implementation
independent fashion that allows comprehension for a REXX programmer.

We at the logical level we will use \href{https://en.wikipedia.org/wiki/Parsing_expression_grammar}{PEG}
grammar to describe REXX syntax, and (for REXX Level C - Classic REXX) convert the
ANSI specification to this format.

For readability, we will assume that the Parser / Grammar can support Left
Recursion.

For Phases 0-2 we will specify (at the Physical Level) how we implement these
grammars in the 3 stages - lexer, parsers, and finally AST Manipulation.

For Phase 3 we intend to use a PEG parser delivered in REXX Level L directly;
likely a \href{https://github.com/lukehutch/pikaparser}{PIKA Parser}. We are therefore
using the PEG format defined/inspired by its reference implementation.

\subsection{Parser Errors}

For all cREXX Phases and REXX levels - parsing errors will be embedded into the
AST tree, and it is expected that a AST Manipulation stage (even for phase 3)
will be needed to ensure that error reporting is useful and standards compliant.

\subsection{REXX PEG Format}

This extends \href{https://en.wikipedia.org/wiki/Parsing_expression_grammar}{the PEG format} to concisely support REXX constructs, and is inspired by the
\href{https://github.com/lukehutch/pikaparser}{PIKA Parser Reference implementation on Github}.

Currently this is only supported by \textquotedbl{}human interpretation\textquotedbl{}. In time it is envisaged that this format be developed and used in a REXX Level L PARSE Instruction to allow language parsing and AST tree generation.

\subsubsection{Whitespace and Comments}

An issue with the PEG Format is the treatment of non-program tokens (i.e. Whitespace and Comments): either they are assumed to have been stripped out (in which case the grammar cannot make use of this information; recalling that REXX uses abuttals as an operator), or they have to be referenced throughout the grammar rules (very messy and not supportive of the common approach of implementing with a separate lexing stage).

Our format defines two master rules:

\begin{itemize}
\item \texttt{\textbackslash{}WHITESPACE} - to indicate whitespace, these are discarded

\item \texttt{\textbackslash{}COMMENT} - to indicate comment tokens. These tokens can be considered to have been created but hidden. For human interpretation not different to /ws but at a later date this difference could be used, as an example, for language transformation applications.

\end{itemize}

In addition, the \texttt{\&\textbar{}} notation for (\texttt{Abuttal}) - see below - allows languages to detect when there was no whitespace (or comments) between two tokens.

String (\textquotedbl{} or \textquotesingle{}) of characters and rule names that begin with a Capital letter are treated a tokens (see below) in this case white space and comments are NOT ignored.

\subsubsection{Rule Format}

The rules are of the form \texttt{RuleName \textless{}- Clause;}.

If a rule name starts with a capital (recommendation: capitalise the whole rule name) then it is considered to be a TOKEN. This just means that whitespaces and comments are included in the matching / token rather than being suppressed.

AST node labels may be specified in the form \texttt{RuleName \textless{}- ASTNodeLabel:Clause;}.

Sub-Rules processed in the order specified by the parent rule to get rid of ambiguities

\emph{Deprecated: The rule name may be followed by optional square brackets containing the precedence of the
rule (as an integer), optionally followed by an associativity modifier (\texttt{,L} or \texttt{,R}).}

Nonterminal clauses can be specified using the following notation:

\begin{itemize}
\item \texttt{X Y Z} for a sequence of matches (\texttt{X} should match, followed by \texttt{Y},
followed by \texttt{Z}), i.e. \texttt{Seq}

\item \texttt{X / Y / Z} for ordered choice (\texttt{X} should match, or if it doesn\textquotesingle{}t, \texttt{Y}
should match, or if it doesn\textquotesingle{}t\textquotesingle{} \texttt{Z} should match) , i.e. \texttt{First}

\item \texttt{X+} to indicate that \texttt{X} must match one or more times, i.e. \texttt{OneOrMore}

\item \texttt{X*} to indicate that \texttt{X} must match zero or more times, i.e. \texttt{ZeroOrMore} - nongreedy

\item \texttt{X**} to indicate that \texttt{X} must match zero or more times, i.e. \texttt{ZeroOrMore} - greedy

\item \texttt{X?} to indicate that \texttt{X} may optionally match, i.e. \texttt{Optional}

\item \texttt{\&X} to look ahead and require \texttt{X} to match without consuming characters, i.e. \texttt{FollowedBy}

\item \texttt{!X} to look ahead and require that there is no match (the logical negation of \texttt{\&X}), i.e. \texttt{NotFollowedBy}

\item \texttt{\&\textbar{}} to look ahead (without consuming characters) and require that there is no whitespace before the next token, i.e. \texttt{Abuttal}

\end{itemize}

Terminal clauses can be specified using the following notation. Standard
character escaping is supported, including for Unicode codepoints.

\begin{itemize}
\item \texttt{\textquotesingle{}{[}\textquotesingle{}} for individual characters

\item \texttt{\textquotedbl{}import\textquotedbl{}} for strings of characters (case sensitive)

\item \texttt{\textquotesingle{}import\textquotesingle{}} for strings of characters (case insensitive)

\item \texttt{{[}0-9{]}} for character ranges

\item \texttt{{[}+\textbackslash{}-*/{]}} for character sets (note \texttt{-} is escaped)

\item \texttt{{[}\^{}\textbackslash{}n{]}} for negated character sets (note that this will slow down the parser,
since a negated matching rule will spuriously match in many more places)

\item \texttt{\textbackslash{}} = Escape character

\item \texttt{.} = wildcard (matches any single character)

\end{itemize}

Logic can be used between clauses

\begin{itemize}
\item \texttt{//} = logical or

\item \texttt{\&\&} = logical and

\item \texttt{\^{}} = not

\end{itemize}

Unordered Sets

\begin{itemize}
\item \texttt{\{a b c\}} - Unordered set 1 of each member in any order

\item \texttt{\{+ a b c\}} - Unordered set 1 or more of each member in any order

\item \texttt{\{* a b c\}} - Unordered set 0 or more of each member in any order

\item \texttt{\{n* a b c\}} - Unordered set n or more of each member in any order

\end{itemize}

\subsubsection{Special Rules}

\begin{itemize}
\item \texttt{\textbackslash{}eof} - Platform specific detection of EOF (or End of Stream)

\item \texttt{\textbackslash{}eol} - Platform specific detection of EOL

\end{itemize}

\subsubsection{AST Generation}

\begin{itemize}
\item Each rule returns a set of ordered nodes (which can be empty)

\item \texttt{rule \textless{}- a:subrule b:subrule -\textgreater{} *} Any sub-rules AST nodes are are kept as a set of sibling nodes. Equivalent to:

\item \texttt{rule \textless{}- a:subrule b:subrule} No AST Node is created, any subrules AST nodes are are kept as a set of sibling nodes (for the use of parent rules)

\item \texttt{... -\textgreater{} TYPE} Creates an AST node of type TYPE made up of the complete rule selection, any subrules AST nodes are added as children. Equivalent to:

\item \texttt{rule \textless{}- a:subrule b:subrule -\textgreater{} (TYPE *)} AST Tree with node type TYPE as root and nodes as children == -\textgreater{} TYPE

\item \texttt{rule \textless{}- subrule subrule -\textgreater{} TYPE *} AST Tree with node type TYPE as root and any nodes as siblings following TYPE

\item \texttt{rule \textless{}- a:subrule b:subrule -\textgreater{} TYPE a b} AST Tree with node type TYPE as root and a and b nodes as siblings following TYPE

\item \texttt{rule \textless{}- a:subrule b:subrule -\textgreater{} a b} AST Tree with nodes a and b as siblings

\item \texttt{... -\textgreater{} TYPE{[}parm{]}} As \texttt{-\textgreater{} TYPE} but with a parameter stored in the node

\item \texttt{rule \textless{}- a:subrule b:subrule c:subrule ... -\textgreater{} (a b c ...)} AST Tree with rule label a as root and b and c (etc.) nodes as children (b and can be null in which case they are \textquotesingle{}\textbackslash{}\textquotesingle{} added.

\item \texttt{rule \textless{}- a:subrule b:subrule* -\textgreater{} (a b)} AST Tree with rule label a as root and many children from subrule b

\item \texttt{rule \textless{}- a:subrule b:subrule -\textgreater{} (TYPE a b)} AST Tree with node type TYPE as root and a and b nodes as children

\item \texttt{rule \textless{}- a:subrule b:subrule c:subrule -\textgreater{} (TYPE (a b) c)} AST Tree with node type TYPE as root and a and c nodes as children b as a grandchild

\item \texttt{rule \textless{}- subrule -\textgreater{} TYPE1 / subrule -\textgreater{} TYPE1} Shows how different subrules may lead to a different AST node

\end{itemize}


\chapter{\crexx{} level C - Classic}
\crexx{} Level C \index{\crexx{} level C}corresponds to The Classic Rexx language as
implemented on z/VM and ANSI/Incits Standard J318. It is implemented
in \crexx{} level B.

\input{../Level-C-Grammar.tex}

\section{\crexx{} level C Statements}
% \subsubsection{Do instruction}\label{refdo}
\index{Instructions,DO}
\index{Flow control,with DO construct}
\index{Group, DO,}
\index{DO group,}
\index{Simple DO group,}
\begin{shaded}
\begin{alltt}
\textbf{do} [\textbf{label} \emph{name}] [\textbf{protect} \emph{term}] [\textbf{binary]};
        \emph{instructionlist}
    [\textbf{catch} [\emph{vare} =] \emph{exception};
        \emph{instructionlist}]...
    [\textbf{finally}[;]
        \emph{instructionlist}]
    \textbf{end} [\emph{name}];

where \emph{name} is a non-numeric \emph{symbol}

and \emph{instructionlist} is zero or more \emph{instruction}s
\end{alltt}
\end{shaded}
 The \keyword{do} instruction is used to group instructions together for
execution; these are executed once.
The group may optionally be given a label, and may protect an object
while the instructions in the group are executed; exceptional conditions
can be handled with \keyword{catch} and \keyword{finally}.
 
The most common use of \keyword{do} is simply for treating a number of
instructions as group.

\textbf{Example:}
\begin{lstlisting}
/* The two instructions between DO and END will both */
/* be executed if A has the value 3.                 */
if a=3 then do
  a=a+2
  say 'Smile!'
  end
\end{lstlisting}
\index{Body,of group}
Here, only the first \emph{instructionlist} is used.
This forms the \emph{body} of the group.
 
The instructions in the \emph{instructionlist}s may be any assignment,
method call, or keyword instruction, including any of the more complex
constructions such as \keyword{loop}, \keyword{if}, \keyword{select}, and
the \keyword{do} instruction itself.
\subsection{Label phrase}
\index{DO instruction,LABEL}
\index{DO group,naming of}
\index{LABEL,on DO instruction}
 
If \keyword{label} is used to specify a \emph{name} for the group,
then a \keyword{leave} which specifies that name may be used to leave the
group, and the \keyword{end} that ends the group may optionally specify
the name of the group for additional checking.

\textbf{Example:}
\begin{lstlisting}
do label sticky
  x=ask
  if x='quit' then leave sticky
  say 'x was' x
  end sticky
\end{lstlisting}
\subsection{Protect phrase}
\index{PROTECT,on DO instruction}
 
If \keyword{protect} is given it must be followed by a \emph{term}
that evaluates to a value that is not just a type and is not of a
primitive type; while the \keyword{do} construct is being executed, the
value (object) is protected - that is, all the instructions in the
\keyword{do} construct have exclusive access to the object.
 
Both \keyword{label} and \keyword{protect} may be specified, in any order,
if required.
\subsection{Exceptions in do groups}
 
\index{CATCH,on DO instruction}
\index{FINALLY,on DO instruction}
Exceptions that are raised by the instructions within a do group may be
caught using one or more \keyword{catch} clauses that name the
\emph{exception} that they will catch.
When an exception is caught, the exception object that holds the details
of the exception may optionally be assigned to a variable,
\emph{vare}.
 
Similarly, a \keyword{finally} clause may be used to introduce
instructions that will always be executed at the end of the group, even
if an exception is raised (whether caught or not).
 
%% The  \emph{Exceptions} section (see page \pageref{refexcep})  has details and
%% examples of \keyword{catch} and \keyword{finally}.

%% \subsection{Binary}
%% A group of one or more statements in a do binary group will
%% follow the semantics of binary statements in binary classes or
%% methods; \marginnote{\color{gray}3.04}the scope is limited to the do binary group.

\subsubsection{Exit instruction}\label{refexit}
\index{EXIT instruction,}
\index{Instructions,EXIT}
\index{Return code, setting on exit,}
\index{Return string, setting on exit,}
\begin{shaded}
\begin{alltt}
\textbf{exit} [\emph{expression}];
\end{alltt}
\end{shaded}
 \keyword{exit} is used to unconditionally leave a program, and
optionally return a result to the caller.
The entire program is terminated immediately.

If an \emph{expression} is given, it is evaluated and the result
of the evaluation is then passed back to the caller in an
implementation-dependent manner when the program terminates.
Typically this value is expected to be a small whole number; most
implementations will accept values in the range 0 through 250.
If no expression is given, a default result (which depends on the
implementation, and is typically zero) is passed back to the caller.

\textbf{Example:}
\begin{lstlisting}
j=3
exit j*4
/* Would exit with the value '12' */
\end{lstlisting}
\index{Running off the end of a program,}
\index{Bottom of program, reaching during execution,}
 "Running off the end" of a program is equivalent to the
instruction \textbf{return;}.  In the case where the program is simply
a stand-alone application with no \keyword{class} or \keyword{method}
instructions, this has the same effect as \textbf{exit;}, in that it
terminates the whole program and returns a default result.

\subsubsection{If instruction}
\index{IF instruction,}
\index{Instructions,IF}
\index{THEN,following IF clause}
\index{,}
\index{Flow control,with IF construct}
\begin{shaded}
\begin{alltt}
\textbf{if} \emph{expression}[;]
     \textbf{then}[;] \emph{instruction}
    [\textbf{else}[;] \emph{instruction}]
\end{alltt}
\end{shaded}
 The \keyword{if} construct is used to conditionally execute an
instruction or group of instructions.
It can also be used to select between two alternatives.
 The expression is evaluated and must result in either 0 or 1.
If the result was 1 (true) then the instruction after the
\keyword{then} is executed.
If the result was 0 (false) and an \keyword{else} was given
then the instruction after the \keyword{else} is executed.
 \textbf{Example:}
\begin{lstlisting}
if answer='Yes' then say 'OK!'
                else say 'Why not?'
\end{lstlisting}
 Remember that if the \keyword{else} clause is on the same line as the
last clause of the \keyword{then} part, then you need a semicolon to
terminate that clause.
 \textbf{Example:}
\begin{lstlisting}
if answer='Yes' then say 'OK!';  else say 'Why not?'
\end{lstlisting}
 The \keyword{else} binds to the nearest \keyword{then} at the same level.
This means that any \keyword{if} that is used as the instruction
following the \keyword{then} in an \keyword{if} construct that has an
\keyword{else} clause, must itself have an \keyword{else} clause (which
may be followed by the dummy instruction, \keyword{nop}).
 \textbf{Example:}
\begin{lstlisting}
if answer='Yes' then if name='Fred' then say 'OK, Fred.'
                                    else say 'OK.'
                else say 'Why not?'
\end{lstlisting}
 
To include more than one instruction following \keyword{then} or
\keyword{else}, use a grouping instruction (\keyword{do}, \keyword{loop},
or \keyword{select}).
 \textbf{Example:}
\begin{lstlisting}
if answer='Yes' then do
  say 'Line one of two'
  say 'Line two of two'
  end
\end{lstlisting}
In this instance, both \keyword{say} instructions are executed when
the result of the \keyword{if} expression is 1.
 
Multiple expressions, separated by commas, can be given on the
\keyword{if} clause, which then has the syntax:
\begin{shaded}
\begin{alltt}
\textbf{if} \emph{expression}[, \emph{expression}]... [;]
\end{alltt}
\end{shaded}
In this case, the expressions are evaluated in turn from left to
right, and if the result of any evaluation is 1 then the test has
succeeded and the instruction following the associated \keyword{then}
clause is executed.
If all the expressions evaluate to 0 and an \keyword{else} was given
then the instruction after the \keyword{else} is executed.
 
Note that once an expression evaluation has resulted in 1, no further
expressions in the clause are evaluated.  So, for example, in:
\begin{lstlisting}
-- assume 'name' is a string
if name=null, name='' then say 'Empty'
\end{lstlisting}
then if \texttt{name} does not refer to an object it will compare equal to
null and the \keyword{say} instruction will be executed without
evaluating the second expression in the \keyword{if} clause.
\begin{shaded}\noindent
\textbf{Notes:}
\begin{enumerate}
\item An \emph{instruction} may be any assignment, method call, or
keyword instruction, including any of the more complex constructions
such as \keyword{do}, \keyword{loop}, \keyword{select}, and the \keyword{if}
instruction itself.
A null clause is not an instruction, however, so putting an extra
semicolon after the \keyword{then} or \keyword{else} is not equivalent to
putting a dummy instruction.
The \keyword{nop} instruction is provided for this purpose.
\item The keyword \keyword{then} is treated specially, in that it need not start a
clause.
This allows the expression on the \keyword{if} clause to be terminated by
the \keyword{then}, without a "\textbf{;}" being required -
were this not so, people used to other computer languages would
be inconvenienced.
Hence the symbol \keyword{then} cannot be used as a variable name within
the expression.
%% \footnote{
%% Strictly speaking, \keyword{then} should only be recognized if not
%% the name of a variable.  In this special case, however, \nr{} language
%% processors are permitted to treat \keyword{then} as reserved in the
%% context of an \keyword{if} clause, to provide better performance and
%% more useful error reporting.
%% }
\end{enumerate}
\end{shaded}\indent

% \input{nr3impor}
\subsubsection{Iterate instruction}
\index{ITERATE instruction,}
\index{Instructions,ITERATE}
\index{ITERATE instruction,use of variable on}
\index{,}
\index{Loops,modification of}
\begin{shaded}
\begin{alltt}
\textbf{iterate} [\emph{name}];

where \emph{name} is a non-numeric \emph{symbol}.
\end{alltt}
\end{shaded}
 \keyword{iterate} alters the flow of control within a \keyword{loop}
construct.
It may only be used in the body (the first \emph{instructionlist})
of the construct.

Execution of the instruction list stops, and control is passed
directly back up to the \keyword{loop} clause just as though the last
clause in the body of the construct had just been executed.
The control variable (if any) is then stepped (iterated) and termination
conditions tested as normal and the instruction list is executed again,
unless the loop is terminated by the \keyword{loop} clause.

If no \emph{name} is specified, then \keyword{iterate} will step
the innermost active loop.
 
If a \emph{name} is specified, then it must be the name of the
label, or control variable if there is no label, of a currently active
loop (which may be the innermost), and this is the loop that is
iterated.
Any active \keyword{do}, \keyword{loop}, or \keyword{select} constructs
inside the loop selected for iteration are terminated (as though by a
\keyword{leave} instruction).

\textbf{Example:}
\begin{lstlisting}
loop i=1 to 4
  if i=2 then iterate i
  say i
  end
/* Would display the numbers:  1, 3, 4  */
\end{lstlisting}
 \textbf{Notes:}
\begin{enumerate}
\index{Active constructs,}
\index{Loops,active}
\index{Names,on ITERATE instructions}
\item A loop is active if it is currently being executed.
If a method (even in the same class) is called during execution of a
loop, then the loop becomes inactive until the method has returned.
\keyword{iterate} cannot be used to step an inactive loop.
\item The \emph{name} symbol, if specified, must exactly match the
label (or the name of the control variable, if there is no label) in the
\keyword{loop} clause in all respects except case.
\end{enumerate}

\subsubsection{Leave instruction}\label{refleave}
\index{LEAVE instruction,}
\index{Instructions,LEAVE}
\index{LEAVE instruction,use of variable on}
\index{,}
\index{Loops,termination of}
\begin{shaded}
\begin{alltt}
\textbf{leave} [\emph{name}];

where \emph{name} is a non-numeric \emph{symbol}.
\end{alltt}
\end{shaded}
 \keyword{leave} causes immediate exit from one or more \keyword{do},
\keyword{loop}, or \keyword{select} constructs.
It may only be used in the body (the first \emph{instructionlist})
of the construct.
 
Execution of the instruction list is terminated, and control is
passed to the \keyword{end} clause of the construct, just as though the
last clause in the body of the construct had just been executed or (if
a loop) the termination condition had been met normally, except that on
exit the control variable (if any) will contain the value it had when
the \keyword{leave} instruction was executed.
 
If no \emph{name} is specified, then \keyword{leave} must be
within an active loop and will terminate the innermost active loop.
 
If a \emph{name} is specified, then it must be the name of the
label (or control variable for a loop with no label), of a currently
active \keyword{do}, \keyword{loop}, or \keyword{select} construct
(which may be the innermost).  That construct (and any active constructs
inside it) is then terminated.  Control then passes to the clause
following the \keyword{end} clause that matches the
\keyword{do}, \keyword{loop}, or \keyword{select} clause identified by the
\emph{name}.

\textbf{Example:}
\begin{lstlisting}
loop i=1 to 5
  say i
  if i=3 then leave
  end i
/* Would display the numbers:  1, 2, 3  */
\end{lstlisting}
 \textbf{Notes:}
\begin{enumerate}
\index{FINALLY,reached by LEAVE}
\item If any construct being left includes a \keyword{finally} clause, the
\emph{instructionlist} following the \keyword{finally} will be
executed before the construct is left.
\index{Active constructs,}
\index{Loops,active}
\index{Constructs,active}
\index{Names,on LEAVE instructions}
\item 
A \keyword{do}, \keyword{loop}, or \keyword{select} construct
is active if it is currently being executed.
If a method (even in the same class) is called during execution of an
active construct, then the construct becomes inactive until the method
has returned.
\keyword{leave} cannot be used to leave an inactive construct.
\item The \emph{name} symbol, if specified, must exactly match the
label (or the name of the control variable, for a loop with no label) in
the \keyword{do}, \keyword{loop}, or \keyword{select} clause in all
respects except case.
\end{enumerate}

\subsubsection{Loop instruction}\label{refloop}
\index{,}
\index{Instructions,LOOP}
\index{,}
\index{FOREVER,repetitor on LOOP instruction}
\index{FOR,repetitor on LOOP instruction}
\index{OVER repetitor on LOOP instruction,}
\index{WHILE phrase of LOOP instruction,}
\index{UNTIL phrase of LOOP instruction,}
\index{BY phrase of LOOP instruction,}
\index{TO phrase of LOOP instruction,}
\index{FOR,phrase of LOOP instruction}
\index{,}
\index{,}
\index{Loops,repetitive}
\index{Conditional loops,}
\index{Infinite loops,}
\index{Numbers,in LOOP instruction}
\index{Indefinite loops,}
\index{Flow control,with LOOP construct}
\index{= equals sign,in LOOP instruction}
\begin{shaded}
\begin{alltt}
\textbf{loop} [\textbf{label} \emph{name}] [\textbf{protect} \emph{termp}] [\emph{repetitor}] [\emph{conditional}];
        \emph{instructionlist}
    [\textbf{catch} [\emph{vare} =] \emph{exception};
        \emph{instructionlist}]...
    [\textbf{finally}[;]
        \emph{instructionlist}]
    \textbf{end} [\emph{name}];

where \emph{repetitor} is one of:

    \emph{varc} = \emph{expri} [\textbf{to} \emph{exprt}] [\textbf{by} \emph{exprb}] [\textbf{for} \emph{exprf}]
    \emph{varo} \textbf{over} \emph{termo}
    \textbf{for} \emph{exprr}
    \textbf{forever}

and \emph{conditional} is either of:

    \textbf{while} \emph{exprw}
    \textbf{until} \emph{expru}

and \emph{name} is a non-numeric \emph{symbol}

and \emph{instructionlist} is zero or more \emph{instruction}s

and \emph{expri}, \emph{exprt}, \emph{exprb}, \emph{exprf}, \emph{exprr}, \emph{exprw}, and \emph{expru} are \emph{expressions}.
\end{alltt}
\end{shaded}
 The \keyword{loop} instruction is used to group instructions together
and execute them repetitively.
The loop may optionally be given a label, and may protect an object
while the instructions in the loop are executed; exceptional conditions
can be handled with \keyword{catch} and \keyword{finally}.
 \keyword{loop} is the most complicated of the \crexx{} keyword
instructions.
It can be used as a simple indefinite loop, a predetermined
repetitive loop, as a loop with a bounding condition that is
recalculated on each iteration, or as a loop that steps over the
contents of a collection of values.

Syntax notes
\begin{itemize}
\item 
The \keyword{label} and \keyword{protect} phrases may be in any order.
They must precede any \emph{repetitor} or \emph{conditional}.
\item 
\index{Body,of a loop}
The first \emph{instructionlist} is known as the \emph{body} of
the loop.
\item 
The \keyword{to}, \keyword{by}, and \keyword{for} phrases in the first form
of \emph{repetitor} may be in any order, if used, and will be
evaluated in the order they are written.
\item 
Any instruction allowed in a method is allowed in an
\emph{instructionlist}, including assignments, method call
instructions, and keyword instructions (including any of the more
complex constructions such as \keyword{if}, \keyword{do}, \keyword{select},
or the \keyword{loop} instruction itself).
\item 
If \keyword{for} or \keyword{forever} start the \emph{repetitor} and
are followed by an "\textbf{=}" character, they are taken as
control variable names, not keywords (as for assignment instructions).
\item 
The expressions \emph{expri}, \emph{exprt}, \emph{exprb}, or
\emph{exprf} will be ended by any of the keywords \keyword{to},
\keyword{by}, \keyword{for}, \keyword{while}, or \keyword{until} (unless
the word is the name of a variable).
\item 
The expressions \emph{exprw} or \emph{expru} will be ended by
either of the keywords \keyword{while} or \keyword{until} (unless the
word is the name of a variable).
\end{itemize}

Indefinite loops

\index{Indefinite loops,}
\index{FOREVER,loops}
 If neither \emph{repetitor} nor \emph{conditional} are
present, or the \emph{repetitor} is the keyword \keyword{forever},
then the loop is an \emph{indefinite loop}.
It will be ended only when some instruction in the first
\emph{instructionlist} causes control to leave the loop.

\textbf{Example:}
\begin{lstlisting}
/* This displays "Go caving!" at least once */
loop forever
  say 'Go caving!'
  if ask='' then leave
  end
\end{lstlisting}

Bounded loops

\index{Bounded loop,}
\index{Repetitive loops,}
\index{Loops,repetitive}
\index{Repetitor phrase,}
\index{Conditional phrase,}
 If a \emph{repetitor} (other than \keyword{forever}) or
\emph{conditional} is given, the first \emph{instructionlist}
forms a \emph{bounded loop}, and the instruction list is executed
according to any \emph{repetitor phrase}, optionally modified by a
\emph{conditional phrase}.
\begin{description}
\item[Simple bounded loops]
\index{Simple repetitor phrase,}
\index{Bounded loop,simple}

When the \emph{repetitor} starts with the keyword \keyword{for},
the expression \emph{exprr} is evaluated immediately
(with \textbf{0} added, to effect any rounding) to give a repetition
count, which must be a whole number that is zero or positive.
The loop is then executed that many times, unless it is terminated by
some other condition.

\textbf{Example:}
\begin{lstlisting}
/* This displays "Hello" five times */
loop for 5
  say 'Hello'
  end
\end{lstlisting}
\item[Controlled bounded loops]
\index{Bounded loop,controlled}
\index{Controlled loops,}
\index{Control variable,}
\index{Variables,controlling loops}

A \emph{controlled loop} begins with an \emph{assignment},
which can be identified by the "\textbf{=}" that follows the name
of a control variable, \emph{varc}.
The control variable is assigned an initial value (the result of
\emph{expri}, formatted as though 0 had been added)
before the first execution of the instruction list.
The control variable is then stepped (by adding the result of
\emph{exprb}) before the second and subsequent times that the
instruction list is executed.
 
The name of the control variable, \emph{varc}, must be a non-numeric
symbol that names an existing or new variable in the current method or a
property in the current class (that is, it cannot be element of an
array, the property of a superclass, or a more complex term).  It is
further restricted in that it must not already be used as the name of a
control variable or label in a loop (or \keyword{do} or \keyword{select}
construct) that encloses the new loop.
 
\index{End condition of a LOOP loop,}
The instruction list in the body of the loop is executed repeatedly
while the end condition (determined by the result of \emph{exprt})
is not met.
If \emph{exprb} is positive or zero, then the loop will be
terminated when \emph{varc} is greater than the result of
\emph{exprt}.
If negative, then the loop will be terminated when \emph{varc} is
less than the result of \emph{exprt}.
 The expressions \emph{exprt} and \emph{exprb} must result in
numbers.
They are evaluated once only (with 0 added, to effect any
rounding), in the order they appear in the instruction, and before the
loop begins and before \emph{expri} (which must also result in a
number) is evaluated and the control variable is set to its initial
value.
 
The default value for \emph{exprb} is 1.
If no \emph{exprt} is given then the loop will execute indefinitely
unless it is terminated by some other condition.
 \textbf{Example:}
\begin{lstlisting}
loop i=3 to -2 by -1
  say i
  end
/* Would display: 3, 2, 1, 0, -1, -2 */
\end{lstlisting}
Note that the numbers do not have to be whole numbers:
 \textbf{Example:}
\begin{lstlisting}
x=0.3
loop y=x to x+4 by 0.7
  say y
  end
/* Would display: 0.3, 1.0, 1.7, 2.4, 3.1, 3.8 */
\end{lstlisting}
 The control variable may be altered within the loop, and this may
affect the iteration of the loop.
Altering the value of the control variable in this way is normally
considered to be suspect programming practice, though it may be
appropriate in certain circumstances.
 Note that the end condition is tested at the start of each iteration
(and after the control variable is stepped, on the second and
subsequent iterations).  It is therefore possible for the body of the
loop to be skipped entirely if the end condition is met immediately.
 The execution of a controlled loop may further be bounded by a
\keyword{for} phrase.
In this case, \emph{exprf} must be given and must evaluate to a
non-negative whole number.
This acts just like the repetition count in a simple bounded loop, and
sets a limit to the number of iterations around the loop if it is not
terminated by some other condition.
 
\emph{exprf} is evaluated along with the expressions
\emph{exprt} and \emph{exprb}.
That is, it is evaluated once only (with \textbf{0} added), when the
\keyword{loop} instruction is first executed and before the control
variable is given its initial value; the three expressions are evaluated
in the order in which they appear.
Like the \keyword{to} condition, the \keyword{for} count is checked at the
start of each iteration, as shown in the  programmer's (see page \pageref{refloopmod}) 
model:ea..
 \textbf{Example:}
\begin{lstlisting}
loop y=0.3 to 4.3 by 0.7 for 3
  say y
  end
/* Would display: 0.3, 1.0, 1.7 */
\end{lstlisting}
 
\index{END clause,specifying control variable}
In a controlled loop, the symbol that describes the control variable may
be specified on the \keyword{end} clause (unless a label is specified,
see below).
\crexx{} will then check that this symbol exactly matches the
\emph{varc} of the control variable in the \keyword{loop} clause (in
all respects except case).
If the symbol does not match, then the program is in error - this
enables the nesting of loops to be checked automatically.
 \textbf{Example:}
\begin{lstlisting}
loop k=1 to 10
  ...
  ...
  end k  /* Checks this is the END for K loop */
\end{lstlisting}
\textbf{Note: }The values taken by the control variable may be affected by the
\keyword{numeric} settings, since normal \crexx{} arithmetic rules apply
to the computation of stepping the control variable.
\item[Over]\label{refloopov}
\index{Bounded loop,over values}
\index{Over loops,}
\index{Control variable,}

When the second token of the \emph{repetitor} is the keyword
\keyword{over}, the control variable, \emph{varo}, is used
to work through the sub-values in the collection of indexed strings
identified by \emph{termo}.
In this case, the \keyword{loop} instruction takes a "snapshot" of
the indexes that exist in the collection at the start of the loop, and
then for each iteration of the loop the control variable is set to the
next available index from the snapshot.
 
The number of iterations of the loop will be the number of indexes in
the collection, unless the loop is terminated by some other condition.
 \textbf{Example:}
\begin{lstlisting}
mycoll=''
mycoll['Tom']=1
mycoll['Dick']=2
mycoll['Harry']=3
loop name over mycoll
  say mycoll[name]
  end
/* might display: 3, 1, 2 */
\end{lstlisting}
 \textbf{Notes:}
\begin{enumerate}
\item 
The order in which the values are returned is undefined; all that
is known is that all indexes available when the loop started will be
recorded and assigned to \emph{varo} in turn as the loop iterates.
\item 
The same restrictions apply to \emph{varo} as apply to
\emph{varc}, the control variable for controlled loops (see above).
\item 
Similarly, the symbol \emph{varo} may be used as a name for the loop
and be specified on the \keyword{end} clause (unless a label is
specified, see below).
\end{enumerate}
 \emph{In the reference implementation, the \keyword{over} form of
repetitor may also be used to step though the contents of any object
that is of a type that is a subclass of \textbf{java.util.Dictionary},
such as an object of type \textbf{java.util.Hashtable}.
In this case, \emph{termo} specifies the dictionary, and a snapshot
(enumeration) of the keys to the Dictionary is taken at the start of the
loop.
Each iteration of the loop then assigns a new key to the control
variable \emph{varo} which must be (or will be given, if it is new)
the type \textbf{java.lang.Object}.
}
\item[Conditional phrases]\label{refloopwu}
\index{Conditional phrase,}

Any of the forms of loop syntax can be followed by a
\emph{conditional} phrase which may cause termination of the loop.
 
If \keyword{while} is specified, \emph{exprw} is evaluated, using the
latest values of all variables in the expression, before the instruction
list is executed on every iteration, and after the control
variable (if any) is stepped.
The expression must evaluate to either 0 or 1, and the instruction list
will be repeatedly executed while the result is 1 (that is, the loop
ends if the expression evaluates to 0).
 \textbf{Example:}
\begin{lstlisting}
loop i=1 to 10 by 2 while i<6
  say i
  end
/* Would display: 1, 3, 5 */
\end{lstlisting}
 
If \keyword{until} is specified, \emph{expru} is evaluated, using the
latest values of all variables in the expression, on the second and
subsequent iterations, and before the control variable (if any) is stepped.
\footnote{
Thus, it appears that the \keyword{until} condition is tested after the
instruction list is executed on each iteration.
However, it is the \keyword{loop} clause that carries out the evaluation.
}
The expression must evaluate to either 0 or 1, and the instruction list
will be repeatedly executed until the result is 1 (that is, the loop
ends if the expression evaluates to 1).
 \textbf{Example:}
\begin{lstlisting}
loop i=1 to 10 by 2 until i>6
  say i
  end
/* Would display: 1, 3, 5, 7 */
\end{lstlisting}
\end{description}
 Note that the execution of loops may also be modified by
using the \keyword{iterate} or \keyword{leave} instructions.

Label phrase

\index{Loops,label}
\index{Loops,naming of}
\index{LABEL,on LOOP instruction}
 
The \keyword{label} phrase may used to specify a \emph{name} for the
loop.  The name can then optionally be used on
\begin{itemize}
\item a \keyword{leave} instruction, to specify the name of the loop to leave
\item an \keyword{iterate} instruction, to specify the name of the loop to
be iterated
\item the \keyword{end} clause of the loop, to confirm the identity of the
loop that is being ended, for additional checking.
\end{itemize}
 \textbf{Example:}
\begin{lstlisting}
loop label pooks i=1 to 10
  loop label hill while j<3
    ...
    if a=b then leave pooks
    ...
    end hill
  end pooks
\end{lstlisting}
In this example, the \keyword{leave} instruction leaves both loops.
 
If a label is specified using the \keyword{label} keyword, it overrides
any name derived from the control variable name (if any).  That is, the
variable name cannot be used to refer to the loop if a label is
specified.
% \section{Protect phrase}
% \index{PROTECT,on LOOP instruction}
 
% The \keyword{protect} phrase may used to specify a term,
% \emph{termp}, that evaluates to a value that is not just a type and
% is not of a primitive type;
% while the \keyword{loop} construct is being executed, the value (object)
% is protected - that is, all the instructions in the \keyword{loop}
% construct have exclusive access to the object.
%  \textbf{Example:}
% \begin{lstlisting}
% loop protect myobject while a<b
%   ...
%   end
% \end{lstlisting}
 
% Both \keyword{label} and \keyword{protect} may be specified, in any order,
% if required.
% \section{Exceptions in loops}
% \index{CATCH,on LOOP instruction}
% \index{FINALLY,on LOOP instruction}
 
% Exceptions that are raised by the instructions within a \keyword{loop}
% construct may be caught using one or more \keyword{catch} clauses that
% name the \emph{exception} that they will catch.  When an exception is
% caught, the exception object that holds the details of the exception may
% optionally be assigned to a variable, \emph{vare}.
 
% Similarly, a \keyword{finally} clause may be used to introduce
% instructions that will always be executed when the loop ends, even if an
% exception is raised (whether caught or not).
 
% The  \emph{Exceptions} section (see page \pageref{refexcep})  has details and
% examples of \keyword{catch} and \keyword{finally}.

Programmer's model - how a typical loop is executed\label{refloopmod}

This model forms part of the definition of the \keyword{loop}
instruction.
\index{Loops,execution model}
\index{Model,of loop execution}
\index{Programmer's model of LOOP,}
 For the following loop:
\begin{alltt}
\keyword{loop} \emph{varc} \keyword{=} \emph{expri} \keyword{to} \emph{exprt} \keyword{by} \emph{exprb} \keyword{while} \emph{exprw}
  ...
  \emph{instruction list}
  ...
  \keyword{end}
\end{alltt}
 \crexx{} will execute the following:
\begin{alltt}
   $tempt=exprt+0   /* ($variables are internal and   */
   $tempb=exprb+0   /*   are not accessible.)         */
   varc=expri+0
   Transfer control to the point identified as $start:

$loop:
   /* An UNTIL expression would be tested here, with: */
   /* if expru then leave                             */
   varc=varc + $tempb
$start:
   if varc > $tempt then leave  /* leave quits a loop */
   /* A FOR count would be checked here               */
   if \textbackslash exprw then leave
      ...
      instruction list
      ...
   Transfer control to the point identified as $loop:
\end{alltt}
 \textbf{Notes:}
\begin{enumerate}
\item 
This example is for \emph{exprb} \textbf{>= 0}.
For a negative \emph{exprb}, the test at the start point of the loop
would use "\textbf{<}" rather than "\textbf{>}".
\item 
The upwards transfer of control takes place at the end of the body of
the loop, immediately before the \keyword{end} clause (or any
\keyword{catch} or \keyword{finally} clause).
The \keyword{end} clause is only reached when the loop is finally
completed.
\end{enumerate}
\index{,}

% \input{nr3meth}
% \input{nr3nop}
% \input{nr3numer}
% \input{nr3optio}
% \input{nr3pack}
% \input{nr3parse}
% \input{nr3prop}
% \input{nr3retur}
\subsubsection{Say instruction}\label{refsay}
\index{SAY,instruction}
\index{Instructions,SAY}
\index{,}
\index{,}
\index{Console, writing to with SAY,}
\index{Terminal, writing to with SAY,}
\index{stdout, writing to with SAY,}
\begin{shaded}
\begin{alltt}
\textbf{say} [\emph{expression}];
\end{alltt}
\end{shaded}
 \keyword{say} writes a string to the default output character
stream.
This typically causes it to be displayed (or spoken, or typed, \emph{etc.}) to
the user.

\textbf{Example:}
\begin{lstlisting}
data=100
say data 'divided by 4 =>' data/4
/* would display:  "100 divided by 4 => 25"  */
\end{lstlisting}
 
The result of evaluating the \emph{expression} is expected to be a
string; if it is not a string, it will be converted to a string.
This result string is written from the program via an
implementation-defined output stream.
 
\index{Line, displaying,}
\begin{shaded}\noindent
By default, the result string is treated as a "line" (an
implementation-dependent mechanism for indicating line termination is
effected after the string is written).
If, however, the string ends in the NUL character
(\textbf{'\textbackslash -'} or \textbf{'\textbackslash 0'}) then that character
is removed and line termination is not indicated.
\end{shaded}\indent
The result string may be of any length.  If no expression is specified,
or the expression result is \textbf{null}, then an empty line is
written (that is, as though the expression resulted in a null string).

% \input{nr3say}
% \input{nr3selec}
% \input{nr3signa}
% \input{nr3trace}

\subsection{\crexx{} level C Built In Functions (BIF's)}
% \chapter{Functions for \crexx{} strings}\label{refbmeth}
% \index{R\textsc{exx},class/functions of}
%  This section describes the set of functions defined for the \crexx{}
% string class, \textbf{R\textsc{exx}}.  These are called \emph{built-in
% functions}, and include character manipulation, word manipulation,
% conversion, and arithmetic functions.
 
% Implementations will also provide other functions for the \textbf{R\textsc{exx}}
% class (for example, to implement the \crexx{} operators or to provide
% constructors with primitive arguments), but these are not part of the
% \crexx{} language.
% \footnote{
% \emph{Details of the functions provided in the reference implementation are
% included in  Appendix C (see page \pageref{refappc}) .}
% }
% \section{General notes on the built-in functions:}
% \begin{enumerate}
% \item All functions work on a \crexx{} string of type \textbf{R\textsc{exx}}; this
% is referred to by the name \emph{string} in the descriptions of the
% functions.  For example, if the \textbf{word} function were invoked using
% the term:
% \begin{lstlisting}
% "Three word phrase".word(2)
% \end{lstlisting}
% then in the description of \textbf{word} the name
% \emph{string} refers to the string "\textbf{Three word
% phrase}", and the name \emph{n} refers to the string
% "\textbf{2}".
% \item All function arguments are of type \textbf{R\textsc{exx}} and all functions
% return a string of type \textbf{R\textsc{exx}}; if a number is returned, it
% will be formatted as though 0 had been added with no rounding.
% \item 
% The first parenthesis in a function call must immediately follow the name
% of the function, with no space in between.
% \item The parentheses in a function call can be omitted if no
% arguments are required and the function call is part of a
%  \emph{compound term} (see page \pageref{refcomterm}) .
% \footnote{
% Unless an implementation-provided option to disallow parenthesis
% omission is in force.
% }
% \item A position in a string is the number of a character in the string,
% where the first character is at position 1, \emph{etc.}
% \item Where arguments are optional, commas may only be included between
% arguments that are present (that is, trailing commas in argument lists
% are not permitted).
% \item A \emph{pad} argument, if specified, must be exactly one
% character long.
% \item If a function has a sub-option selected by the first character of a
% string, that character may be in upper or lowercase.
% \item Conversion between character encodings and decimal or hexadecimal
% is dependent on the machine representation (encoding) of characters
% and hence will return appropriately different results for Unicode,
% ASCII, EBCDIC, and other implementations.
% \end{enumerate}
% \subsection{The built-in functions}\label{builtinfunctions}
\begin{description}
\item[abbrev(info [,length{]})]\label{refabbrev}
\index{ABBREV function,}
\index{Function, built-in,ABBREV}
\index{Abbreviations,testing with ABBREV function}
returns 1 if \emph{info} is equal to the leading characters of
\emph{string} and \emph{info} is not less than
the minimum length, \emph{length}; 0 is returned
if either of these conditions is not met.
\emph{length} must be a non-negative whole number; the default is
the length of \emph{info}.
 \textbf{Examples:}
\begin{lstlisting}
'Print'.abbrev('Pri')   == 1
'PRINT'.abbrev('Pri')   == 0
'PRINT'.abbrev('PRI',4) == 0
'PRINT'.abbrev('PRY')   == 0
'PRINT'.abbrev('')      == 1
'PRINT'.abbrev('',1)    == 0
\end{lstlisting}
\textbf{Note: }A null string will always match if a length of 0 (or the default)
is used.
This allows a default keyword to be selected automatically if desired.
 \textbf{Example:}
\begin{lstlisting}
say 'Enter option:';  option=ask
select  /* keyword1 is to be the default */
  when 'keyword1'.abbrev(option) then ...
  when 'keyword2'.abbrev(option) then ...
     ...
  otherwise ...
  end
\end{lstlisting}
\item[abs()]\label{refabs}
\index{ABS function,}
\index{Function, built-in,ABS}
\index{Mathematical function,ABS}
\index{Absolute,value, finding using ABS function}
returns the absolute value of \emph{string}, which must be a
number.
 Any sign is removed from the number, and it is then formatted by adding
zero with a digits setting that is either nine or, if greater, the
number of digits in the mantissa of the number (excluding leading
insignificant zeros).
Scientific notation is used, if necessary.
 
\textbf{Examples:}
\begin{lstlisting}
'12.3'.abs              == 12.3
' -0.307'.abs           == 0.307
'123.45E+16'.abs        == 1.2345E+18
'- 1234567.7654321'.abs == 1234567.7654321
\end{lstlisting}

\item[b2d([n{]})]\label{refb2x}
\index{B2D function}
\index{Packing a string,with B2D}
\index{Function, built-in,B2D}
\index{Conversion,binary to decimal}
\index{Binary,conversion to decimal}
Binary to decimal.
Converts \emph{string}, a string of at least one binary
(\textbf{0} and/or \textbf{1}) digits, to an equivalent string of
decimal characters (a number), without rounding.
The returned string will use digits,
and will not include any blanks.
 If the number of binary digits in the string is not a multiple of four,
then up to three \textbf{'0'} digits will be added on the left before
conversion to make a total that is a multiple of four.
 If \emph{string} is the null string, 0 is returned. If n is not specified, \emph{string} is taken to be an unsigned number.

\textbf{Examples:}
\begin{lstlisting}
'01110'.b2d == 14 
'10000001'.b2d == 129 
'111110000001'.b2d == 3969 
'1111111110000001'.b2d == 65409 
'1100011011110000'.b2d == 50928 
\end{lstlisting}
If n is specified, string is taken as a signed number expressed in n binary characters. If the most significant (left-most) bit is zero then the number is positive; otherwise it is a negative number in twos-complement form. In both cases it is converted to a NetRexx number which may, therefore, be negative. If n is 0, 0 is always returned.

If necessary, string is padded on the left with '0' characters (note, not “signextended”), or truncated on the left, to length n characters; (that is, as though string.right(n, '0') had been executed.)

\textbf{Examples:}
\begin{lstlisting}
'10000001'.b2d(8) == -127 
'10000001'.b2d(16) == 129 
'1111000010000001'.b2d(16) == -3967 
'1111000010000001'.b2d(12) == 129 
'1111000010000001'.b2d(8) == -127 
'1111000010000001'.b2d(4) == 1 
'0000000000110001'.b2d(0) == 0
\end{lstlisting} 
\item[b2x()]\label{refb2x}
\index{B2X function}
\index{Packing a string,with B2X}
\index{Function, built-in,B2X}
\index{Conversion,binary to hexadecimal}
\index{Binary,conversion to hexadecimal}
Binary to hexadecimal.
Converts \emph{string}, a string of at least one binary
(\textbf{0} and/or \textbf{1}) digits, to an equivalent string of
hexadecimal characters.
The returned string will use uppercase Roman letters for the values A-F,
and will not include any blanks.
 If the number of binary digits in the string is not a multiple of four,
then up to three \textbf{'0'} digits will be added on the left before
conversion to make a total that is a multiple of four.
 
\textbf{Examples:}
\begin{lstlisting}
'11000011'.b2x  == 'C3'
'10111'.b2x     == '17'
'0101'.b2x      == '5'
'101'.b2x       == '5'
'111110000'.b2x == '1F0'
\end{lstlisting}
  
\item[center(length [,pad{]})]\label{refcenter}

\emph{or}
\item[centre(length [,pad{]})]\label{refcentre}
\index{CENTRE function,}
\index{CENTER function,}
\index{Function, built-in,CENTRE}
\index{Function, built-in,CENTER}
\index{Formatting,text centering}

returns a string of length \emph{length} with \emph{string}
centered in it, with \emph{pad} characters added as necessary to
make up the required length.
\emph{length} must be a non-negative whole number.
The default \emph{pad} character is blank.
If the string is longer than \emph{length}, it will be truncated at
both ends to fit.
If an odd number of characters are truncated or added, the right hand
end loses or gains one more character than the left hand end.
 
\textbf{Examples:}
\begin{lstlisting}
'ABC'.centre(7)          == '  ABC  '
'ABC'.center(8,'-')      == '--ABC---'
'The blue sky'.centre(8) == 'e blue s'
'The blue sky'.center(7) == 'e blue '
\end{lstlisting}
\textbf{Note: }This function may be called either \textbf{centre} or \textbf{center},
which avoids difficulties due to the difference between the British and
American spellings.
\item[changestr(needle, new)]\label{refchastr}
\index{CHANGESTR function,}
\index{Function, built-in,CHANGESTR}
\index{Replacing strings,using CHANGESTR}
\index{Changing strings,using CHANGESTR}

returns a copy of \emph{string} in which each occurrence of the
\emph{needle} string is replaced by the \emph{new} string.
Each unique (non-overlapping) occurrence of the \emph{needle} string
is changed, searching from left to right and starting from the first
(leftmost) position in \emph{string}.
Only the original \emph{string} is searched for the
\emph{needle}, and each character in \emph{string} can only be
included in one match of the \emph{needle}.
 
If the \emph{needle} is the null string, the result is a copy of
\emph{string}, unchanged.
 
\textbf{Examples:}
\begin{lstlisting}
'elephant'.changestr('e','X')    == 'XlXphant'
'elephant'.changestr('ph','X')   == 'eleXant'
'elephant'.changestr('ph','hph') == 'elehphant'
'elephant'.changestr('e','')     == 'lphant'
'elephant'.changestr('','!!')    == 'elephant'
\end{lstlisting}
 The  \textbf{countstr} function (see page \pageref{refcoustr})  can be used to
count the number of changes that could be made to a string in this
fashion.
\item[compare(target [,pad{]})]\label{refcompar}
\index{COMPARE function,}
\index{Function, built-in,COMPARE}
\index{Finding a mismatch using COMPARE,}
\index{Comparison,of strings/using COMPARE}

returns 0 if \emph{string} and \emph{target}
are the same.
If they are not, the returned number is positive and is the position of
the first character that is not the same in both strings.
If one string is shorter than the other, one or more \emph{pad}
characters are added on the right to make it the same length for the
comparison.
The default \emph{pad} character is a blank.
 
\textbf{Examples:}
\begin{lstlisting}
'abc'.compare('abc')      == 0
'abc'.compare('ak')       == 2
'ab '.compare('ab')       == 0
'ab '.compare('ab',' ')   == 0
'ab '.compare('ab','x')   == 3
'ab-- '.compare('ab','-') == 5
\end{lstlisting}
\item[copies(n)]\label{refcopies}
\index{COPIES function,}
\index{Function, built-in,COPIES}
\index{Copying a string using COPIES,}
\index{Repeating a string with COPIES,}
returns \emph{n} directly concatenated copies of
\emph{string}.
\emph{n} must be positive or 0; if 0, the null string is returned.
 
\textbf{Examples:}
\begin{lstlisting}
'abc'.copies(3) == 'abcabcabc'
'abc'.copies(0) == ''
''.copies(2)    == ''
\end{lstlisting}
\item[copyindexed(sub)]\label{refcopyind}
\index{COPYINDEXED function,}
\index{Function, built-in,COPYINDEXED}
\index{Indexed strings,merging}
\index{Indexed strings,copying}
\index{Copying indexed variables,}
\index{Merging indexed variables,}
copies the collection of indexed  sub-values (see page \pageref{refinstr}) 
of \emph{sub} into the collection associated with
\emph{string}, and returns the modified \emph{string}.  The
resulting collection is the union of the two collections (that is,
it contains the indexes and their values from both collections).
If a given index exists in both collections then the sub-value of
\emph{string} for that index is replaced by the sub-value from
\emph{sub}.
 
The non-indexed value of \emph{string} is not affected.
 
\textbf{Example:}
 Following the instructions:
\begin{lstlisting}
foo='def'
foo['a']=1
foo['b']=2
bar='ghi'
bar['b']='B'
bar['c']='C'
merged=foo.copyIndexed(bar)
\end{lstlisting}
then:
\begin{lstlisting}
merged['a'] == '1'
merged['b'] == 'B'
merged['c'] == 'C'
merged['d'] == 'def'
\end{lstlisting}


\index{COUNTSTR function,}
\index{Function, built-in,COUNTSTR}
\index{Counting,strings, using COUNTSTR}
\item[countstr(needle)]\label{refcoustr}
returns the count of non-overlapping occurrences of the
\emph{needle} string in \emph{string}, searching from left to
right and starting from the first (leftmost) position in
\emph{string}.
 
If the \emph{needle} is the null string, \textbf{0} is returned.
 
\textbf{Examples:}
\begin{lstlisting}
'elephant'.countstr('e')  == '2'
'elephant'.countstr('ph') == '1'
'elephant'.countstr('')   == '0'
\end{lstlisting}
 The  \textbf{changestr} function (see page \pageref{refchastr})  can be used to
change occurrences of \emph{needle} to some other string.
\item[c2d()]\label{refc2d}
\index{C2D function,}
\index{Function, built-in,C2D}
\index{Conversion,coded character to decimal}
\index{Conversion,character to decimal}
\index{Character,conversion to decimal}
\index{Coded character,conversion to decimal}

Coded character to decimal.
Converts the encoding of the character in \emph{string} (which must be
exactly one character) to its decimal representation.
The returned string will be a non-negative number that represents
the encoding of the character and will not include any sign, blanks,
insignificant leading zeros, or decimal part.
 
\textbf{Examples:}
\begin{lstlisting}
'M'.c2d  == '77'  -- ASCII or Unicode
'7'.c2d  == '247' -- EBCDIC
'\textbackslash{}r'.c2d == '13'  -- ASCII or Unicode
'\textbackslash{}0'.c2d == '0'
\end{lstlisting}
 The  \textbf{c2x} function (see page \pageref{refc2x})  can be used to
convert the encoding of a character to a hexadecimal representation.
\item[c2x()]\label{refc2x}
\index{C2X function,}
\index{Unpacking a string,with C2X}
\index{Function, built-in,C2X}
\index{Conversion,coded character to hexadecimal}
\index{Conversion,character to hexadecimal}
\index{Character,conversion to hexadecimal}
\index{Coded character,conversion to hexadecimal}

Coded character to hexadecimal.
Converts the encoding of the character in \emph{string} (which must be
exactly one character) to its hexadecimal representation (unpacks).
The returned string will use uppercase Roman letters for the values A-F,
and will not include any blanks.
Insignificant leading zeros are removed.
 
\textbf{Examples:}
\begin{lstlisting}
'M'.c2x  == '4D' -- ASCII or Unicode
'7'.c2x  == 'F7' -- EBCDIC
'\textbackslash{}r'.c2x == 'D'  -- ASCII or Unicode
'\textbackslash{}0'.c2x == '0'
\end{lstlisting}
 The  \textbf{c2d} function (see page \pageref{refc2d})  can be used to
convert the encoding of a character to a decimal number.
\item[datatype(option)]\label{refdataty}
\index{DATATYPE function,}
\index{Function, built-in,DATATYPE}
\index{Mathematical function,DATATYPE options}
\index{Types,checking with DATATYPE}
\index{Numbers,checking with DATATYPE}
\index{Whole numbers,checking with DATATYPE}
\index{Alphanumerics,checking with DATATYPE}
\index{Alphabetics,checking with DATATYPE}
\index{Letters,checking with DATATYPE}
\index{Bits,checking with DATATYPE}
\index{Binary,checking with DATATYPE}
\index{Digits,checking with DATATYPE}
\index{Lowercase,checking with DATATYPE}
\index{Mixed case,checking with DATATYPE}
\index{Whole numbers,checking with DATATYPE}
\index{Numbers,checking with DATATYPE}
\index{Symbol characters,checking with DATATYPE}
\index{Uppercase,checking with DATATYPE}
\index{Hexadecimal,checking with DATATYPE}
\index{,}
\index{,}
\index{,}

returns 1 if \emph{string} matches the description requested with
the \emph{option}, or 0 otherwise.
If \emph{string} is the null string, 0 is always returned.
 
Only the first character of \emph{option} is significant, and it may
be in either uppercase or lowercase.
The following \emph{option} characters are recognized:
\begin{description}
\item[A]\label{refdta}
(Alphanumeric); returns 1 if \emph{string} only contains
characters from the ranges "a-z", "A-Z", and "0-9".
\item[B]\label{refdtb}
(Binary); returns 1 if \emph{string} only contains the
characters "0" and/or "1".
\item[D]\label{refdtd}
(Digits); returns 1 if \emph{string} only contains
characters from the range "0-9".
\item[L]\label{refdtl}
(Lowercase); returns 1 if \emph{string} only contains
characters from the range "a-z".
\item[M]\label{refdtm}
(Mixed case); returns 1 if \emph{string} only contains
characters from the ranges "a-z" and "A-Z".
\item[N]\label{refdtn}
(Number); returns 1 if \emph{string} is a syntactically valid
\crexx{} number that could be added to \textbf{'0'} without error,
\item[S]\label{refdts}
(Symbol); returns 1 if \emph{string} only contains characters
that are valid in non-numeric symbols (the alphanumeric characters and
underscore), and does not start with a digit.  Note that both uppercase
and lowercase letters are permitted.
\item[U]\label{refdtu}
(Uppercase); returns 1 if \emph{string} only contains
characters from the range "A-Z".
\item[W]\label{refdtw}
(Whole Number); returns 1 if \emph{string} is a syntactically valid
\crexx{} number that can be added to \textbf{'0'} without error, and
whose decimal part after that addition, with no rounding, is zero.
\item[X]\label{refdtx}
(heXadecimal); returns 1 if \emph{string} only contains
characters from the ranges "a-f", "A-F", and "0-9".
\end{description}
 
\textbf{Examples:}
\begin{lstlisting}
'101'.datatype('B')    == 1
'12.3'.datatype('D')   == 0
'12.3'.datatype('N')   == 1
'12.3'.datatype('W')   == 0
'LaArca'.datatype('M') == 1
''.datatype('M')       == 0
'Llanes'.datatype('L') == 0
'3 d'.datatype('s')    == 0
'BCd3'.datatype('X')   == 1
'BCgd3'.datatype('X')  == 0
\end{lstlisting}
\textbf{Note: }The \textbf{datatype} function tests the meaning of the characters
in a string, independent of the encoding of those characters.  Extra
letters and Extra digits cause \textbf{datatype} to return 0 except
for the number tests ("\textbf{N}" and "\textbf{W}"),
which treat extra digits whose value is in the range 0-9 as though they
were the corresponding Arabic numeral.
\item[date()] see page \pageref{refrexxdate}
\item[delstr(n [,length{]})]\label{refdelstr}
\index{DELSTR function,}
\index{Function, built-in,DELSTR}
\index{Deleting,part of a string}

returns a copy of \emph{string} with the sub-string of
\emph{string} that begins at the \emph{n}\emph{th} character, and is
of length \emph{length} characters, deleted.
If \emph{length} is not specified, or is greater than the number of
characters from \emph{n} to the end of the string, the rest of the
string is deleted (including the \emph{n}\emph{th} character).
\emph{length} must be a non-negative whole number, and \emph{n}
must be a positive whole number.  If \emph{n} is greater than the
length of \emph{string}, the string is returned unchanged.
 
\textbf{Examples:}
\begin{lstlisting}
'abcd'.delstr(3)    == 'ab'
'abcde'.delstr(3,2) == 'abe'
'abcde'.delstr(6)   == 'abcde'
\end{lstlisting}
\item[delword(n [,length])]\label{refdelword}
\index{DELWORD function,}
\index{Function, built-in,DELWORD}
\index{Deleting,words from a string}
\index{Words,deleting from a string}

returns a copy of \emph{string} with the sub-string of
\emph{string} that starts at the \emph{n}\emph{th} word, and is of
length \emph{length} blank-delimited words, deleted.
If \emph{length} is not specified, or is greater than number of
remaining words in the string, it defaults to be the remaining words
in the string (including the \emph{n}\emph{th} word).
\emph{length} must be a non-negative whole number, and \emph{n}
must be a positive whole number.  If \emph{n} is greater than the
number of words in \emph{string}, the string is returned unchanged.
The string deleted includes any blanks following the final word
involved, but none of the blanks preceding the first word involved.
 
\textbf{Examples:}
\begin{lstlisting}
'Now is the  time'.delword(2,2) == 'Now time'
'Now is the time '.delword(3)   == 'Now is '
'Now  time'.delword(5)          == 'Now  time'
\end{lstlisting}

\index{D2B function,}
\index{Function, built-in,D2B}
\index{Conversion,decimal to binary}
\index{Decimal,conversion to binary}
\index{Binary,from decimal}
\index{Binary,from decimal}
\item[d2b([n{]})]\label{refd2b}
Decimal to binary.
Returns a string of binary characters of length as needed or of length
n, which is the binary representation of the decimal number. The
returned string will use 0 and 1 characters for binary values. string
must be a whole number, and must be non-negative unless n is
specified, or an error will result. If n is not specified, the length
of the result returned is such that there are no leading 0 characters,
unless string was equal to 0 (in which case '0' is returned).

If n is specified it is the length of the final result in characters;
that is, after conversion the input string will be sign-extended to
the required length (negative numbers are converted assuming
twos-complement form). If the number is too big to fit into n
characters, it will be truncated on the left. n must be a nonnegative
whole number.

 \textbf{Examples:}
\begin{lstlisting}
'0'.d2b == 0 
'9'.d2b == 1001 
'19'.d2b == 10011 
'129'.d2b == 10000001 
'129'.d2b(1) == 1 
'129'.d2b(8) == 10000001 
'127'.d2b(12) == 000001111111 
'129'.d2b(16) == 0000000010000001 
'257'.d2b(8) == 00000001 
'-127'.d2b(8) == 10000001 
'-127'.d2b(16) == 1111111110000001 
'12'.d2b(0) == 
\end{lstlisting}

\index{D2C function,}
\index{Function, built-in,D2C}
\index{Conversion,decimal to character}
\index{Decimal,conversion to character}
\index{Coded character,from decimal}
\index{Character,from decimal}
\index{Character,from a number}
\index{Numbers,conversion to character}
\item[d2c()]\label{refd2c}
Decimal to coded character.
Converts the \emph{string} (a \crexx{} \emph{number}) to a
single character, where the number is used as the encoding of the
character.
 
\emph{string} must be a non-negative whole number.
An error results if the encoding described does not produce a valid
character for the implementation (for example, if it has more
significant bits than the implementation's encoding for characters).
 
\textbf{Examples:}
\begin{lstlisting}
'77'.d2c  == 'M' -- ASCII or Unicode
'+77'.d2c == 'M' -- ASCII or Unicode
'247'.d2c == '7' -- EBCDIC
'0'.d2c   == '\textbackslash 0'
\end{lstlisting}
\item[d2x([n])]\label{refd2x}
\index{D2X function,}
\index{Function, built-in,D2X}
\index{Conversion,decimal to hexadecimal}
\index{Decimal,conversion to hexadecimal}
\index{Numbers,conversion to hexadecimal}

Decimal to hexadecimal.
Returns a string of hexadecimal characters of length as needed or of
length \emph{n}, which is the hexadecimal (unpacked) representation
of the decimal number.  The returned string will use uppercase
Roman letters for the values A-F, and will not include any blanks.
 \emph{string} must be a whole number, and must be non-negative
unless \emph{n} is specified, or an error will result.
If \emph{n} is not specified, the length of the result returned is
such that there are no leading 0 characters, unless \emph{string}
was equal to 0 (in which case \textbf{'0'} is returned).
 
If \emph{n} is specified it is the length of the final result in
characters; that is, after conversion the input string will be
sign-extended to the required length (negative numbers are converted
assuming twos-complement form).
If the number is too big to fit into \emph{n} characters, it will be
truncated on the left.
\emph{n} must be a non-negative whole number.
 
\textbf{Examples:}
\begin{lstlisting}
'9'.d2x       == '9'
'129'.d2x     == '81'
'129'.d2x(1)  == '1'
'129'.d2x(2)  == '81'
'127'.d2x(3)  == '07F'
'129'.d2x(4)  == '0081'
'257'.d2x(2)  == '01'
'-127'.d2x(2) == '81'
'-127'.d2x(4) == 'FF81'
'12'.d2x(0)   == ''
\end{lstlisting}

\item[exists(index)]\label{refexists}
\index{EXISTS function,}
\index{Function, built-in,EXISTS}
\index{Index strings,testing for}
\index{Indexed strings,testing for}
\index{Testing for indexed variables,}

returns 1 if \emph{index} names a  sub-value (see page \pageref{refinstr})  of
\emph{string} that has explicitly been assigned a value, or 0
otherwise.
 
\textbf{Example:}
 Following the instructions:
\begin{lstlisting}
vowel=0
vowel['a']=1
vowel['b']=1
vowel['b']=null -- drops previous assignment
\end{lstlisting}
then:
\begin{lstlisting}
vowel.exists('a') == '1'
vowel.exists('b') == '0'
vowel.exists('c') == '0'
\end{lstlisting}

\item[format([before [,after]])]\label{refformat}
\index{FORMAT,function}
\index{Function, built-in,FORMAT}
\index{Mathematical function,FORMAT}
\index{Formatting,numbers for display}
\index{Numbers,formatting for display}
\index{Numbers,rounding}
\index{Conversion,formatting numbers}

formats (lays out) \emph{string}, which must be a number.
 
The number, \emph{string}, is first formatted by adding zero with a
digits setting that is either nine or, if greater, the number of digits
in the mantissa of the number (excluding leading insignificant zeros).
If no arguments are given, the result is precisely that of this
operation.
 
The arguments \emph{before} and \emph{after} may be specified to
control the number of characters to be used for the integer part and
decimal part of the result respectively.  If either of these is omitted
(with no arguments specified to its right), or is \textbf{null}, the
number of characters used will be as many as are needed for that part.
 
\emph{before} must be a positive number; if it is larger than is
needed to contain the integer part, that part is padded on the left with
blanks to the requested length.
If \emph{before} is not large enough to contain the integer part
of the number (including the sign, for negative numbers), an error
results.
 
\emph{after} must be a non-negative number; if it is not the same
size as the decimal part of the number, the number will be rounded (or
extended with zeros) to fit.  Specifying 0 for \emph{after} will
cause the number to be rounded to an integer (that is, it will have no
decimal part or decimal point).
 
\textbf{Examples:}
\begin{lstlisting}
' - 12.73'.format         == '-12.73'
'0.000'.format            == '0'
'3'.format(4)             == '   3'
'1.73'.format(4,0)        == '   2'
'1.73'.format(4,3)        == '   1.730'
'-.76'.format(4,1)        == '  -0.8'
'3.03'.format(4)          == '   3.03'
' - 12.73'.format(null,4) == '-12.7300'
\end{lstlisting}
 
Further arguments may be passed to the \keyword{format} function to control
the use of exponential notation.
The full syntax of the function is then:
 
\keyword{format([before[,after[,explaces[,exdigits[,exform]]]]])}
 The first two arguments are as already described.  The other three
(\emph{explaces}, \emph{exdigits}, and \emph{exform})
control the exponent part of the result.  The default for any of the
arguments may be selected by omitting them (if there are no arguments to
be specified to their right) or by using the value \textbf{null}.
 
\emph{explaces} must be a positive number; it sets the number of
places (digits after the sign of the exponent) to be used for any
exponent part, the default being to use as many as are needed.
If \emph{explaces} is specified and is not large enough to contain
the exponent, an error results.
If \emph{explaces} is specified and the exponent will be 0,
then \emph{explaces}+2 blanks are supplied for the exponent
part of the result.
 
\emph{exdigits} sets the trigger point for use of exponential
notation.
If, after the first formatting, the number of places needed before the
decimal point exceeds \emph{exdigits}, or if the absolute value of
the result is less than \textbf{0.000001}, then exponential form will
be used, provided that \emph{exdigits} was specified.
When \emph{exdigits} is not specified, exponential notation
will never be used.
The current setting of \keyword{numeric digits} may be used for
\emph{exdigits} by specifying the special word
 \textbf{digits} (see page \pageref{refswdigit}) .
If 0 is specified for \emph{exdigits}, exponential
notation is always used unless the exponent would be 0.
 
\emph{exform} sets the form for exponential notation (if needed).
\emph{exform} may be either \textbf{'Scientific'} (the default)
or \textbf{'Engineering'}.  Only the first character of
\emph{exform} is significant and it may be in uppercase or in
lowercase.
The current setting of \keyword{numeric form} may be used by specifying
the special word  \textbf{form} (see page \pageref{refswform}) .
If engineering form is in effect, up to three digits (plus sign) may be
needed for the integer part of the result (\emph{before}).
 
\textbf{Examples:}
\begin{lstlisting}
'12345.73'.format(null,null,2,2) == '1.234573E+04'
'12345.73'.format(null,3,null,0) == '1.235E+4'
'1.234573'.format(null,3,null,0) == '1.235'
'123.45'.format(null,3,2,0)      == '1.235E+02'
'1234.5'.format(null,3,2,0,'e')  == '1.235E+03'
'1.2345'.format(null,3,2,0)      == '1.235    '
'12345.73'.format(null,null,3,6) == '12345.73     '
'12345e+5'.format(null,3)        == '1234500000.000'
\end{lstlisting}
 \textbf{Implementation minimum:} If exponents are supported in an
implementation, then they must be supported for exponents whose
absolute value is at least as large as the largest number that can be
expressed as an exact integer in default precision, \emph{i.e.}, 999999999.
Therefore, values for \emph{explaces} of up to 9 should also be
supported.
\item[insert(new [,n [,length [,pad{]]]})]\label{refinsert}
\index{INSERT function,}
\index{Function, built-in,INSERT}
\index{Inserting a string into another,}

inserts the string \emph{new}, padded or truncated to length
\emph{length}, into a copy of the target \emph{string} after the
\emph{n}\emph{th} character; the string with any inserts is returned.
\emph{length} and \emph{n} must be a non-negative whole numbers.
If \emph{n} is greater than the length of the target string,
padding is added before the \emph{new} string also.
The default value for \emph{n} is 0, which means insert before the
beginning of the string.  The default value for \emph{length} is
the length of \emph{new}.  The default \emph{pad} character is
a blank.
 
\textbf{Examples:}
\begin{lstlisting}
'abc'.insert('123')         == '123abc'
'abcdef'.insert(' ',3)      == 'abc def'
'abc'.insert('123',5,6)     == 'abc  123   '
'abc'.insert('123',5,6,'+') == 'abc++123+++'
'abc'.insert('123',0,5,'-') == '123--abc'
\end{lstlisting}

\index{LASTPOS function,}
\index{Function, built-in,LASTPOS}
\index{Finding a string in another string,}
\index{Locating,a string in another string}
\item[lastpos(needle [,start{]})]\label{reflastpos}
returns the position of the last occurrence of the string
\emph{needle} in \emph{string} (the "haystack"), searching
from right to left.
If the string \emph{needle} is not found, or is the null string,
0 is returned.
By default the search starts at the last character of
\emph{string} and scans backwards.
This may be overridden by specifying \emph{start}, the point at
which to start the backwards scan.
\emph{start} must be a positive whole number, and defaults to the
value \emph{string}\textbf{.length} if larger than that
value or if not specified (with a minimum default value of one).
 
\textbf{Examples:}
\begin{lstlisting}
'abc def ghi'.lastpos(' ')   == 8
'abc def ghi'.lastpos(' ',7) == 4
'abcdefghi'.lastpos(' ')     == 0
'abcdefghi'.lastpos('cd')    == 3
''.lastpos('?')              == 0
\end{lstlisting}

\index{LEFT function,}
\index{Function, built-in,LEFT}
\index{Formatting,text left justification}
\item[left(length [,pad{]})]\label{refleft}
returns a string of length \emph{length} containing the
left-most \emph{length} characters of \emph{string}.
The string is padded with \emph{pad} characters (or truncated) on
the right as needed.
The default \emph{pad} character is a blank.
\emph{length} must be a non-negative whole number.
This function is exactly equivalent to
\emph{string}\textbf{.substr(1}, \emph{length}
[, \emph{pad}]\textbf{)}.
 
\textbf{Examples:}
\begin{lstlisting}
'abc d'.left(8)     == 'abc d   '
'abc d'.left(8,'.') == 'abc d...'
'abc defg'.left(6)  == 'abc de'
\end{lstlisting}

\index{LENGTH,function}
\index{Function, built-in,LENGTH}
\index{Strings,length of}
\index{Data,length of}
\item[length()]\label{reflength}
returns the number of characters in \emph{string}.
 
\textbf{Examples:}
\begin{lstlisting}
'abcdefgh'.length == 8
''.length         == 0
\end{lstlisting}
\item[lower([n [,length{]}])]\label{reflower}
\index{LOWER function,}
\index{Function, built-in,LOWER}
\index{Strings,lowercasing}
\index{Lowercasing strings,}

returns a copy of \emph{string} with any uppercase characters in
the sub-string of \emph{string} that begins at the \emph{n}\emph{th}
character, and is of length \emph{length} characters, replaced by
their lowercase equivalent.
 
\emph{n} must be a positive whole number, and defaults to 1 (the
first character in \emph{string}).  If \emph{n} is greater than
the length of \emph{string}, the string is returned unchanged.
 
\emph{length} must be a non-negative whole number.
If \emph{length} is not specified, or is greater than the number of
characters from \emph{n} to the end of the string, the rest of the
string (including the \emph{n}\emph{th} character) is assumed.
 
\textbf{Examples:}
\begin{lstlisting}
'SumA'.lower      == 'suma'
'SumA'.lower(2)   == 'Suma'
'SuMB'.lower(1,1) == 'suMB'
'SUMB'.lower(2,2) == 'SumB'
''.lower          == ''
\end{lstlisting}

\item[max(number)]\label{refmax}
\index{MAX function,}
\index{Function, built-in,MAX}
\index{Mathematical function,MAX}

returns the larger of \emph{string} and \emph{number}, which
must both be numbers.  If they compare equal (that is, when subtracted,
the result is 0), then \emph{string} is selected for the result.
 
The comparison is effected using a numerical comparison with a digits
setting that is either nine or, if greater, the larger of the number of
digits in the mantissas of the two numbers (excluding leading
insignificant zeros).
 
The selected result is formatted by adding zero to the selected number
with a digits setting that is either nine or, if greater, the number of
digits in the mantissa of the number (excluding leading insignificant
zeros).
Scientific notation is used, if necessary.
 
\textbf{Examples:}
\begin{lstlisting}
0.max(1)          ==1
'-1'.max(1)       ==1
'+1'.max(-1)      ==1
'1.0'.max(1.00)   =='1.0'
'1.00'.max(1.0)   =='1.00'
'123456700000'.max(1234567E+5)   == '123456700000'
'1234567E+5'.max('123456700000') == '1.234567E+11'
\end{lstlisting}

\item[min(number)]\label{refmin}
\index{MIN function,}
\index{Function, built-in,MIN}
\index{Mathematical function,MIN}

returns the smaller of \emph{string} and \emph{number}, which
must both be numbers.  If they compare equal (that is, when subtracted,
the result is 0), then \emph{string} is selected for the result.
 
The comparison is effected using a numerical comparison with a digits
setting that is either nine or, if greater, the larger of the number of
digits in the mantissas of the two numbers (excluding leading
insignificant zeros).
 
The selected result is formatted by adding zero to the selected number
with a digits setting that is either nine or, if greater, the number of
digits in the mantissa of the number (excluding leading insignificant
zeros).
Scientific notation is used, if necessary.
 
\textbf{Examples:}
\begin{lstlisting}
0.min(1)          ==0
'-1'.min(1)       =='-1'
'+1'.min(-1)      =='-1'
'1.0'.min(1.00)   =='1.0'
'1.00'.min(1.0)   =='1.00'
'123456700000'.min(1234567E+5)   == '123456700000'
'1234567E+5'.min('123456700000') == '1.234567E+11'
\end{lstlisting}

\index{OVERLAY function,}
\index{Function, built-in,OVERLAY}
\index{Overlaying a string onto another,}
\item[overlay(new [,n [,length [,pad{]]]})]\label{refoverlay}
overlays the string \emph{new}, padded or truncated to length
\emph{length}, onto a copy of the target \emph{string} starting
at the \emph{n}\emph{th} character; the string with any overlays is
returned.  Overlays may extend beyond the end of the original
\emph{string}.
If \emph{length} is specified it must be a non-negative whole
number.
If \emph{n} is greater than the length of
the target string, padding is added before the \emph{new} string
also.
The default \emph{pad} character is a blank, and the default value
for \emph{n} is 1.
\emph{n} must be greater than 0.
The default value for \emph{length} is the length of \emph{new}.
 
\textbf{Examples:}
\begin{lstlisting}
'abcdef'.overlay(' ',3)      == 'ab def'
'abcdef'.overlay('.',3,2)    == 'ab. ef'
'abcd'.overlay('qq')         == 'qqcd'
'abcd'.overlay('qq',4)       == 'abcqq'
'abc'.overlay('123',5,6,'+') == 'abc+123+++'
\end{lstlisting}

\index{POS position function,}
\index{Function, built-in,POS}
\index{Finding a string in another string,}
\index{Locating,a string in another string}
\index{Searching a string for a word or phrase,}
\item[pos(needle [,start{]})]\label{refpos}
returns the position of the string \emph{needle}, in
\emph{string} (the "haystack"), searching from left to right.
If the string \emph{needle} is not found, or is the null string,
0 is returned.
By default the search starts at the first character of
\emph{string} (that is, \emph{start} has the value 1).
This may be overridden by specifying \emph{start} (which must be a
positive whole number), the point at which to start the search; if
\emph{start} is greater than the length of \emph{string} then 0
is returned.
 \textbf{Examples:}
\begin{lstlisting}
'Saturday'.pos('day')    == 6
'abc def ghi'.pos('x')   == 0
'abc def ghi'.pos(' ')   == 4
'abc def ghi'.pos(' ',5) == 8
\end{lstlisting}

\index{REVERSE function,}
\index{Function, built-in,REVERSE}
\item[reverse()]\label{refreverse}
returns a copy of \emph{string}, swapped end for end.
 
\textbf{Examples:}
\begin{lstlisting}
'ABc.'.reverse        == '.cBA'
'XYZ '.reverse        == ' ZYX'
'Tranquility'.reverse == 'ytiliuqnarT'
\end{lstlisting}

\item[right(length [,pad{]})]\label{refright}
\index{RIGHT function,}
\index{Function, built-in,RIGHT}
\index{Formatting,text right justification}
\index{Leading zeros,adding with the RIGHT function}
\index{Zeros,adding on the left}
\index{Zeros,padding}

returns a string of length \emph{length} containing the
right-most \emph{length} characters of \emph{string} -
that is, padded with \emph{pad} characters (or truncated) on the
left as needed.  The default \emph{pad} character is a blank.
\emph{length} must be a non-negative whole number.
 
\textbf{Examples:}
\begin{lstlisting}
'abc  d'.right(8)  == '  abc  d'
'abc def'.right(5) == 'c def'
'12'.right(5,'0')  == '00012'
\end{lstlisting}

\index{SEQUENCE function,}
\index{Function, built-in,SEQUENCE}
\index{Collating sequence, using SEQUENCE,}
\item[sequence(final)]\label{refsequen}
 returns a string of all characters, in ascending order of encoding,
between and including the character in \emph{string} and the
character in \emph{final}.
\emph{string} and \emph{final} must be single characters;
if \emph{string} is greater than \emph{final}, an error is
reported.
 
\textbf{Examples:}
\begin{lstlisting}
'a'.sequence('f')           == 'abcdef'
'\\0'.sequence('\\x03')       == '\\x00\\x01\\x02\\x03'
'\\ufffe'.sequence('\\uffff') == '\\ufffe\\uffff'
\end{lstlisting}

\index{SIGN function,}
\index{Function, built-in,SIGN}
\index{Mathematical function,SIGN}
\item[sign()]\label{refsign}
returns a number that indicates the sign of \emph{string}, which
must be a number.
\emph{string} is first formatted, just as though the operation
"\textbf{string+0}" had been carried out with sufficient digits
to avoid rounding.
If the number then starts with \textbf{'-'} then \textbf{'-1'} is
returned; if it is \textbf{'0'} then \textbf{'0'} is returned; and
otherwise \textbf{'1'} is returned.
 
\textbf{Examples:}
\begin{lstlisting}
'12.3'.sign    ==  1
'0.0'.sign     ==  0
' -0.307'.sign == -1
\end{lstlisting}

\index{SOUNDEX function,}
\index{Function, built-in,SOUNDEX}
\index{Normalizing a string by its sound,SOUNDEX}
\item[soundex()]\label{refsoundex}
 returns the normalized soundex value of the string. This implementation
is for the English language.  

\textbf{Examples:}
\begin{lstlisting}
'EULER'.soundex()  == 'E460'
\end{lstlisting}

\index{SPACE function,}
\index{Function, built-in,SPACE}
\index{Formatting,text spacing}
\index{Blank,removal with SPACE function}
\item[space([n [,pad{]]})]\label{refspace}
returns a copy of \emph{string} with the blank-delimited words in
\emph{string} formatted with \emph{n} (and only \emph{n})
\emph{pad} characters between each word.
\emph{n} must be a non-negative whole number.
If \emph{n} is 0, all blanks are removed.
Leading and trailing blanks are always removed.
The default for \emph{n} is 1, and the default \emph{pad}
character is a blank.
 
\textbf{Examples:}
\begin{lstlisting}
'abc  def  '.space        == 'abc def'
'  abc def '.space(3)     == 'abc   def'
'abc  def  '.space(1)     == 'abc def'
'abc  def  '.space(0)     == 'abcdef'
'abc  def  '.space(2,'+') == 'abc++def'
\end{lstlisting}

\index{STRIP function,}
\index{Function, built-in,STRIP}
\index{Leading blanks,removal with STRIP function}
\index{Blank,removal with STRIP function}
\index{Zeros,removal with STRIP function}
\index{Leading zeros,removal with STRIP function}
\index{Character,removal with STRIP function}
\index{Trailing blanks,removal with STRIP function}
\item[strip([option [,char{]]}])]\label{refstrip}
returns a copy of \emph{string} with Leading, Trailing, or Both
leading and trailing characters removed, when the first character of
\emph{option} is L, T, or B respectively (these may be given in
either uppercase or lowercase).  The default is B.
The second argument, \emph{char}, specifies the character to be
removed, with the default being a blank.
If given, \emph{char} must be exactly one character long.
 
\textbf{Examples:}
\begin{lstlisting}
'  ab c  '.strip        == 'ab c'
'  ab c  '.strip('L')   == 'ab c  '
'  ab c  '.strip('t')   == '  ab c'
'12.70000'.strip('t',0) == '12.7'
'0012.700'.strip('b',0) == '12.7'
\end{lstlisting}

\index{SUBSTR function,}
\index{Function, built-in,SUBSTR}
\index{Extracting,a sub-string}
\index{Sub-string, extracting,}
\item[substr(n [,length [,pad{]]})]\label{refsubstr}
returns the sub-string of \emph{string} that begins at the
\emph{n}\emph{th} character, and is of length \emph{length}, padded
with \emph{pad} characters if necessary.
\emph{n} must be a positive whole number, and \emph{length} must
be a non-negative whole number.
If \emph{n} is greater than \emph{string}\textbf{.length},
then only pad characters can be returned.
 If \emph{length} is omitted it defaults to be the rest of the
string (or 0 if \emph{n} is greater than the length of the string).
The default \emph{pad} character is a blank.
 
\textbf{Examples:}
\begin{lstlisting}
'abc'.substr(2)       == 'bc'
'abc'.substr(2,4)     == 'bc  '
'abc'.substr(5,4)     == '    '
'abc'.substr(2,6,'.') == 'bc....'
'abc'.substr(5,6,'.') == '......'
\end{lstlisting}
\textbf{Note: }In some situations the positional (numeric) patterns of parsing
templates are more convenient for selecting sub-strings, especially if
more than one sub-string is to be extracted from a string.

\index{SUBWORD function,}
\index{Function, built-in,SUBWORD}
\index{Extracting,words from a string}
\index{Words,extracting from a string}
\item[subword(n [,length{]})]\label{refsubword}
returns the sub-string of \emph{string} that starts at the
\emph{n}\emph{th} word, and is up to \emph{length} blank-delimited
words long.
\emph{n} must be a positive whole number; if greater than the number
of words in the string then the null string is returned.
\emph{length} must be a non-negative whole number.
If \emph{length} is omitted it defaults to be the remaining words
in the string.
The returned string will never have leading or trailing blanks, but
will include all blanks between the selected words.
 
\textbf{Examples:}
\begin{lstlisting}
'Now is the  time'.subword(2,2) == 'is the'
'Now is the  time'.subword(3)   == 'the  time'
'Now is the  time'.subword(5)   == ''
\end{lstlisting}

\item[time()] see page \pageref{refrexxtime}
\index{TRANSLATE function,}
\index{Function, built-in,TRANSLATE}
\index{Translation,with TRANSLATE function}
\index{Re-ordering characters,with TRANSLATE function}
\index{Moving characters, with TRANSLATE function,}
\index{Strings,moving with TRANSLATE function}
\index{,}
\index{Replacing strings,using TRANSLATE}
\index{Changing strings,using TRANSLATE}
\item[translate(tableo, tablei [,pad{]})]\label{reftrans}
returns a copy of \emph{string} with each character in
\emph{string} either unchanged or translated to another character.
 
The \textbf{translate} function acts by searching the input translate
table, \emph{tablei}, for each character in \emph{string}.
If the character is found in \emph{tablei} (the first, leftmost,
occurrence being used if there are duplicates) then the corresponding
character in the same position in the output translate table,
\emph{tableo}, is used in the result string; otherwise the original
character found in \emph{string} is used.
The result string is always the same length as \emph{string}.
 
The translate tables may be of any length, including the null string.
The output table, \emph{tableo}, is padded with \emph{pad} or
truncated on the right as necessary to be the same length as
\emph{tablei}.
The default \emph{pad} is a blank.
 
\textbf{Examples:}
\begin{lstlisting}
'abbc'.translate('\&','b')           == 'a\&\&c'
'abcdef'.translate('12','ec')       == 'ab2d1f'
'abcdef'.translate('12','abcd','.') == '12..ef'
'4123'.translate('abcd','1234')     == 'dabc'
'4123'.translate('hods','1234')     == 'shod'
\end{lstlisting}
\textbf{Note: }The last two examples show how the \textbf{translate} function
may be used to move around the characters in a string.
In these examples, any 4-character string could be specified as the
first argument and its last character would be moved to the beginning of
the string.
Similarly, the term:
\begin{lstlisting}
'gh.ef.abcd'.translate(19970827,'abcdefgh')
\end{lstlisting}
(which returns "\textbf{27.08.1997}") shows how a string (in
this case perhaps a date) might be re-formatted and merged with other
characters using the \textbf{translate} function.

\index{TRUNC function,}
\index{Function, built-in,TRUNC}
\index{Truncating numbers,}
\index{Numbers,truncating}
\index{Formatting,numbers with TRUNC}
\item[trunc([n{]})]\label{reftrunc}
returns the integer part of \emph{string}, which must be a
number, with \emph{n} decimal places (digits after the decimal
point).
\emph{n} must be a non-negative whole number, and defaults to zero.
 
The number \emph{string} is formatted by adding zero with a digits
setting that is either nine or, if greater, the number of digits in the
mantissa of the number (excluding leading insignificant zeros).
It is then truncated to \emph{n} decimal places (or trailing zeros
are added if needed to make up the specified length).
If \emph{n} is 0 (the default) then an integer with no decimal
point is returned.
The result will never be in exponential form.
 
\textbf{Examples:}
\begin{lstlisting}
'12.3'.trunc         == 12
'127.09782'.trunc(3) == 127.097
'127.1'.trunc(3)     == 127.100
'127'.trunc(2)       == 127.00
'0'.trunc(2)         == 0.00
\end{lstlisting}

\index{UPPER function,}
\index{Function, built-in,UPPER}
\index{Strings,uppercasing}
\index{Uppercasing strings,}
\item[upper([n [,length{]]})]\label{refupper}
returns a copy of \emph{string} with any lowercase characters in
the sub-string of \emph{string} that begins at the \emph{n}\emph{th}
character, and is of length \emph{length} characters, replaced by
their uppercase equivalent.
 
\emph{n} must be a positive whole number, and defaults to 1 (the
first character in \emph{string}).  If \emph{n} is greater than
the length of \emph{string}, the string is returned unchanged.
 
\emph{length} must be a non-negative whole number.
If \emph{length} is not specified, or is greater than the number of
characters from \emph{n} to the end of the string, the rest of the
string (including the \emph{n}\emph{th} character) is assumed.
 
\textbf{Examples:}
\begin{lstlisting}
'Fou-Baa'.upper        == 'FOU-BAA'
'Mad Sheep'.upper      == 'MAD SHEEP'
'Mad sheep'.upper(5)   == 'Mad SHEEP'
'Mad sheep'.upper(5,1) == 'Mad Sheep'
'Mad sheep'.upper(5,4) == 'Mad SHEEp'
'tinganon'.upper(1,1)  == 'Tinganon'
''.upper               == ''
\end{lstlisting}

\index{VERIFY function,}
\index{Function, built-in,VERIFY}
\index{Strings,verifying contents of}
\item[verify(reference [,option [,start{]]})]\label{refverify}
verifies that \emph{string} is composed only of characters
from \emph{reference}, by returning the position of the first
character in \emph{string} that is not also in
\emph{reference}.  If all the characters were found in
\emph{reference}, 0 is returned.
 The \emph{option} may be either \textbf{'Nomatch'} (the
default) or \textbf{'Match'}.  Only the first character of
\emph{option} is significant and it may be in uppercase or in
lowercase.
If \textbf{'Match'} is specified, the position of the first character
in \emph{string} that \textbf{is} in \emph{reference} is
returned, or 0 is returned if none of the characters were found.
 The default for \emph{start} is 1 (that is, the search starts at
the first character of \emph{string}).
This can be overridden by giving a different \emph{start} point,
which must be positive.
 If \emph{string} is the null string, the function returns 0,
regardless of the value of the \emph{option}.
Similarly if \emph{start} is greater than
\emph{string}\textbf{.length}, 0 is returned.
 If \emph{reference} is the null string, then the returned value
is the same as the value used for \emph{start},
unless \textbf{'Match'} is specified as the \emph{option}, in
which case 0 is returned.
 
\textbf{Examples:}
\begin{lstlisting}
'123'.verify('1234567890')          == 0
'1Z3'.verify('1234567890')          == 2
'AB4T'.verify('1234567890','M')     == 3
'1P3Q4'.verify('1234567890','N',3)  == 4
'ABCDE'.verify('','n',3)            == 3
'AB3CD5'.verify('1234567890','m',4) == 6
\end{lstlisting}

\index{WORD function,}
\index{Function, built-in,WORD}
\index{Words,extracting from a string}
\item[word(n)]\label{refword}
returns the \emph{n}\emph{th} blank-delimited word in
\emph{string}.
\emph{n} must be positive.
If there are fewer than \emph{n} words in \emph{string}, the
null string is returned.
This function is exactly equivalent to
\emph{string}\textbf{.subword(}\emph{n},\textbf{1)}.
 
\textbf{Examples:}
\begin{lstlisting}
'Now is the time'.word(3) == 'the'
'Now is the time'.word(5) == ''
\end{lstlisting}

\index{WORDINDEX function,}
\index{Function, built-in,WORDINDEX}
\index{Words,locating in a string}
\item[wordindex(n)]\label{refwordind}
returns the character position of the \emph{n}\emph{th}
blank-delimited word in \emph{string}.
\emph{n} must be positive.
If there are fewer than \emph{n} words in the string, 0 is returned.
 
\textbf{Examples:}
\begin{lstlisting}
'Now is the time'.wordindex(3) == 8
'Now is the time'.wordindex(6) == 0
\end{lstlisting}

\index{WORDLENGTH function,}
\index{Function, built-in,WORDLENGTH}
\index{Words,finding length of}
\item[wordlength(n)]\label{refwordlen}
returns the length of the \emph{n}\emph{th} blank-delimited word in
\emph{string}.
\emph{n} must be positive.
If there are fewer than \emph{n} words in the string, 0 is returned.
 
\textbf{Examples:}
\begin{lstlisting}
'Now is the time'.wordlength(2)    == 2
'Now comes the time'.wordlength(2) == 5
'Now is the time'.wordlength(6)    == 0
\end{lstlisting}

\index{WORDPOS function,}
\index{Function, built-in,WORDPOS}
\index{Searching a string for a word or phrase,}
\index{Locating,a word or phrase in a string}
\index{Words,finding in a string}
\item[wordpos(phrase [,start{]})]\label{refwordpos}
searches \emph{string} for the first occurrence of the sequence
of blank-delimited words \emph{phrase}, and returns the word number
of the first word of \emph{phrase} in \emph{string}.  Multiple
blanks between words in either \emph{phrase} or \emph{string}
are treated as a single blank for the comparison, but otherwise the
words must match exactly.  Similarly, leading or trailing blanks on
either string are ignored.
If \emph{phrase} is not found, or contains no words, 0 is returned.
 By default the search starts at the first word in \emph{string}.
This may be overridden by specifying \emph{start} (which must be
positive), the word at which to start the search.
 
\textbf{Examples:}
\begin{lstlisting}
'now is the time'.wordpos('the')       == 3
'now is the time'.wordpos('The')       == 0
'now is the time'.wordpos('is the')    == 2
'now is the time'.wordpos('is    the') == 2
'now is the time'.wordpos('is  time')  == 0
'To be or not to be'.wordpos('be')     == 2
'To be or not to be'.wordpos('be',3)   == 6
\end{lstlisting}

\index{WORDS function,}
\index{Function, built-in,WORDS}
\index{Words,counting, using WORDS}
\index{Counting,words, using WORDS}
\item[words()]\label{refwords}
returns the number of blank-delimited words in \emph{string}.
 
\textbf{Examples:}
\begin{lstlisting}
'Now is the time'.words == 4
' '.words               == 0
''.words                == 0
\end{lstlisting}

\index{X2B function,}
\index{Unpacking a string,with X2B}
\index{Function, built-in,X2B}
\index{Conversion,hexadecimal to binary}
\index{Hexadecimal,conversion to binary}
\item[x2b()]\label{refx2b}
Hexadecimal to binary.
Converts \emph{string} (a string of at least one hexadecimal
characters) to an equivalent string of binary digits.
Hexadecimal characters may be any decimal digit character (0-9) or any
of the first six alphabetic characters (a-f), in either lowercase or
uppercase.
 \emph{string} may be of any length; each hexadecimal character
with be converted to a string of four binary digits.
The returned string will have a length that is a multiple of four, and
will not include any blanks.
 
\textbf{Examples:}
\begin{lstlisting}
'C3'.x2b  == '11000011'
'7'.x2b   == '0111'
'1C1'.x2b == '000111000001'
\end{lstlisting}

\index{X2C function,}
\index{Packing a string,with X2C}
\index{Function, built-in,X2C}
\index{Conversion,hexadecimal to character}
\index{Hexadecimal,conversion to character}
\index{Coded character,from hexadecimal}
\index{Character,from hexadecimal}
\index{Character,from a number}
\index{Numbers,conversion to character}
\item[x2c()]\label{refx2c}
Hexadecimal to coded character.
Converts the \emph{string} (a string of hexadecimal characters) to
a single character (packs).
Hexadecimal characters may be any decimal digit character (0-9) or any
of the first six alphabetic characters (a-f), in either lowercase or
uppercase.
 
\emph{string} must contain at least one hexadecimal character;
insignificant leading zeros are removed, and the string is then padded
with leading zeros if necessary to make a sufficient number of
hexadecimal digits to describe a character encoding for the
implementation.
 
An error results if the encoding described does not produce a valid
character for the implementation (for example, if it has more
significant bits than the implementation's encoding for characters).
 \textbf{Examples:}
\begin{lstlisting}
'004D'.x2c == 'M' -- ASCII or Unicode
'4d'.x2c   == 'M' -- ASCII or Unicode
'A2'.x2c   == 's' -- EBCDIC
'0'.x2c    == '\textbackslash 0'
\end{lstlisting}
 The  \textbf{d2c} function (see page \pageref{refd2c})  can be used to
convert a \crexx{} number to the encoding of a character.

\index{X2D function,}
\index{Function, built-in,X2D}
\index{Conversion,hexadecimal to decimal}
\index{Hexadecimal,conversion to decimal}
\item[x2d([n{]})]\label{refx2d}
Hexadecimal to decimal.
Converts the \emph{string} (a string of hexadecimal characters) to
a decimal number, without rounding.
If \emph{string} is the null string, 0 is returned.
 
If \emph{n} is not specified, \emph{string} is taken to
be an unsigned number.
 
\textbf{Examples:}
\begin{lstlisting}
'0E'.x2d    == 14
'81'.x2d    == 129
'F81'.x2d   == 3969
'FF81'.x2d  == 65409
'c6f0'.x2d  == 50928
\end{lstlisting}
 
If \emph{n} is specified, \emph{string} is taken as a signed
number expressed in \emph{n} hexadecimal characters.
If the most significant (left-most) bit is zero then the number is
positive; otherwise it is a negative number in twos-complement form.
In both cases it is converted to a \crexx{} number which may,
therefore, be negative.
If \emph{n} is 0, 0 is always returned.
 
If necessary, \emph{string} is padded on the left
with \textbf{'0'} characters (note, not "sign-extended"), or
truncated on the left, to length \emph{n} characters; (that is, as
though \emph{string}\textbf{.right(}\emph{n}, \textbf{'0')}
had been executed.)
 
\textbf{Examples:}
\begin{lstlisting}
'81'.x2d(2)   == -127
'81'.x2d(4)   == 129
'F081'.x2d(4) == -3967
'F081'.x2d(3) == 129
'F081'.x2d(2) == -127
'F081'.x2d(1) == 1
'0031'.x2d(0) == 0
\end{lstlisting}
 The  \textbf{c2d} function (see page \pageref{refc2d})  can be used to convert
a character to a decimal representation of its encoding.
\end{description}



\chapter{\crexx{} level B}
\crexx{} Level B is a Rexx version that can use native data types and
has a statement to include \emph{rxas} assembler statements in Rexx
modules. It is used as the implementation language of \crexx{} level
C. Level B has no \keyword{decimal} type but has \keyword{float} and
\keyword{int} types which correspond to the native C data types on the platform. 

\section{Level B Grammar Specification (Phase 0 PoC)}

\emph{Page Status: This page is ready for review for Phase 0 (PoC). It may have errors
and be changed based on feedback and implementation experience}

\subsection{Notes}

\begin{itemize}
\item This document \textbf{only} covers Phase 0 (PoC Prototype) Scope

\item CREXX Level B is \emph{NOT} Compatible with Classic REXX (Level A/C)

\item See {the REXX Language Level} descriptions

\item The {REXX PEG format} is used to define the grammar (implementation independent)

\end{itemize}

\subsection{Global Rules - File / Line Endings / Whitespace / Comments}

\subsubsection{Keywords}

\begin{verbatim}
keywords <- 'ADDRESS' / 'ARG' / 'BY' / 'CALL' / 'DO' / 'ELSE' / 'END' /
            'FOR' / 'IF' / 'ITERATE' / 'LEAVE' / 'NOP' / 'OTHERWISE' /
            'PARSE' / 'PULL' / 'PROCEDURE' / 'RETURN' / 'REXXLEVEL' /
            'REXXOPTION' / 'SAY' / 'THEN' / 'TO' / 'WHEN' / 'UPPER' 
            'MOD' / 'IDIV';
\end{verbatim}

\subsubsection{Symbols}

\begin{itemize}
\item misc symbols: \textquotesingle{},\textquotesingle{} / \textquotesingle{})\textquotesingle{} / \textquotesingle{}(\textquotesingle{}

\item logic operator: \textquotesingle{}\textbar{}\textquotesingle{} / \textquotesingle{}\&\&\textquotesingle{} / \textquotesingle{}\&\textquotesingle{}

\item normal compare: \textquotesingle{}=\textquotesingle{} / \textquotesingle{}\textbackslash{}=\textquotesingle{} / \textquotesingle{}\textless{}\textgreater{}\textquotesingle{} / \textquotesingle{}\textgreater{}\textless{}\textquotesingle{} / \textquotesingle{}\textgreater{}\textquotesingle{} / \textquotesingle{}\textless{}\textquotesingle{} / \textquotesingle{}\textgreater{}=\textquotesingle{} / \textquotesingle{}\textless{}=\textquotesingle{} / \textquotesingle{}\textbackslash{}\textgreater{}\textquotesingle{} / \textquotesingle{}\textbackslash{}\textless{}\textquotesingle{}

\item strict compare: \textquotesingle{}==\textquotesingle{} / \textquotesingle{}\textbackslash{}==\textquotesingle{} / \textquotesingle{}\textgreater{}\textgreater{}\textquotesingle{} / \textquotesingle{}\textless{}\textless{}\textquotesingle{} / \textquotesingle{}\textgreater{}\textgreater{}=\textquotesingle{} / \textquotesingle{}\textless{}\textless{}=\textquotesingle{} / \textquotesingle{}\textbackslash{}\textgreater{}\textgreater{}\textquotesingle{} / \textquotesingle{}\textbackslash{}\textless{}\textless{}\textquotesingle{}

\item concat operator: \textquotesingle{}\textbar{}\textbar{}\textquotesingle{}

\item additive operator:  \textquotesingle{}+\textquotesingle{} / \textquotesingle{}-\textquotesingle{}

\item multiplicative operator: \textquotesingle{}*\textquotesingle{} / \textquotesingle{}/\textquotesingle{} / \textquotesingle{}//\textquotesingle{} / \textquotesingle{}\%\textquotesingle{}

\item power operator: \textquotesingle{}**\textquotesingle{}

\item prefix operator: \textquotesingle{}+\textquotesingle{} / \textquotesingle{}-\textquotesingle{} / \textquotesingle{}\textquotesingle{}

\end{itemize}

\subsubsection{Whitespace \& Comments}

\begin{verbatim}
\COMMENT <- '/*' .* \COMMENT* .*  ('*/' / eos->ERROR[6.1]); # Get rid of comments
WS <- [ \t\v\f]+;                                           # Simple Whitespace Characters
CONTINUATION <- ',' WS* \EOL ^\EOF;                         # Continuation - trailing ','. Note: Comments already removed
\WHITESPACE <- WS / CONTINUATION;                           # Whitespace includes continuation
\end{verbatim}

\subsubsection{EOS}

\begin{verbatim}
EOS <- \EOF; # Platform specific detection of EOF (which is labelled End of Stream in ANSI REXX)
\end{verbatim}

\subsubsection{EOL}

\begin{verbatim}
EOL <- \EOL; # e.g. '\r\n' / '\n';  # Platform specific detection of EOL
\end{verbatim}

\subsubsection{Characters}

\begin{verbatim}
DIGIT <- [0-9];
LETTER <- [_!?A-Za-z];
VAR_CHAR <- LETTER / DIGIT;
\end{verbatim}

\subsubsection{Strings}

\begin{verbatim}
STRING <- ( ( '"'  ( (EOL -> ERROR[6.3]) / [^"] / '""'  )* '"' ) /
            ( '\'' ( (EOL -> ERROR[6.3]) / [^'] / '\'\'')* '\'') )
       -> STRING; 
\end{verbatim}

\subsubsection{Number}

\begin{verbatim}
NUMBER <- DIGIT+ -> NUMBER;
\end{verbatim}

\subsubsection{Symbols}

\begin{verbatim}
CONST_SYMBOL <- !keyword VAR_CHAR+ -> CONST_SYMBOL;
VAR_SYMBOL <- !keyword LETTER VAR_CHAR* -> VAR_SYMBOL;
LABEL <- (!keyword LETTER VAR_CHAR*) ':' -> LABEL;
value <- VAR_SYMBOL / CONST_SYMBOL / NUMBER / STRING;
taken_constant <- (VAR_SYMBOL / CONST_SYMBOL);
\end{verbatim}

\subsection{Program \& Structure}

\begin{verbatim}
program <- level_b_options instruction_list? EOS -> PROGRAM_FILE;
end_of_clause <- ';' / EOL;
ncl <- end_of_clause / ( e:.->(ERROR["21.1"] e) resync ); 
resync <- .* &end_of_clause; # Resync after an error

instruction_list <- i:(procedure / labeled_instruction)+ 
                 -> (INSTRUCTIONS i?);

labeled_instruction <- LABEL? / group / ( single_instruction ncl ) / ncl;

instruction <- group / ( single_instruction ncl );

single_instruction <- assignment / keyword_instruction / command;

procedure <- LABEL 'PROCEDURE' ncl 
             ( !(EOS / procedure) i:labeled_instruction )*
          -> (PROCEDURE LABEL (INSTRUCTIONS i));

assignment <-  v:var_symbol '=' e:expression -> (ASSIGN v e)
             / t:NUMBER '=' resync -> (ERROR[31.1] t) 
         / &digit t:CONST_SYMBOL '=' resync -> (ERROR[Msg31.2] t)
	 / &'.' t:CONST_SYMBOL '=' resync -> (ERROR[Msg31.3] t);

keyword_instruction <- address / arg / call / iterate 
         / leave / nop / parse / pull / return / say  
	     / t:'THEN' resync -> (ERROR[8.1] t) 
	     / t:'ELSE' resync -> (ERROR[8.2] t) 
    	 / t:'WHEN' resync -> (ERROR[9.1] t)
	     / t:'OTHERWISE' resync -> (ERROR[9.2] t);
         / t:'END' resync -> ERROR[10.1] t);

command <- c:expression -> (ADDRESS c);

group <- simple_do / do / if ;
\end{verbatim}

\subsection{Language Options (Level B)}

\begin{verbatim}
level_b_options <- 'REXXLEVEL' level:CONST_SYMBOL end_of_clause
                   'REXXOPTION' options:CONST_SYMBOL* end_of_clause
                    -> (REXX level options);
\end{verbatim}

\subsection{Groups}

\subsubsection{Simple DO Group}

\begin{verbatim}
do <- 'DO' (ncl / t:. resync -> (ERROR[27.1] t)) i:instruction_list? simple_do_ending 
   -> i;

simple_do_ending <- 'END' ncl 
	 / EOS -> ERROR[14.1]
	 / t:. resync -> (ERROR[35.1] t);
\end{verbatim}

\subsubsection{DO Group}

\begin{verbatim}
do <- 'DO' r:dorep (ncl / t:. resync -> (ERROR[27.1] t)) i:instruction_list? do_ending 
   -> (DO r i);

do_ending <- 'END' VAR_SYMBOL? ncl 
	 / EOS -> ERROR[14.1]
	 / t:. resync -> (ERROR[35.1] t);

dorep <- ( a:assignment {? t:dot b:dob f:dof} ) 
      -> (REPEAT a t? b? f?);
dot <- 'TO' e:expression -> (TO e);
dob <- 'BY' e:expression -> (BY e);
dof <- 'FOR' e:expression -> (FOR e);
\end{verbatim}

\subsubsection{IF Group}

\begin{verbatim}
if <- 'IF' e:expression ncl* (t:then / ((. -> ERROR[18.1]) resync )) f:else? 
-> (IF e t f?);

then <- 'THEN' ncl* (i:instruction -> (INSTRUCTIONS i) / eos -> ERROR{14.3] / 'END' -> ERROR[10.5] ) ;

else <- 'ELSE' ncl* (i:instruction -> (INSTRUCTIONS i) / eos -> ERROR[14.4] / 'END' -> ERROR[10.6] ) ;
\end{verbatim}

\subsection{Instructions}

\subsubsection{ADDRESS}

\begin{verbatim}
address <- 'ADDRESS' e:taken_constant c:expression? -> (ADDRESS ENVIRONMENT[e] c);
\end{verbatim}

\subsubsection{Arg}

\begin{verbatim}
arg <- 'ARG' t:template_list?
    -> (PARSE (OPTIONS UPPER?) ARG t?)
\end{verbatim}

\subsubsection{Call}

\begin{verbatim}
call <- 'CALL' (f:taken_constant / ( (. -> ERROR[19.2]) resync) ) e:expression_list?
     -> (CALL CONST_SYMBOL[f] e);
expression_list <- expr (',' expr)*;
\end{verbatim}

\subsubsection{Iterate}

\begin{verbatim}
iterate <- 'ITERATE' ( v:VAR_SYMBOL / (. -> ERROR[20.2]) resync) )?
        -> (ITERATE v?)
\end{verbatim}

\subsubsection{Leave}

\begin{verbatim}
leave <- 'LEAVE' ( v:VAR_SYMBOL / (. -> ERROR[20.2]) resync) )?
      -> (LEAVE v?);
\end{verbatim}

\subsubsection{Nop}

\begin{verbatim}
nop <- 'NOP';
\end{verbatim}

\subsubsection{Parse}

\begin{verbatim}
parse <- ('PARSE' (in:parse_type / (. -> ERROR[25.12]) resync)) out:template_list?)
         -> (PARSE OPTIONS in out)

       / ('PARSE' 'op:UPPER' (in:parse_type / (. -> ERROR[25.13]) resync)) out:template_list?)
         -> (PARSE (OPTIONS op) in out);

parse_type <- parse_key;
parse_key <- 'ARG'->ARG / 'PULL'->PULL;
\end{verbatim}

\subsubsection{Pull}

\begin{verbatim}
pull <- 'PULL' t:template_list?
     -> (PARSE (OPTIONS UPPER?) PULL t?);
\end{verbatim}

\subsubsection{Return}

\begin{verbatim}
return <- 'RETURN' e:expression?
       -> (RETURN e?);
\end{verbatim}

\subsubsection{Say}

\begin{verbatim}
say <- 'SAY' e:expression?
    -> (SAY e?);
\end{verbatim}

\subsubsection{Parse Templates}

\begin{verbatim}
template_list <- t:template (',' t:template)* 
              -> (TEMPLATES t+);
template <- (trigger / target / ((. -> ERROR[38.1]) resync)+;
target <- (VAR_SYMBOL / '.') 
       -> TARGET;
trigger <- pattern / positional;
pattern <- STRING / vrefp 
        -> PATTERN;
vrefp <- '(' 
            ( VAR_SYMBOL / ((. -> ERROR[19.7]) resync) ) 
            ( ')' / ((. -> ERROR[46.1]) resync) );
positional <- absolute_positional / relative_positional; 
absolute_positional <- (NUMBER / '=' position)
                    -> ABS_POS;
position <- NUMBER / vrefp / ((. -> ERROR[38.2]) resync);
relative_positional <- s:('+' / '-') position
                    -> (REL_POS SIGN[s] position);
\end{verbatim}

\subsection{Expressions}

\begin{verbatim}
expression <- expr  
           ( (',' -> ERROR[37.1] resync) / (')' -> ERROR[37.2] resync) )? ;

expr <- and_expression / 
        ( (a:expr op:or_operator b:and_expression)->(op a b) );
or_operator <- ('|' / '&&') -> OP_OR ;

and_expression <- comparison / 
        ( (a:and_expression op:and_operator b:comparison)->(op a b) );
and_operator <- ('&') -> OP_AND ;

comparison <- concatenation / 
        ( (a: comparison op:comparison_operator b:concatenation)->(op a b) );
comparison_operator <- (normal_compare / strict_compare) -> OP_COMPARE;
normal_compare<- '=' / '\\=' / '<>' / '><' / '>' / '<' / '>=' / '<=' / '\\>' / '\\<';
strict_compare<- '==' / '\\==' / '>>' / '<<' / '>>=' / '<<=' / '\\>>' / '\\<<';

concatenation <- addition 
                / ( (a:concatenation &| b:addition)->(OP_CONCAT a b) ) 
                / ( (a:concatenation b:addition)->(OP_SCONCAT a b) )  
                / ( (a:concatenation op:concat_operator b:addition)->(op a b) );
concat_operator <- '||' -> OP_CONCAT ;

addition <- multiplication 
                / ( (a:addition op:additive_operator b:multiplication)->(op a b) );
additive_operator <- ('+' / '\-')->OP_ADD;

multiplication <- power_expression 
                / (a:multiplication op:multiplicative_operator b:power_expression)->(op a b) );
multiplicative_operator <- ('*' / '/' / '//' / '%')->OP_MULT;

power_expression <- prefix_expression 
                / ( (a:power_expression op:power_operator b:prefix_expression)->(op a b) );
power_operator <- '**' -> OP_POWER ;

prefix_expression <- ( (op:prefix_operator a:prefix_expression)->(op a) ) 
           / term 
           / ( ( e:.->(ERROR["35.1"] e) ) resync);
prefix_operator <- ('+' / '-' / '\') -> OP_PREFIX ;

term <- value
      / function 
      / '(' expr ( (e:',' -> (ERROR["37.1"] e) ) resync) / ')' 
      / ( ( e:. -> (ERROR["36"] e) ) resync) );

function <- (f:taken_constant '(' p:expression_list? (')') -> (FUNCTION[f] p)
          / ((e:. -> (ERROR["36"] e)) resync);
\end{verbatim}



\section{The Toolchain}
\subsection{Prerequisites for building the \crexx{} translator}
\subsection{Building \crexx{} Applications}

\section{The Ahead-of-time compiler (AOT)}
The AOT has \emph{rxas} source as input and LLVM instructions as
output. This enables \crexx{} to have optimized native targets for the
platforms LLVM is available on. The AOT compiler is not in the plans
for the first delivery.

\appendix
\chapter{Platform differences}

\chapter{Notices}

\chapter{Instructions by Mnemonic}
\begin{longtable}{lll}
  \toprule
  \texttt{Opcode} & \texttt{Instruction} & \texttt{parameters} \\
  \texttt{ 18  } & \texttt{ ADDF        } & \texttt{  {REG,REG,REG}        } \\
\texttt{ 19  } & \texttt{ ADDF        } & \texttt{  {REG,REG,FLOAT}      } \\
\texttt{ 03  } & \texttt{ ADDI        } & \texttt{  {REG,REG,REG}        } \\
\texttt{ 04  } & \texttt{ ADDI        } & \texttt{  {REG,REG,INT}        } \\
\texttt{ 90  } & \texttt{ AMAP        } & \texttt{  {REG,REG}            } \\
\texttt{ 91  } & \texttt{ AMAP        } & \texttt{  {REG,INT}            } \\
\texttt{ 88  } & \texttt{ AND         } & \texttt{  {REG,REG,REG}        } \\
\texttt{ 47  } & \texttt{ APPEND      } & \texttt{  {REG,REG}            } \\
\texttt{ 44  } & \texttt{ APPENDCHAR  } & \texttt{  {REG,REG}            } \\
\texttt{ D5  } & \texttt{ BCF         } & \texttt{  {ID,REG}             } \\
\texttt{ D6  } & \texttt{ BCF         } & \texttt{  {ID,REG,REG}         } \\
\texttt{ D1  } & \texttt{ BCT         } & \texttt{  {ID,REG}             } \\
\texttt{ D2  } & \texttt{ BCT         } & \texttt{  {ID,REG,REG}         } \\
\texttt{ D3  } & \texttt{ BCTNM       } & \texttt{  {ID,REG}             } \\
\texttt{ D4  } & \texttt{ BCTNM       } & \texttt{  {ID,REG,REG}         } \\
\texttt{ E1  } & \texttt{ BEQ         } & \texttt{  {ID,REG,REG}         } \\
\texttt{ E2  } & \texttt{ BEQ         } & \texttt{  {ID,REG,INT}         } \\
\texttt{ D9  } & \texttt{ BGE         } & \texttt{  {ID,REG,REG}         } \\
\texttt{ DA  } & \texttt{ BGE         } & \texttt{  {ID,REG,INT}         } \\
\texttt{ D7  } & \texttt{ BGT         } & \texttt{  {ID,REG,REG}         } \\
\texttt{ D8  } & \texttt{ BGT         } & \texttt{  {ID,REG,INT}         } \\
\texttt{ DD  } & \texttt{ BLE         } & \texttt{  {ID,REG,REG}         } \\
\texttt{ DE  } & \texttt{ BLE         } & \texttt{  {ID,REG,INT}         } \\
\texttt{ DB  } & \texttt{ BLT         } & \texttt{  {ID,REG,REG}         } \\
\texttt{ DC  } & \texttt{ BLT         } & \texttt{  {ID,REG,INT}         } \\
\texttt{ DF  } & \texttt{ BNE         } & \texttt{  {ID,REG,REG}         } \\
\texttt{ E0  } & \texttt{ BNE         } & \texttt{  {ID,REG,INT}         } \\
\texttt{ A4  } & \texttt{ BR          } & \texttt{  {ID}                 } \\
\texttt{ A6  } & \texttt{ BRF         } & \texttt{  {ID,REG}             } \\
\texttt{ A5  } & \texttt{ BRT         } & \texttt{  {ID,REG}             } \\
\texttt{ A7  } & \texttt{ BRTF        } & \texttt{  {ID,ID,REG}          } \\
\texttt{ EE  } & \texttt{ BRTPANDT    } & \texttt{  {ID,REG,INT}         } \\
\texttt{ ED  } & \texttt{ BRTPT       } & \texttt{  {ID,REG}             } \\
\texttt{ 9C  } & \texttt{ CALL        } & \texttt{  {REG,FUNC}           } \\
\texttt{ 9D  } & \texttt{ CALL        } & \texttt{  {REG,FUNC,REG}       } \\
\texttt{ 9B  } & \texttt{ CALL        } & \texttt{  {FUNC}               } \\
\texttt{ CE  } & \texttt{ CNOP        } & \texttt{  NO OPERAND           } \\
\texttt{ 41  } & \texttt{ CONCAT      } & \texttt{  {REG,REG,REG}        } \\
\texttt{ 42  } & \texttt{ CONCAT      } & \texttt{  {REG,REG,STRING}     } \\
\texttt{ 43  } & \texttt{ CONCAT      } & \texttt{  {REG,STRING,REG}     } \\
\texttt{ 45  } & \texttt{ CONCCHAR    } & \texttt{  {REG,REG,REG}        } \\
\texttt{ AA  } & \texttt{ COPY        } & \texttt{  {REG,REG}            } \\
\texttt{ 2B  } & \texttt{ DEC         } & \texttt{  {REG}                } \\
\texttt{ 2D  } & \texttt{ DEC0        } & \texttt{  NO OPERAND           } \\
\texttt{ 2F  } & \texttt{ DEC1        } & \texttt{  NO OPERAND           } \\
\texttt{ 31  } & \texttt{ DEC2        } & \texttt{  NO OPERAND           } \\
\texttt{ 27  } & \texttt{ DIVF        } & \texttt{  {REG,REG,REG}        } \\
\texttt{ 28  } & \texttt{ DIVF        } & \texttt{  {REG,REG,FLOAT}      } \\
\texttt{ 29  } & \texttt{ DIVF        } & \texttt{  {REG,FLOAT,REG}      } \\
\texttt{ 11  } & \texttt{ DIVI        } & \texttt{  {REG,REG,REG}        } \\
\texttt{ 12  } & \texttt{ DIVI        } & \texttt{  {REG,REG,INT}        } \\
\texttt{ C9  } & \texttt{ DROPCHAR    } & \texttt{  {REG,REG,REG}        } \\
\texttt{ BB  } & \texttt{ EXIT        } & \texttt{  NO OPERAND           } \\
\texttt{ BC  } & \texttt{ EXIT        } & \texttt{  {REG}                } \\
\texttt{ BD  } & \texttt{ EXIT        } & \texttt{  {INT}                } \\
\texttt{ 16  } & \texttt{ FADD        } & \texttt{  {REG,REG,REG}        } \\
\texttt{ 17  } & \texttt{ FADD        } & \texttt{  {REG,REG,FLOAT}      } \\
\texttt{ AC  } & \texttt{ FCOPY       } & \texttt{  {REG,REG}            } \\
\texttt{ 24  } & \texttt{ FDIV        } & \texttt{  {REG,REG,REG}        } \\
\texttt{ 26  } & \texttt{ FDIV        } & \texttt{  {REG,FLOAT,REG}      } \\
\texttt{ 25  } & \texttt{ FDIV        } & \texttt{  {REG,REG,FLOAT}      } \\
\texttt{ 67  } & \texttt{ FEQ         } & \texttt{  {REG,REG,FLOAT}      } \\
\texttt{ 66  } & \texttt{ FEQ         } & \texttt{  {REG,REG,REG}        } \\
\texttt{ C5  } & \texttt{ FFORMAT     } & \texttt{  {REG,REG,REG}        } \\
\texttt{ 6A  } & \texttt{ FGT         } & \texttt{  {REG,REG,REG}        } \\
\texttt{ 6B  } & \texttt{ FGT         } & \texttt{  {REG,REG,FLOAT}      } \\
\texttt{ 6C  } & \texttt{ FGT         } & \texttt{  {REG,FLOAT,REG}      } \\
\texttt{ 6E  } & \texttt{ FGTE        } & \texttt{  {REG,REG,FLOAT}      } \\
\texttt{ 6D  } & \texttt{ FGTE        } & \texttt{  {REG,REG,REG}        } \\
\texttt{ 6F  } & \texttt{ FGTE        } & \texttt{  {REG,FLOAT,REG}      } \\
\texttt{ 71  } & \texttt{ FLT         } & \texttt{  {REG,REG,FLOAT}      } \\
\texttt{ 70  } & \texttt{ FLT         } & \texttt{  {REG,REG,REG}        } \\
\texttt{ 72  } & \texttt{ FLT         } & \texttt{  {REG,FLOAT,REG}      } \\
\texttt{ 75  } & \texttt{ FLTE        } & \texttt{  {REG,FLOAT,REG}      } \\
\texttt{ 73  } & \texttt{ FLTE        } & \texttt{  {REG,REG,REG}        } \\
\texttt{ 74  } & \texttt{ FLTE        } & \texttt{  {REG,REG,FLOAT}      } \\
\texttt{ 20  } & \texttt{ FMULT       } & \texttt{  {REG,REG,REG}        } \\
\texttt{ 21  } & \texttt{ FMULT       } & \texttt{  {REG,REG,FLOAT}      } \\
\texttt{ E3  } & \texttt{ FNDBLNK     } & \texttt{  {REG,REG,REG}        } \\
\texttt{ E4  } & \texttt{ FNDNBLNK    } & \texttt{  {REG,REG,REG}        } \\
\texttt{ 68  } & \texttt{ FNE         } & \texttt{  {REG,REG,REG}        } \\
\texttt{ 69  } & \texttt{ FNE         } & \texttt{  {REG,REG,FLOAT}      } \\
\texttt{ E6  } & \texttt{ FSEX        } & \texttt{  {REG}                } \\
\texttt{ 1A  } & \texttt{ FSUB        } & \texttt{  {REG,REG,REG}        } \\
\texttt{ 1B  } & \texttt{ FSUB        } & \texttt{  {REG,REG,FLOAT}      } \\
\texttt{ 1C  } & \texttt{ FSUB        } & \texttt{  {REG,FLOAT,REG}      } \\
\texttt{ C2  } & \texttt{ FTOB        } & \texttt{  {REG}                } \\
\texttt{ C1  } & \texttt{ FTOI        } & \texttt{  {REG}                } \\
\texttt{ BF  } & \texttt{ FTOS        } & \texttt{  {REG}                } \\
\texttt{ CD  } & \texttt{ GETBYTE     } & \texttt{  {REG,REG,REG}        } \\
\texttt{ 54  } & \texttt{ GETSTRPOS   } & \texttt{  {REG,REG}            } \\
\texttt{ E7  } & \texttt{ GETTP       } & \texttt{  {REG,REG}            } \\
\texttt{ 94  } & \texttt{ GMAP        } & \texttt{  {REG,REG}            } \\
\texttt{ 95  } & \texttt{ GMAP        } & \texttt{  {REG,STRING}         } \\
\texttt{ 51  } & \texttt{ HEXCHAR     } & \texttt{  {REG,REG,REG}        } \\
\texttt{ 01  } & \texttt{ IADD        } & \texttt{  {REG,REG,REG}        } \\
\texttt{ 02  } & \texttt{ IADD        } & \texttt{  {REG,REG,INT}        } \\
\texttt{ 32  } & \texttt{ IAND        } & \texttt{  {REG,REG,REG}        } \\
\texttt{ 33  } & \texttt{ IAND        } & \texttt{  {REG,REG,INT}        } \\
\texttt{ AB  } & \texttt{ ICOPY       } & \texttt{  {REG,REG}            } \\
\texttt{ 0E  } & \texttt{ IDIV        } & \texttt{  {REG,REG,REG}        } \\
\texttt{ 0F  } & \texttt{ IDIV        } & \texttt{  {REG,REG,INT}        } \\
\texttt{ 10  } & \texttt{ IDIV        } & \texttt{  {REG,INT,REG}        } \\
\texttt{ 56  } & \texttt{ IEQ         } & \texttt{  {REG,REG,REG}        } \\
\texttt{ 57  } & \texttt{ IEQ         } & \texttt{  {REG,REG,INT}        } \\
\texttt{ 5A  } & \texttt{ IGT         } & \texttt{  {REG,REG,REG}        } \\
\texttt{ 5B  } & \texttt{ IGT         } & \texttt{  {REG,REG,INT}        } \\
\texttt{ 5C  } & \texttt{ IGT         } & \texttt{  {REG,INT,REG}        } \\
\texttt{ 5D  } & \texttt{ IGTE        } & \texttt{  {REG,REG,REG}        } \\
\texttt{ 5E  } & \texttt{ IGTE        } & \texttt{  {REG,REG,INT}        } \\
\texttt{ 5F  } & \texttt{ IGTE        } & \texttt{  {REG,INT,REG}        } \\
\texttt{ 60  } & \texttt{ ILT         } & \texttt{  {REG,REG,REG}        } \\
\texttt{ 61  } & \texttt{ ILT         } & \texttt{  {REG,REG,INT}        } \\
\texttt{ 62  } & \texttt{ ILT         } & \texttt{  {REG,INT,REG}        } \\
\texttt{ 63  } & \texttt{ ILTE        } & \texttt{  {REG,REG,REG}        } \\
\texttt{ 64  } & \texttt{ ILTE        } & \texttt{  {REG,REG,INT}        } \\
\texttt{ 65  } & \texttt{ ILTE        } & \texttt{  {REG,INT,REG}        } \\
\texttt{ 13  } & \texttt{ IMOD        } & \texttt{  {REG,REG,REG}        } \\
\texttt{ 15  } & \texttt{ IMOD        } & \texttt{  {REG,INT,REG}        } \\
\texttt{ 14  } & \texttt{ IMOD        } & \texttt{  {REG,REG,INT}        } \\
\texttt{ 0A  } & \texttt{ IMULT       } & \texttt{  {REG,REG,REG}        } \\
\texttt{ 0B  } & \texttt{ IMULT       } & \texttt{  {REG,REG,INT}        } \\
\texttt{ 2A  } & \texttt{ INC         } & \texttt{  {REG}                } \\
\texttt{ 2C  } & \texttt{ INC0        } & \texttt{  NO OPERAND           } \\
\texttt{ 2E  } & \texttt{ INC1        } & \texttt{  NO OPERAND           } \\
\texttt{ 30  } & \texttt{ INC2        } & \texttt{  NO OPERAND           } \\
\texttt{ 58  } & \texttt{ INE         } & \texttt{  {REG,REG,REG}        } \\
\texttt{ 59  } & \texttt{ INE         } & \texttt{  {REG,REG,INT}        } \\
\texttt{ 3C  } & \texttt{ INOT        } & \texttt{  {REG,REG}            } \\
\texttt{ 3D  } & \texttt{ INOT        } & \texttt{  {REG,INT}            } \\
\texttt{ 34  } & \texttt{ IOR         } & \texttt{  {REG,REG,REG}        } \\
\texttt{ 35  } & \texttt{ IOR         } & \texttt{  {REG,REG,INT}        } \\
\texttt{ CF  } & \texttt{ IPOW        } & \texttt{  {REG,REG,REG}        } \\
\texttt{ D0  } & \texttt{ IPOW        } & \texttt{  {REG,REG,INT}        } \\
\texttt{ E5  } & \texttt{ ISEX        } & \texttt{  {REG}                } \\
\texttt{ 38  } & \texttt{ ISHL        } & \texttt{  {REG,REG,REG}        } \\
\texttt{ 39  } & \texttt{ ISHL        } & \texttt{  {REG,REG,INT}        } \\
\texttt{ 3A  } & \texttt{ ISHR        } & \texttt{  {REG,REG,REG}        } \\
\texttt{ 3B  } & \texttt{ ISHR        } & \texttt{  {REG,REG,INT}        } \\
\texttt{ 05  } & \texttt{ ISUB        } & \texttt{  {REG,REG,REG}        } \\
\texttt{ 06  } & \texttt{ ISUB        } & \texttt{  {REG,REG,INT}        } \\
\texttt{ 07  } & \texttt{ ISUB        } & \texttt{  {REG,INT,REG}        } \\
\texttt{ C0  } & \texttt{ ITOF        } & \texttt{  {REG}                } \\
\texttt{ BE  } & \texttt{ ITOS        } & \texttt{  {REG}                } \\
\texttt{ 36  } & \texttt{ IXOR        } & \texttt{  {REG,REG,REG}        } \\
\texttt{ 37  } & \texttt{ IXOR        } & \texttt{  {REG,REG,INT}        } \\
\texttt{ AE  } & \texttt{ LINK        } & \texttt{  {REG,REG}            } \\
\texttt{ B1  } & \texttt{ LOAD        } & \texttt{  {REG,INT}            } \\
\texttt{ B4  } & \texttt{ LOAD        } & \texttt{  {REG,CHAR}           } \\
\texttt{ B3  } & \texttt{ LOAD        } & \texttt{  {REG,STRING}         } \\
\texttt{ B2  } & \texttt{ LOAD        } & \texttt{  {REG,FLOAT}          } \\
\texttt{ EB  } & \texttt{ LOADSETTP   } & \texttt{  {REG,STRING,INT}     } \\
\texttt{ EA  } & \texttt{ LOADSETTP   } & \texttt{  {REG,FLOAT,INT}      } \\
\texttt{ E9  } & \texttt{ LOADSETTP   } & \texttt{  {REG,INT,INT}        } \\
\texttt{ 8E  } & \texttt{ MAP         } & \texttt{  {REG,REG}            } \\
\texttt{ 8F  } & \texttt{ MAP         } & \texttt{  {REG,STRING}         } \\
\texttt{ A8  } & \texttt{ MOVE        } & \texttt{  {REG,REG}            } \\
\texttt{ 8C  } & \texttt{ MTIME       } & \texttt{  {REG}                } \\
\texttt{ 22  } & \texttt{ MULTF       } & \texttt{  {REG,REG,REG}        } \\
\texttt{ 23  } & \texttt{ MULTF       } & \texttt{  {REG,REG,FLOAT}      } \\
\texttt{ 0C  } & \texttt{ MULTI       } & \texttt{  {REG,REG,REG}        } \\
\texttt{ 0D  } & \texttt{ MULTI       } & \texttt{  {REG,REG,INT}        } \\
\texttt{ 8A  } & \texttt{ NOT         } & \texttt{  {REG,REG}            } \\
\texttt{ 98  } & \texttt{ NSMAP       } & \texttt{  {REG,STRING,STRING}  } \\
\texttt{ 96  } & \texttt{ NSMAP       } & \texttt{  {REG,REG,REG}        } \\
\texttt{ 99  } & \texttt{ NSMAP       } & \texttt{  {REG,STRING,REG}     } \\
\texttt{ 97  } & \texttt{ NSMAP       } & \texttt{  {REG,REG,STRING}     } \\
\texttt{ B0  } & \texttt{ NULL        } & \texttt{  {REG}                } \\
\texttt{ 89  } & \texttt{ OR          } & \texttt{  {REG,REG,REG}        } \\
\texttt{ CC  } & \texttt{ PADSTR      } & \texttt{  {REG,REG,REG}        } \\
\texttt{ 92  } & \texttt{ PMAP        } & \texttt{  {REG,REG}            } \\
\texttt{ 93  } & \texttt{ PMAP        } & \texttt{  {REG,STRING}         } \\
\texttt{ 52  } & \texttt{ POSCHAR     } & \texttt{  {REG,REG,REG}        } \\
\texttt{ 9E  } & \texttt{ RET         } & \texttt{  NO OPERAND           } \\
\texttt{ 9F  } & \texttt{ RET         } & \texttt{  {REG}                } \\
\texttt{ A0  } & \texttt{ RET         } & \texttt{  {INT}                } \\
\texttt{ A1  } & \texttt{ RET         } & \texttt{  {FLOAT}              } \\
\texttt{ A2  } & \texttt{ RET         } & \texttt{  {CHAR}               } \\
\texttt{ A3  } & \texttt{ RET         } & \texttt{  {STRING}             } \\
\texttt{ 78  } & \texttt{ RSEQ        } & \texttt{  {REG,REG,REG}        } \\
\texttt{ 79  } & \texttt{ RSEQ        } & \texttt{  {REG,REG,STRING}     } \\
\texttt{ 46  } & \texttt{ SAPPEND     } & \texttt{  {REG,REG}            } \\
\texttt{ B5  } & \texttt{ SAY         } & \texttt{  {REG}                } \\
\texttt{ B7  } & \texttt{ SAY         } & \texttt{  {INT}                } \\
\texttt{ B8  } & \texttt{ SAY         } & \texttt{  {FLOAT}              } \\
\texttt{ BA  } & \texttt{ SAY         } & \texttt{  {CHAR}               } \\
\texttt{ B9  } & \texttt{ SAY         } & \texttt{  {STRING}             } \\
\texttt{ 3E  } & \texttt{ SCONCAT     } & \texttt{  {REG,REG,REG}        } \\
\texttt{ 3F  } & \texttt{ SCONCAT     } & \texttt{  {REG,REG,STRING}     } \\
\texttt{ 40  } & \texttt{ SCONCAT     } & \texttt{  {REG,STRING,REG}     } \\
\texttt{ AD  } & \texttt{ SCOPY       } & \texttt{  {REG,REG}            } \\
\texttt{ 76  } & \texttt{ SEQ         } & \texttt{  {REG,REG,REG}        } \\
\texttt{ 77  } & \texttt{ SEQ         } & \texttt{  {REG,REG,STRING}     } \\
\texttt{ EC  } & \texttt{ SETORTP     } & \texttt{  {REG,INT}            } \\
\texttt{ 53  } & \texttt{ SETSTRPOS   } & \texttt{  {REG,REG}            } \\
\texttt{ E8  } & \texttt{ SETTP       } & \texttt{  {REG,INT}            } \\
\texttt{ 7C  } & \texttt{ SGT         } & \texttt{  {REG,REG,REG}        } \\
\texttt{ 7D  } & \texttt{ SGT         } & \texttt{  {REG,REG,STRING}     } \\
\texttt{ 7E  } & \texttt{ SGT         } & \texttt{  {REG,STRING,REG}     } \\
\texttt{ 7F  } & \texttt{ SGTE        } & \texttt{  {REG,REG,REG}        } \\
\texttt{ 80  } & \texttt{ SGTE        } & \texttt{  {REG,REG,STRING}     } \\
\texttt{ 81  } & \texttt{ SGTE        } & \texttt{  {REG,STRING,REG}     } \\
\texttt{ 82  } & \texttt{ SLT         } & \texttt{  {REG,REG,REG}        } \\
\texttt{ 83  } & \texttt{ SLT         } & \texttt{  {REG,REG,STRING}     } \\
\texttt{ 84  } & \texttt{ SLT         } & \texttt{  {REG,STRING,REG}     } \\
\texttt{ 85  } & \texttt{ SLTE        } & \texttt{  {REG,REG,REG}        } \\
\texttt{ 86  } & \texttt{ SLTE        } & \texttt{  {REG,REG,STRING}     } \\
\texttt{ 87  } & \texttt{ SLTE        } & \texttt{  {REG,STRING,REG}     } \\
\texttt{ 7B  } & \texttt{ SNE         } & \texttt{  {REG,REG,STRING}     } \\
\texttt{ 7A  } & \texttt{ SNE         } & \texttt{  {REG,REG,REG}        } \\
\texttt{ B6  } & \texttt{ SSAY        } & \texttt{  {REG}                } \\
\texttt{ C3  } & \texttt{ STOF        } & \texttt{  {REG}                } \\
\texttt{ C4  } & \texttt{ STOI        } & \texttt{  {REG}                } \\
\texttt{ 50  } & \texttt{ STRCHAR     } & \texttt{  {REG,REG}            } \\
\texttt{ 4F  } & \texttt{ STRCHAR     } & \texttt{  {REG,REG,REG}        } \\
\texttt{ 4E  } & \texttt{ STRLEN      } & \texttt{  {REG,REG}            } \\
\texttt{ C6  } & \texttt{ STRLOWER    } & \texttt{  {REG,REG}            } \\
\texttt{ C7  } & \texttt{ STRUPPER    } & \texttt{  {REG,REG}            } \\
\texttt{ 1D  } & \texttt{ SUBF        } & \texttt{  {REG,REG,REG}        } \\
\texttt{ 1E  } & \texttt{ SUBF        } & \texttt{  {REG,REG,FLOAT}      } \\
\texttt{ 1F  } & \texttt{ SUBF        } & \texttt{  {REG,FLOAT,REG}      } \\
\texttt{ 08  } & \texttt{ SUBI        } & \texttt{  {REG,REG,REG}        } \\
\texttt{ 09  } & \texttt{ SUBI        } & \texttt{  {REG,REG,INT}        } \\
\texttt{ CB  } & \texttt{ SUBSTCUT    } & \texttt{  {REG,REG}            } \\
\texttt{ 55  } & \texttt{ SUBSTR      } & \texttt{  {REG,REG,REG}        } \\
\texttt{ CA  } & \texttt{ SUBSTRING   } & \texttt{  {REG,REG,REG}        } \\
\texttt{ A9  } & \texttt{ SWAP        } & \texttt{  {REG,REG}            } \\
\texttt{ 8B  } & \texttt{ TIME        } & \texttt{  {REG}                } \\
\texttt{ C8  } & \texttt{ TRANSCHAR   } & \texttt{  {REG,REG,REG}        } \\
\texttt{ 48  } & \texttt{ TRIML       } & \texttt{  {REG,REG}            } \\
\texttt{ 4B  } & \texttt{ TRIML       } & \texttt{  {REG,REG,REG}        } \\
\texttt{ 49  } & \texttt{ TRIMR       } & \texttt{  {REG,REG}            } \\
\texttt{ 4C  } & \texttt{ TRIMR       } & \texttt{  {REG,REG,REG}        } \\
\texttt{ 4A  } & \texttt{ TRUNC       } & \texttt{  {REG,REG}            } \\
\texttt{ 4D  } & \texttt{ TRUNC       } & \texttt{  {REG,REG,REG}        } \\
\texttt{ AF  } & \texttt{ UNLINK      } & \texttt{  {REG}                } \\
\texttt{ 9A  } & \texttt{ UNMAP       } & \texttt{  {REG}                } \\
\texttt{ 8D  } & \texttt{ XTIME       } & \texttt{  {REG,STRING}         } \\

\end{longtable}


\chapter{Instructions by Opcode}
\begin{longtable}{lll}
  \toprule
  \texttt{Opcode} & \texttt{Instruction} & \texttt{parameters} \\
  \texttt{     } & \texttt{             } & \texttt{                       } \\
\texttt{     } & \texttt{             } & \texttt{                       } \\
\texttt{ XX  } & \texttt{ ASSEMBLY IN } & \texttt{ NSTRUCTION List       } \\
\texttt{ 01  } & \texttt{ IADD        } & \texttt{  {REG,REG,REG}        } \\
\texttt{ 02  } & \texttt{ IADD        } & \texttt{  {REG,REG,INT}        } \\
\texttt{ 03  } & \texttt{ ADDI        } & \texttt{  {REG,REG,REG}        } \\
\texttt{ 04  } & \texttt{ ADDI        } & \texttt{  {REG,REG,INT}        } \\
\texttt{ 05  } & \texttt{ ISUB        } & \texttt{  {REG,REG,REG}        } \\
\texttt{ 06  } & \texttt{ ISUB        } & \texttt{  {REG,REG,INT}        } \\
\texttt{ 07  } & \texttt{ SUBI        } & \texttt{  {REG,REG,REG}        } \\
\texttt{ 08  } & \texttt{ SUBI        } & \texttt{  {REG,REG,INT}        } \\
\texttt{ 09  } & \texttt{ IMULT       } & \texttt{  {REG,REG,REG}        } \\
\texttt{ 0A  } & \texttt{ IMULT       } & \texttt{  {REG,REG,INT}        } \\
\texttt{ 0B  } & \texttt{ MULTI       } & \texttt{  {REG,REG,REG}        } \\
\texttt{ 0C  } & \texttt{ MULTI       } & \texttt{  {REG,REG,INT}        } \\
\texttt{ 0D  } & \texttt{ IDIV        } & \texttt{  {REG,REG,REG}        } \\
\texttt{ 0E  } & \texttt{ IDIV        } & \texttt{  {REG,REG,INT}        } \\
\texttt{ 0F  } & \texttt{ DIVI        } & \texttt{  {REG,REG,REG}        } \\
\texttt{ 10  } & \texttt{ DIVI        } & \texttt{  {REG,REG,INT}        } \\
\texttt{ 11  } & \texttt{ FADD        } & \texttt{  {REG,REG,REG}        } \\
\texttt{ 12  } & \texttt{ FADD        } & \texttt{  {REG,REG,FLOAT}      } \\
\texttt{ 13  } & \texttt{ ADDF        } & \texttt{  {REG,REG,REG}        } \\
\texttt{ 14  } & \texttt{ ADDF        } & \texttt{  {REG,REG,FLOAT}      } \\
\texttt{ 15  } & \texttt{ FSUB        } & \texttt{  {REG,REG,REG}        } \\
\texttt{ 16  } & \texttt{ FSUB        } & \texttt{  {REG,REG,FLOAT}      } \\
\texttt{ 17  } & \texttt{ FSUB        } & \texttt{  {REG,FLOAT,REG}      } \\
\texttt{ 18  } & \texttt{ SUBF        } & \texttt{  {REG,REG,REG}        } \\
\texttt{ 19  } & \texttt{ SUBF        } & \texttt{  {REG,REG,FLOAT}      } \\
\texttt{ 1A  } & \texttt{ SUBF        } & \texttt{  {REG,FLOAT,REG}      } \\
\texttt{ 1B  } & \texttt{ FMULT       } & \texttt{  {REG,REG,REG}        } \\
\texttt{ 1C  } & \texttt{ FMULT       } & \texttt{  {REG,REG,FLOAT}      } \\
\texttt{ 1D  } & \texttt{ MULTF       } & \texttt{  {REG,REG,REG}        } \\
\texttt{ 1E  } & \texttt{ MULTF       } & \texttt{  {REG,REG,FLOAT}      } \\
\texttt{ 1F  } & \texttt{ FDIV        } & \texttt{  {REG,REG,REG}        } \\
\texttt{ 20  } & \texttt{ FDIV        } & \texttt{  {REG,REG,FLOAT}      } \\
\texttt{ 21  } & \texttt{ FDIV        } & \texttt{  {REG,FLOAT,REG}      } \\
\texttt{ 22  } & \texttt{ DIVF        } & \texttt{  {REG,REG,REG}        } \\
\texttt{ 23  } & \texttt{ DIVF        } & \texttt{  {REG,REG,FLOAT}      } \\
\texttt{ 24  } & \texttt{ DIVF        } & \texttt{  {REG,FLOAT,REG}      } \\
\texttt{ 25  } & \texttt{ INC         } & \texttt{  {REG}                } \\
\texttt{ 26  } & \texttt{ DEC         } & \texttt{  {REG}                } \\
\texttt{ 27  } & \texttt{ INC0        } & \texttt{  NO OPERAND           } \\
\texttt{ 28  } & \texttt{ DEC0        } & \texttt{  NO OPERAND           } \\
\texttt{ 29  } & \texttt{ INC1        } & \texttt{  NO OPERAND           } \\
\texttt{ 2A  } & \texttt{ DEC1        } & \texttt{  NO OPERAND           } \\
\texttt{ 2B  } & \texttt{ INC2        } & \texttt{  NO OPERAND           } \\
\texttt{ 2C  } & \texttt{ DEC2        } & \texttt{  NO OPERAND           } \\
\texttt{ 2D  } & \texttt{ SCONCAT     } & \texttt{  {REG,REG,REG}        } \\
\texttt{ 2E  } & \texttt{ SCONCAT     } & \texttt{  {REG,REG,STRING}     } \\
\texttt{ 2F  } & \texttt{ SCONCAT     } & \texttt{  {REG,STRING,REG}     } \\
\texttt{ 30  } & \texttt{ CONCAT      } & \texttt{  {REG,REG,REG}        } \\
\texttt{ 31  } & \texttt{ CONCAT      } & \texttt{  {REG,REG,STRING}     } \\
\texttt{ 32  } & \texttt{ CONCAT      } & \texttt{  {REG,STRING,REG}     } \\
\texttt{ 33  } & \texttt{ APPENDCHAR  } & \texttt{  {REG,REG}            } \\
\texttt{ 34  } & \texttt{ TRIML       } & \texttt{  {REG,REG}            } \\
\texttt{ 35  } & \texttt{ TRIMR       } & \texttt{  {REG,REG}            } \\
\texttt{ 36  } & \texttt{ TRUNC       } & \texttt{  {REG,REG}            } \\
\texttt{ 37  } & \texttt{ TRIML       } & \texttt{  {REG,REG,REG}        } \\
\texttt{ 38  } & \texttt{ TRIMR       } & \texttt{  {REG,REG,REG}        } \\
\texttt{ 39  } & \texttt{ TRUNC       } & \texttt{  {REG,REG,REG}        } \\
\texttt{ 3A  } & \texttt{ STRLEN      } & \texttt{  {REG,REG}            } \\
\texttt{ 3B  } & \texttt{ STRCHAR     } & \texttt{  {REG,REG,REG}        } \\
\texttt{ 3C  } & \texttt{ STRCHAR     } & \texttt{  {REG,REG}            } \\
\texttt{ 3D  } & \texttt{ SETSTRPOS   } & \texttt{  {REG,REG}            } \\
\texttt{ 3E  } & \texttt{ GETSTRPOS   } & \texttt{  {REG,REG}            } \\
\texttt{ 3F  } & \texttt{ SUBSTR      } & \texttt{  {REG,REG,REG}        } \\
\texttt{ 40  } & \texttt{ IEQ         } & \texttt{  {REG,REG,REG}        } \\
\texttt{ 41  } & \texttt{ IEQ         } & \texttt{  {REG,REG,INT}        } \\
\texttt{ 42  } & \texttt{ INE         } & \texttt{  {REG,REG,REG}        } \\
\texttt{ 43  } & \texttt{ INE         } & \texttt{  {REG,REG,INT}        } \\
\texttt{ 44  } & \texttt{ IGT         } & \texttt{  {REG,REG,REG}        } \\
\texttt{ 45  } & \texttt{ IGT         } & \texttt{  {REG,REG,INT}        } \\
\texttt{ 46  } & \texttt{ IGT         } & \texttt{  {REG,INT,REG}        } \\
\texttt{ 47  } & \texttt{ IGTE        } & \texttt{  {REG,REG,REG}        } \\
\texttt{ 48  } & \texttt{ IGTE        } & \texttt{  {REG,REG,INT}        } \\
\texttt{ 49  } & \texttt{ IGTE        } & \texttt{  {REG,INT,REG}        } \\
\texttt{ 4A  } & \texttt{ ILT         } & \texttt{  {REG,REG,REG}        } \\
\texttt{ 4B  } & \texttt{ ILT         } & \texttt{  {REG,REG,INT}        } \\
\texttt{ 4C  } & \texttt{ ILT         } & \texttt{  {REG,INT,REG}        } \\
\texttt{ 4D  } & \texttt{ ILTE        } & \texttt{  {REG,REG,REG}        } \\
\texttt{ 4E  } & \texttt{ ILTE        } & \texttt{  {REG,REG,INT}        } \\
\texttt{ 4F  } & \texttt{ ILTE        } & \texttt{  {REG,INT,REG}        } \\
\texttt{ 50  } & \texttt{ FEQ         } & \texttt{  {REG,REG,REG}        } \\
\texttt{ 51  } & \texttt{ FEQ         } & \texttt{  {REG,REG,FLOAT}      } \\
\texttt{ 52  } & \texttt{ FNE         } & \texttt{  {REG,REG,REG}        } \\
\texttt{ 53  } & \texttt{ FNE         } & \texttt{  {REG,REG,FLOAT}      } \\
\texttt{ 54  } & \texttt{ FGT         } & \texttt{  {REG,REG,REG}        } \\
\texttt{ 55  } & \texttt{ FGT         } & \texttt{  {REG,REG,FLOAT}      } \\
\texttt{ 56  } & \texttt{ FGT         } & \texttt{  {REG,FLOAT,REG}      } \\
\texttt{ 57  } & \texttt{ FGTE        } & \texttt{  {REG,REG,REG}        } \\
\texttt{ 58  } & \texttt{ FGTE        } & \texttt{  {REG,REG,FLOAT}      } \\
\texttt{ 59  } & \texttt{ FGTE        } & \texttt{  {REG,FLOAT,REG}      } \\
\texttt{ 5A  } & \texttt{ FLT         } & \texttt{  {REG,REG,REG}        } \\
\texttt{ 5B  } & \texttt{ FLT         } & \texttt{  {REG,REG,FLOAT}      } \\
\texttt{ 5C  } & \texttt{ FLT         } & \texttt{  {REG,FLOAT,REG}      } \\
\texttt{ 5D  } & \texttt{ FLTE        } & \texttt{  {REG,REG,REG}        } \\
\texttt{ 5E  } & \texttt{ FLTE        } & \texttt{  {REG,REG,FLOAT}      } \\
\texttt{ 5F  } & \texttt{ FLTE        } & \texttt{  {REG,FLOAT,REG}      } \\
\texttt{ 60  } & \texttt{ SEQ         } & \texttt{  {REG,REG,REG}        } \\
\texttt{ 61  } & \texttt{ SEQ         } & \texttt{  {REG,REG,STRING}     } \\
\texttt{ 62  } & \texttt{ SNE         } & \texttt{  {REG,REG,REG}        } \\
\texttt{ 63  } & \texttt{ SNE         } & \texttt{  {REG,REG,STRING}     } \\
\texttt{ 64  } & \texttt{ SGT         } & \texttt{  {REG,REG,REG}        } \\
\texttt{ 65  } & \texttt{ SGT         } & \texttt{  {REG,REG,STRING}     } \\
\texttt{ 66  } & \texttt{ SGT         } & \texttt{  {REG,STRING,REG}     } \\
\texttt{ 67  } & \texttt{ SGTE        } & \texttt{  {REG,REG,REG}        } \\
\texttt{ 68  } & \texttt{ SGTE        } & \texttt{  {REG,REG,STRING}     } \\
\texttt{ 69  } & \texttt{ SGTE        } & \texttt{  {REG,STRING,REG}     } \\
\texttt{ 6A  } & \texttt{ SLT         } & \texttt{  {REG,REG,REG}        } \\
\texttt{ 6B  } & \texttt{ SLT         } & \texttt{  {REG,REG,STRING}     } \\
\texttt{ 6C  } & \texttt{ SLT         } & \texttt{  {REG,STRING,REG}     } \\
\texttt{ 6D  } & \texttt{ SLTE        } & \texttt{  {REG,REG,REG}        } \\
\texttt{ 6E  } & \texttt{ SLTE        } & \texttt{  {REG,REG,STRING}     } \\
\texttt{ 6F  } & \texttt{ SLTE        } & \texttt{  {REG,STRING,REG}     } \\
\texttt{ 70  } & \texttt{ AND         } & \texttt{  {REG,REG,REG}        } \\
\texttt{ 71  } & \texttt{ OR          } & \texttt{  {REG,REG,REG}        } \\
\texttt{ 72  } & \texttt{ TIME        } & \texttt{  {REG}                } \\
\texttt{ 73  } & \texttt{ MAP         } & \texttt{  {REG,REG}            } \\
\texttt{ 74  } & \texttt{ MAP         } & \texttt{  {REG,STRING}         } \\
\texttt{ 75  } & \texttt{ AMAP        } & \texttt{  {REG,REG}            } \\
\texttt{ 76  } & \texttt{ AMAP        } & \texttt{  {REG,INT}            } \\
\texttt{ 77  } & \texttt{ PMAP        } & \texttt{  {REG,REG}            } \\
\texttt{ 78  } & \texttt{ PMAP        } & \texttt{  {REG,STRING}         } \\
\texttt{ 79  } & \texttt{ GMAP        } & \texttt{  {REG,REG}            } \\
\texttt{ 7A  } & \texttt{ GMAP        } & \texttt{  {REG,STRING}         } \\
\texttt{ 7B  } & \texttt{ NSMAP       } & \texttt{  {REG,REG,REG}        } \\
\texttt{ 7C  } & \texttt{ NSMAP       } & \texttt{  {REG,REG,STRING}     } \\
\texttt{ 7D  } & \texttt{ NSMAP       } & \texttt{  {REG,STRING,STRING}  } \\
\texttt{ 7E  } & \texttt{ NSMAP       } & \texttt{  {REG,STRING,REG}     } \\
\texttt{ 7F  } & \texttt{ UNMAP       } & \texttt{  {REG}                } \\
\texttt{ 80  } & \texttt{ CALL        } & \texttt{  {FUNC}               } \\
\texttt{ 81  } & \texttt{ CALL        } & \texttt{  {REG,FUNC}           } \\
\texttt{ 82  } & \texttt{ CALL        } & \texttt{  {REG,FUNC,REG}       } \\
\texttt{ 83  } & \texttt{ RET         } & \texttt{  NO OPERAND           } \\
\texttt{ 84  } & \texttt{ RET         } & \texttt{  {REG}                } \\
\texttt{ 85  } & \texttt{ RET         } & \texttt{  {INT}                } \\
\texttt{ 86  } & \texttt{ RET         } & \texttt{  {FLOAT}              } \\
\texttt{ 87  } & \texttt{ RET         } & \texttt{  {CHAR}               } \\
\texttt{ 88  } & \texttt{ RET         } & \texttt{  {STRING}             } \\
\texttt{ 89  } & \texttt{ BR          } & \texttt{  {ID}                 } \\
\texttt{ 8A  } & \texttt{ BRT         } & \texttt{  {ID,REG}             } \\
\texttt{ 8B  } & \texttt{ BRF         } & \texttt{  {ID,REG}             } \\
\texttt{ 8C  } & \texttt{ MOVE        } & \texttt{  {REG,REG}            } \\
\texttt{ 8D  } & \texttt{ COPY        } & \texttt{  {REG,REG}            } \\
\texttt{ 8E  } & \texttt{ ICOPY       } & \texttt{  {REG,REG}            } \\
\texttt{ 8F  } & \texttt{ LINK        } & \texttt{  {REG,REG}            } \\
\texttt{ 90  } & \texttt{ UNLINK      } & \texttt{  {REG}                } \\
\texttt{ 91  } & \texttt{ NULL        } & \texttt{  {REG}                } \\
\texttt{ 92  } & \texttt{ LOAD        } & \texttt{  {REG,INT}            } \\
\texttt{ 93  } & \texttt{ LOAD        } & \texttt{  {REG,FLOAT}          } \\
\texttt{ 94  } & \texttt{ LOAD        } & \texttt{  {REG,STRING}         } \\
\texttt{ 95  } & \texttt{ LOAD        } & \texttt{  {REG,CHAR}           } \\
\texttt{ 96  } & \texttt{ SAY         } & \texttt{  {REG}                } \\
\texttt{ 97  } & \texttt{ SSAY        } & \texttt{  {REG}                } \\
\texttt{ 98  } & \texttt{ SAY         } & \texttt{  {INT}                } \\
\texttt{ 99  } & \texttt{ SAY         } & \texttt{  {FLOAT}              } \\
\texttt{ 9A  } & \texttt{ SAY         } & \texttt{  {STRING}             } \\
\texttt{ 9B  } & \texttt{ SAY         } & \texttt{  {CHAR}               } \\
\texttt{ 9C  } & \texttt{ EXIT        } & \texttt{  NO OPERAND           } \\
\texttt{ 9D  } & \texttt{ EXIT        } & \texttt{  {REG}                } \\
\texttt{ 9E  } & \texttt{ EXIT        } & \texttt{  {INT}                } \\
\texttt{ 9F  } & \texttt{ ITOS        } & \texttt{  {REG}                } \\
\texttt{ A0  } & \texttt{ FTOS        } & \texttt{  {REG}                } \\
\texttt{ A1  } & \texttt{ ITOF        } & \texttt{  {REG}                } \\

\end{longtable}


\newcommand{\bibTitle}[1]{``#1''}

\begin{thebibliography}{9}

  % Terence Parr, Language Implementation Patterns
  % Mike Cowlishaw, The Rexx Language
  % ANSI/INCITS J318, The Rexx Language Standard
  
% \bibitem{tlshandling}
%   Ulrich Drepper,
%   \bibTitle{ELF Handling For Thread-Local Storage,}
%   \url{https://akkadia.org/drepper/tls.pdf};
%   2013
% \bibitem{sysvabi}
%   \bibTitle{System V Application Binary Interface,}
%   edition 4.1,
%   \url{http://www.sco.com/developers/devspecs/gabi41.pdf};
%   1997
% \bibitem{sysvabidraft}
%   \bibTitle{System V Application Binary Interface,}
%   chapters 4 and 5, latest snapshot,
%   \url{http://www.sco.com/developers/gabi/latest/contents.html};
%   2013
% \bibitem{gnu-vec}
%   \bibTitle{Using the GNU Compiler Collection,}
%   vector extensions,
%   \url{https://gcc.gnu.org/onlinedocs/gcc/Vector-Extensions.html}
% \bibitem{sa22}
%   \bibTitle{\ARCH{} Principles of Operation,}
%   IBM Publication No.~SA22-{\ifzseries 7832-12\else 7201-08\fi};
%   2019
\end{thebibliography}

\printindex

\end{document}
