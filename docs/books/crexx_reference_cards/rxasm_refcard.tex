\documentclass[10pt]{article}
\usepackage{supertabular}
\usepackage{ascii}
\usepackage[orthodox,l2tabu,abort]{nag}
\usepackage{xcolor}
\definecolor{nrblue}{RGB}{38,139,210}
\definecolor{nrgreen}{RGB}{65,133,153}
\definecolor{nrcyan}{RGB}{42,161,152 }
\definecolor{nrorange}{RGB}{203 ,75,22}
\definecolor{nrgrey}{RGB}{101,123,131}

\usepackage{pagecolor}
\pagecolor{yellow!50}
\usepackage{auto-pst-pdf,pst-barcode}
% Page layout
\usepackage[landscape,margin=0.5in]{geometry}
\usepackage{multicol}
 \setlength{\columnsep}{20pt}
% Title area
\usepackage{titling} % Allows for use of date, author, etc. after \maketitle
% Ref: http://tex.stackexchange.com/questions/3988/titlesec-versus-titling-mangling-thetitle
\let\oldtitle\title
\renewcommand{\title}[1]{\oldtitle{#1}\newcommand{\mythetitle}{#1}}
\renewcommand{\maketitle}{%
{\begin{center}\Large \mythetitle\end{center}}
}
% fonts
\usepackage{fontspec}

\setmainfont[Mapping=tex-text]{TeX Gyre Termes}
\setmonofont[Mapping=tex-text,Scale=0.80]{Source Code Pro}

\newcommand{\cRexx}{cR\textsc{exx}}
\newcommand{\Rexx}{R\textsc{exx}}
\newcommand{\RxAsm}{R\textsc{xAsm}}

\usepackage{color}
\usepackage{xcolor}
\usepackage{listings}
\lstdefinelanguage{rxas}
{morekeywords={
appendchar,
addf,
addi,
addi,
amap,
amap,
and,
append,
addf,
bcf,
bcf,
bct,
bct,
bctnm,
bctnm,
beq,
bge,
bge,
bgt,
bgt,
ble,
ble,
blt,
blt,
bne,
bne,
bpoff,
bpon,
br,
brf,
brt,
brtpandt,
beq,
brtpt,
brtf,
call,
cnop,
call,
concchar,
concat,
concat,
concat,
call,
copy,
divf,
dec,
dec0,
dcall,
dec2,
dec1,
divf,
divi,
divi,
dllparms,
dropchar,
divf,
exit,
exit,
exit,
fadd,
fadd,
fcopy,
fdiv,
fdiv,
fdiv,
feq,
fformat,
fgt,
fgt,
fgt,
fgte,
fgte,
fgte,
flt,
flt,
flt,
flte,
flte,
flte,
fmult,
fmult,
fndblnk,
fndnblnk,
fne,
fne,
fpow,
fpow,
fsex,
fsub,
fsub,
fsub,
ftob,
ftoi,
ftos,
feq,
getstrpos,
gmap,
gettp,
gmap,
getbyte,
hexchar,
iadd,
iadd,
igt,
iand,
icopy,
idiv,
idiv,
idiv,
ieq,
ieq,
igt,
igte,
iand,
igte,
igte,
ilt,
ilt,
ilt,
ilte,
ilte,
ilte,
imod,
imod,
imod,
imult,
imult,
inc,
inc0,
inc1,
ine,
igt,
ine,
inot,
inot,
ior,
ior,
ipow,
ipow,
ipow,
irand,
irand,
isex,
ishl,
ishl,
ishr,
ishr,
isub,
isub,
isub,
itof,
itos,
ixor,
ixor,
inc2,
linkattr,
load,
linkattr,
loadsettp,
load,
load,
load,
link,
loadsettp,
loadsettp,
metaloadedmodules,
map,
map,
metadecodeinst,
metaloadcalleraddr,
metaloaddata,
metaloadedprocs,
metaloadfoperand,
metaloadinst,
metaloadioperand,
metaloadmodule,
metaloadpoperand,
metaloadsoperand,
move,
mtime,
multf,
multi,
metalinkpreg,
multi,
multf,
nsmap,
nsmap,
nsmap,
null,
nsmap,
not,
opendll,
or,
padstr,
pmap,
pmap,
poschar,
ret,
ret,
ret,
ret,
ret,
ret,
rseq,
rseq,
readline,
sappend,
say,
say,
say,
say,
sayx,
sayx,
sconcat,
sconcat,
sconcat,
scopy,
seq,
seq,
setortp,
setstrpos,
settp,
sgt,
sgt,
sgt,
sgte,
sgte,
sgte,
slt,
slt,
slt,
slte,
slte,
slte,
sne,
sne,
stof,
stoi,
strchar,
strchar,
strlen,
strlower,
strupper,
subf,
subf,
subf,
subi,
subi,
substcut,
substr,
substring,
swap,
say,
time,
transchar,
triml,
triml,
trimr,
trimr,
trunc,
trunc,
unlink,
unmap,
xtime,
  },
sensitive=false,
extendedchars=true,
morecomment=[s]={/*}{*/},
morecomment=[l]{*},
morecomment=[s]{/**}{*/},
morestring=[b]",
morestring=[d]",
morestring=[b]',
morestring=[d]'}

\lstset{language=rxas,
  captionpos=none,
  tabsize=3,
  keywordstyle=\color{nrorange},
  commentstyle=\color{nrgrey},
  stringstyle=\color{nrgreen},
  numbers=left,
  numberstyle=\tiny,
  numbersep=5pt,
  breaklines=true,
  showstringspaces=false,
  index=[1][keywords],
  columns=fixed,
  basicstyle=\fontsize{10}{10}\fontspec{Hack},emph={label}
}


%for leader dots in the TOC
\usepackage{tocloft}
\renewcommand{\cftsecleader}{\cftdotfill{\cftdotsep}}

% Document divisions
\usepackage{titlesec}
\setcounter{secnumdepth}{0}
\titlespacing{\section}{0pt}{0pt}{0pt}
\titlespacing{\subsection}{0pt}{0pt}{0pt}
\usepackage{nopageno} % To keep \section from resetting page style

\setlength{\parindent}{0pt} % disabling indentation by default

% Lists
\usepackage{enumitem} % for consistent formatting of lists
\newlist{ttdesc}{description}{1}
\setlist[ttdesc]{font=\ttfamily,noitemsep}
\usepackage{calc} % for \widthof

% Code
\usepackage{listings}
\lstset{language=[LaTeX]TeX,%
  basicstyle=\itshape,%
  keywordstyle=\normalfont\ttfamily,%
  morekeywords={part,chapter,subsection,subsubsection,paragraph,subparagraph}%
  }

 %this formats the description headings
\renewcommand{\descriptionlabel}[1]{\hspace{\labelsep}\texttt{#1}}

\title{\fontspec{Helvetica}\textbf{RXASM Reference Summary}}
\author{The \cRexx{} project}
\date{2021}


% ----------------------------
% to make supertabular work in multicolumn)
\newcount\n
\n=0
\def\tablebody{}
\makeatletter
\loop\ifnum\n<100
        \advance\n by1
        \protected@edef\tablebody{\tablebody
                \textbf{\number\n.}& shortText
                \tabularnewline
        }
\repeat

\makeatletter
\let\mcnewpage=\newpage
\newcommand{\TrickSupertabularIntoMulticols}{%
  \renewcommand\newpage{%
    \if@firstcolumn
      \hrule width\linewidth height0pt
      \columnbreak
    \else
      \mcnewpage
    \fi
  }%
}
\makeatother
%------------------------------


\begin{document}
\begin{multicols}{3}
  \TrickSupertabularIntoMulticols
  \lstset{language=rxas}
  \lstset{basicstyle=\ttfamily}
\hspace*{4cm}  \maketitle


  \vspace*{3cm}
\hspace*{-1.5cm}\rule{9.95cm}{1cm}




  
\hrulefill 

\textbf{First Edition (July 2021) - for cREXX-Phase-0 v0.1.6}


\tableofcontents
\hrulefill

\RxAsm{} assembly language is generated by the \emph{rxc} \cRexx{} translator from
\Rexx{} source and produces binary .rxbin files, when
assembled using the \emph{rxas} assembler. It can be used to write
programs to be interpreted by the \emph{rxvm} virtual machine. This
reference summary contains the machine and assembler language statements and a
short introduction to the use of the \cRexx{} commandline language tools.

\section{\fontspec{Helvetica} Machine Instructions}
These are instructions for the \emph{rxvm} virtual machine, which is 
implemented on various operating environments.

\raggedright
\begin{ttdesc}[labelwidth=\widthof{\ttfamily{letterpaper/a4paper}}]
\item[ASSEMBLY IN  STRUCTION List       ]        
\item[ADDF         {REG,REG,REG}        ]        Convert and Add to Float (op1=op2+op3) (Deprecated)
\item[ADDF         {REG,REG,FLOAT}      ]        Convert and Add to Float (op1=op2+op3) (Deprecated)
\item[ADDI         {REG,REG,REG}        ]        Convert and Add to Integer (op1=op2+op3) (Deprecated)
\item[ADDI         {REG,REG,INT}        ]        Convert and Add to Integer (op1=op2+op3) (Deprecated)
\item[AMAP         {REG,REG}            ]        Map op1 to arg register index in op2
\item[AMAP         {REG,INT}            ]        Map op1 to arg register index  op2
\item[AND          {REG,REG,REG}        ]        Logical (int) and op1=(op2 \&\& op3)
\item[APPEND       {REG,REG}            ]        String Append (op1=op1||op2)
\item[APPENDCHAR   {REG,REG}            ]        Append Concat Char op2 (as int) on op1
\item[BCF          {ID,REG}             ]        if op2=0 goto op1(if false) else dec op2
\item[BCF          {ID,REG,REG}         ]        if op2=0 goto op1(if false) else dec op2 and inc op3
\item[BCT          {ID,REG}             ]        dec op2; if op2>0; goto op1(if true)
\item[BCT          {ID,REG,REG}         ]        dec op2; inc op3, if op2>0; goto op1(if true)
\item[BCTNM        {ID,REG}             ]        dec op2; if op2>=0; goto op1(if true)
\item[BCTNM        {ID,REG,REG}         ]        dec op2; inc op3, if op2>=0; goto op1(if true)
\item[BEQ          {ID,REG,REG}         ]        if op2==op3 then goto op1
\item[BEQ          {ID,REG,INT}         ]        if op2==op3 then goto op1
\item[BGE          {ID,REG,REG}         ]        if op2>=op3 then goto op1
\item[BGE          {ID,REG,INT}         ]        if op2>=op3 then goto op1
\item[BGT          {ID,REG,REG}         ]        if op2>op3 then goto op1
\item[BGT          {ID,REG,INT}         ]        if op2>op3 then goto op1
\item[BLE          {ID,REG,REG}         ]        if op2<=op3 then goto op1
\item[BLE          {ID,REG,INT}         ]        if op2<=op3 then goto op1
\item[BLT          {ID,REG,REG}         ]        if op2<op3 then goto op1
\item[BLT          {ID,REG,INT}         ]        if op2<op3 then goto op1
\item[BNE          {ID,REG,REG}         ]        if op2!=op3 then goto op1
\item[BNE          {ID,REG,INT}         ]        if op2!=op3 then goto op1
\item[BR           {ID}                 ]        Branch to op1
\item[BRF          {ID,REG}             ]        Branch to op1 if op2 false
\item[BRT          {ID,REG}             ]        Branch to op1 if op2 true
\item[BRTF         {ID,ID,REG}          ]        Branch to op1 if op3 true, otherwise branch to op2
\item[BRTPANDT     {ID,REG,INT}         ]        if op2.typeflag \&\& op3 true then goto op1
\item[BRTPT        {ID,REG}             ]        if op2.typeflag true then goto op1
\item[CALL         {REG,FUNC}           ]        Call procedure (op1=op2())
\item[CALL         {REG,FUNC,REG}       ]        Call procedure (op1=op2(op3...) )
\item[CALL         {FUNC}               ]        Call procedure (op1())
\item[CNOP         NO OPERAND           ]        no operation
\item[CONCAT       {REG,REG,REG}        ]        String Concat (op1=op2||op3)
\item[CONCAT       {REG,REG,STRING}     ]        String Concat (op1=op2||op3)
\item[CONCAT       {REG,STRING,REG}     ]        String Concat (op1=op2||op3)
\item[CONCCHAR     {REG,REG,REG}        ]        Concat Char op1 from op2 position op3
\item[COPY         {REG,REG}            ]        Copy op2 to op1
\item[DEC          {REG}                ]        Decrement Int (op1--)
\item[DEC0         NO OPERAND           ]        Decrement R0-- Int
\item[DEC1         NO OPERAND           ]        Decrement R0-- Int
\item[DEC2         NO OPERAND           ]        Decrement R0-- Int
\item[DIVF         {REG,REG,REG}        ]        Convert and Divide to Float (op1=op2/op3) (Deprecated)
\item[DIVF         {REG,REG,FLOAT}      ]        Convert and Divide to Float (op1=op2/op3) (Deprecated)
\item[DIVF         {REG,FLOAT,REG}      ]        Convert and Divide to Float (op1=op2/op3) (Deprecated)
\item[DIVI         {REG,REG,REG}        ]        Convert and Divide to Integer (op1=op2/op3) (Deprecated)
\item[DIVI         {REG,REG,INT}        ]        Convert and Divide to Integer (op1=op2/op3) (Deprecated)
\item[DROPCHAR     {REG,REG,REG}        ]        set op1 from op2 after dropping all chars from op3
\item[EXIT         NO OPERAND           ]        Exit
\item[EXIT         {REG}                ]        Exit op1
\item[EXIT         {INT}                ]        Exit op1
\item[FADD         {REG,REG,REG}        ]        Float Add (op1=op2+op3)
\item[FADD         {REG,REG,FLOAT}      ]        Float Add (op1=op2+op3)
\item[FCOPY        {REG,REG}            ]        Copy Float op2 to op1
\item[FDIV         {REG,REG,REG}        ]        Float Divide (op1=op2/op3)
\item[FDIV         {REG,REG,FLOAT}      ]        Float Divide (op1=op2/op3)
\item[FDIV         {REG,FLOAT,REG}      ]        Float Divide (op1=op2/op3)
\item[FEQ          {REG,REG,REG}        ]        Float Equals op1=(op2==op3)
\item[FEQ          {REG,REG,FLOAT}      ]        Float Equals op1=(op2==op3)
\item[FFORMAT      {REG,REG,REG}        ]        Set string value from float value using a format string
\item[FGT          {REG,REG,REG}        ]        Float Greater than op1=(op2>op3)
\item[FGT          {REG,REG,FLOAT}      ]        Float Greater than op1=(op2>op3)
\item[FGT          {REG,FLOAT,REG}      ]        Float Greater than op1=(op2>op3)
\item[FGTE         {REG,REG,REG}        ]        Float Greater than equals op1=(op2>=op3)
\item[FGTE         {REG,REG,FLOAT}      ]        Float Greater than equals op1=(op2>=op3)
\item[FGTE         {REG,FLOAT,REG}      ]        Float Greater than equals op1=(op2>=op3)
\item[FLT          {REG,REG,REG}        ]        Float Less than op1=(op2<op3)
\item[FLT          {REG,REG,FLOAT}      ]        Float Less than op1=(op2<op3)
\item[FLT          {REG,FLOAT,REG}      ]        Float Less than op1=(op2<op3)
\item[FLTE         {REG,REG,REG}        ]        Float Less than equals op1=(op2<=op3)
\item[FLTE         {REG,REG,FLOAT}      ]        Float Less than equals op1=(op2<=op3)
\item[FLTE         {REG,FLOAT,REG}      ]        Float Less than equals op1=(op2<=op3)
\item[FMULT        {REG,REG,REG}        ]        Float Multiply (op1=op2*op3)
\item[FMULT        {REG,REG,FLOAT}      ]        Float Multiply (op1=op2*op3)
\item[FNDBLNK      {REG,REG,REG}        ]        op1 = find next blank in op2[op3] and behind
\item[FNDNBLNK     {REG,REG,REG}        ]        op1 = find next next non blank in op2[op3] and behind
\item[FNE          {REG,REG,REG}        ]        Float Not equals op1=(op2!=op3)
\item[FNE          {REG,REG,FLOAT}      ]        Float Not equals op1=(op2!=op3)
\item[FSEX         {REG}                ]        float op1 = -op1 (sign change)
\item[FSUB         {REG,REG,REG}        ]        Float Subtract (op1=op2-op3)
\item[FSUB         {REG,REG,FLOAT}      ]        Float Subtract (op1=op2-op3)
\item[FSUB         {REG,FLOAT,REG}      ]        Float Subtract (op1=op2-op3)
\item[FTOB         {REG}                ]        Set register boolean (int 1 or 0) value from its float value
\item[FTOI         {REG}                ]        Set register int value from its float value
\item[FTOS         {REG}                ]        Set register string value from its float value
\item[GETBYTE      {REG,REG,REG}        ]        get byte  (op1=op2(op3)
\item[GETSTRPOS    {REG,REG}            ]        Get String (op2) charpos into op1
\item[GETTP        {REG,REG}            ]        gets the register type flag (op1 = op2.typeflag)
\item[GMAP         {REG,REG}            ]        Map op1 to global var name in op2
\item[GMAP         {REG,STRING}         ]        Map op1 to global var name op2
\item[HEXCHAR      {REG,REG,REG}        ]        op1 (as hex) = op2[op3]
\item[IADD         {REG,REG,REG}        ]        Integer Add (op1=op2+op3)
\item[IADD         {REG,REG,INT}        ]        Integer Add (op1=op2+op3)
\item[IAND         {REG,REG,REG}        ]        bit wise and of 2 integers (op1=op2\&op3)
\item[IAND         {REG,REG,INT}        ]        bit wise and of 2 integers (op1=op2\&op3)
\item[ICOPY        {REG,REG}            ]        Copy Integer op2 to op1
\item[IDIV         {REG,REG,REG}        ]        Integer Divide (op1=op2/op3)
\item[IDIV         {REG,REG,INT}        ]        Integer Divide (op1=op2/op3)
\item[IDIV         {REG,INT,REG}        ]        Integer Divide (op1=op2/op3)
\item[IEQ          {REG,REG,REG}        ]        Int Equals op1=(op2==op3)
\item[IEQ          {REG,REG,INT}        ]        Int Equals op1=(op2==op3)
\item[IGT          {REG,REG,REG}        ]        Int Greater than op1=(op2>op3)
\item[IGT          {REG,REG,INT}        ]        Int Greater than op1=(op2>op3)
\item[IGT          {REG,INT,REG}        ]        Int Greater than op1=(op2>op3)
\item[IGTE         {REG,REG,REG}        ]        Int Greater than equals op1=(op2>=op3)
\item[IGTE         {REG,REG,INT}        ]        Int Greater than equals op1=(op2>=op3)
\item[IGTE         {REG,INT,REG}        ]        Int Greater than equals op1=(op2>=op3)
\item[ILT          {REG,REG,REG}        ]        Int Less than op1=(op2<op3)
\item[ILT          {REG,REG,INT}        ]        Int Less than op1=(op2<op3)
\item[ILT          {REG,INT,REG}        ]        Int Less than op1=(op2<op3)
\item[ILTE         {REG,REG,REG}        ]        Int Less than equals op1=(op2<=op3)
\item[ILTE         {REG,REG,INT}        ]        Int Less than equals op1=(op2<=op3)
\item[ILTE         {REG,INT,REG}        ]        Int Less than equals op1=(op2<=op3)
\item[IMOD         {REG,REG,REG}        ]        Integer Modulo (op1=op2%op3)
\item[IMOD         {REG,REG,INT}        ]        Integer Modulo (op1=op2%op3)
\item[IMOD         {REG,INT,REG}        ]        Integer Modulo (op1=op2\&op3)
\item[IMULT        {REG,REG,REG}        ]        Integer Multiply (op1=op2*op3)
\item[IMULT        {REG,REG,INT}        ]        Integer Multiply (op1=op2*op3)
\item[INC          {REG}                ]        Increment Int (op1++)
\item[INC0         NO OPERAND           ]        Increment R0++ Int
\item[INC1         NO OPERAND           ]        Increment R0++ Int
\item[INC2         NO OPERAND           ]        Increment R0++ Int
\item[INE          {REG,REG,REG}        ]        Int Not equals op1=(op2!=op3)
\item[INE          {REG,REG,INT}        ]        Int Not equals op1=(op2!=op3)
\item[INOT         {REG,REG}            ]        inverts all bits of an integer (op1=~op2)
\item[INOT         {REG,INT}            ]        inverts all bits of an integer (op1=~op2)
\item[IOR          {REG,REG,REG}        ]        bit wise or of 2 integers (op1=op2|op3)
\item[IOR          {REG,REG,INT}        ]        bit wise or of 2 integers (op1=op2|op3)
\item[IPOW         {REG,REG,REG}        ]        op1=op2**op3
\item[IPOW         {REG,REG,INT}        ]        op1=op2**op3
\item[ISEX         {REG}                ]        dec op1 = -op1 (sign change)
\item[ISHL         {REG,REG,REG}        ]        bit wise shift logical left of integer (op1=op2<<op3)
\item[ISHL         {REG,REG,INT}        ]        bit wise shift logical left of integer (op1=op2<<op3)
\item[ISHR         {REG,REG,REG}        ]        bit wise shift logical right of integer (op1=op2>>op3)
\item[ISHR         {REG,REG,INT}        ]        bit wise shift logical right of integer (op1=op2>>op3)
\item[ISUB         {REG,REG,REG}        ]        Integer Subtract (op1=op2-op3)
\item[ISUB         {REG,REG,INT}        ]        Integer Subtract (op1=op2-op3)
\item[ISUB         {REG,INT,REG}        ]        Integer Subtract (op1=op2-op3)
\item[ITOF         {REG}                ]        Set register float value from its int value
\item[ITOS         {REG}                ]        Set register string value from its int value
\item[IXOR         {REG,REG,REG}        ]        bit wise exclusive OR of 2 integers (op1=op2^op3)
\item[IXOR         {REG,REG,INT}        ]        bit wise exclusive OR of 2 integers (op1=op2^op3)
\item[LINK         {REG,REG}            ]        Link op2 to op1
\item[LOAD         {REG,INT}            ]        Load op1 with op2
\item[LOAD         {REG,FLOAT}          ]        Load op1 with op2
\item[LOAD         {REG,CHAR}           ]        Load op1 with op2
\item[LOAD         {REG,STRING}         ]        Load op1 with op2
\item[LOADSETTP    {REG,INT,INT}        ]        load register and sets the register type flag load op1=op2 (op1.typeflag = op3)
\item[LOADSETTP    {REG,FLOAT,INT}      ]        load register and sets the register type flag load op1=op2 (op1.typeflag = op3)
\item[LOADSETTP    {REG,STRING,INT}     ]        load register and sets the register type flag load op1=op2 (op1.typeflag = op3)
\item[MAP          {REG,REG}            ]        Map op1 to var name in op2
\item[MAP          {REG,STRING}         ]        Map op1 to var name op2
\item[MOVE         {REG,REG}            ]        Move op2 to op1 (Deprecated use swap)
\item[MTIME        {REG}                ]        Put time in microseconds into op1
\item[MULTF        {REG,REG,REG}        ]        Convert and Multiply to Float (op1=op2*op3) (Deprecated)
\item[MULTF        {REG,REG,FLOAT}      ]        Convert and Multiply to Float (op1=op2*op3) (Deprecated)
\item[MULTI        {REG,REG,REG}        ]        Convert and Multiply to Integer (op1=op2*op3) (Deprecated)
\item[MULTI        {REG,REG,INT}        ]        Convert and Multiply to Integer (op1=op2*op3) (Deprecated)
\item[NOT          {REG,REG}            ]        Logical (int) not op1=!op2
\item[NSMAP        {REG,REG,REG}        ]        Map op1 to namespace in op2 var name in op3
\item[NSMAP        {REG,REG,STRING}     ]        Map op1 to namespace in op2 var name op3
\item[NSMAP        {REG,STRING,REG}     ]        Map op1 to namespace op2 var name in op3
\item[NSMAP        {REG,STRING,STRING}  ]        Map op1 to namespace op2 var name op3
\item[NULL         {REG}                ]        Null op1
\item[OR           {REG,REG,REG}        ]        Logical (int) or op1=(op2 || op3)
\item[PADSTR       {REG,REG,REG}        ]        set op1=op2[repeated op3 times]
\item[PMAP         {REG,REG}            ]        Map op1 to parent var name in op2
\item[PMAP         {REG,STRING}         ]        Map op1 to parent var name op2
\item[POSCHAR      {REG,REG,REG}        ]        op1 = position of op3 in op2
\item[RET          NO OPERAND           ]        Return VOID
\item[RET          {REG}                ]        Return op1
\item[RET          {INT}                ]        Return op1
\item[RET          {FLOAT}              ]        Return op1
\item[RET          {CHAR}               ]        Return op1
\item[RET          {STRING}             ]        Return op1
\item[RSEQ         {REG,REG,REG}        ]        non strict String Equals op1=(op2=op3)
\item[RSEQ         {REG,REG,STRING}     ]        non strict String Equals op1=(op2=op3)
\item[SAPPEND      {REG,REG}            ]        String Append with space (op1=op1||op2)
\item[SAY          {REG}                ]        Say op1 (as string)
\item[SAY          {INT}                ]        Say op1
\item[SAY          {FLOAT}              ]        Say op1
\item[SAY          {CHAR}               ]        Say op1
\item[SAY          {STRING}             ]        Say op1
\item[SCONCAT      {REG,REG,REG}        ]        String Concat with space (op1=op2||op3)
\item[SCONCAT      {REG,REG,STRING}     ]        String Concat with space (op1=op2||op3)
\item[SCONCAT      {REG,STRING,REG}     ]        String Concat with space (op1=op2||op3)
\item[SCOPY        {REG,REG}            ]        Copy String op2 to op1
\item[SEQ          {REG,REG,REG}        ]        String Equals op1=(op2==op3)
\item[SEQ          {REG,REG,STRING}     ]        String Equals op1=(op2==op3)
\item[SETORTP      {REG,INT}            ]        or the register type flag (op1.typeflag = op1.typeflag || op2)
\item[SETSTRPOS    {REG,REG}            ]        Set String (op1) charpos set to op2
\item[SETTP        {REG,INT}            ]        sets the register type flag (op1.typeflag = op2)
\item[SGT          {REG,REG,REG}        ]        String Greater than op1=(op2>op3)
\item[SGT          {REG,REG,STRING}     ]        String Greater than op1=(op2>op3)
\item[SGT          {REG,STRING,REG}     ]        String Greater than op1=(op2>op3)
\item[SGTE         {REG,REG,REG}        ]        String Greater than equals op1=(op2>=op3)
\item[SGTE         {REG,REG,STRING}     ]        String Greater than equals op1=(op2>=op3)
\item[SGTE         {REG,STRING,REG}     ]        String Greater than equals op1=(op2>=op3)
\item[SLT          {REG,REG,REG}        ]        String Less than op1=(op2<op3)
\item[SLT          {REG,REG,STRING}     ]        String Less than op1=(op2<op3)
\item[SLT          {REG,STRING,REG}     ]        String Less than op1=(op2<op3)
\item[SLTE         {REG,REG,REG}        ]        String Less than equals op1=(op2<=op3)
\item[SLTE         {REG,REG,STRING}     ]        String Less than equals op1=(op2<=op3)
\item[SLTE         {REG,STRING,REG}     ]        String Less than equals op1=(op2<=op3)
\item[SNE          {REG,REG,REG}        ]        String Not equals op1=(op2!=op3)
\item[SNE          {REG,REG,STRING}     ]        String Not equals op1=(op2!=op3)
\item[SSAY         {REG}                ]        String Say op1 (Deprecated use 'say reg')
\item[STOF         {REG}                ]        Set register float value from its string value
\item[STOI         {REG}                ]        Set register int value from its string value
\item[STRCHAR      {REG,REG}            ]        op1 (as int) = op2[charpos]
\item[STRCHAR      {REG,REG,REG}        ]        op1 (as int) = op2[op3]
\item[STRLEN       {REG,REG}            ]        String Length op1 = length(op2)
\item[STRLOWER     {REG,REG}            ]        Set string to lower case value
\item[STRUPPER     {REG,REG}            ]        Set string to upper case value
\item[SUBF         {REG,REG,REG}        ]        Convert and Subtract to Float (op1=op2-op3) (Deprecated)
\item[SUBF         {REG,REG,FLOAT}      ]        Convert and Subtract to Float (op1=op2-op3) (Deprecated)
\item[SUBF         {REG,FLOAT,REG}      ]        Convert and Subtract to Float (op1=op2-op3) (Deprecated)
\item[SUBI         {REG,REG,REG}        ]        Convert and Subtract to Integer (op1=op2-op3) (Deprecated)
\item[SUBI         {REG,REG,INT}        ]        Convert and Subtract to Integer (op1=op2-op3) (Deprecated)
\item[SUBSTCUT     {REG,REG}            ]        set op1=substr(op1,,op2) cuts off op1 after position op3
\item[SUBSTR       {REG,REG,REG}        ]        op1 = op2[charpos]...op2[charpos+op3-1]
\item[SUBSTRING    {REG,REG,REG}        ]        set op1=substr(op2,op3) remaining string
\item[SWAP         {REG,REG}            ]        Swap op1 and op2
\item[TIME         {REG}                ]        Put time into op1
\item[TRANSCHAR    {REG,REG,REG}        ]        replace op1 if it is in op3-list by char in op2-list
\item[TRIML        {REG,REG}            ]        Trim String (op1) from Left by (op2) Chars
\item[TRIML        {REG,REG,REG}        ]        Trim String (op2) from Left by (op3) Chars into op1
\item[TRIMR        {REG,REG}            ]        Trim String (op1) from Right by (op2) Chars
\item[TRIMR        {REG,REG,REG}        ]        Trim String (op2) from Right by (op3) Chars into op1
\item[TRUNC        {REG,REG}            ]        Trunc String (op1) to (op2) Chars
\item[TRUNC        {REG,REG,REG}        ]        Trunc String (op2) to (op3) Chars into op1
\item[UNLINK       {REG}                ]        Unlink op1
\item[UNMAP        {REG}                ]        Unmap op1
\item[XTIME        {REG,STRING}         ]        put special time properties into op1

\end{ttdesc}
\section{\fontspec{Helvetica} Commandline Tools}
\raggedright
\begin{ttdesc}[labelwidth=\widthof{\ttfamily{letterpaper/a4paper}}]
\item[rxc]  translate a .rexx source file to a .rxasm source file
\item[rxas]  compile a .rxasm source file to an .rxbin binary file
\item[rxvm] run a .rxbin binary in the threaded interpreter (highest performance)
\item[rxbvm] run a .rxbin binary in the conventional interpreter (for
  VM/370)
\item[rxdas] disassemble a .rxbin binary to an .rxasm source file
\end{ttdesc}
To run a program consisting of multiple dependent functions, run
\emph{rxvm} with the first qualifiers of all required .rxbin files;
these will be automatically resolved.

\section{\fontspec{Helvetica} Assembler Instructions}
Each line has an optional \emph{label}, an instruction (machine or
assembler) and a number of operands (0 or more) of different types
(see next paragraph).

\begin{ttdesc}[labelwidth=\widthof{\ttfamily{letterpaper/a4paper}}]
\item[*]  comment until end of line
\item[.globals=\emph{n}] specifies the number of globals in this
  source file
\item[.expose] expose (make globally known) the preceding label
\item[.locals]  specifies the number of locals in this source file or
  function
\end{ttdesc}


\section{\fontspec{Helvetica} Operand Types}
\begin{ttdesc}[labelwidth=\widthof{\ttfamily{letterpaper/a4paper}}]
\item[REG] a register
\item[STRING] a sequence of Unicode characters
\item[FUNC] a global or local function
\item[ID] a label
\item[INT] an integer type (a whole number)
\item[FLOAT] a floating point number
\end{ttdesc}



\section{\fontspec{Helvetica} Calling functions}
\section{\fontspec{Helvetica} Limits}
\pagebreak
\section{\fontspec{Helvetica} Instructions by Opcode}
\begin{supertabular}{|lll|}
  \tabletail{\emph{cont. ...}}
  \tablelasttail{\emph{end}}
  \texttt{Opcode} & \texttt{Instruction} & \texttt{parameters} \\
  % \shrinkheight{2cm}
\texttt{     } & \texttt{             } & \texttt{                       } \\
\texttt{     } & \texttt{             } & \texttt{                       } \\
\texttt{ XX  } & \texttt{ ASSEMBLY IN } & \texttt{ NSTRUCTION List       } \\
\texttt{ 01  } & \texttt{ IADD        } & \texttt{  {REG,REG,REG}        } \\
\texttt{ 02  } & \texttt{ IADD        } & \texttt{  {REG,REG,INT}        } \\
\texttt{ 03  } & \texttt{ ADDI        } & \texttt{  {REG,REG,REG}        } \\
\texttt{ 04  } & \texttt{ ADDI        } & \texttt{  {REG,REG,INT}        } \\
\texttt{ 05  } & \texttt{ ISUB        } & \texttt{  {REG,REG,REG}        } \\
\texttt{ 06  } & \texttt{ ISUB        } & \texttt{  {REG,REG,INT}        } \\
\texttt{ 07  } & \texttt{ SUBI        } & \texttt{  {REG,REG,REG}        } \\
\texttt{ 08  } & \texttt{ SUBI        } & \texttt{  {REG,REG,INT}        } \\
\texttt{ 09  } & \texttt{ IMULT       } & \texttt{  {REG,REG,REG}        } \\
\texttt{ 0A  } & \texttt{ IMULT       } & \texttt{  {REG,REG,INT}        } \\
\texttt{ 0B  } & \texttt{ MULTI       } & \texttt{  {REG,REG,REG}        } \\
\texttt{ 0C  } & \texttt{ MULTI       } & \texttt{  {REG,REG,INT}        } \\
\texttt{ 0D  } & \texttt{ IDIV        } & \texttt{  {REG,REG,REG}        } \\
\texttt{ 0E  } & \texttt{ IDIV        } & \texttt{  {REG,REG,INT}        } \\
\texttt{ 0F  } & \texttt{ DIVI        } & \texttt{  {REG,REG,REG}        } \\
\texttt{ 10  } & \texttt{ DIVI        } & \texttt{  {REG,REG,INT}        } \\
\texttt{ 11  } & \texttt{ FADD        } & \texttt{  {REG,REG,REG}        } \\
\texttt{ 12  } & \texttt{ FADD        } & \texttt{  {REG,REG,FLOAT}      } \\
\texttt{ 13  } & \texttt{ ADDF        } & \texttt{  {REG,REG,REG}        } \\
\texttt{ 14  } & \texttt{ ADDF        } & \texttt{  {REG,REG,FLOAT}      } \\
\texttt{ 15  } & \texttt{ FSUB        } & \texttt{  {REG,REG,REG}        } \\
\texttt{ 16  } & \texttt{ FSUB        } & \texttt{  {REG,REG,FLOAT}      } \\
\texttt{ 17  } & \texttt{ FSUB        } & \texttt{  {REG,FLOAT,REG}      } \\
\texttt{ 18  } & \texttt{ SUBF        } & \texttt{  {REG,REG,REG}        } \\
\texttt{ 19  } & \texttt{ SUBF        } & \texttt{  {REG,REG,FLOAT}      } \\
\texttt{ 1A  } & \texttt{ SUBF        } & \texttt{  {REG,FLOAT,REG}      } \\
\texttt{ 1B  } & \texttt{ FMULT       } & \texttt{  {REG,REG,REG}        } \\
\texttt{ 1C  } & \texttt{ FMULT       } & \texttt{  {REG,REG,FLOAT}      } \\
\texttt{ 1D  } & \texttt{ MULTF       } & \texttt{  {REG,REG,REG}        } \\
\texttt{ 1E  } & \texttt{ MULTF       } & \texttt{  {REG,REG,FLOAT}      } \\
\texttt{ 1F  } & \texttt{ FDIV        } & \texttt{  {REG,REG,REG}        } \\
\texttt{ 20  } & \texttt{ FDIV        } & \texttt{  {REG,REG,FLOAT}      } \\
\texttt{ 21  } & \texttt{ FDIV        } & \texttt{  {REG,FLOAT,REG}      } \\
\texttt{ 22  } & \texttt{ DIVF        } & \texttt{  {REG,REG,REG}        } \\
\texttt{ 23  } & \texttt{ DIVF        } & \texttt{  {REG,REG,FLOAT}      } \\
\texttt{ 24  } & \texttt{ DIVF        } & \texttt{  {REG,FLOAT,REG}      } \\
\texttt{ 25  } & \texttt{ INC         } & \texttt{  {REG}                } \\
\texttt{ 26  } & \texttt{ DEC         } & \texttt{  {REG}                } \\
\texttt{ 27  } & \texttt{ INC0        } & \texttt{  NO OPERAND           } \\
\texttt{ 28  } & \texttt{ DEC0        } & \texttt{  NO OPERAND           } \\
\texttt{ 29  } & \texttt{ INC1        } & \texttt{  NO OPERAND           } \\
\texttt{ 2A  } & \texttt{ DEC1        } & \texttt{  NO OPERAND           } \\
\texttt{ 2B  } & \texttt{ INC2        } & \texttt{  NO OPERAND           } \\
\texttt{ 2C  } & \texttt{ DEC2        } & \texttt{  NO OPERAND           } \\
\texttt{ 2D  } & \texttt{ SCONCAT     } & \texttt{  {REG,REG,REG}        } \\
\texttt{ 2E  } & \texttt{ SCONCAT     } & \texttt{  {REG,REG,STRING}     } \\
\texttt{ 2F  } & \texttt{ SCONCAT     } & \texttt{  {REG,STRING,REG}     } \\
\texttt{ 30  } & \texttt{ CONCAT      } & \texttt{  {REG,REG,REG}        } \\
\texttt{ 31  } & \texttt{ CONCAT      } & \texttt{  {REG,REG,STRING}     } \\
\texttt{ 32  } & \texttt{ CONCAT      } & \texttt{  {REG,STRING,REG}     } \\
\texttt{ 33  } & \texttt{ APPENDCHAR  } & \texttt{  {REG,REG}            } \\
\texttt{ 34  } & \texttt{ TRIML       } & \texttt{  {REG,REG}            } \\
\texttt{ 35  } & \texttt{ TRIMR       } & \texttt{  {REG,REG}            } \\
\texttt{ 36  } & \texttt{ TRUNC       } & \texttt{  {REG,REG}            } \\
\texttt{ 37  } & \texttt{ TRIML       } & \texttt{  {REG,REG,REG}        } \\
\texttt{ 38  } & \texttt{ TRIMR       } & \texttt{  {REG,REG,REG}        } \\
\texttt{ 39  } & \texttt{ TRUNC       } & \texttt{  {REG,REG,REG}        } \\
\texttt{ 3A  } & \texttt{ STRLEN      } & \texttt{  {REG,REG}            } \\
\texttt{ 3B  } & \texttt{ STRCHAR     } & \texttt{  {REG,REG,REG}        } \\
\texttt{ 3C  } & \texttt{ STRCHAR     } & \texttt{  {REG,REG}            } \\
\texttt{ 3D  } & \texttt{ SETSTRPOS   } & \texttt{  {REG,REG}            } \\
\texttt{ 3E  } & \texttt{ GETSTRPOS   } & \texttt{  {REG,REG}            } \\
\texttt{ 3F  } & \texttt{ SUBSTR      } & \texttt{  {REG,REG,REG}        } \\
\texttt{ 40  } & \texttt{ IEQ         } & \texttt{  {REG,REG,REG}        } \\
\texttt{ 41  } & \texttt{ IEQ         } & \texttt{  {REG,REG,INT}        } \\
\texttt{ 42  } & \texttt{ INE         } & \texttt{  {REG,REG,REG}        } \\
\texttt{ 43  } & \texttt{ INE         } & \texttt{  {REG,REG,INT}        } \\
\texttt{ 44  } & \texttt{ IGT         } & \texttt{  {REG,REG,REG}        } \\
\texttt{ 45  } & \texttt{ IGT         } & \texttt{  {REG,REG,INT}        } \\
\texttt{ 46  } & \texttt{ IGT         } & \texttt{  {REG,INT,REG}        } \\
\texttt{ 47  } & \texttt{ IGTE        } & \texttt{  {REG,REG,REG}        } \\
\texttt{ 48  } & \texttt{ IGTE        } & \texttt{  {REG,REG,INT}        } \\
\texttt{ 49  } & \texttt{ IGTE        } & \texttt{  {REG,INT,REG}        } \\
\texttt{ 4A  } & \texttt{ ILT         } & \texttt{  {REG,REG,REG}        } \\
\texttt{ 4B  } & \texttt{ ILT         } & \texttt{  {REG,REG,INT}        } \\
\texttt{ 4C  } & \texttt{ ILT         } & \texttt{  {REG,INT,REG}        } \\
\texttt{ 4D  } & \texttt{ ILTE        } & \texttt{  {REG,REG,REG}        } \\
\texttt{ 4E  } & \texttt{ ILTE        } & \texttt{  {REG,REG,INT}        } \\
\texttt{ 4F  } & \texttt{ ILTE        } & \texttt{  {REG,INT,REG}        } \\
\texttt{ 50  } & \texttt{ FEQ         } & \texttt{  {REG,REG,REG}        } \\
\texttt{ 51  } & \texttt{ FEQ         } & \texttt{  {REG,REG,FLOAT}      } \\
\texttt{ 52  } & \texttt{ FNE         } & \texttt{  {REG,REG,REG}        } \\
\texttt{ 53  } & \texttt{ FNE         } & \texttt{  {REG,REG,FLOAT}      } \\
\texttt{ 54  } & \texttt{ FGT         } & \texttt{  {REG,REG,REG}        } \\
\texttt{ 55  } & \texttt{ FGT         } & \texttt{  {REG,REG,FLOAT}      } \\
\texttt{ 56  } & \texttt{ FGT         } & \texttt{  {REG,FLOAT,REG}      } \\
\texttt{ 57  } & \texttt{ FGTE        } & \texttt{  {REG,REG,REG}        } \\
\texttt{ 58  } & \texttt{ FGTE        } & \texttt{  {REG,REG,FLOAT}      } \\
\texttt{ 59  } & \texttt{ FGTE        } & \texttt{  {REG,FLOAT,REG}      } \\
\texttt{ 5A  } & \texttt{ FLT         } & \texttt{  {REG,REG,REG}        } \\
\texttt{ 5B  } & \texttt{ FLT         } & \texttt{  {REG,REG,FLOAT}      } \\
\texttt{ 5C  } & \texttt{ FLT         } & \texttt{  {REG,FLOAT,REG}      } \\
\texttt{ 5D  } & \texttt{ FLTE        } & \texttt{  {REG,REG,REG}        } \\
\texttt{ 5E  } & \texttt{ FLTE        } & \texttt{  {REG,REG,FLOAT}      } \\
\texttt{ 5F  } & \texttt{ FLTE        } & \texttt{  {REG,FLOAT,REG}      } \\
\texttt{ 60  } & \texttt{ SEQ         } & \texttt{  {REG,REG,REG}        } \\
\texttt{ 61  } & \texttt{ SEQ         } & \texttt{  {REG,REG,STRING}     } \\
\texttt{ 62  } & \texttt{ SNE         } & \texttt{  {REG,REG,REG}        } \\
\texttt{ 63  } & \texttt{ SNE         } & \texttt{  {REG,REG,STRING}     } \\
\texttt{ 64  } & \texttt{ SGT         } & \texttt{  {REG,REG,REG}        } \\
\texttt{ 65  } & \texttt{ SGT         } & \texttt{  {REG,REG,STRING}     } \\
\texttt{ 66  } & \texttt{ SGT         } & \texttt{  {REG,STRING,REG}     } \\
\texttt{ 67  } & \texttt{ SGTE        } & \texttt{  {REG,REG,REG}        } \\
\texttt{ 68  } & \texttt{ SGTE        } & \texttt{  {REG,REG,STRING}     } \\
\texttt{ 69  } & \texttt{ SGTE        } & \texttt{  {REG,STRING,REG}     } \\
\texttt{ 6A  } & \texttt{ SLT         } & \texttt{  {REG,REG,REG}        } \\
\texttt{ 6B  } & \texttt{ SLT         } & \texttt{  {REG,REG,STRING}     } \\
\texttt{ 6C  } & \texttt{ SLT         } & \texttt{  {REG,STRING,REG}     } \\
\texttt{ 6D  } & \texttt{ SLTE        } & \texttt{  {REG,REG,REG}        } \\
\texttt{ 6E  } & \texttt{ SLTE        } & \texttt{  {REG,REG,STRING}     } \\
\texttt{ 6F  } & \texttt{ SLTE        } & \texttt{  {REG,STRING,REG}     } \\
\texttt{ 70  } & \texttt{ AND         } & \texttt{  {REG,REG,REG}        } \\
\texttt{ 71  } & \texttt{ OR          } & \texttt{  {REG,REG,REG}        } \\
\texttt{ 72  } & \texttt{ TIME        } & \texttt{  {REG}                } \\
\texttt{ 73  } & \texttt{ MAP         } & \texttt{  {REG,REG}            } \\
\texttt{ 74  } & \texttt{ MAP         } & \texttt{  {REG,STRING}         } \\
\texttt{ 75  } & \texttt{ AMAP        } & \texttt{  {REG,REG}            } \\
\texttt{ 76  } & \texttt{ AMAP        } & \texttt{  {REG,INT}            } \\
\texttt{ 77  } & \texttt{ PMAP        } & \texttt{  {REG,REG}            } \\
\texttt{ 78  } & \texttt{ PMAP        } & \texttt{  {REG,STRING}         } \\
\texttt{ 79  } & \texttt{ GMAP        } & \texttt{  {REG,REG}            } \\
\texttt{ 7A  } & \texttt{ GMAP        } & \texttt{  {REG,STRING}         } \\
\texttt{ 7B  } & \texttt{ NSMAP       } & \texttt{  {REG,REG,REG}        } \\
\texttt{ 7C  } & \texttt{ NSMAP       } & \texttt{  {REG,REG,STRING}     } \\
\texttt{ 7D  } & \texttt{ NSMAP       } & \texttt{  {REG,STRING,STRING}  } \\
\texttt{ 7E  } & \texttt{ NSMAP       } & \texttt{  {REG,STRING,REG}     } \\
\texttt{ 7F  } & \texttt{ UNMAP       } & \texttt{  {REG}                } \\
\texttt{ 80  } & \texttt{ CALL        } & \texttt{  {FUNC}               } \\
\texttt{ 81  } & \texttt{ CALL        } & \texttt{  {REG,FUNC}           } \\
\texttt{ 82  } & \texttt{ CALL        } & \texttt{  {REG,FUNC,REG}       } \\
\texttt{ 83  } & \texttt{ RET         } & \texttt{  NO OPERAND           } \\
\texttt{ 84  } & \texttt{ RET         } & \texttt{  {REG}                } \\
\texttt{ 85  } & \texttt{ RET         } & \texttt{  {INT}                } \\
\texttt{ 86  } & \texttt{ RET         } & \texttt{  {FLOAT}              } \\
\texttt{ 87  } & \texttt{ RET         } & \texttt{  {CHAR}               } \\
\texttt{ 88  } & \texttt{ RET         } & \texttt{  {STRING}             } \\
\texttt{ 89  } & \texttt{ BR          } & \texttt{  {ID}                 } \\
\texttt{ 8A  } & \texttt{ BRT         } & \texttt{  {ID,REG}             } \\
\texttt{ 8B  } & \texttt{ BRF         } & \texttt{  {ID,REG}             } \\
\texttt{ 8C  } & \texttt{ MOVE        } & \texttt{  {REG,REG}            } \\
\texttt{ 8D  } & \texttt{ COPY        } & \texttt{  {REG,REG}            } \\
\texttt{ 8E  } & \texttt{ ICOPY       } & \texttt{  {REG,REG}            } \\
\texttt{ 8F  } & \texttt{ LINK        } & \texttt{  {REG,REG}            } \\
\texttt{ 90  } & \texttt{ UNLINK      } & \texttt{  {REG}                } \\
\texttt{ 91  } & \texttt{ NULL        } & \texttt{  {REG}                } \\
\texttt{ 92  } & \texttt{ LOAD        } & \texttt{  {REG,INT}            } \\
\texttt{ 93  } & \texttt{ LOAD        } & \texttt{  {REG,FLOAT}          } \\
\texttt{ 94  } & \texttt{ LOAD        } & \texttt{  {REG,STRING}         } \\
\texttt{ 95  } & \texttt{ LOAD        } & \texttt{  {REG,CHAR}           } \\
\texttt{ 96  } & \texttt{ SAY         } & \texttt{  {REG}                } \\
\texttt{ 97  } & \texttt{ SSAY        } & \texttt{  {REG}                } \\
\texttt{ 98  } & \texttt{ SAY         } & \texttt{  {INT}                } \\
\texttt{ 99  } & \texttt{ SAY         } & \texttt{  {FLOAT}              } \\
\texttt{ 9A  } & \texttt{ SAY         } & \texttt{  {STRING}             } \\
\texttt{ 9B  } & \texttt{ SAY         } & \texttt{  {CHAR}               } \\
\texttt{ 9C  } & \texttt{ EXIT        } & \texttt{  NO OPERAND           } \\
\texttt{ 9D  } & \texttt{ EXIT        } & \texttt{  {REG}                } \\
\texttt{ 9E  } & \texttt{ EXIT        } & \texttt{  {INT}                } \\
\texttt{ 9F  } & \texttt{ ITOS        } & \texttt{  {REG}                } \\
\texttt{ A0  } & \texttt{ FTOS        } & \texttt{  {REG}                } \\
\texttt{ A1  } & \texttt{ ITOF        } & \texttt{  {REG}                } \\

\end{supertabular}

\section{\fontspec{Helvetica} Example Programs}
\lstinputlisting[]{main.rxas}
\lstinputlisting[]{func1.rxas}
\lstinputlisting[]{func2.rxas}
\pagebreak


\hrulefill

\noindent Copyright \textcopyright{} \thedate{} \theauthor{}
\begin{pspicture}(1in,1in)
    \psbarcode{https://github.com/adesutherland/CREXX/wiki/Logical-REXX-VM-Assembler}{eclevel=L}{qrcode}
\end{pspicture}

\end{multicols}
\end{document}
