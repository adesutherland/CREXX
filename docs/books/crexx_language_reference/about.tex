\def\tightlist{}


\chapter*{About This Book}
The \crexx{} language is a further development, and variant of the
\textsc{Rexx} language\footnote{Cowlishaw, 1979}.

This language reference provides a comprehensive and detailed
description of the syntax, semantics, and features of the \crexx{}
programming language. It usually includes a complete listing of all
the language constructs, such as statements, expressions, functions,
classes, and modules, along with their syntax, parameters, and usage.

\section*{Audience}
The \crexx{} language reference is aimed at experienced programmers who
are familiar with the \textsc{Rexx} language. It is an
essential resource for developers who need to write code in the
language, debug programs, or create tools, while the \emph{\crexx{} User's Guide} serves as an introduction to
the language and is intended for new users.

\section*{History}

\begin{description}
\item[mvp] This version documents the Minimally Viable Product
  release, Q1 2022 and is intended for developers only. It documents \crexx{} level B, which is a typed
  subset of Classic Rexx.
\item[f29] First version, Git feature [F0029]
\end{description}

\section*{Document Structure}
This document is in three parts:
\begin{enumerate}
\item The objectives of the \crexx{} language and the concepts underlying its design, and acknowledgements.
\item An overview and introduction to the \crexx{} language.
\item The definition of the language.
\end{enumerate}

