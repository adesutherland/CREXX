\hypertarget{building-and-running-crexx}{%
\subsection{Building and Running
cRexx}\label{building-and-running-crexx}}

This paragraph aims to show all you need to know about how to build
cRexx from scratch, and then run it. It will show what you need, what to
do and when it is build, how to make working programs with it.

\hypertarget{requirements}{%
\subsubsection{Requirements}\label{requirements}}

Currently cRexx is known to build on Windows, macOS and
Linux\footnote{There is a separate instruction for VM/370 (and later)
  mainframe operating systems.}. You need one of those and:

\begin{longtable}[]{@{}ll@{}}
\toprule()
Tool & Function \\
\midrule()
\endhead
Git & source code version management \\
CMake & build tool \\
Rexx & used during build process (brexx, ooRexx, Regina will all do) \\
C compiler & gcc, clang (install g++ as some C++ elements are used) \\
Bison & parser generator \\
\bottomrule()
\end{longtable}

and either

\begin{longtable}[]{@{}ll@{}}
\toprule()
Tool & Function \\
\midrule()
\endhead
Make & conventional build tool \\
Ninja & fast build tool \\
\bottomrule()
\end{longtable}

\hypertarget{platform-specific-info}{%
\paragraph{Platform specific info}\label{platform-specific-info}}

On Linux and macOS, this instruction is identical. For macOS, Xcode
batch tools need to be installed, which will provide you with git, make
and the compiler. Brew will give easy access to regina-rexx\footnote{ooRexx
  and brexx will also work, one of those needs to be on the path. For
  Linux, you will need to install git (which will be there on most
  distributions), cmake and gcc or clang.} and Ninja-build.

On Windows, you will need a compatibility layer like
\href{https://www.msys2.org}{msys} - installing this has the additional
advantage of easy access to git, gcc, cmake and the rest of the
necessary tools. On more modern Windows, the WSL{[}\^{}3\}{]} and Ubuntu
is not a bad choice.

\hypertarget{process}{%
\subsubsection{Process}\label{process}}

Here it is assumed that all tools are installed and working, and
available on your PATH environment variable.

Choose or make a suitably named directory on your system to contain the
source code. Note that the cRexx source is kept in a different directory
on you system than where it is built in, or will run from. Now run this
command:

\begin{verbatim}
git clone https://github.com/adesutherland/CREXX.git
\end{verbatim}

This will give you a CREXX subdirectory in the current directory,
containing the source of cRexx and its dependencies. This is the
`develop' branch, which is the one you would normally want to use. All
of these are written in the C99 version of the C programming language,
which should be widely supported by C compilers on most platforms.

Make a new subdirectory in the current directory (not in CREXX, but in
the one that contains it), like `crexx-build'.

\begin{verbatim}
mkdir crexx-build
\end{verbatim}

and cd into that directory. Now issue the following command (we assume
that you installed ninja, otherwise subsitute `make' for the two
instances of `ninja'):

\begin{verbatim}
cmake -G Ninja -DCMAKE_BUILD_TYPE=Release ../CREXX && ninja && ctest --output-on-failure
\end{verbatim}

This will do a lot of things. In fact, if all goes well, you will have a
built and tested cRexx system. \#\#\# Explaining the build process Let's
zoom in a little on what we did. The first step is to tell CMake to
validate your system environment and generate a build script (and a test
script) for the chosen build tool. Cmake will read the file
CmakeLists.txt and validate that your system can do what it asks it to
do. This can yield error messages, for example if the C compiler lacks
certain functions or header files. (When that happens, open an
\href{https://github.com/adesutherland/CREXX/issues}{issue} and someone
will have a look at it - or peruse
\href{https://stackoverflow.com}{stack overflow} which is what we
probably also will do).

After CMake has successfully validated the build environment, it will
generate a build script (a Makefile in the case of Make and a
build.ninja file in the case of Ninja). This is specified after the `-G'
flag. The `-DCMAKE\_BUILD\_TYPE=Release' flag makes sure we do an
optimized build, which means we specify an `-o3' flag to the C compiler,
which then will spend some time optimizing the executable modules, which
makes them run faster (they do!). The alternative is a `debug' build
which will yield slower executables, but with more debugging information
in them.

The two ampersands (\&\&) mean we do the next part only if the previous
step was successful. This is a `ninja' statement, which will build
everything in the build.ninja specification file. These are a lot of
parts, and the good news is, when they are built once, only the changed
source will be built, which will be fast.

After this, the generated test suite is run with the `ctest' command.
This knows what to do, and will show you successes and failures. If what
you checked out if git is not a released version, there is a small
change that some test cases fail, but generally these should indicate
success.

\hypertarget{use-of-rexx-to-build-crexx}{%
\subsubsection{Use of Rexx to build
cRexx}\label{use-of-rexx-to-build-crexx}}

Rexx code is used twice during the build process. The first time is in
an early stage: as the Rexx VM `machine code' instructions are generated
from a file, a Rexx script needs to work on these to generate two C
sources. (The Rexx engine for this needs to be in working order on the
building system - this is one of the things CMake checks, and it can
flag down the build if it cannot locate a working `rexx' executable).

The second time Rexx is used, it is already our working cRexx instance:
the library of built-in functions is written in Rexx (and Rexx
Assembler) and needs to be compiled (carefully observing the
dependencies on other Rexx built-in functions) before it is added to the
library and the cRexx executables.

\hypertarget{what-do-we-have-after-a-successful-build-process}{%
\subsubsection{What do we have after a successful build
process}\label{what-do-we-have-after-a-successful-build-process}}

When all went well, we have a set of native executables for the platform
we built cRexx on. These are

\begin{longtable}[]{@{}ll@{}}
\toprule()
Name & Function \\
\midrule()
\endhead
rxc & cRexx compiler \\
rxas & cRexx assembler \\
rxdas & cRexx disassembler \\
rxvm & cRexx virtual machine \\
rxbvm & cRexx virtual machine, non-threaded version \\
rxvme & cRexx virtual machine with linked-in Rexx library \\
rxdb & cRexx debugger \\
rxcpack & cRexx C-generator for native executables \\
\bottomrule()
\end{longtable}

You read that right, cRexx already can compile your Rexx script into an
executable file, that can be run standalone, for example, on a computer
that has no cRexx and/or C compiler installed.

Another salient fact from the above table is that `rxdb', the cRexx
debugger, is entirely written in Rexx and compiled and packaged into an
executable file.

The executables in the above table need to be on the PATH environment
variable. These are to be found in the compiler, assembler,
disassembler, debugger and cpacker directories of the crexx-build
directory we created earlier. It is up to you to add all these
separately to the PATH environment, or to just collect them all into one
directory that is already on the PATH.

\hypertarget{running-crexx}{%
\subsection{Running cRexx}\label{running-crexx}}

All Rexx scripts you run with cRexx are compiled by `rxc' into Rexx
assembler code (.rxas) and then assembled into an .rxbin file, which
contains the Rexx bytecode for execution by the Rexx Virtual Machine
`rxvm'. This rxvm executable takes care of linking separately compiled
modules together and executing them, so one function can find another.

\hypertarget{an-easy-test---hello-world}{%
\subsubsection{An easy test - Hello
World}\label{an-easy-test---hello-world}}

Let's say you have a Rexx exec you would like to run. To not have any
surprises, it is of the `hello world' kind. We have a file called
hello.rexx, containing:

\begin{verbatim}
/* rexx */
options levelb
say 'hello cRexx world!'
return 0
\end{verbatim}

When cRexx level `C' (for `Classic') is available, the `options levelb'
(on line 2) statement can be left out; for the moment, level B is all we
have, and the compiler will refuse to compile without it.

With all our cRexx executables on the PATH, we only need to do:

\begin{verbatim}
rxc hello
rxas hello
rxvm hello
\end{verbatim}

to see `hello cRexx world!' on the console.

It might be a good idea to make a shell script to execute these three
programs in succession, and perhaps call it `crexx'. But take into
account that this really is a very simple case, in which no built-in
functions are called. You can look into the generated rexx assembler
(hello.rxas) file:

\begin{verbatim}
➜  crexx git:(master) ✗ cat hello.rxas
/*
 * cREXX COMPILER VERSION : cREXX F0043
 * SOURCE                 : hello
 * BUILT                  : 2022-12-03 22:27:52
 */

.srcfile="hello"
.globals=0

main() .locals=1 .expose=hello.main
   .meta "hello.main"="b" ".int" main() "" ""
   .src 3:1="say 'hello cRexx world!'"
   say "hello cRexx world!"
   .src 4:1="return 0"
   ret 0
\end{verbatim}

and you can see here that the compiler actually has generated a `say'
assembler instruction for the Rexx `say' instruction. (Assembler became
a whole lot easier with Rexx assembler.) But we did not yet call any
function.

\hypertarget{using-built-in-functions}{%
\subsubsection{Using built-in
functions}\label{using-built-in-functions}}

Most Rexx programs use the extremely well designed built-in functions.
Now with these functions written in Rexx, and not hidden in the compiler
somewhere, we must tell it to import those from the library where we put
them earlier during the build process. Let's say we want to add a
display of the current weekday to our hello program. This will now be:

\begin{verbatim}
/* rexx */
options levelb
import rxfnsb
say 'hello cRexx world!'
say 'today is' date('w')
return 0
\end{verbatim}

Never mind the import statement, which you will not need when cRexx
`Classic' level C is available. But in level B, we need this, because we
need the flexibility it affords our plans for the future.

We must tell the compiler where to find the signature of the date()
function, so it can check if we call it in the correct way, with the
right parameters. This is done with the -i switch, which points to the
directory containing the library - which is called `library', by the
way.

\begin{verbatim}
rxc -i ~/crexx-build/lib/rxfns hello
\end{verbatim}

The assembler runs unchanged, because it trusts the compiler to have
checked if the called function really sits in that library, and has the
right parameters - the right code to call it has been generated.

\begin{verbatim}
rxas hello
\end{verbatim}

To run it, we can employ the `rxvme' executable - this one is extended
with linked-in versions of all the functions in the library:

\begin{verbatim}
rxvme hello
\end{verbatim}

which yields:

\begin{verbatim}
hello cRexx world!
today is Saturday
\end{verbatim}

\hypertarget{building-a-standalone-executable}{%
\subsubsection{Building a standalone
executable}\label{building-a-standalone-executable}}

To end this short tour of building and running cRexx, we are taking that
last example and will build a standalone executable out of it. If your
neighbor, family member or sports club runs the same OS as you, they can
now run your rexx program from a USB stick, without ever installing
anything.

For this, we need the next set of commands, expressed as a Rexx exec:

\begin{verbatim}
/* rexx compile a rexx exec to a native executable */
/* Classic Rexx and NetRexx compatible             */
crexx_home='~/crexx-build'
if arg='' then do
  say 'exec name expected.'
  exit 99
end
parse arg execName'.'extension
if extension<>'' then say 'filename extension ignored.'
'rxc  -i' crexx_home'/lib/rxfns' execName
'rxas' execName
'rxcpack' execName crexx_home'/lib/rxfns/library'
'gcc -o' execName,
'-lrxvml -lmachine -lavl_tree -lplatform -lm -L',
crexx_home'/interpreter -L'crexx_home'/machine -L',
crexx_home'/avl_tree -L'crexx_home'/platform'  execName'.c'
\end{verbatim}

This exec is delivered in the source tree, bin directory. At the moment
it can be run by classic rexx interpreters and NetRexx, it will soon be
runnable by cRexx itself. The exec works by compiling the Rexx program
specified (again without the .rexx file extension) to an .rxbin Rexx
bytecode file, which is then serialized to a C source file, containing
the cRexx Virtual Machine and the library rxbin files, by the rxcpack
command. It is a rather peculiar looking C source, but nevertheless it
will compile to a working executable, which is done by the last step in
the exec, here using the gcc compiler. And you will be able to run it
without the overhead of checking, compiling, tokenizing to bytecode and
linking, so it will be quite fast:

\begin{verbatim}
➜  crexx git:(master) ✗ time ./hello
hello cRexx world!
today is Saturday
./hello  0.00s user 0.00s system 61% cpu 0.009 total
\end{verbatim}
