\hypertarget{building-the-toolchain}{%
\section{\texorpdfstring{Building the \crexx{}
toolchain}{Building the  toolchain}}\label{building-the-toolchain}}

This paragraph aims to show all you need to know about how to build
cRexx from scratch, and then run it. It will show what you need, what to
do and when it is build, how to make working programs with it.

\hypertarget{requirements}{%
\section{Requirements}\label{requirements}}

Currently cRexx is known to build on Windows, macOS and
Linux\footnote{There is a separate instruction for VM/370 (and later)
  mainframe operating systems.}. You need one of those and:

\begin{longtable}[]{@{}ll@{}}
\toprule()
Tool & Function \\
\midrule()
\endhead
Git & source code version management \\
CMake & build tool \\
Rexx & used during build process (brexx, ooRexx, Regina will all do) \\
C compiler & gcc, clang (install g++ as some C++ elements are used) \\
Bison & parser generator \\
Make & conventional build tool \textbf{or} \\
Ninja & fast build tool \\
\bottomrule()
\end{longtable}

\hypertarget{platform-specific-info}{%
\section{Platform specific info}\label{platform-specific-info}}

On Linux and macOS, this instruction is identical. For macOS, Xcode
batch tools need to be installed, which will provide you with git, make
and the compiler. Brew will give easy access to regina-rexx\footnote{ooRexx
  and brexx will also work, one of those needs to be on the path. For
  Linux, you will need to install git (which will be there on most
  distributions), cmake and gcc or clang.} and Ninja-build.

On Windows, you will need a compatibility layer like
\href{https://www.msys2.org}{msys} - installing this has the additional
advantage of easy access to git, gcc, cmake and the rest of the
necessary tools. On more modern Windows, the WSL\footnote{WSL: Windows
  subsystem for Linux.} and Ubuntu is not a bad choice.

\hypertarget{process}{%
\section{Process}\label{process}}

Here it is assumed that all tools are installed and working, and
available on your PATH environment variable.

Choose or make a suitably named directory on your system to contain the
source code. Note that the cRexx source is kept in a different directory
on you system than where it is built in, or will run from. Now run this
command:

\begin{verbatim}
git clone https://github.com/adesutherland/CREXX.git
\end{verbatim}

This will give you a CREXX subdirectory in the current directory,
containing the source of cRexx and its dependencies. This is the
`develop' branch, which is the one you would normally want to use. All
of these are written in the C99 version of the C programming language,
which should be widely supported by C compilers on most platforms.

Make a new subdirectory in the current directory (not in CREXX, but in
the one that contains it), like `crexx-build'.

\begin{verbatim}
mkdir crexx-build
\end{verbatim}

and cd into that directory. Now issue the following command (we assume
that you installed ninja, otherwise subsitute `make' for the two
instances of `ninja'):

\begin{verbatim}
cmake -G Ninja -DCMAKE_BUILD_TYPE=Release ../CREXX && ninja && ctest
\end{verbatim}

This will do a lot of things. In fact, if all goes well, you will have a
built and tested cRexx system.

\hypertarget{explaining-the-build-process}{%
\section{Explaining the build
process}\label{explaining-the-build-process}}

Let's zoom in a little on what we did. The first step is to tell CMake
to validate your system environment and generate a build script (and a
test script) for the chosen build tool. Cmake will read the file
CmakeLists.txt and validate that your system can do what it asks it to
do. This can yield error messages, for example if the C compiler lacks
certain functions or header files. (When that happens, open an
\href{https://github.com/adesutherland/CREXX/issues}{issue} and someone
will have a look at it - or peruse
\href{https://stackoverflow.com}{stack overflow} which is what we
probably also will do).

After CMake has successfully validated the build environment, it will
generate a build script (a Makefile in the case of Make and a
build.ninja file in the case of Ninja). This is specified after the `-G'
flag. The `-DCMAKE\_BUILD\_TYPE=Release' flag makes sure we do an
optimized build, which means we specify an `-o3' flag to the C compiler,
which then will spend some time optimizing the executable modules, which
makes them run faster (they do!). The alternative is a `debug' build
which will yield slower executables, but with more debugging information
in them.

The two ampersands (\&\&) mean we do the next part only if the previous
step was successful. This is a `ninja' statement, which will build
everything in the build.ninja specification file. These are a lot of
parts, and the good news is, when they are built once, only the changed
source will be built, which will be fast.

After this, the generated test suite is run with the `ctest' command.
This knows what to do, and will show you successes and failures. If what
you checked out if git is not a released version, there is a small
change that some test cases fail, but generally these should indicate
success.

\hypertarget{use-of-rexx-to-build-crexx}{%
\section{Use of Rexx to build cRexx}\label{use-of-rexx-to-build-crexx}}

Rexx code is used twice during the build process. The first time is in
an early stage: as the Rexx VM `machine code' instructions are generated
from a file, a Rexx script needs to work on these to generate two C
sources. (The Rexx engine for this needs to be in working order on the
building system - this is one of the things CMake checks, and it can
flag down the build if it cannot locate a working `rexx' executable).

The second time Rexx is used, it is already our working cRexx instance:
the library of built-in functions is written in Rexx (and Rexx
Assembler) and needs to be compiled (carefully observing the
dependencies on other Rexx built-in functions) before it is added to the
library and the cRexx executables.

\hypertarget{what-do-we-have-after-a-successful-build-process}{%
\section{What do we have after a successful build
process}\label{what-do-we-have-after-a-successful-build-process}}

When all went well, we have a set of native executables for the platform
we built cRexx on. These are

\begin{longtable}[]{@{}ll@{}}
\toprule()
Name & Function \\
\midrule()
\endhead
rxc & cRexx compiler \\
rxas & cRexx assembler \\
rxdas & cRexx disassembler \\
rxvm & cRexx virtual machine \\
rxbvm & cRexx virtual machine, non-threaded version \\
rxvme & cRexx virtual machine with linked-in Rexx library \\
rxdb & cRexx debugger \\
rxcpack & cRexx C-generator for native executables \\
\bottomrule()
\end{longtable}

You read that right, cRexx already can compile your Rexx script into an
executable file, that can be run standalone, for example, on a computer
that has no cRexx and/or C compiler installed.

Another salient fact from the above table is that `rxdb', the cRexx
debugger, is entirely written in Rexx and compiled and packaged into an
executable file.

The executables in the above table need to be on the PATH environment
variable. These are to be found in the compiler, assembler,
disassembler, debugger and cpacker directories of the crexx-build
directory we created earlier. It is up to you to add all these
separately to the PATH environment, or to just collect them all into one
directory that is already on the PATH.
