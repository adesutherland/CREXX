\section{Level C Grammar Specification}

\emph{Page Status: This page is work in progress, incomplete, inconsistent and full of errors ... read no further!}

\subsection{Notes}

\begin{itemize}
\item CREXX Level C is Classic REXX!

\item See {the REXX Language Level} descriptions

\item The {REXX PEG format} is used to define the grammar (implementation independent)

\item This specification includes cross references to the REXX standard and to Phase 1 Implementation details

\end{itemize}

\subsection{Global Rules - File / Line Endings / Whitespace / Comments}

\subsubsection{Whitespace \& Comments}

\begin{verbatim}
\comment <- '/*' .* \comment* .*  ('*/' / eos->ERROR[6.1]); # Get rid of comments
ws <- [ \t\v\f]+;                                           # Simple Whitespace Characters
continuation <- ',' ws* \eol ^\eof;                         # Continuation - trailing ','. Note: Comments already removed
\whitespace <- ws / continuation;                           # Whitespace includes continuation
\end{verbatim}

\subsubsection{eos}

\begin{verbatim}
eos <- \eof; # Platform specific detection of EOF (which is labelled End of Stream in ANSI REXX)
\end{verbatim}

\textbar{} Cross References \textbar{} {ANSI} \textbar{} {Phase 1} \textbar{}
\textbar{} -{}-{}- \textbar{} -{}-{}- \textbar{} -{}-{}- \textbar{}

\subsubsection{eol}

\begin{verbatim}
eol <= \eol; # e.g. '\r\n' / '\n';  # Platform specific detection of EOL
\end{verbatim}

\subsubsection{Characters}

\begin{verbatim}
digit <- [0-9];
special <- [,:;)(]
operator <- [+\-%/&=><*\\]
general_letter <- [_!?A-Za-z]
valid <- digit / special / operator / general_letter / ws / [.'"\[\]] ;
invalid <- ^valid -> ERROR[13.1];
char <- valid / invalid;
\end{verbatim}

\subsubsection{Strings}

\begin{verbatim}
quotation_mark_string <- '"' char* '"' !'"'; 
apostrophe_string <- '\'' char* '\'' !'\'';
quoted_string_error  <- ['"] char* eol -> ERROR[6.3]; 
string <- quotation_mark_string / apostrophe_string / quoted_string_error;
hex_string <- string [xX] ^(general_letter / digit / [.]);
binary_string <- string [bB] ^(general_letter / digit / [.]);
string_literal <- hex_string / binary_string / string -> STRING;
\end{verbatim}

\subsubsection{Symbols and Labels}

\begin{verbatim}
var_symbol <- general_letter var_symbol_char* -> VAR_SYMBOL;
var_symbol_char <- general_letter / digit / [.];
label <- general_letter var_symbol_char* ':' -> LABEL;
\end{verbatim}

\subsubsection{Number}

\begin{verbatim}
number <- plain_number exponent?;
plain_number <- '.'? digit+ / digit+ '.' digit*;
exponent <- 'E' [+\-] digit+; # NOTE the '\' is escaping the '-'

const_symbol <- (  ( plain_number 'E' [+\-] digit+ !(general_letter / [.]) )     # It could be an exponent number
             / plain_number                                                      # If not it could be a plain number
             / ('.' ( 'MN' &ws / 'RESULT' &ws / 'RC' &ws / 'RS' &ws / 'SIGL' &ws # Starts with a period and is one of the acceptable strings
             / var_symbol_char+ ->ERROR[50.1] ) )                                # Otherwise add an error token to the AST
             / ( digit var_symbol_char+)) -> CONST_SYMBOL;                       # Finally could be something like 12BLAH
\end{verbatim}

\subsubsection{Operators}

\begin{verbatim}
operator_char <- [+\-%/&=></*];
Operator <- operator_char 
/ '/' '/'
/ '/' '/' 
/ '*' '*' 
/ '\\' '='
/ '>' '<' 
/ '<' '>' 
/ '>' '='
/ '\\' '<' 
/ '<' '=' 
/ '\\' '>'
/ '=' '=' 
/ '\\' '=' '='
/ '>' '>' 
/ '<' '<'
/ '>' '>' '='
/ '\\' '<' '<'
/ '<' '<' '='
/ '\\' '>' '>'
/ '&' '&';	
\end{verbatim}

\subsubsection{TODO - are these needed?}

\begin{verbatim}
hex_string <- (hex_digit [break_hex_digit_pair+] / [hex_digit hex_digit [break_hex_digit_pair+]]) / (Msg15.1 / Msg15.3)
hex_digit <- digit / 'a' / 'A' / 'b' / 'B' / 'c' / 'C' / 'd' / 'D' / 'e' / 'E' / 'f' / 'F'
break_hex_digit_pair <- ws hex_digit hex_digit
binary_string <- (binary_digit [break_binary_digit_quad+] / binary_digit binary_digit [break_binary_digit_quad+] / binary_digit binary_digit         
binary_digit [break_binary_digit_quad+] / [binary_digit binary_digit binary_digit binary_digit [break_binary_digit_quad+]]) / (Msg15.2 / Msg15.4)
binary_digit <- '0' / '1' 
break_binary_digit_quad <- ws binary_digit binary_digit binary_digit binary_digit
\end{verbatim}

\subsubsection{Program}

\begin{verbatim}
program <- null_clause* instruction_list? ('END'->ERROR[10.1])? -> PROGRAM;
end_of_clause <- ';' / eol;
null_clause <- end_of_clause (label end_of_clause)* ;
ncl <- ( null_clause+ ) / ( .->ERROR[21.1] resync ); 
\end{verbatim}

\emph{Resync after an error to the next clause - an addition to the specification}
resync \textless{}- .* end\_of\_clause;

\begin{verbatim}
instruction_list<- instruction+ -> INSTRUCTIONS;

instruction <- group / (single_instruction ncl);

single_instruction <- assignment / keyword_instruction / command;

assignment <-  v:var_symbol '=' e:expression -> (ASSIGNMENT v e)
         / t:NUMBER '=' resync -> (ERROR[31.1] t) 
		 / &digit t:CONST_SYMBOL '=' resync -> (ERROR[Msg31.2] t)
		 / &'.' t:CONST_SYMBOL '=' resync -> (ERROR[Msg31.3] t);

keyword_instruction <- address / arg / call / drop / exit / interpret / iterate 
                 / leave / nop / numeric / options / parse / procedure / pull 
	     / push / queue / return / say / signal / trace 
	     / t:'THEN' resync -> (ERROR[8.1] t) 
	     / t:'ELSE' resync -> (ERROR[8.2] t) 
    	     / t:'WHEN' resync -> (ERROR[9.1] t)
	     / t:'OTHERWISE' resync -> (ERROR[9.2] t);

command <- expression -> EXPRESSION;

group <- do / if / select ; 
\end{verbatim}

\subsubsection{DO Group}

\emph{TODO - When should Msg21.1 be raised?}
do \textless{}- do\_specification (ncl / t:.+ resync -\textgreater{} (ERROR{[}27.1{]} t)) instruction\_list? do\_ending -\textgreater{} DO ;

\begin{verbatim}
do_ending <-   'END' var_symbol? ncl 
		 / eos -> ERROR[14.1]
		 / t:.+ resync -> (ERROR[35.1] t);

do_specification <- 'DO' dorep 
		      / 'DO' docond 
		      / 'DO' dorep docond 
		      / 'DO' 'FOREVER' docond      
		      / 'DO' 'FOREVER' t:.+ resync -> (ERROR[25.16, "WHILE UNTIL"] t)
		      / 'DO' 'FOREVER' -> FOREVER;
		      / 'DO' ;

docond <-   'WHILE' e:expression -> (WHILE e)
          / 'UNTIL' e:expression -> (UNTIL e);

dorep <- ( assignment {? dot dob dof} / expression ) -> REPEAT;

dot <- 'TO' e:expression -> (TO e);

dob <- 'BY' e:expression -> (BY e);

dof <- 'FOR' e:expression -> (FOR e);
\end{verbatim}

\subsubsection{IF Group}

\begin{verbatim}
if <- 'IF' expression ncl? (then / ((. -> ERROR[18.1]) resync )) else? -> IF;

then <- 'THEN' ncl (instruction -> INSTRUCTIONS / eos -> ERROR{14.3] / 'END' -> ERROR[10.5] ) ;

else <- 'ELSE' ncl (instruction -> INSTRUCTIONS / eos -> ERROR[14.4] / 'END' -> ERROR[10.6] ) ;
\end{verbatim}

\subsubsection{SELECT Group}

\begin{verbatim}
select <- 'SELECT' ncl select_body ('END' (var_symbol -> ERROR[10.4])? ncl -> SELECT
                                 / eos -> ERROR[14.2] 
            		 / . -> ERROR[7.2] resync );

select_body <- (when / . -> ERROR[7.1] resync) when* otherwise?

when <- 'WHEN' expression ncl? (then / ((. -> ERROR[18.2]) resync)) -> WHEN;

otherwise <- 'OTHERWISE' ncl instruction_list? -> OTHERWISE;
\end{verbatim}

\subsubsection{ADDRESS}

\begin{verbatim}
address <- 'ADDRESS' [(taken_constant [expression] / Msg19.1 / valueexp) [ 'WITH' connection]]

taken_constant <- symbol / STRING

valueexp <- 'VALUE' expression

connection <- error [adio] / input [adeo] / output [adei] / Msg25.5

adio <- input [output] / output [input]

input <- 'INPUT' (resourcei / Msg25.6)

resourcei <- resources / 'NORMAL'

output <- 'OUTPUT' (resourceo / Msg25.7)

resourceo <- 'APPEND' (resources / Msg25.8) / 'REPLACE' (resources / Msg25.9) / resources / 'NORMAL'

adeo <- error [output] / output [error]

error <- 'ERROR' (resourceo / Msg25.14)

adei <- error [input] / input [error]

resources <- 'STREAM' (var_symbol / Msg53.1) / 'STEM' (var_symbol / Msg53.2)

vref <- '(' (var_symbol / (.->ERROR[20.1]) ) ( ')' / . -> ERROR[46.1] resync)
\end{verbatim}

\subsubsection{Arg}

\begin{verbatim}
arg <- 'ARG' [template_list]

call <- 'CALL' (callon_spec/ (taken_constant/Msg19.2)[expression_list])

callon_spec <- 'ON' (callable_condition / Msg25.1) ['NAME' (taken_constant / Msg19.3)] / 'OFF' (callable_condition / Msg25.2)

callable_condition<- 'ERROR' / 'FAILURE' / 'HALT' / 'NOTREADY'

expression_list <- expr / [expr] ',' [expression_list]
\end{verbatim}

\subsubsection{Drop}

\begin{verbatim}
drop <- 'DROP' variable_list

variable_list <- (vref / var_symbol)+
\end{verbatim}

\subsubsection{Exit}

\begin{verbatim}
exit <- 'EXIT' [expression]
\end{verbatim}

\subsubsection{Interpret}

\begin{verbatim}
interpret <- 'INTERPRET' expression
\end{verbatim}

\subsubsection{Iterate}

\begin{verbatim}
iterate <- 'ITERATE' [VAR_SYMBOL / Msg20.2 ]
\end{verbatim}

\subsubsection{Leave}

\begin{verbatim}
leave <- 'LEAVE' [VAR_SYMBOL / Msg20.2 ]
\end{verbatim}

\subsubsection{Nop}

\begin{verbatim}
nop <- 'NOP'
\end{verbatim}

\subsubsection{Numric}

\begin{verbatim}
numeric <- 'NUMERIC' (numeric_digits / numeric_form / numeric_fuzz / Msg25.15)

numeric_digits <- 'DIGITS' [expression]

numeric_form <- 'FORM' ('ENGINEERING' / 'SCIENTIFIC' / valueexp / Msg25.11)

numeric_fuzz <- 'FUZZ' [expression]
\end{verbatim}

\subsubsection{Options}

\begin{verbatim}
options <- 'OPTIONS' expression
\end{verbatim}

\subsubsection{Parse}

\begin{verbatim}
parse <- 'PARSE'(parse_type / Msg25.12)[template_list] / 'PARSE' 'UPPER' (parse_type / Msg25.13) [template_list]

parse_type <- parse_key / parse_value / parse_var

parse_key <- 'ARG' / 'PULL' / 'SOURCE' / 'LINEIN' / 'VERSION'

parse_value <- 'VALUE' [expression] ('WITH' / Msg38.3)

parse_var <- 'VAR' var_symbol
\end{verbatim}

\subsubsection{Procedure}

\begin{verbatim}
procedure <- 'PROCEDURE' ['EXPOSE' variable_list / Msg25.17]
\end{verbatim}

\subsubsection{Pull}

\begin{verbatim}
pull <- 'PULL' [template_list]
\end{verbatim}

\subsubsection{Push}

\begin{verbatim}
push <- 'PUSH' [expression]
\end{verbatim}

\subsubsection{Queue}

\begin{verbatim}
queue <- 'QUEUE' [expression]
\end{verbatim}

\subsubsection{Return}

\begin{verbatim}
return <- 'RETURN' [expression]
\end{verbatim}

\subsubsection{Say}

\begin{verbatim}
say <- 'SAY' [expression]
\end{verbatim}

\subsubsection{Signal}

\begin{verbatim}
signal <- 'SIGNAL' (signal_spec / valueexp / taken_constant / Msg19.4)

signal_spec <- 'ON' (condition / Msg25.3) ['NAME' (taken_constant / Msg19.3)] / 'OFF' (condition / Msg25.4)

condition <- callable_condition / 'NOVALUE' / 'SYNTAX' / 'LOSTDIGITS'
\end{verbatim}

\subsubsection{Trace}

\begin{verbatim}
trace <- 'TRACE' [(taken_constant / Msg19.6) / valueexp]
\end{verbatim}

\subsubsection{Parse Templates}

\begin{verbatim}
template_list <- template / [template] ',' [template_list]

template <- (trigger / target / Msg38.1)+

target <- VAR_SYMBOL / '.'

trigger <- pattern / positional

pattern <- STRING / vrefp

vrefp <- '(' (VAR_SYMBOL / Msg19.7) (')' / Msg46.1)

positional <- absolute_positional / relative_positional absolute_positional<- NUMBER / '=' position

position <- NUMBER / vrefp / Msg38.2

relative_positional<- ('+' / '-') position

symbol <- VAR_SYMBOL / CONST_SYMBOL / NUMBER
\end{verbatim}

\subsubsection{Expressions}

\begin{verbatim}
expression <- expr [(',' Msg37.1) / (')' Msg37.2 )]

expr <- expr_alias

expr_alias <- and_expression / expr_alias or_operator and_expression

or_operator <- '/' / '&&'

and_expression <- comparison / and_expression '&' comparison

comparison <- concatenation / comparison comparison_operator concatenation

comparison_operator<- normal_compare / strict_compare

normal_compare<- '=' / '\=' / '<>' / '><' / '>' / '<' / '>=' / '<=' / '\>' / '\<'

strict_compare<- '==' / '\==' / '>>' / '<<' / '>>=' / '<<=' / '\>>' / '\<<'

concatenation <- addition / concatenation (' ' / '//') addition
addition <- multiplication / addition additive_operator multiplication

additive_operator<- '+' / '-'

multiplication <- power_expression / multiplication multiplicative_operator power_expression

multiplicative_operator<- '*' / '/' / '//' / '%'

power_expression <- prefix_expression / power_expression '**' prefix_expression

prefix_expression <- ('+' / '-' / '\') prefix_expression / term / Msg35.1

term <- symbol / STRING / function / '(' expr_alias (',' Msg37.1 / ')' / Msg36)

function <- taken_constant '(' [expression_list] (')' / Msg3632
\end{verbatim}
