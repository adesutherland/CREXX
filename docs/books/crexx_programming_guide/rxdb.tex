\chapter{rxdb - the \crexx{} Debugger}\label{debugger}
The debugger is one of the programs in the toolchain delivered with \textsc{Rexx}
as its source code; most other programs, at the moment, are compiled
from C. It is easily adaptable and can be regarded a \emph{debugger
  construction set}. By adapting and recompiling the user can
implement their own wishes for a debugger. In this sense, it can be
seen as an open-ended complement to the \textsc{Rexx} \code{trace}
statement. Because it has modes for \textsc{Rexx} as well as
\code{rxas} Assembler, it is a useful tool for debugging
high-level as well as low-level problems.
\section{Command Line Options}
\fontspec{IBM Plex Mono}
\begin{shaded}
  \small
  \obeylines \splice{rxdb -h | sed "s/&/\and/g"}
 \end{shaded}
\fontspec{TeX Gyre Pagella}
\section{Runtime Options}
After the \code{rxdb} program is started, a few runtime options appear in the
delivered version. This is an example session:
