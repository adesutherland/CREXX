\chapter{Interlanguage calls}
A \crexx{} program is able to call programs written in the \textsc{Rexx}
language, but also programs native to the platform, using a number of
calling conventions: \[todo: checkrelease\]
\begin{description}
  \item[Address] the \code{address} statement can use the shell and
    I/O indirection to start native executables and provide input, and
    retrieve the output.
    \item[RexxSaa] the traditional RexxSAA calling convention can be
      used for direct interfaces to executables that are designed to
      function as a Rexx library. In its most simple form, these can
      return \textsc{Rexx} strings to the calling program.
      \item[Generic Call Interface] In this RexxSAA extension, the
      type and length of the parameters can be specified by the
      caller\footnote{Which is considered unsafe but sometimes the
        only possibility for programs not designed to be called by \crexx{}}.
    \end{description}
