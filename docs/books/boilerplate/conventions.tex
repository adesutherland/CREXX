\chapter{Typographical conventions}
In general, the following conventions have  been observed in the \crexx{} publications:
\begin{itemize}
\item Body text is in this font
\item Examples of language statements are in a \keyword{keyword} or \textbf{bold} type
\item Variables or strings as mentioned in source code, or things that appear on the console, are in a \texttt{typewriter} type
\item Items that are introduced, or emphasized, are in an \emph{italic} type
\item Included program fragments are listed in this fashion:
\begin{lstlisting}[label=example,caption=Example Listing]
-- salute the reader
say 'hello reader'
\end{lstlisting}
\end{itemize}
The small numbers in the left margin of the listing are meant for easy
reference to the source lines, and are not part of the program.
%%% Local Variables: 
%%% mode: latex
%%% TeX-master: t
%%% End: 
