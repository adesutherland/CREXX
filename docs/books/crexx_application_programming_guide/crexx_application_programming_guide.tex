\documentclass[10pt]{book}
\usepackage[FINAL]{../../../boilerplate/rexx} 
\usepackage{hyperref}
\usepackage{graphics}
\usepackage{geometry}
\usepackage{setspace}
\usepackage{etoolbox}
\usepackage{fontspec}
\usepackage{tocloft}
\usepackage{titlesec}
\setmainfont[Mapping=tex-text]{Georgia Pro}
\setmonofont[Mapping=tex-text,Scale=0.82]{IBM Plex Mono}
\usepackage{tabularx}
\usepackage{booktabs}
\usepackage{makeidx}
\usepackage{color}
\usepackage{xcolor}
\usepackage{listings}

%fancyhdr for page headers, footers, numbers
\usepackage{fancyhdr}

\pagestyle{fancy}
\fancyhf{} % Clear default header/footer settings

\fancyheadoffset{0pt}
\addtolength{\headsep}{15pt}

% Define the headers
\fancyhead[RO]{\normalfont \nouppercase{\rightmark}}  % Chapter title on right pages, aligned right
\fancyhead[LE]{\normalfont \nouppercase{\leftmark}}   % Section title on left pages, aligned right

% Ensure chapter and section marks update correctly
%% \renewcommand{\chaptermark}[1]{\markright{#1}\normalfont \thechapter.\ #1}
%% \renewcommand{\sectionmark}[1]{\markleft{#1}\normalfont \thesection.\ #1}

\fancyfoot[C]{\thepage}

% avoid headers on empty pages
\usepackage{emptypage}

\usepackage{caption}
\usepackage{longtable}
\usepackage{colortbl}
\usepackage{framed}
\usepackage{fancyvrb}
\definecolor{shadecolor}{rgb}{0.9,0.9,0.9}
\definecolor{nrblue}{RGB}{38,139,210}
\definecolor{nrgreen}{RGB}{65,133,153}
\definecolor{nrcyan}{RGB}{42,161,152 }
\definecolor{nrorange}{RGB}{203 ,75,22}
\definecolor{nrgrey}{RGB}{101,123,131}
\usepackage{alltt}
%% \DeclareCaptionFont{white}{\color{white}}
%% \DeclareCaptionFormat{listing}{\colorbox{gray}{\parbox{\textwidth}{#1#2#3}}}
%% \captionsetup[lstlisting]{format=listing,labelfont=white,textfont=white}
\usepackage[official]{eurosym}
%% \makeatletter
%% \@empty\z@\@empty
%% \lst@CCPutMacro\lst@ProcessOther {"2D}{\lst@ttfamily{-{}}{-{}}} 
%% \makeatother
%%\renewcommand*\familydefault{\sfdefault}

\lstdefinelanguage{NetRexx}
{morekeywords={abstract,adapter,binary,case,catch,class,constant,dependent,deprecated,digits,do,else,end,engineering,extends,final,finally,for,forever,if,implements,indirect,import,indirect,inheritable,interface,iterate,label,leave,loop,method,native,nop,numeric,options,otherwise,over,package,parent,parse,private,properties,protect,public,return,returns,rexx,say,scientific,set,digits,form,select,shared,signal,signals,sourceline,static,super,then,this,until,used,upper,volatile,when,where,while},
sensitive=false,
extendedchars=true,
morecomment=[s]={/*}{*/},
morecomment=[l]{--},
morecomment=[s]{/**}{*/},
morestring=[b]",
morestring=[d]",
morestring=[b]',
morestring=[d]'}

\lstset{language=NetRexx,
  captionpos=none,
  tabsize=3,
  alsolanguage=Rexx,
  keywordstyle=\color{nrorange},
  commentstyle=\color{nrgrey},
  stringstyle=\color{nrgreen},
  numbers=none,
  numberstyle=\tiny,
  numbersep=5pt,
  breaklines=true,
  showstringspaces=false,
  index=[1][keywords],
  columns=fixed,
  basicstyle=\fontsize{10}{10}\fontspec{Hack},emph={label}
}

%% \lstdefinelanguage
%%    {x64Assembler}     % add a "x64" dialect of Assembler
%%    [x86masm]{Assembler} % based on the "x86masm" dialect
%%    % with these extra keywords:
%%    {morekeywords={CDQE,CQO,CMPSQ,CMPXCHG16B,JRCXZ,LODSQ,MOVSXD, %
%%                   POPFQ,PUSHFQ,SCASQ,STOSQ,IRETQ,RDTSCP,SWAPGS, %
%%                   rax,rdx,rcx,rbx,rsi,rdi,rsp,rbp, %
%%                   r8,r8d,r8w,r8b,r9,r9d,r9w,r9b, %
%%                   r10,r10d,r10w,r10b,r11,r11d,r11w,r11b, %
%%                   r12,r12d,r12w,r12b,r13,r13d,r13w,r13b, %
%%                   r14,r14d,r14w,r14b,r15,r15d,r15w,r15b}} % etc.

%% \lstset{language=x64Assembler}

\lstdefinelanguage
  [x86]{Assembler}%
  {moreemph=[3]{al,ah,ax,eax,bl,bh,bx,ebx,cl,ch,cx,ecx,dl,dh,dx,edx,%
      si,esi,di,edi,bp,ebp,sp,esp,cs,ds,es,ss,fs,gs,cr0,cr1,cr2,cr3,%
      db0,db1,db2,db3,db4,db5,db6,db7,tr0,tr1,tr2,tr3,tr4,tr5,tr6,tr7,%
      st},
   morekeywords=[1]{aaa,aad,aam,aas,adc,add,and,arpl,bound,bsf,bsr,bswap,bt,btc,%
      btr,bts,call,cbw,cdq,clc,cld,cli,clts,cmc,cmp,cmps,cmpsb,cmpsw,%
      cmpsd,cmpxchg,cwd,cwde,daa,das,dec,div,enter,hlt,idiv,imul,in,%
      inc,ins,int,into,invd,invlpg,iret,ja,jae,jb,jbe,jc,jcxz,jecxz,%
      je,jg,jge,jl,jle,jna,jnae,jnb,jnbe,jnc,jne,jng,jnge,jnl,jnle,%
      jno,jnp,jns,jnz,jo,jp,jpe,jpo,js,jz,jmp,lahf,lar,lea,leave,lgdt,%
      lidt,lldt,lmsw,lock,lods,lodsb,lodsw,lodsd,loop,loopz,loopnz,%
      loope,loopne,lds,les,lfs,lgs,lss,lsl,ltr,mov,movs,movsb,movsw,%
      movsd,movsx,movzx,mul,neg,nop,not,or,out,outs,pop,popl,popq,popa,popad,%
      popf,popfd,push,pusha,pushad,pushf,pushfd,rcl,rcr,rep,repe,%
      repne,repz,repnz,ret,retf,rol,ror,sahf,sal,sar,sbb,scas,seta,%
      setae,setb,setbe,setc,sete,setg,setge,setl,setle,setna,setnae,%
      setnb,setnbe,setnc,setne,setng,setnge,setnl,setnle,setno,setnp,%
      setns,setnz,seto,setp,setpe,setpo,sets,setz,sgdt,shl,shld,shr,%
      shrd,sidt,sldt,smsw,stc,std,sti,stos,stosb,stosw,stosd,str,sub,%
      test,testl,testq,verr,verw,wait,wbinvd,xadd,xchg,xlatb,xor,xorl,xorq,fabs,fadd,fbld,%
      fbstp,fchs,fclex,fcom,fcos,fdecstp,fdiv,fdivr,ffree,fiadd,ficom,%
      fidiv,fidivr,fild,fimul,fincstp,finit,fist,fisub,fisubr,fld,fld1,%
      fldl2e,fldl2t,fldlg2,fldln2,fldpi,fldz,fldcw,fldenv,fmul,fnop,%
      fpatan,fprem,fprem1,fptan,frndint,frstor,fsave,fscale,fsetpm,%
      fsin,fsincos,fsqrt,fst,fstcw,fstenv,fstsw,fsub,fsubr,ftst,fucom,%
      fwait,fxam,fxch,fxtract,fyl2x,fyl2xp1,f2xm1},%
   morekeywords=[2]{.align,.alpha,assume,byte,code,comm,comment,.const,%
      .cref,.data,.data?,db,dd,df,dosseg,dq,dt,dw,dword,else,end,endif,%
      endm,endp,ends,eq,equ,.err,.err1,.err2,.errb,.errdef,.errdif,%
      .erre,.erridn,.errnb,.errndef,.errnz,event,exitm,extrn,far,%
      .fardata,.fardata?,function,fword,ge,group,gt,high,if,if1,if2,ifb,ifdef,%
      ifdif,ife,ifidn,ifnb,ifndef,include,includelib,irp,irpc,label,%
      .lall,le,length,.lfcond,.list,local,low,lt,macro,mask,mod,.model,%
      name,ne,near,offset,org,out,page,proc,ptr,public,purge,qword,.%
      radix,record,rept,.sall,seg,segment,.seq,.sfcond,short,size,%
      .stack,struc,subttl,tbyte,.tfcond,this,title,type,.type,width,%
      word,.xall,.xcref,.xlist},%
   alsoletter=.,alsodigit=?,alsodigit=\$,%
   otherkeywords={\%,\$,@},%
   sensitive=f,%
   morestring=[b]",%
   morestring=[b]',%
   morecomment=[l];%
   morecomment=[l]##,%
   }[keywords,comments,strings,emph]

\lstdefinelanguage
   [x86_64]{Assembler}     % add a "x86_64" dialect of Assembler
   [x86]{Assembler}        % based on the "x86" dialect
   % with these extra keywords:
   {morekeywords={cdqe,cqo,cmpsq,cmpxchg16b,jrcxz,lodsq,movsxd, %
                  popfq,pushfq,scasq,stosq,iretq,rdtscp,swapgs, %
                  movl,movq,pushl,pushq,subl,subq,cmpl,cmpq,    %
                  addl,addq,incl,incq,leal,leaq,                %
                  vmovsd, vaddsd, vsubsd, vmulsd, vdivsd,       %
                  vfmadd132sd,vfmadd213sd,vfmadd231sd,          %
                },                                   %
   moreemph=[3]{rax,rdx,rcx,rbx,rsi,rdi,rsp,rbp,rip, %
                r8,r8d,r8w,r8b,r9,r9d,r9w,r9b,       %
                r10,r10d,r10w,r10b,r11,r11d,r11w,r11b,%
                r12,r12d,r12w,r12b,r13,r13d,r13w,r13b,%
                r14,r14d,r14w,r14b,r15,r15d,r15w,r15b,%
      xmm0, xmm1, xmm2, xmm3, xmm4, xmm5, xmm6, xmm7,%
      ymm0, ymm1, ymm2, ymm3, ymm4, ymm5, ymm6, ymm7,%
    }} % etc.


\lstdefinelanguage
   {zAssembler}     % add a s390 dialect of Assembler
   [x86masm]{Assembler} 
   % with these extra keywords:
   {morekeywords={PRINT,GEN,END,LTORG,AMODE,RMODE,USING
                 }}

\lstset{language=zAssembler,
  tabsize=3,
  alsolanguage=rexx,
  keywordstyle=\color{blue},
  commentstyle=\color{nrgrey},
  stringstyle=\color{nrgreen},
  numbers=left,
  numberstyle=\tiny,
  numbersep=5pt,
  numberbychapter=false,
  breaklines=true,
  showstringspaces=false,
  index=[1][keywords],
  columns=fixed,
  basicstyle=\fontsize{9}{9}\fontspec{JuliaMono},emph={label}
}



% MIPS Assembly language definition for the LaTeX `listings' package
%
% The list of instructions and directives are those understood by the
% MARS MIPS Simulator [http://courses.missouristate.edu/KenVollmar/MARS/]
%
% Author: Eric Walkingshaw <eric@walkingshaw.net>
%
% This code is in the public domain.
%
% Here is an example style. I like it for slides, but you might want 
% something a bit less noisy for print. :)
%
% \definecolor{CommentGreen}{rgb}{0,.6,0}
% \lstset{
%   language=[mips]Assembler,
%   escapechar=@, % include LaTeX code between `@' characters
%   keepspaces,   % needed to preserve spacing with lstinline
%   basicstyle=\footnotesize\ttfamily\bfseries,
%   commentstyle=\color{CommentGreen},
%   stringstyle=\color{cyan},
%   showstringspaces=false,
%   keywordstyle=[1]\color{blue},    % instructions
%   keywordstyle=[2]\color{magenta}, % directives
%   keywordstyle=[3]\color{red},     % registers
% }

%% \ProvidesPackage{mips}

%% \RequirePackage{listings}

\lstdefinelanguage[mips]{Assembler}{%
  % so listings can detect directives and register names
  alsoletter={.\$},
  % strings, characters, and comments
  morestring=[b]",
  morestring=[b]',
  morecomment=[l]\#,
  % instructions
  morekeywords={[1]abs,abs.d,abs.s,add,add.d,add.s,addi,addiu,addu,%
    and,andi,b,bc1f,bc1t,beq,beqz,bge,bgeu,bgez,bgezal,bgt,bgtu,%
    bgtz,ble,bleu,blez,blt,bltu,bltz,bltzal,bne,bnez,break,c.eq.d,%
    c.eq.s,c.le.d,c.le.s,c.lt.d,c.lt.s,ceil.w.d,ceil.w.s,clo,clz,%
    cvt.d.s,cvt.d.w,cvt.s.d,cvt.s.w,cvt.w.d,cvt.w.s,div,div.d,div.s,%
    divu,eret,floor.w.d,floor.w.s,j,jal,jalr,jr,l.d,l.s,la,lb,lbu,%
    ld,ldc1,lh,lhu,li,ll,lui,lw,lwc1,lwl,lwr,madd,maddu,mfc0,mfc1,%
    mfc1.d,mfhi,mflo,mov.d,mov.s,move,movf,movf.d,movf.s,movn,movn.d,%
    movn.s,movt,movt.d,movt.s,movz,movz.d,movz.s,msub,msubu,mtc0,mtc1,%
    mtc1.d,mthi,mtlo,mul,mul.d,mul.s,mulo,mulou,mult,multu,mulu,neg,%
    neg.d,neg.s,negu,nop,nor,not,or,ori,rem,remu,rol,ror,round.w.d,%
    round.w.s,s.d,s.s,sb,sc,sd,sdc1,seq,sge,sgeu,sgt,sgtu,sh,sle,%
    sleu,sll,sllv,slt,slti,sltiu,sltu,sne,sqrt.d,sqrt.s,sra,srav,srl,%
    srlv,sub,sub.d,sub.s,subi,subiu,subu,sw,swc1,swl,swr,syscall,teq,%
    teqi,tge,tgei,tgeiu,tgeu,tlt,tlti,tltiu,tltu,tne,tnei,trunc.w.d,%
    trunc.w.s,ulh,ulhu,ulw,ush,usw,xor,xori},
  % assembler directives
  morekeywords={[2].align,.ascii,.asciiz,.byte,.data,.double,.extern,%
    .float,.globl,.half,.kdata,.ktext,.set,.space,.text,.word},
  % register names
  morekeywords={[3]\$0,\$1,\$2,\$3,\$4,\$5,\$6,\$7,\$8,\$9,\$10,\$11,%
    \$12,\$13,\$14,\$15,\$16,\$17,\$18,\$19,\$20,\$21,\$22,\$23,\$24,%
    \$25,\$26,\$27,\$28,\$29,\$30,\$31,%
    \$zero,\$at,\$v0,\$v1,\$a0,\$a1,\$a2,\$a3,\$t0,\$t1,\$t2,\$t3,\$t4,
    \$t5,\$t6,\$t7,\$s0,\$s1,\$s2,\$s3,\$s4,\$s5,\$s6,\$s7,\$t8,\$t9,%
    \$k0,\$k1,\$gp,\$sp,\$fp,\$ra},
}[strings,comments,keywords]

\lstdefinelanguage{ini}{
  morecomment=[l]\#,
  morecomment=[l]\;,
  morestring=[b]",
  moredelim=[s][\bfseries\color{blue}]{[}{]},
  alsoletter={=},
  morekeywords={true,false,on,off,yes,no},
  keywordstyle=\color{purple}\bfseries,
}

\lstdefinestyle{iniStyle}{
  language=ini,
  basicstyle=\ttfamily\small,
  commentstyle=\color{gray}\itshape,
  stringstyle=\color{green!50!black},
  showstringspaces=false,
  breaklines=true
}

\lstdefinelanguage{rxas}
{morekeywords={
appendchar,
addf,
addi,
addi,
amap,
amap,
and,
append,
addf,
bcf,
bcf,
bct,
bct,
bctnm,
bctnm,
beq,
bge,
bge,
bgt,
bgt,
ble,
ble,
blt,
blt,
bne,
bne,
bpoff,
bpon,
br,
brf,
brt,
brtpandt,
beq,
brtpt,
brtf,
call,
cnop,
call,
concchar,
concat,
concat,
concat,
call,
copy,
divf,
dec,
dec0,
dcall,
dec2,
dec1,
divf,
divi,
divi,
dllparms,
dropchar,
divf,
exit,
exit,
exit,
fadd,
fadd,
fcopy,
fdiv,
fdiv,
fdiv,
feq,
fformat,
fgt,
fgt,
fgt,
fgte,
fgte,
fgte,
flt,
flt,
flt,
flte,
flte,
flte,
fmult,
fmult,
fndblnk,
fndnblnk,
fne,
fne,
fpow,
fpow,
fsex,
fsub,
fsub,
fsub,
ftob,
ftoi,
ftos,
feq,
getstrpos,
gmap,
gettp,
gmap,
getbyte,
hexchar,
iadd,
iadd,
igt,
iand,
icopy,
idiv,
idiv,
idiv,
ieq,
ieq,
igt,
igte,
iand,
igte,
igte,
ilt,
ilt,
ilt,
ilte,
ilte,
ilte,
imod,
imod,
imod,
imult,
imult,
inc,
inc0,
inc1,
ine,
igt,
ine,
inot,
inot,
ior,
ior,
ipow,
ipow,
ipow,
irand,
irand,
isex,
ishl,
ishl,
ishr,
ishr,
isub,
isub,
isub,
itof,
itos,
ixor,
ixor,
inc2,
linkattr,
load,
linkattr,
loadsettp,
load,
load,
load,
link,
loadsettp,
loadsettp,
metaloadedmodules,
map,
map,
metadecodeinst,
metaloadcalleraddr,
metaloaddata,
metaloadedprocs,
metaloadfoperand,
metaloadinst,
metaloadioperand,
metaloadmodule,
metaloadpoperand,
metaloadsoperand,
move,
mtime,
multf,
multi,
metalinkpreg,
multi,
multf,
nsmap,
nsmap,
nsmap,
null,
nsmap,
not,
opendll,
or,
padstr,
pmap,
pmap,
poschar,
ret,
ret,
ret,
ret,
ret,
ret,
rseq,
rseq,
readline,
sappend,
say,
say,
say,
say,
sayx,
sayx,
sconcat,
sconcat,
sconcat,
scopy,
seq,
seq,
setortp,
setstrpos,
settp,
sgt,
sgt,
sgt,
sgte,
sgte,
sgte,
slt,
slt,
slt,
slte,
slte,
slte,
sne,
sne,
stof,
stoi,
strchar,
strchar,
strlen,
strlower,
strupper,
subf,
subf,
subf,
subi,
subi,
substcut,
substr,
substring,
swap,
say,
time,
transchar,
triml,
triml,
trimr,
trimr,
trunc,
trunc,
unlink,
unmap,
xtime,
  },
sensitive=false,
extendedchars=true,
morecomment=[s]={/*}{*/},
morecomment=[l]{*},
morecomment=[s]{/**}{*/},
morestring=[b]",
morestring=[d]",
morestring=[b]',
morestring=[d]'}

\lstset{language=rxas,
  captionpos=none,
  tabsize=3,
  keywordstyle=\color{nrorange},
  commentstyle=\color{nrgrey},
  stringstyle=\color{nrgreen},
  numbers=none,
  numberstyle=\tiny,
  numbersep=5pt,
  breaklines=true,
  showstringspaces=false,
  index=[1][keywords],
  columns=fixed,
  basicstyle=\fontsize{10}{10}\fontspec{Hack},emph={label}
}
%% \input{../../../boilerplate/jclformat}
\usepackage{pst-barcode,pstricks-add}
\usepackage{bashful}
\usepackage{metalogo}
\usepackage{marginnote}
\usepackage{pdfpages}
\usepackage{svg}
\usepackage{float}
\makeindex
\DeclareGraphicsExtensions{.jpg,.png}
\setlength{\parskip}{8pt}
\setlength{\parindent}{0pt}
\usepackage{enumitem}
\usepackage{underscore}
\usepackage{babel}
\usepackage{graphicx}
\usepackage{array}
% \usepackage{tabularx}
\usepackage{multirow}
\usepackage{threeparttable}
\usepackage{longtable}
\usepackage{booktabs}
\usepackage{makeidx}
\usepackage{framed}
\usepackage{alltt}
\usepackage{barracuda}
\usepackage{longtable}
\usepackage{graphicx}
\usepackage{wrapfig}
\usepackage{multicol}
\usepackage[CJK]{ucharclasses}
\usepackage[comma]{natbib}           % CSLI Pubs favored bibliography package.
%% \bibliographystyle{cslipubs-natbib}  % CSLI Pubs bibliography format.
\bibliographystyle{plainnat}  % CSLI Pubs bibliography format.
% defines for consistent use
\newcommand{\crexx}{\textsc{crexx}}
\newcommand{\rexx}{R\textsc{exx}}
%% \newcommand{\nr}{Net\textsc{rexx}}
%% \newcommand{\nrpackagename}{\splice{java GetPackageName}}
%% \newcommand{\minimalJVMversion}{8}
%% \newcommand{\maximalJVMversion}{19ea}
\newcommand{\keyword}[1]{\texttt{#1}}
\newcommand{\code}[1]{\texttt{#1}}
%% \newcommand{\thisyear}{\splice{java TexYear}}
\newcommand{\tightlist}{}
\newcommand{\Shaded}{\shaded}
\newcommand{\msd}[1]{\msdhelper#1\relax}
\newcommand{\msdhelper}[1]
  {\ifx\relax#1\else
    \ifx-#1--{}\else#1\fi
    \expandafter\msdhelper\fi}
  \newcommand{\doublehyphen}{\mbox{\msd{``-~-''}}}
  \newcommand{\doublehyphenunquoted}{\mbox{\msd{-~-}}}

\newcommand{\simularelease}{12.00, date of release  9 aug 1985}
%% This books starts at chapter 0  
\setcounter{chapter}{-1}
%% the font used for tables
\AtBeginEnvironment{longtable}{\fontspec{1403 Vintage Mono Pro}\small}

%% \setlength{\labelsep}{2em} % Distance between label and text
\setlength{\labelwidth}{75pt} % Width of the label box  

\setlist[description]{font=\normalfont}

\usepackage{sectsty}
%% \partfont{\color{nrgreen}}
\chapterfont{\fontspec{Avenir Light}}
\sectionfont{\fontspec{Avenir Light}}
\subsectionfont{\fontspec{Avenir Light}}
\subsubsectionfont{\fontspec{Avenir Light}}

\usepackage[Sonny]{fncychap}

\raggedbottom

\geometry{left=1in, top=1in, bottom=1in, right=1in}
\setstretch{1.2}
\titlespacing*{\section}{0pt}{*1.5}{*0.5}
\titlespacing*{\subsection}{0pt}{*1.2}{*0.4}

\setlength{\parskip}{0.5em}
\raggedbottom

\makeatletter
\patchcmd{\@startsection}{\@afterindenttrue}{\@afterindentfalse}{}{}
\makeatother

%% heading number left overhang
\makeatletter
\def\@seccntformat#1{\protect\makebox[0pt][r]{\csname
the#1\endcsname\quad}}
\makeatother

%% use the heading fonts in the TOC
%% \renewcommand{\cftsecfont}{\sffamily\bfseries} % Matches section font
%% \renewcommand{\cftsubsecfont}{\sffamily} % Example for subsections

\renewcommand{\cftpartfont}{\sffamily}
\renewcommand{\cftchapfont}{\sffamily} % Matches section font
\renewcommand{\cftchappagefont}{\sffamily} % Matches section font
\renewcommand{\cftsubsecfont}{\sffamily} % Example for subsections
\renewcommand{\cftsubsubsecfont}{\sffamily} % Example for subsections
\renewcommand{\cftsecfont}{\sffamily} % Matches section font

\titleformat{name=\chapter,numberless}[display]
  %{\fontspec{Avenir Light}\filcenter} % formatting applied to the
  % heading text
    {\filcenter} % formatting applied to the heading text
  {}
  {0pt}
  %% {\titlerule \vspace{1ex} \fontspec{Avenir Light} } 
 {\titlerule \vspace{1ex} }  % code before the heading text

\titlespacing*{name=\chapter,numberless}
  {0pt}{-1cm}{1cm}  % left, before, after

  \renewcommand{\contentsname}{\fontspec{Avenir Light}Contents}
  \renewcommand{\indexname}{\fontspec{Avenir Light}Index}
  \renewcommand{\listtablename}{\fontspec{Avenir Light}List of tables}
  \renewcommand{\listfigurename}{\fontspec{Avenir Light}List of figures}
  
  \sectionfont{%
    \hrule                % place a horizontal line
    \vspace{1ex}%         % add some vertical space
    \fontspec{Avenir Light}       % set section title size & style
  }


  


\begin{document}
\renewcommand{\isbn}{978-90-819090-1-3}
\setcounter{tocdepth}{1}
\title{\fontspec{TeX Gyre
    Pagella}\textsc{crexx}\protect\fontspec{Bodoni URW
    Light}\\Application Programming Guide}

\author{The \crexx{} team}
%\date{\null\hfill Version \splice{java org.netrexx.process.NrVersion} of \today}
\date{\null\hfill \today}
\maketitle
\pagenumbering{Roman}
\pagestyle{plain}
\frontmatter
\pagenumbering{Roman}
\pagestyle{plain}
\section*{Publication Data}
\textcopyright  Copyright The Rexx Language Association, 2011-\splice{java TexYear}
%\\

All original material in this publication is published under the Creative Commons - Share Alike 3.0 License as stated at \url{http://creativecommons.org/licenses/by-nc-sa/3.0/us/legalcode}.\\[0.5cm]
The responsible publisher of this edition is identified as \emph{IBizz IT Services and Consultancy}, Amsteldijk 14, 1074 HR Amsterdam, a registered company governed by the laws of the Kingdom of The Netherlands.\\[1cm]
This edition is registered under ISBN \isbn \\[1cm]
\psset{unit=1in}
\begin{pspicture}(3.5,1in)
  \psbarcode{\isbn}{includetext guardwhitespace}{isbn}
\end{pspicture}
\newpage
%%% Local Variables:
%%% mode: latex
%%% TeX-master: t
%%% End:

\tableofcontents

\newpage
\pagenumbering{arabic}
\frontmatter
\large
\chapter{The \crexx{} Programming Series}
This book is part of a library, the \emph{\crexx{} Programming Series}, documenting the \crexx{} programming language and its use and applications. This section lists the other publications in this series, and their roles. These books can be ordered in convenient hardcopy and electronic formats from the Rexx Language Association.
\newline
\newline
\begin{tabularx}{\textwidth}{>{\bfseries}lX}
\toprule
User Guide & This guide is meant for an audience that has done some programming and wants to start quickly. It starts with a quick tour of the language, and a section on installing the \crexx{} translator and how to run it. It also contains help for troubleshooting if anything in the installation does not work as designed, and states current limits and restrictions of the open source reference implementation.
\\\midrule
Application Programming Guide & The Application Programming Guide
explains the working of the tools and has examples for building
programs on the platforms it supports.
\\\midrule
Language Reference & Referred to as the CRL, this is meant as the formal definition for the language, documenting its syntax and semantics, and prescribing minimal functionality for language implementers.
\\\midrule
The \crexx{} VM Specification & The \crexx{} VM
Specification, documents the \crexx{} Assembly Language and its execution
by the \crexx{} Virtual Machine. It also contains low level virtual
machine and ABI specifications.
\\\bottomrule
\end{tabularx}
%%% Local Variables: 
%%% mode: latex
%%% TeX-master: t
%%% End: 

\chapter{Typographical conventions}
In general, the following conventions have  been observed in the \crexx{} publications:
\begin{itemize}
\item Body text is in this font
\item Examples of language statements are in a \keyword{keyword} or \textbf{bold} type
\item Variables or strings as mentioned in source code, or things that appear on the console, are in a \texttt{typewriter} type
\item Items that are introduced, or emphasized, are in an \emph{italic} type
\item Included program fragments are listed in this fashion:
\begin{lstlisting}[label=example,caption=Example Listing]
-- salute the reader
say 'hello reader'
\end{lstlisting}
\end{itemize}
The small numbers in the left margin of the listing are meant for easy
reference to the source lines, and are not part of the program.
%%% Local Variables: 
%%% mode: latex
%%% TeX-master: t
%%% End: 


\mainmatter
\def\tightlist{}


\chapter*{About This Book}
The \crexx{} language is a further development, and variant of the
\textsc{Rexx} language\footnote{Cowlishaw, 1979}. This book aims to
document the workings of this implementation and serves as reference
for users and implementors alike.

\section*{Application Programming Guide}

This \emph{Application Programming Guide} focuses on documenting the
tools delivered with \crexx{}, from a usage perspective. The technical
background and design documentation is in the \emph{\crexx{} VM
  Specification} (this includes information on how to build the tools
from source), while the programming language itself is defined in
the \emph{\crexx{} Language Definition} document.

This document includes practical examples, best practices, and code
snippets that illustrate how to perform common tasks, such as setting
up the environment, handling data input and output, processing user
input, and interacting with other software components.

\section*{Audience}
This application programming guide is aimed at developers who want to use \crexx{} to create an application or integrate it into an existing system.

\section*{History}

\begin{description}
\item[mvp] This version documents the Minimally Viable Product
  release, Q1 2022 and is intended for developers only. It documents
  the toolchain for \crexx{} level B, which is a typed
  subset of Classic Rexx.
\item[f29] First version, Git feature [F0029]
\end{description}


\part{Guide}
\chapter{Running cRexx on Linux and macOS}
Linux and other Unix-like operating systems like Apple macOS behave in
an identical way when compiling, linking and running a \crexx{}
program.
\hypertarget{running-crexx-entails-compiling}{%
\section{Running cRexx entails
compiling}\label{running-crexx-entails-compiling}}

All Rexx scripts you run with cRexx are compiled by \texttt{rxc} into
Rexx assembler code (.rxas) and then assembled into an .rxbin file,
which contains the Rexx bytecode for execution by the Rexx Virtual
Machine \texttt{rxvm}. During both the compliation and assembler steps
optimization of the code takes place. The rxvm executable takes care of
linking separately compiled modules together and executing them, so one
function can find another.

\hypertarget{a-first-program---hello-world}{%
\section{A first program - Hello
World}\label{a-first-program---hello-world}}

Let's say you have a Rexx exec you would like to run. To not have any
surprises, it is of the \emph{hello world} kind. We have a file called
hello.rexx, containing:

\lstinputlisting[language=rexx,label=hello_example]{examples/hello.rexx}
\fontspec{IBM Plex Mono}
\splice{rxc examples/hello}
\splice{rxas examples/hello}
\begin{shaded}
  \small
\obeylines \splice{rxvm examples/hello}
\end{shaded}
\fontspec{TeX Gyre Pagella}

When cRexx level `C' (for `Classic') is available, the `options levelb'
(on line 2) statement can be left out; for the moment, level B is all we
have, and the compiler will refuse to compile without it.

With all our cRexx executables on the PATH, we only need to do:

\begin{verbatim}
rxc hello
rxas hello
rxvm hello
\end{verbatim}

to see `hello cRexx world!' on the console, as include above this, run
from the included program. Like all the programs in the \crexx{}
documentation, these programs are compiled and run from the included
source by the process that builds the document.

It might be a good idea to make a shell script to execute these three
programs in succession, and perhaps call it `crexx'. But take into
account that this really is a very simple case, in which no built-in
functions are called. We can have a look at the generated rexx assembler
(hello.rxas) file:

\lstinputlisting[language=rxas,label=hellorxas_example]{examples/hello.rxas}

and you can see here that the compiler actually has generated a `say'
assembler instruction for the Rexx `say' instruction. (Assembler became
a whole lot easier with Rexx assembler.) But we did not yet call any
function.

\hypertarget{using-built-in-functions}{%
\section{Using built-in functions}\label{using-built-in-functions}}

Most Rexx programs use the extremely well designed built-in functions.
Now with these functions written in Rexx, and not hidden in the compiler
somewhere, we must tell it to import those from the library where we put
them earlier during the build process. Let's say we want to add a
display of the current weekday to our hello program. This will now be:

\lstinputlisting[language=rexx,label=hellodateexample]{examples/hellodate.rexx}

Never mind the import statement, which you will not need when cRexx
`Classic' level C is available. But in level B, we need this, because we
need the flexibility it affords our plans for the future. See more
about \code{import} on page \pageref{intraImport}.

We must tell the compiler where to find the signature of the date()
function, so it can check if we call it in the correct way, with the
right parameters. This is done with the -i switch, which points to the
directory containing the library - which is called `library', by the
way.

\begin{verbatim}
rxc -i ~/crexx-build/lib/rxfns hellodate
\end{verbatim}

[todo]
% \splice{rxc -i /Users/apps/crexx_release/lib/rxfns examples/hellodate}
% \splice{rxas examples/hellodate}


The assembler runs unchanged, because it trusts the compiler to have
checked if the called function really sits in that library, and has the
right parameters - the right code to call it has been generated.

\begin{verbatim}
rxas hellodate
\end{verbatim}

To run it, we can employ the \texttt{rxvme} executable - this one is
extended with linked-in versions of all the functions in the library:

\begin{verbatim}
rxvme hellodate
\end{verbatim}

which yields:

\fontspec{IBM Plex Mono}
\begin{shaded}
  \small
\obeylines \splice{rxvme examples/hellodate}
\end{shaded}
\fontspec{TeX Gyre Pagella}

\hypertarget{building-a-standalone-executable}{%
\section{Building a standalone
executable}\label{building-a-standalone-executable}}

It is possible to build a standalone executable of this program. It is
possible to run your compiled Rexx program, e.g.~from a USB stick,
without ever installing crexx, on the same OS and instruction set
architecture.

For this, we need the next set of commands, expressed as a Rexx exec:

\begin{verbatim}
/* rexx compile a rexx exec to a native executable */
/* Classic Rexx and NetRexx compatible             */
crexx_home='~/crexx-build'
if arg='' then do
  say 'exec name expected.'
  exit 99
end
parse arg execName'.'extension
if extension<>'' then say 'filename extension ignored.'
'rxc  -i' crexx_home'/lib/rxfns' execName
'rxas' execName
'rxcpack' execName crexx_home'/lib/rxfns/library'
'gcc -o' execName,
'-lrxvml -lmachine -lavl_tree -lplatform -lm -L',
crexx_home'/interpreter -L'crexx_home'/machine -L',
crexx_home'/avl_tree -L'crexx_home'/platform'  execName'.c'
\end{verbatim}

This exec is delivered in the source tree, bin directory.

The exec works by compiling the Rexx program specified (again without
the .rexx file extension) to an .rxbin Rexx bytecode file, which is then
serialized to a C source file, containing the cRexx Virtual Machine and
the library rxbin files, by the rxcpack command. It is a rather peculiar
looking C source, but nevertheless it will compile to a working
executable, which is done by the last step in the exec, here using the
gcc compiler. And you will be able to run it without the overhead of
checking, compiling, tokenizing to bytecode and linking, so it will be
quite fast:

\begin{verbatim}
hello cRexx world!
today is Saturday
./hello  0.00s user 0.00s system 61% cpu 0.009 total
\end{verbatim}

\chapter{Running cRexx on Windows operating systems}
\chapter{Running cRexx on VM/370CE}
\chapter{Interlanguage operability}
\part{Reference}
\chapter{\crexx{} Compiler}
\section{Command Line Options}
\fontspec{IBM Plex Mono}
\begin{shaded}
  \small
  \obeylines \splice{rxc -h | sed "s/&/\and/g"}
 \end{shaded}
\fontspec{TeX Gyre Pagella}
\chapter{\crexx{} Assembler}
\section{Overview}
The purpose of the \crexx{} assembler is to translate a text file with
\emph{rxvm} assembler instructions to a file with binary contents containing these
instructions in their binary, executable form. Its main use is to
translate an \emph{.rxas} file produced by the \crexx{} compiler
\emph{rxc} to a binary \emph{.rxbin} file.

\section{Program Structure}

 Describe the structure of the program, including the main components, how they interact, and any key algorithms or data structures used.

\section{Input/Output}

: Specify the format and structure of any input and output files used by the program, and provide examples of how to use them.

\section{Command Line Arguments}
When the command line argument -h is specified the options are shown:\\
\fontspec{IBM Plex Mono}
\begin{shaded}
  \small
  \obeylines \splice{rxas -h | sed "s/&/\and/g"}
 \end{shaded}
\fontspec{TeX Gyre Pagella}
\section{Assembler Directives}
For machine instructions, see the \emph{\crexx{} VM
  Specification}. This section discusses instructions to the
assembler, which are called \emph{directives} to clearly distinguish
them from virtual machine instruction. These are necessary to pass information into the compiled
\emph{.rxbin} binary file, to enable execution by the \emph{rxvm}
virtual machine.

\section{Examples}

: Provide several examples of how to use the assembler program to
create assembly code, assemble it, and link it into executable code.

\section{In-line assembly}
The \crexx{} compiler \emph{rxc} enables\footnote{In \crexx{} level b} inline assembly through the
\textbf{assembler} statement. When used in this way, a lot of
the complications of the assembly language can be handled by the
\crexx{} compiler, like assigning registers to variables, and the
conversion of datatypes like \emph{integer} for display as \emph{string}.

\lstinputlisting[language=rexx,label=ipow_example,caption=ipow
example.]{examples/pow.rexx}
\fontspec{IBM Plex Mono}
\begin{shaded}
  \small
\splice{rxc examples/pow} \obeylines
\splice{rxas examples/pow} \obeylines \splice{rxvm examples/pow}
 \end{shaded}
\fontspec{TeX Gyre Pagella}

This is a simple and straightforward way to complement the low level
assembler instructions with the power of the \textsc{Rexx}
language. The following example intends to explain how this is
implemented; it can be skipped without consequences.

In this example, the compiler generates the following assembler source:
\lstinputlisting[language=rxas,label=ipow_rxas_example,caption=ipow
rxas example.]{examples/pow.rxas}

The \code{.src} directives indicate where the work is done. The
variables are assigned, as integers, to the registers \code{r1} and
\code{r2}. The line \code{ipow, number, power} becomes \code{ipow
  r3,r1,r2}, and the display on the terminal is handled by the
  \code{itos},\code{sconcat} and \code{say} instructions.

\section{Troubleshooting}

: Include a section on troubleshooting common issues with the assembler program, along with tips for resolving them.

\section{References}

: Provide a list of references used in creating the documentation, such as manuals or online resources for the assembler program, and any other relevant sources.

\chapter{\crexx{} Disassembler}
A disassembler reverses the actions of an assembler; in this case it
delivers a disassembly which in itself can be re-assembled - and still works.

\chapter{\crexx{} Debugger}
\chapter{\crexx{} C Packer}
The C Packer program converts the \emph{.rxbin} files into a C
language structure which links together all needed modules, and a
large part of the Virtual Machine infrastructure, which file then can
be compiled and link edited by the C compiler. GCC and Clang are the
targeted compiler toolchains for Linux, macOS and Windows.
\section{Command Line Options}
\fontspec{IBM Plex Mono}
\begin{shaded}
  \small
  \obeylines \splice{rxcpack -h | sed "s/&/\and/g"}
 \end{shaded}
\fontspec{TeX Gyre Pagella}

\backmatter
\listoftables
%\listoffigures
%\lstlistoflistings
\printindex
\clearpage
\psset{unit=1in}
\begin{pspicture}(3.5,1in)
  \psbarcode{\isbn}{includetext guardwhitespace}{isbn}
\end{pspicture}
\end{document}
