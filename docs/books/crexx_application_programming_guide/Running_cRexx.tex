\hypertarget{running-crexx-entails-compiling}{%
\section{Running cRexx entails
compiling}\label{running-crexx-entails-compiling}}

All Rexx scripts you run with cRexx are compiled by \texttt{rxc} into
Rexx assembler code (.rxas) and then assembled into an .rxbin file,
which contains the Rexx bytecode for execution by the Rexx Virtual
Machine \texttt{rxvm}. During both the compliation and assembler steps
optimization of the code takes place. The rxvm executable takes care of
linking separately compiled modules together and executing them, so one
function can find another.

\hypertarget{a-first-program---hello-world}{%
\section{A first program - Hello
World}\label{a-first-program---hello-world}}

Let's say you have a Rexx exec you would like to run. To not have any
surprises, it is of the \emph{hello world} kind. We have a file called
hello.rexx, containing:

\lstinputlisting[language=rexx,label=hello_example]{examples/hello.rexx}
\fontspec{IBM Plex Mono}
\splice{rxc examples/hello}
\splice{rxas examples/hello}
\begin{shaded}
  \small
\obeylines \splice{rxvm examples/hello}
\end{shaded}
\fontspec{TeX Gyre Pagella}

When cRexx level `C' (for `Classic') is available, the `options levelb'
(on line 2) statement can be left out; for the moment, level B is all we
have, and the compiler will refuse to compile without it.

With all our cRexx executables on the PATH, we only need to do:

\begin{verbatim}
rxc hello
rxas hello
rxvm hello
\end{verbatim}

to see `hello cRexx world!' on the console, as include above this, run
from the included program. Like all the programs in the \crexx{}
documentation, these programs are compiled and run from the included
source by the process that builds the document.

It might be a good idea to make a shell script to execute these three
programs in succession, and perhaps call it `crexx'. But take into
account that this really is a very simple case, in which no built-in
functions are called. We can have a look at the generated rexx assembler
(hello.rxas) file:

\lstinputlisting[language=rxas,label=hellorxas_example]{examples/hello.rxas}

and you can see here that the compiler actually has generated a `say'
assembler instruction for the Rexx `say' instruction. (Assembler became
a whole lot easier with Rexx assembler.) But we did not yet call any
function.

\hypertarget{using-built-in-functions}{%
\section{Using built-in functions}\label{using-built-in-functions}}

Most Rexx programs use the extremely well designed built-in functions.
Now with these functions written in Rexx, and not hidden in the compiler
somewhere, we must tell it to import those from the library where we put
them earlier during the build process. Let's say we want to add a
display of the current weekday to our hello program. This will now be:

\lstinputlisting[language=rexx,label=hellodateexample]{examples/hellodate.rexx}

Never mind the import statement, which you will not need when cRexx
`Classic' level C is available. But in level B, we need this, because we
need the flexibility it affords our plans for the future. See more
about \code{import} on page \pageref{intraImport}.

We must tell the compiler where to find the signature of the date()
function, so it can check if we call it in the correct way, with the
right parameters. This is done with the -i switch, which points to the
directory containing the library - which is called `library', by the
way.

\begin{verbatim}
rxc -i ~/crexx-build/lib/rxfns hellodate
\end{verbatim}

[todo]
% \splice{rxc -i /Users/apps/crexx_release/lib/rxfns examples/hellodate}
% \splice{rxas examples/hellodate}


The assembler runs unchanged, because it trusts the compiler to have
checked if the called function really sits in that library, and has the
right parameters - the right code to call it has been generated.

\begin{verbatim}
rxas hellodate
\end{verbatim}

To run it, we can employ the \texttt{rxvme} executable - this one is
extended with linked-in versions of all the functions in the library:

\begin{verbatim}
rxvme hellodate
\end{verbatim}

which yields:

\fontspec{IBM Plex Mono}
\begin{shaded}
  \small
\obeylines \splice{rxvme examples/hellodate}
\end{shaded}
\fontspec{TeX Gyre Pagella}

\hypertarget{building-a-standalone-executable}{%
\section{Building a standalone
executable}\label{building-a-standalone-executable}}

It is possible to build a standalone executable of this program. It is
possible to run your compiled Rexx program, e.g.~from a USB stick,
without ever installing crexx, on the same OS and instruction set
architecture.

For this, we need the next set of commands, expressed as a Rexx exec:

\begin{verbatim}
/* rexx compile a rexx exec to a native executable */
/* Classic Rexx and NetRexx compatible             */
crexx_home='~/crexx-build'
if arg='' then do
  say 'exec name expected.'
  exit 99
end
parse arg execName'.'extension
if extension<>'' then say 'filename extension ignored.'
'rxc  -i' crexx_home'/lib/rxfns' execName
'rxas' execName
'rxcpack' execName crexx_home'/lib/rxfns/library'
'gcc -o' execName,
'-lrxvml -lmachine -lavl_tree -lplatform -lm -L',
crexx_home'/interpreter -L'crexx_home'/machine -L',
crexx_home'/avl_tree -L'crexx_home'/platform'  execName'.c'
\end{verbatim}

This exec is delivered in the source tree, bin directory.

The exec works by compiling the Rexx program specified (again without
the .rexx file extension) to an .rxbin Rexx bytecode file, which is then
serialized to a C source file, containing the cRexx Virtual Machine and
the library rxbin files, by the rxcpack command. It is a rather peculiar
looking C source, but nevertheless it will compile to a working
executable, which is done by the last step in the exec, here using the
gcc compiler. And you will be able to run it without the overhead of
checking, compiling, tokenizing to bytecode and linking, so it will be
quite fast:

\begin{verbatim}
hello cRexx world!
today is Saturday
./hello  0.00s user 0.00s system 61% cpu 0.009 total
\end{verbatim}
